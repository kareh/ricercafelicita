\chapter{Prefazione}
Da ragazzo sapevo che esisteva qualcosa oltre la monota routine della scuola, dei compiti, delle esauste battute e dei logori scherzi tra compagni. Cosi', mi ero messo a cercare la ``vita vera'' nell'informatica e nella matematica, dove, ogni giorno che passava, conoscevo cose arcane e stupende. Ma non sarebbe bastata tutta la conoscenza di Eulero ed Erdos per parlare al mio cuore della vera vita. Ne' era poi cosi' difficile avvicinarsi ad essa. Bastava ``arrendersi a Dio'' e mettersi in gioco per quanto la propria ed altrui anima chiedeva. Oggi, dopo venti anni, capisco com'e' la ``vita vera''. Non l'ho pienamente raggiunta, ma la capisco e, con il costo di costanti piccoli e grandi impegni e sacrifici, la amo sempre piu'.


La vita vera esiste. Ce ne ricordiamo quando leggiamo un bel libro, che ci coinvolge e ci guida lungo una riflessione, lungo un viaggio.  La vita vera e' contenuta dentro di noi. Non occorre andare ad abitare lontano, conquistare un titolo altisonante od ottenere gli oggetti dei propri desideri. E' sufficiente diventare grandi, senza dover essere super-man. Fermarsi e fare un dono a chi e' piu' piccolo, nonostante si stia attraversando un periodo stressante, e' lo stesso gesto di un eroe che, nonostante sia dedito e preoccupato per la sua rischiosa impresa, si ferma nel suo cammino per esaudire il desiderio di un bambino.

Raggiungere la vera vita e', tuttavia, l'impresa piu' difficile, stressante e piu' lunga che si possa immaginare. Praticamente nessuno sa' come raggiungerla, altrimenti il mondo sarebbe gia' in Pace, ma ognuno e' in grado di mettersi in cammino per trovarla, di impegnarsi per costruirla, di pazientare per coltivarla.

Un esempio, e' cercare la vera vita nel lavoro.

Chi ricerca la vera vita lavorando, si alza anche la ventesima mattina, dopo averlo fatto gia' 19 giorni nello stesso mese, e va a lavorare, nonostante farebbe bene al suo fegato fare una semplice passeggiata dal fruttivendolo o
stringere la mano del suo amore e stare cosi' una mezz'ora.

Andare a lavoro perche' 
\begin{enumerate}
    \item il proprio capo probabilmente non capirebbe, dato che non ha cognizione delle nostre esigenze. E avere cognizione delle esigenze di ogni dipendente non e' una cosa facile ed e' molto impegnativa.
    \item Perche', noi non abbiamo cognizione completa dei piani dei nostri capi e di quello che stanno facendo i colleghi. Delle difficolta' che sta' affrontando l'azienda, di quello che vuole fare per il prossimo futuro.\\
        Magari piu' o meno lo sappiamo, ma non sappiamo nel dettaglio le cose, e nel lavoro i dettagli sono fondamentali.\\
        Se potessimo farlo e ci assentassimo o ritardassimo a nostro arbitrio, 
        e se lo facessero tutti i dipendenti, piu' spesso mancherebbe la persona che proprio quel giorno serve, e cosi' la produttivita' calerebbe.
    \item Perche' molti clienti, ad ogni minimo calo di qualita' od aumento del costo passerebbero subito ad altre aziende.
    \item Perche' le altre aziende non aspettano altro che un momento debole della nostra azienda per superla e prendere la sua fetta di mercato.
\end{enumerate}
Andando a lavorare credendo in quello che si fa, si ha la possibilita' di stare con altre persone, che come noi cercano di sopravvivere impegnandosi; di rinforzare la fiducia in noi stessi tramite i piccoli quotidiani successi; di conoscere persone diverse da noi, al di fuori del nostro cerchio di amicizie.\\
Di stare con i colleghi ed avere la possibilita' di trasmettere le proprie tecniche e i propri trucchi, di mettersi a disposizione delle loro piccole difficolta'.\\
Di sopportare chi ci e' antipatico, perche' noi in tutta la nostra vita abbiamo fatto di tutto per fare tacere quelle parti del vivere che, invece, a nostro avviso, lui lascia totalmente libere o vizia. E poi, rendersi conto che, in realta', il suo modo di vivere ci interessa sotto certi aspetti, e che ne' continuando ad essere come noi siamo, ne' essendo esattamente come lui e' arriveremo alla Meta. E, capito cio', cambiando nel tempo, ritrovare la pace di stare con lui.\\
Di impegnarsi su problemi che solo noi sappiamo o dobbiamo risolvere. Ricorrere a tutta la propria esperienza, resistenza e astuzia. Paura, poi fede, speranza, poi impegno e ancora impegno, e pazienza e ancora impegno, e poi il momento fatidico della messa in atto, e poi il successo. La gioia con i colleghi, le lodi del capo. E poi, l'indomani, la solita routine :)\\
Di tornare la sera, sapendo di aver fatto di meglio che si poteva nella giornata per se stessi e per tutti, mangiare un piatto semplice e nutriente, parlare con chi si vive con le poche forze che si hanno, chiudere gli occhi e ritornare nella nostra dimora fuori dal tempo e dello spazio.\\

Non esiste solo il lavoro. Esistono molti altri modi per ricercare la vera vita: la maestria nell'arte, o nella tecnica, o nello sport o nella scienza; la ricerca e la coltivazione dell'amore; la tranquillita' nella famiglia e la sua protezione.\\

Per quanto difficile o inauspicabile, accettare, curare e amare la propria e altrui vita, qualunque sia il punto di partenza e' la vera vita.\\

Il piacere della vita vera e' ineguagliabile.\\
E' genuino e ricco, e' raffinato e inarrestabile.\\

Vedere e sentire che la propria o altrui vita cresce e poi fiorisce e' il piacere piu' grande.
Tutti gli altri piaceri sono declinazioni di questo.\\

Chi capisce e nutrisce nel tempo questa consapevolezza, impara ad amare se stesso e gli altri,

\centerpoemon{senza aspettare soldi,}
\begin{poem}
a mani nude,\\
senza aspettare soldi,\\
assistenza,\\
luoghi, occasioni,\\
altro.\\
\end{poem}

\leavevmode\\
Per ogni persona che si alza per ricercare la vera vita,
la vita di tutti migliora esponenzialmente.\\

Questi racconti, poesie e pensieri, sono tracce dei risultati che ho raggiunto con i miei sforzi, con il contributo e con l'affetto di chi mi e' stato vicino.\\
Vogliono essere una piccola dimostrazione dell'esistenza della vera vita.

\begin{flushright}
    \vspace*{\fill}
    Kareh, \finishDate
\end{flushright}


\section{Links}

Pagina web del libro:\\
\url{https://kareh.github.io}\\

\leavevmode\\
Codice sorgente (LaTeX):\\
\url{https://github.com/kareh/ricercafelicita}

