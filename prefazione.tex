\chapter{Prefazione}
Da ragazzo sapevo che esisteva qualcosa oltre la monota routine della scuola, dei compiti, e dei soliti scherzi tra compagni. Cosi', mi ero messo a cercare la ``vita vera'' nell'informatica e nella matematica, dove, ogni giorno che passava, conoscevo cose arcane e stupende. Ma non sarebbe bastata tutta la conoscenza di Eulero ed Erdos per parlare al mio cuore della Vita. Oggi, dopo venti anni, la vedo piu' nitidamente. Non l'ho ancora raggiunta come Chi Veramente Ama desidererebbe, ma la capisco e, con il costo di costanti piccoli e grandi sforzi, la amo sempre piu'.

La vita vera esiste. Ce ne ricordiamo quando leggiamo un bel libro o vediamo un film, che ci coinvolge, che ci fa ritornare a sperare o ci fa riflettere, oppure quando facciamo un sogno che ci rimane, che ci vuole parlare e che ci rimanda a significati che ancora non capiamo e che da tempo avevamo dimenticato. La vita vera non e' lontana. Non occorre conquistare un titolo altisonante od ottenere cio' che non si ha. Per darle realta', e' sufficiente crescere, costantemente, per vivere la Pace con se stessi e con gli altri. 

La ricerca della vera vita e' un cammino in cui si smette di ricercare cose vane, che al piu' approssimano la meta, e dove smettendo di credere nel mondo, in cio' in cui tutti comunemente credono, smettendo di vedere alcun uomo come superiore agli altri, che sia te stesso (te stessa), che sia un nostro genitore, che sia un uomo famoso della societa', si abbandonano gli idoli del nostro cuore, e si cammina nella strada solitaria e buia dell'anima che brama la vera luce. In questo cammino si ricerca Dio, al di la' delle scritture e delle istituzioni. Si ricerca Dio che e' inscritto nelle fibre dei nostri dolori e pianti piu' cupi, nelle bellezze e nelle altezze delle nostre gioie. Dove si ci inchina anche alla persona piu' lontana da noi stessi, riconoscendo la sacralita' della sua esistenza, della sua natura che e' la nostra stessa natura, e sopportando, e cambiando noi stessi, si prega per la sua Pace.

Raggiungere la vera vita e' cosi' l'impresa piu' difficile, stressante e piu' lunga che si possa immaginare. Praticamente nessuno sa' come raggiungerla, ma ognuno e' in grado di mettersi in cammino per trovarla, di impegnarsi per costruirla, di pazientare per coltivarla.

Per dare concretezza a quanto fin'ora detto, facciamo l'esempio del lavoro come strumento per la ricerca della vera vita. Oltre a realizzare una mera logica di sussistenza, andando a lavorare si ha la possibilita' di stare con altre persone, che come noi cercano di sopravvivere impegnandosi onestamente e che sono al di fuori del nostro cerchio di amicizie. Di rinforzare la fiducia in noi stessi tramite i piccoli quotidiani successi. Di trasmettere le proprie tecniche e i propri trucchi per i successi dei colleghi. Di sopportare chi ci e' antipatico, per rendersi conto che, in realta', il suo modo di vivere ci interessa sotto certi aspetti, e che ne' continuando ad essere come noi siamo, ne' essendo esattamente come lui e' arriveremo alla Meta. E, capito cio', cambiando nel tempo, ritrovare la pace di stare con lui e dare spunti di cambiamento a lui. Di impegnarsi su problemi che solo noi sappiamo o dobbiamo risolvere. Ricorrere a tutta la propria esperienza, resistenza e astuzia. Vivere la paura del fallimento, poi ritrovare fede, speranza, e soffrire tanto impegno, e pazienza e ancora impegno, e poi arrivare al momento fatidico della messa in atto. Nel caso di successo, condividere la gioia con i colleghi, ornarsi dei complimenti dei superiori. Nel caso di fallimento, riconoscere la bonta' del lavoro fatto, trarre esperienza dagli errori commessi, e ritrovare il coraggio dai consigli di chi ci vuole bene, per poi rimettersi in gioco nel lavoro.  Infine, andando a lavorare si ha la possibilita' di tornare a casa, dimenticandosi della veste del dovere indossata tutta la giornata, e apprezzare la ricerca delle semplici gioie della vita, come un piatto caldo e nutriente, mangiato in compagnia di chi vogliamo bene.

Per tutto questo, chi ricerca la vera vita lavorando, si alza anche la ventesima mattina, dopo averlo fatto gia' 19 giorni nello stesso mese, e va a lavorare, nonostante, quel giorno, farebbe piu' bene per il suo fegato fare una semplice passeggiata per comprare la frutta o stringere la mano del suo amore e guardare il cielo. Chi ricerca la vera vita, non si vede schiavo ne' accetta il mondo come perfetto. Comprende il suo cuore e quello degli altri e capisce che la societa' e' il risultato degli egoismi e delle eccellenze di ognuno. Sa' quindi che non puo' assentarsi perfino la ventesima mattina, perche'
\begin{enumerate}
    \item il suo capo o manager non ha cognizione delle esigenze personali di ogni suo dipendente, e averla non e' facile e sarebbe molto impegnativo per chiunque. Inoltre, e' gia' impegnato ad affrontare le problematiche di produzione aziendali, e quindi, anche desiderandolo, non potrebbe dedicarsi serenamente ad ognuno.
    \item Perche', il vero lavoratore sa che non ha cognizione completa dei piani dei suoi capi e di quello che stanno facendo i colleghi, delle difficolta' che sta' affrontando l'azienda e di quello che vuole fare per il prossimo futuro.
   Forse ha una visione di cio', ma non precisa, quindi non puo' concludere correttamente, autonomamente, se e' giusto fare qualcosa a riguardo del suo lavoro o no. Nello specifico, non puo' decidere se assentarsi o ritardare a suo arbitrio. Inoltre, anche se la sua assenza e' ininfluente, non puo' ragionare individualmente, perche' se poi ogni altro lavoratore facesse cosi', piu' spesso mancherebbe la persona che proprio quel giorno serve, e cosi' la produttivita' calerebbe.
    \item Perche' le aziende clienti dell'azienda del lavoratore, ad ogni minimo calo di qualita' od aumento del costo passerebbero subito ad altre aziende, perche', altrimenti, non riuscerebbero a soddisfare le richieste dei loro stessi clienti. Di conseguenza, l'azienda del lavoratore diventerebbe piu' debole nel mercato e le aziende competitrici ne approfitterebbero per superarla e godere dei benefici che ha goduto quando era avanti a loro, e che, seguendo una logica capitalista, non ha mai condiviso con loro.

\end{enumerate}
Il lavoro, ad ogni modo, non e' il fine, e' un mezzo per mettere a fuoco la ricerca della vera vita. Esistono molti altri modi per ricercarla, come l'arte, lo sport, la politica, la religione, la scienza, la famiglia. Tutti, seppure diversi, hanno gli stessi ingredienti: 1. apprezzare cio' che si e' e gli altri sono, 2. impegnarsi per dare a tutti, 3. non dare immotivata priorita' a nessuno, compresi se stessi, al costo di rinunciare a molti privilegi.

Vale la pena ricercare la vera vita. Il piacere della vita vera e' ineguagliabile. E' genuino e ricco, e' raffinato e inarrestabile. Vedere e sentire che la propria e altrui vita cresce e poi fiorisce e' il piacere piu' grande. Tutti gli altri piaceri sono declinazioni di questo.

Per quanto difficile o inauspicabile, bisogna accettare la propria e altrui vita, e spendere tutto quanto possibile di noi stessi per essa, qualunque sia il punto di partenza. Non pretendendo niente da nessuno, soprattutto da chi riteniamo dovrebbe fare qualcosa per noi e non lo fa. Costruendo la vita, cosi', solo tramite la Carita', il disinteressato amore che sorge nel cuore degli uomini. Andare avanti nel Cammino, senza aspettare soldi, assistenza, luoghi, occasioni od altro, ma solo ricchi del Suo amore. Lui che ci ha dato vita fin da quando nostra madre nella sua pancia, mettendo in atto il Suo amore col proprio corpo, ci tesseva, cellula dopo cellula, e che poi ha sempre sostenuto i nostri passi, ogni volta che oscillavamo tra cio' che dava vita e cio' che dava morte, ha vinto riportandoci alla vita.\\

I racconti, poesie e pensieri di questo libro sono tutti incentrati sul tema della ricerca della vita, che e' difficile nel mondo presente senza valori, frammentato in migliaia di bandiere e di ideali, schiavo di una logica di sopravvivenza individuale e di competizione, isolato nel deserto della paura ed dell'indifferenza metropolitana. 
Pur essendo difficile da raggiungere, la vita vera esiste se facciamo tutto l'opposto di cio' che troviamo scuro e triste nel mondo, e lo facciamo al meglio di noi stessi, al costo di noi stessi. Quindi, coltiviamo valori incarnandoli nel nostro essere, concretizziamo con le nostre piccole, importanti, scelte l'Unico Ideale che comprende tutti e non dimentica nessuno, lavoriamo come diligenti servi delle persone di tutte le nazioni e non come schiavi del potere, che per dimenticare tramite falsi piaceri e benesseri i dolori profondi dell'esistenza, tradiscono chi amano. Desideriamo non il nostro bene, ma quello piu' pregiato che si ottiene quando e' ottenuto il bene dei nostri vicini, amici o estranei, bene che fa' scomparire immediatamente il deserto della solitudine.

\begin{flushright}
    \vspace*{\fill}
    Kareh, \finishDate
\end{flushright}

\section{Links}

Pagina web del libro:\\
\url{https://kareh.github.io}\\

\leavevmode\\
Codice sorgente (LaTeX):\\
\url{https://github.com/kareh/ricercafelicita}

