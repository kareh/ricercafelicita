\chapter{Prefazione}
Da ragazzo sapevo che esisteva qualcosa oltre la monota routine della scuola, dei compiti, e dei soliti scherzi tra compagni. Cosi', mi ero messo a cercare la ``vita vera'' nell'informatica e nella matematica, dove, ogni giorno che passava, conoscevo cose arcane e stupende. Ma non sarebbe bastata tutta la conoscenza di Eulero ed Erdos per parlare al mio cuore della vera vita. Oggi, dopo venti anni, intuisco com'e' la ``vita vera''. Non l'ho ancora raggiunta come Chi Veramente Ama desidererebbe, ma la capisco e, con il costo di costanti piccoli e grandi impegni e sacrifici, la amo sempre piu'.

La vita vera esiste. Ce ne ricordiamo quando leggiamo un bel libro, che ci coinvolge e ci guida lungo una riflessione, lungo un viaggio.  La vita vera e' la sostanza che forma noi stessi e gli altri. Non occorre andare ad abitare lontano, conquistare un titolo altisonante od ottenere gli oggetti che desideriamo ma mai avremo senza tradire noi stessi. Per darle realta', e' sufficiente diventare grandi, ma non onnipotenti o super-eroi. Riuscire a vivere la Pace con se stessi e con gli altri, vivere per portare Pace a se stessi ed agli altri, nel proprio ambiente naturale. 

Non c'e' alcuna potenza che puo' ostacolare il raggiungimento personale della Pace e il desiderio di amore che vuole far raggiungere la Pace agli altri. 

E' un cammino in cui si smette di ricercare cose vane, che al piu' approssimano la vera vita, e dove smettendo di credere nel mondo, in cio' in cui tutti comunemente credono, smettendo di vedere alcun uomo come superiore agli altri, che sia te stesso (te stessa), che sia un nostro genitore, che sia un uomo famoso della societa', si abbandonano gli idoli del nostro cuore, e si cammina nella strada solitaria e buia dell'anima che brama la vera luce. In questo cammino si ricerca Dio, al di la' delle scritture e delle istituzioni. Si ricerca Dio che e' inscritto nelle fibre dei nostri dolori e pianti piu' cupi, nelle bellezze e nelle altezze delle nostre gioie. Dove si ci inchina anche alla persona piu' lontana da noi stessi, riconoscendo la sacralita' della sua esistenza, della stessa natura nostra, e sopportando, e cambiando noi stessi, si prega per la sua Pace.

Raggiungere la vera vita e' cosi' l'impresa piu' difficile, stressante e piu' lunga che si possa immaginare. Nessuno sa' come raggiungerla, altrimenti il mondo sarebbe gia' in Pace, ma ognuno e' in grado di mettersi in cammino per trovarla, di impegnarsi per costruirla, di pazientare per coltivarla.

Per dare concretezza a quanto fin'ora detto, facciamo l'esempio del lavoro come metafora della ricerca della vera vita.

Chi ricerca la vera vita lavorando, si alza anche la ventesima mattina, dopo averlo fatto gia' 19 giorni nello stesso mese, e va a lavorare, nonostante farebbe bene al suo fegato fare una semplice passeggiata dal fruttivendolo o
stringere la mano del suo amore e stare cosi' una mezz'ora.

Andare a lavoro perche' 
\begin{enumerate}
    \item il proprio capo probabilmente non capirebbe, dato che non ha cognizione delle nostre esigenze. E avere cognizione delle esigenze di ogni dipendente non e' una cosa facile ed e' molto impegnativa.
    \item Perche', noi non abbiamo cognizione completa dei piani dei nostri capi e di quello che stanno facendo i colleghi. Delle difficolta' che sta' affrontando l'azienda, di quello che vuole fare per il prossimo futuro.

        Magari piu' o meno lo sappiamo, ma non sappiamo nel dettaglio le cose, e nel lavoro i dettagli sono fondamentali.

        Se potessimo assentarci o ritardare a nostro arbitrio, e se lo facessero tutti i dipendenti, piu' spesso mancherebbe la persona che proprio quel giorno serve, e cosi' la produttivita' calerebbe.
    \item Perche' molti clienti, ad ogni minimo calo di qualita' od aumento del costo passerebbero subito ad altre aziende. E questo e' naturale, perche' altrimenti non riuscirebbero ad ottenere quello che cercano dalla nostra azienda.
    \item Perche' le altre aziende non aspettano altro che un momento debole della nostra azienda per superla e prendere una maggiore fetta di mercato.

\end{enumerate}
Andando a lavorare credendo in quello che si fa, si ha la possibilita' di stare con altre persone, che come noi cercano di sopravvivere impegnandosi; di rinforzare la fiducia in noi stessi tramite i piccoli quotidiani successi; di conoscere persone diverse da noi, al di fuori del nostro cerchio di amicizie.

Di stare con i colleghi ed avere la possibilita' di trasmettere le proprie tecniche e i propri trucchi, di mettersi a disposizione delle loro piccole difficolta'.

Di sopportare chi ci e' antipatico, perche' noi in tutta la nostra vita abbiamo fatto di tutto per fare tacere quelle parti del vivere che, invece, a nostro avviso, lui lascia totalmente libere o vizia. E poi, rendersi conto che, in realta', il suo modo di vivere ci interessa sotto certi aspetti, e che ne' continuando ad essere come noi siamo, ne' essendo esattamente come lui e' arriveremo alla Meta. E, capito cio', cambiando nel tempo, ritrovare la pace di stare con lui.

Di impegnarsi su problemi che solo noi sappiamo o dobbiamo risolvere. Ricorrere a tutta la propria esperienza, resistenza e astuzia. Paura, poi fede, speranza, poi impegno e ancora impegno, e pazienza e ancora impegno, e poi il momento fatidico della messa in atto, e poi il successo. La gioia con i colleghi, le lodi del capo. E poi, l'indomani, la solita routine :)

Di tornare la sera, sapendo di aver fatto di meglio che si poteva nella giornata per se stessi e per tutti, mangiare un piatto semplice e nutriente, parlare con chi si vive con le poche forze che si hanno, chiudere gli occhi e ritornare nella nostra dimora fuori dal tempo e dello spazio.

Non esiste solo il lavoro. Esistono molti altri modi per ricercare la vera vita: la maestria nell'arte, o nella tecnica, o nello sport o nella scienza; la ricerca e la coltivazione dell'amore; la tranquillita' nella famiglia e la sua protezione.\\

Per quanto difficile o inauspicabile, accettare, curare e amare la propria e altrui vita, qualunque sia il punto di partenza e' la vera vita.\\

Il piacere della vita vera e' ineguagliabile.\\
E' genuino e ricco, e' raffinato e inarrestabile.\\

Vedere e sentire che la propria o altrui vita cresce e poi fiorisce e' il piacere piu' grande.
Tutti gli altri piaceri sono declinazioni di questo.\\

Chi capisce e nutrisce nel tempo questa consapevolezza, impara ad amare se stesso e gli altri,

\centerpoemon{senza aspettare soldi,}
\begin{poem}
a mani nude,\\
senza aspettare soldi,\\
assistenza,\\
luoghi, occasioni,\\
altro.\\
\end{poem}

\leavevmode\\
Per ogni persona che si alza per ricercare la vera vita,
la vita di tutti migliora esponenzialmente.\\

Questi racconti, poesie e pensieri, sono tracce dei risultati che ho raggiunto con i miei sforzi, con il contributo e con l'affetto di chi mi e' stato vicino.\\
Vogliono essere una piccola dimostrazione dell'esistenza della vera vita.

\begin{flushright}
    \vspace*{\fill}
    Kareh, \finishDate
\end{flushright}


\section{Links}

Pagina web del libro:\\
\url{https://kareh.github.io}\\

\leavevmode\\
Codice sorgente (LaTeX):\\
\url{https://github.com/kareh/ricercafelicita}

