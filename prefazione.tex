\chapter{Prefazione}
Da ragazzo sapevo che esisteva qualcosa oltre la monotona routine della scuola, dei compiti, e dei soliti scherzi tra compagni. Per questo motivo, sentivo l'esigenza di trovare la ``vita vera''. Non sapevo dove era, e inizialmente la cercai nello sport, nell'informatica e nella matematica. Ogni giorno che passava, conoscevo cose arcane e stupende. La cercai anche nell'amore umano, splendido come un'alba, forte come una tempesta, irrimediabile come il tramonto di un giorno. Ma non sarebbe bastata tutta la conoscenza di Eulero ed Erdos per parlare al mio cuore della vita, ne' sarebbe bastata tutta la sapienza di Jung o lo charm di un prestante attore famoso per condurre una relazione serena, profonda e felice. Oggi, dopo piu' di venti anni, vedo la vita piu' nitidamente. Non l'ho ancora raggiunta come Chi Veramente Ama desidererebbe, ma la capisco e, con il costo di costanti piccoli e grandi sforzi, la amo sempre piu'.

La vita vera esiste. Ce ne ricordiamo quando leggiamo un bel libro o vediamo un film, che ci coinvolge, che ci fa ritornare a sperare o ci fa riflettere, oppure quando facciamo un sogno che ci rimane, che ci vuole parlare e che ci rimanda a significati che ancora non capiamo e che da tempo avevamo dimenticato. La vita vera non e' lontana. Non occorre conquistare un titolo, aspettare il nuovo grande progresso tecnologico od ottenere cio' che non si ha.  La vita in tutta la sua grandezza e bellezza e' essere in pace con se stessi e con gli altri, qui e ora. E' potenzialmente immediata da raggiungere, ma praticamente difficilissima e costosissima da affrontare se il nostro cuore e' ferito o distratto da illusioni.

Raggiungerla e' un cammino in cui si mettono in discussione i propri valori piu' profondi e radicati, per setacciarli e scegliere solo quelli piu' fondamentali e veri; 
e' un cammino in cui si smette di vedere anche solo un uomo come superiore agli altri, che sia il se, che sia un genitore, che sia un uomo famoso della societa', e si abbandonano gli idoli del nostro cuore. Si cammina nella strada solitaria e buia dell'anima che brama la vera luce. In questo cammino si ricerca Dio, unica vera identita' d'amore superiore a tutte le altre. Dio, al di la' delle scritture e delle istituzioni, si trova nelle fibre dei nostri dolori e pianti piu' cupi, nelle bellezze e nelle altezze delle nostre gioie. Nel cammino della vera vita, si ci inchina anche alla persona piu' lontana da noi stessi, riconoscendo la sacralita' della sua esistenza, della sua natura che e' la nostra stessa natura, per raggiungere la pace.

Vale la pena ricercare la vera vita. Il piacere della vita vera e' ineguagliabile. E' genuino e ricco, e' raffinato e inarrestabile. Vedere e sentire che la propria e altrui vita cresce e poi fiorisce e' il piacere piu' grande. Tutti gli altri piaceri sono declinazioni di questo.
Per quanto difficile o inauspicabile, bisogna accettare la propria vita, e spendere tutto quanto possibile di noi stessi per essa, qualunque sia il punto di partenza. Non pretendendo niente da nessuno, soprattutto da noi stessi e da chi riteniamo dovrebbe fare qualcosa e non lo fa. Costruendo la vita, cosi', solo tramite la Carita', il disinteressato amore che sorge nel cuore degli uomini. Andando avanti nel cammino, senza aspettare soldi, assistenza, luoghi, occasioni od altro, ma solo ricchi del Suo amore.

I racconti, poesie e pensieri di questo libro sono tutti incentrati sul tema della ricerca della vita, che e' difficile nel mondo presente senza valori, frammentato in migliaia di bandiere e di ideali, schiavo di una logica di sopravvivenza individuale e di competizione, isolato nel deserto della paura ed dell'indifferenza metropolitana. 
Pur essendo difficile da raggiungere, la vita vera esiste se facciamo tutto l'opposto di cio' che troviamo scuro e triste nel mondo, e lo facciamo al meglio di noi stessi, al costo di noi stessi. Quindi, coltiviamo valori incarnandoli nel nostro essere, concretizziamo con le nostre piccole, importanti, scelte un unico ideale che comprende tutti e non dimentica nessuno, lavoriamo come diligenti servi delle persone di tutte le nazioni e non come schiavi del potere, che per dimenticare tramite falsi piaceri e benesseri i dolori profondi dell'esistenza, tradiscono chi amano. Desideriamo il bene, e soprattutto quello piu' pregiato che si ottiene tramite l'amore per i nostri vicini, amici o estranei, bene che fa' scomparire immediatamente il deserto della solitudine.

\begin{flushright}
    \vspace*{\fill}
    \finishDate\\
    aggiornato il \lastUpdateDate
\end{flushright}

\pagebreak
\section{Links}

Pagina web del libro:\\
\url{https://kareh.github.io}\\

\leavevmode\\
Codice sorgente (LaTeX):\\
\url{https://github.com/kareh/ricercafelicita}

