\chapter{Prefazione}
Da ragazzo sapevo che esisteva qualcosa oltre la monota routine della scuola, dei compiti, e dei soliti scherzi tra compagni. Per questo motivo, sentivo l'esigenza di trovare la ``vita vera''. Non sapevo dove era, e inizialmente la cercai nello sport, nell'informatica e nella matematica. In queste ultime discipline, ogni giorno che passava, conoscevo cose arcane e stupende. Ma non sarebbe bastata tutta la conoscenza di Eulero ed Erdos per parlare al mio cuore della Vita. Oggi, dopo venti anni, la vedo piu' nitidamente. Non l'ho ancora raggiunta come Chi Veramente Ama desidererebbe, ma la capisco e, con il costo di costanti piccoli e grandi sforzi, la amo sempre piu'.

La vita vera esiste. Ce ne ricordiamo quando leggiamo un bel libro o vediamo un film, che ci coinvolge, che ci fa ritornare a sperare o ci fa riflettere, oppure quando facciamo un sogno che ci rimane, che ci vuole parlare e che ci rimanda a significati che ancora non capiamo e che da tempo avevamo dimenticato. La vita vera non e' lontana. Non occorre conquistare un titolo altisonante od ottenere cio' che non si ha. Per darle realta', e' sufficiente crescere, costantemente, per vivere la Pace con se stessi e con gli altri. 

La ricerca della vera vita e' un cammino in cui si smette di ricercare cose vane, che al piu' approssimano la meta, e dove smettendo di credere nel mondo, in cio' in cui tutti comunemente credono, smettendo di vedere alcun uomo come superiore agli altri, che sia te stesso (te stessa), che sia un genitore, che sia un uomo famoso della societa', si abbandonano gli idoli del nostro cuore, e si cammina nella strada solitaria e buia dell'anima che brama la vera luce. In questo cammino si ricerca Dio, al di la' delle scritture e delle istituzioni. Si ricerca Dio che e' inscritto nelle fibre dei nostri dolori e pianti piu' cupi, nelle bellezze e nelle altezze delle nostre gioie. Dove si ci inchina anche alla persona piu' lontana da noi stessi, riconoscendo la sacralita' della sua esistenza, della sua natura che e' la nostra stessa natura, e sopportando, e cambiando noi stessi, si prega per la sua Pace.

Vale la pena ricercare la vera vita. Il piacere della vita vera e' ineguagliabile. E' genuino e ricco, e' raffinato e inarrestabile. Vedere e sentire che la propria e altrui vita cresce e poi fiorisce e' il piacere piu' grande. Tutti gli altri piaceri sono declinazioni di questo.
Per quanto difficile o inauspicabile, bisogna accettare la propria e altrui vita, e spendere tutto quanto possibile di noi stessi per essa, qualunque sia il punto di partenza. Non pretendendo niente da nessuno, soprattutto da chi riteniamo dovrebbe fare qualcosa per noi e non lo fa. Costruendo la vita, cosi', solo tramite la Carita', il disinteressato amore che sorge nel cuore degli uomini. Andare avanti nel Cammino, senza aspettare soldi, assistenza, luoghi, occasioni od altro, ma solo ricchi del Suo amore. Lui che ci ha dato vita fin da quando nostra madre nella sua pancia ci tesseva, cellula dopo cellula. Lui che sempre sostiene i nostri passi, e che ogni volta che abbiamo oscillato tra cio' che dava vita e cio' che dava morte, ha vinto riportandoci alla vita.\\

I racconti, poesie e pensieri di questo libro sono tutti incentrati sul tema della ricerca della vita, che e' difficile nel mondo presente senza valori, frammentato in migliaia di bandiere e di ideali, schiavo di una logica di sopravvivenza individuale e di competizione, isolato nel deserto della paura ed dell'indifferenza metropolitana. 
Pur essendo difficile da raggiungere, la vita vera esiste se facciamo tutto l'opposto di cio' che troviamo scuro e triste nel mondo, e lo facciamo al meglio di noi stessi, al costo di noi stessi. Quindi, coltiviamo valori incarnandoli nel nostro essere, concretizziamo con le nostre piccole, importanti, scelte un unico ideale che comprende tutti e non dimentica nessuno, lavoriamo come diligenti servi delle persone di tutte le nazioni e non come schiavi del potere, che per dimenticare tramite falsi piaceri e benesseri i dolori profondi dell'esistenza, tradiscono chi amano. Desideriamo non il nostro bene, ma quello piu' pregiato che si ottiene quando e' ottenuto il bene dei nostri vicini, amici o estranei, bene che fa' scomparire immediatamente il deserto della solitudine.

\begin{flushright}
    \vspace*{\fill}
    Kareh, \finishDate\\
    aggiornato il \lastUpdateDate
\end{flushright}

\section{Links}

Pagina web del libro:\\
\url{https://kareh.github.io}\\

\leavevmode\\
Codice sorgente (LaTeX):\\
\url{https://github.com/kareh/ricercafelicita}

