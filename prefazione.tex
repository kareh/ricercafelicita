\chapter{Prefazione}
Da ragazzo sapevo che esisteva qualcosa oltre la monota routine della scuola, dei compiti, e dei soliti scherzi tra compagni. Cosi', mi ero messo a cercare la ``vita vera'' nell'informatica e nella matematica, dove, ogni giorno che passava, conoscevo cose arcane e stupende. Ma non sarebbe bastata tutta la conoscenza di Eulero ed Erdos per parlare al mio cuore della vera vita. Oggi, dopo venti anni, intuisco com'e' la ``vita vera''. Non l'ho ancora raggiunta come Chi Veramente Ama desidererebbe, ma la capisco e, con il costo di costanti piccoli e grandi impegni e sacrifici, la amo sempre piu'.

La vita vera esiste. Ce ne ricordiamo quando leggiamo un bel libro, che ci coinvolge e ci guida lungo una riflessione, lungo un viaggio.  La vita vera e' la sostanza che forma noi stessi e gli altri. Non occorre andare ad abitare lontano, conquistare un titolo altisonante od ottenere gli oggetti che desideriamo ma mai avremo senza tradire noi stessi. Per darle realta', e' sufficiente diventare grandi, ma non onnipotenti o super-eroi. Riuscire a vivere la Pace con se stessi e con gli altri, vivere per portare Pace a se stessi ed agli altri, nel proprio ambiente naturale. 

Non c'e' alcuna potenza che puo' ostacolare il raggiungimento personale della Pace e il desiderio di amore che vuole far raggiungere la Pace agli altri. 

E' un cammino in cui si smette di ricercare cose vane, che al piu' approssimano la vera vita, e dove smettendo di credere nel mondo, in cio' in cui tutti comunemente credono, smettendo di vedere alcun uomo come superiore agli altri, che sia te stesso (te stessa), che sia un nostro genitore, che sia un uomo famoso della societa', si abbandonano gli idoli del nostro cuore, e si cammina nella strada solitaria e buia dell'anima che brama la vera luce. In questo cammino si ricerca Dio, al di la' delle scritture e delle istituzioni. Si ricerca Dio che e' inscritto nelle fibre dei nostri dolori e pianti piu' cupi, nelle bellezze e nelle altezze delle nostre gioie. Dove si ci inchina anche alla persona piu' lontana da noi stessi, riconoscendo la sacralita' della sua esistenza, della sua natura che e' la nostra stessa natura, e sopportando, e cambiando noi stessi, si prega per la sua Pace.

Raggiungere la vera vita e' cosi' l'impresa piu' difficile, stressante e piu' lunga che si possa immaginare. Nessuno sa' come raggiungerla, altrimenti il mondo sarebbe gia' in Pace, ma ognuno e' in grado di mettersi in cammino per trovarla, di impegnarsi per costruirla, di pazientare per coltivarla.

Per dare concretezza a quanto fin'ora detto, facciamo l'esempio del lavoro come metafora della ricerca della vera vita.

Chi ricerca la vera vita lavorando, si alza anche la ventesima mattina, dopo averlo fatto gia' 19 giorni nello stesso mese, e va a lavorare, nonostante, quel giorno, farebbe piu' bene per il suo fegato fare una semplice passeggiata per comprare la frutta o stringere la mano del suo amore e guardare il cielo. Chi ricerca la vera vita, non si vede schiavo ne' accetta il mondo come perfetto. Comprende il suo cuore e quello degli altri e capisce che la societa' e' il risultato degli egoismi e delle eccellenze di ognuno. Sa' quindi che non puo' assentarsi perfino nella ventesima mattina del mese, perche'
\begin{enumerate}
    \item il proprio capo non puo' capire, dato che non ha cognizione delle esigenze di ogni dipendente, e averla non e' facile e sarebbe molto impegnativo per chiunque, e praticamente impossibile per una sola persona se l'azienda ha gia' piu' di 10 dipendenti.
        Per questo e' necessario adottare delle regole condivise da tutti i dipendenti dell'azienda. Nello specifico, assentarsi solo per malattia o per ferie o permessi.
    \item Perche', noi non abbiamo cognizione completa dei piani dei nostri capi e di quello che stanno facendo i colleghi. Delle difficolta' che sta' affrontando l'azienda, di quello che vuole fare per il prossimo futuro.
   Forse abbiamo una visione di cio', ma non precisa, ed essendo nel lavoro i dettagli fondamentali, potremmo pensare che sia giusto fare una cosa quando invece non lo e'. Nello specifico, se potessimo assentarci o ritardare a nostro arbitrio, e se lo facessero tutti i dipendenti, piu' spesso mancherebbe la persona che proprio quel giorno serve, e cosi' la produttivita' calerebbe.
    \item Perche' molti clienti, ad ogni minimo calo di qualita' od aumento del costo passerebbero subito ad altre aziende. E questo e' naturale, perche' altrimenti non riuscirebbero ad ottenere quello che cercano dalla nostra azienda, e perderebbero a loro volta i loro clienti.
    \item Le altre aziende non aspettano altro che un momento debole della nostra azienda per superarla e godere dei benefici che ha goduto quando era avanti a loro, e che, seguendo una logica capitalista, non ha mai condiviso con loro.

\end{enumerate}
Andando a lavorare credendo in quello che si fa, si ha la possibilita' di stare con altre persone, che come noi cercano di sopravvivere impegnandosi; di rinforzare la fiducia in noi stessi tramite i piccoli quotidiani successi; di conoscere persone diverse da noi, al di fuori del nostro cerchio di amicizie.

Di stare con i colleghi ed avere la possibilita' di trasmettere le proprie tecniche e i propri trucchi, di mettersi a disposizione delle loro piccole difficolta'.

Di sopportare chi ci e' antipatico, perche' noi in tutta la nostra vita abbiamo fatto di tutto per fare tacere quelle parti del vivere che, invece, a nostro avviso, lui lascia totalmente libere o vizia. E poi, rendersi conto che, in realta', il suo modo di vivere ci interessa sotto certi aspetti, e che ne' continuando ad essere come noi siamo, ne' essendo esattamente come lui e' arriveremo alla Meta. E, capito cio', cambiando nel tempo, ritrovare la pace di stare con lui.

Di impegnarsi su problemi che solo noi sappiamo o dobbiamo risolvere. Ricorrere a tutta la propria esperienza, resistenza e astuzia. Vivere la paura del fallimento, poi ritrovare fede, speranza, e soffrire tanto impegno, e pazienza e ancora impegno, e poi arrivare al momento fatidico della messa in atto. Nel caso di successo, condividere la gioia con i colleghi, ornarsi dei complimenti dei superiori. Nel caso di fallimento, riconoscere la bonta' del lavoro fatto, trarre esperienza dagli errori commessi, e ritrovare il corraggio dai consigli di chi ci vuole bene, per poi rimettersi in gioco nel lavoro.

Di tornare a casa, dimenticandosi della veste del dovere indossata tutta la giornata, e ritornare a cercare le semplici gioie della vita, un piatto caldo e nutriente, mangiato in compagnia di chi vogliamo bene.\\

Il lavoro, ad ogni modo, non e' il fine, e' un mezzo per mettere a fuoco la ricerca della vera vita. Esistono molti altri modi per ricercarla, come l'arte, lo sport, la politica, la religione, la scienza, la famiglia. Tutti, seppure diversi, hanno gli stessi ingredienti: 1. apprezzare cio' che si e' e gli altri sono, 2. impegnarsi per dare a tutti, 3. non dare immotivata priorita' a nessuno, compresi se stessi, al costo di rinunciare a molti privilegi.

Vale la pena ricercare la vera vita. Il piacere della vita vera e' ineguagliabile. E' genuino e ricco, e' raffinato e inarrestabile. Vedere e sentire che la propria e altrui vita cresce e poi fiorisce e' il piacere piu' grande. Tutti gli altri piaceri sono declinazioni di questo.

Per quanto difficile o inauspicabile, bisogna accettare la propria e altrui vita, e spendere tutto quanto possibile di noi stessi per essa, qualunque sia il punto di partenza. Non pretendendo niente da nessuno, soprattutto da chi riteniamo dovrebbe fare qualcosa per noi e non lo fa. Costruendo, cosi', solo tramite la Carita', il disinteressato amore che sorge nel cuore degli uomini. Andare avanti nel Cammino, senza aspettare soldi, assistenza, luoghi, occasioni od altro, ma solo ricchi del Suo amore, che fin da quando la pancia di nostra madre ci tesseva, cellula dopo cellula, ci ha dato vita, ha sostenuto i nostri passi, e ogni volta che oscillavamo tra cio' che dava vita e cio' che dava morte, ha vinto riportandoci alla vita. \\

I racconti, poesie e pensieri di questo libro sono tutti incentrati sul tema della ricerca della vita, che e' difficile nel mondo presente senza valori, frammentato in migliaia di bandiere ed ideali, schiavo di una logica di sopravvivenza individuale e di competizione, isolato nel deserto della paura ed indifferenza metropolitana.

La vita vera esiste, se coltiviamo valori incarnandoli nel nostro essere, se concretizziamo con le nostre piccole, importanti azioni e scelte l'Unico Ideale che comprende tutti e non dimentica nessuno, se lavoriamo come servi delle persone di tutte le nazioni e non come schiavi del potere per dimenticare tramite falsi piaceri e benesseri i dolori profondi della nostra esistenza, se desideriamo non il nostro bene, ma quello piu' pregiato che otteniamo quando e' ottenuto il bene dei nostri vicini, conoscenti o estranei, bene che fa' scomparire immediatamente il deserto della solitudine.

\begin{flushright}
    \vspace*{\fill}
    Kareh, \finishDate
\end{flushright}

\section{Links}

Pagina web del libro:\\
\url{https://kareh.github.io}\\

\leavevmode\\
Codice sorgente (LaTeX):\\
\url{https://github.com/kareh/ricercafelicita}

