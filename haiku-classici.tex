\section{Classici}

Poesie e citazioni di autori classici.

\leavevmode\\[0.25in]

\vfill

\centerpoemon{cio' che penso foot}
\begin{haiku}
Il suono\\
dell'acqua\\
dice\\
cio' che penso \footnote{Haiku Zen}\\
\end{haiku}

\label{dolorosoMondo}
\centerpoemon{di questo mondo doloroso. foot}
\begin{haiku}
Vieni, guarda\\
i veri fiori\\
di questo mondo doloroso.  \footnote{Haiku Zen}\\
\end{haiku}

\centerpoemon{Non importa quale via percorro,}
\begin{haiku}
Occorre veramente preoccuparsi\\
dell'illuminazione?\\
Non importa quale via percorro,\\
sto andando a casa.  \footnote{Haiku Zen}\\
\end{haiku}

\centerpoemon{la seconda sfiorò la fiamma con le sue ali e disse:}
\begin{vcentered}
    \begin{poem}
Gli uomini sono come tre farfalle\\
davanti alla fiamma di una candela.\\
La prima si avvicinò e disse:\\
io conosco l'amore,\\
la seconda sfiorò la fiamma con le sue ali e disse:\\
io conosco ``la scottatura'' dell’amore.\\
la terza si gettò in mezzo alla fiamma\\
e si bruciò. \\
Solo lei conosce il vero Amore.\\
    \end{poem}
    \footnote{Massima sufista. \url{https://it.wikipedia.org/wiki/Bab\%27Aziz\_-\_Il\_principe\_che\_contemplava\_la\_sua\_anima}}
\end{vcentered}

\begin{vcentered}
    Tardi ti amai, bellezza così antica e così nuova, tardi ti amai. Sì, perché tu eri dentro di me e io fuori. Lì ti cercavo. Deforme, mi gettavo sulle belle forme delle tue creature. Eri con me, e non ero con te. Mi tenevano lontano da te le tue creature, inesistenti se non esistessero in te. Mi chiamasti, e il tuo grido sfondò la mia sordità; balenasti, e il tuo splendore dissipò la mia cecità; diffondesti la tua fragranza, e respirai e anelo verso di te, gustai e ho fame e sete; mi toccasti, e arsi di desiderio della tua pace. \footnote{Sant'Agostino, Confessioni 10, 27}
\end{vcentered}

\begin{vcentered}
    La verità raggiunta per via di riflessione filosofica e la verità della Rivelazione non si confondono, né l'una rende superflua l'altra: Esistono due ordini di conoscenza, distinti non solo per il loro principio, ma anche per il loro oggetto: per il loro principio, perché nell'uno conosciamo con la ragione naturale, nell'altro con la fede divina; per l'oggetto, perché oltre le verità che la ragione naturale può capire, ci è proposto di vedere i misteri nascosti in Dio.\footnote{Lettera enciclica, Fides et Ratio, del Papa Giovanni Paolo II}
\end{vcentered}
