\section{Autori}

Poesie e citazioni di autori vari.

\leavevmode\\[0.25in]

\vfill

\centerpoemon{cio' che penso foot}
\begin{haiku}
Il suono\\
dell'acqua\\
dice\\
cio' che penso \footnote{Haiku Zen}\\
\end{haiku}

\label{dolorosoMondo}
\centerpoemon{di questo mondo doloroso. foot}
\begin{haiku}
Vieni, guarda\\
i veri fiori\\
di questo mondo doloroso.  \footnote{Haiku Zen}\\
\end{haiku}

\centerpoemon{Non importa quale via percorro,}
\begin{haiku}
Occorre veramente preoccuparsi\\
dell'illuminazione?\\
Non importa quale via percorro,\\
sto andando a casa.  \footnote{Haiku Zen}\\
\end{haiku}

\centerpoemon{la seconda sfiorò la fiamma con le sue ali e disse:}
\begin{vcentered}
    \begin{poem}
Gli uomini sono come tre farfalle\\
davanti alla fiamma di una candela.\\
La prima si avvicinò e disse:\\
io conosco l'amore,\\
la seconda sfiorò la fiamma con le sue ali e disse:\\
io conosco ``la scottatura'' dell’amore.\\
la terza si gettò in mezzo alla fiamma\\
e si bruciò. \\
Solo lei conosce il vero Amore.\\
    \end{poem}
    \footnote{Massima sufista. \url{https://it.wikipedia.org/wiki/Bab\%27Aziz\_-\_Il\_principe\_che\_contemplava\_la\_sua\_anima}}
\end{vcentered}

\begin{vcentered}
    Tardi ti amai, bellezza così antica e così nuova, tardi ti amai. Sì, perché tu eri dentro di me e io fuori. Lì ti cercavo. Deforme, mi gettavo sulle belle forme delle tue creature. Eri con me, e non ero con te. Mi tenevano lontano da te le tue creature, inesistenti se non esistessero in te. Mi chiamasti, e il tuo grido sfondò la mia sordità; balenasti, e il tuo splendore dissipò la mia cecità; diffondesti la tua fragranza, e respirai e anelo verso di te, gustai e ho fame e sete; mi toccasti, e arsi di desiderio della tua pace. \footnote{Sant'Agostino, Confessioni 10, 27}
\end{vcentered}

\begin{vcentered}

    L'uomo con la luce della ragione sa riconoscere la sua strada, ma la può percorrere in maniera spedita, senza ostacoli e fino alla fine, se con animo retto inserisce la sua ricerca nell'orizzonte della fede.\\

    La verità raggiunta per via di riflessione filosofica e la verità della Rivelazione non si confondono, né l'una rende superflua l'altra. [...] nell'una conosciamo con la ragione naturale, nell'altra con la fede divina; [...] oltre le verità che la ragione naturale può capire, ci è proposto di vedere i misteri nascosti in Dio.\\

    \footnote{Lettera enciclica, Fides et Ratio, del Papa Giovanni Paolo II}

\end{vcentered}

\begin{vcentered}
    La comunità cristiana non è formata da persone esemplari o eccezionali, ma da piccoli e perduti, da peccatori perdonati [da Gesù] che a loro volta perdonano. \footnote{Messalino, in ascolto, agosto/anno B}
\end{vcentered}

\centerpoemon{Tu sei speranza, Tu sei giustizia,}
\begin{haiku}
Tu sei Santo Signore Dio,\\
Tu sei forte, Tu sei grande,\\
Tu sei l'Altissimo l'Onnipotente,\\
Tu Padre Santo, Re del cielo.\\
\leavevmode\\
Tu sei trino, uno Signore,\\
Tu sei il bene, tutto il bene,\\
Tu sei l'Amore, Tu sei il vero,\\
Tu sei umiltà, Tu sei sapienza.\\
\leavevmode\\
Tu sei bellezza, Tu sei la pace,\\
la sicurezza il gaudio la letizia,\\
Tu sei speranza, Tu sei giustizia,\\
Tu temperanza e ogni ricchezza.\\
\leavevmode\\
Tu sei il Custode, Tu sei mitezza,\\
Tu sei rifugio, Tu sei fortezza,\\
Tu carità, fede e speranza,\\
Tu sei tutta la nostra dolcezza.\\
\leavevmode\\
Tu sei la Vita eterno gaudio\\
Signore grande Dio ammirabile,\\
Onnipotente o Creatore\\
    o Salvatore di misericordia.\footnote{\url{https://www.youtube.com/watch?v=yfC285eSiXU}}\\
\end{haiku}

\centerpoemon{La campana del tempio tace,}
\begin{haiku}
La campana del tempio tace,\\
ma il suono continua\\
ad uscire dai fiori\footnote{Haiku Zen}\\
\end{haiku}

\begin{vcentered}
Il regno dei cieli non e' un premio da conquistare o una destinazione da cercare o una meta da raggiungere. Nulla di cio'. Il regno dei cieli e' una festa di nozze, che ci va a cercare dovunque ci troviamo o qualunque cosa noi facciamo, per invitarci a partecipare alla festa e a prendere parte a questa straordinaria gioia. Sembra semplice, eppure il Vangelo ci annuncia che tutti noi snobbiamo questo invito. Che strani che siamo! Alla fine entrano solo quelli spinti quasi per costrizione, perche' la festa nuziale va fatta in ogni caso. Pero', anche quando si e' dentro alla festa, qualcuno entra senza abito nuziale. Come e' possibile esserci con l'abito giusto con un invito arrivato all'improvviso? L'abito e' il segno della gioia di essere parte alle nozze. Oggi viviamo tutti la nostra vita senza abito nuziale, perche' desideriamo essere felici, ma in realta' siamo ignoranti della vera gioia che solo il vangelo sa mostrarci.

    \footnote{Padre Salvo Bucolo, commento al vangelo Mt 22,1-14, \url{https://www.chiesacattolica.it/liturgia-del-giorno/?data-liturgia=20250821}}
\end{vcentered}
