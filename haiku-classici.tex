\section{Classici}

\leavevmode\\[0.25in]

Seguono haiku classici, scritti da autori che vanno dal 1600 fino ai giorni nostri, presi dal libro ``Col saldatore alle due di notte'' di Asbesto Gabriele Zaverio,\\
\url{http://freaknet.org/asbesto/libro.html}

\vfill

\centerpoemon{cio' che penso}
\begin{haiku}
Il suono\\
dell'acqua\\
dice\\
cio' che penso\\
\end{haiku}

\centerpoemon{Il tetto si e' bruciato:}
\begin{haiku}
Il tetto si e' bruciato:\\
ora\\
posso vedere la luna\\
\end{haiku}

\centerpoemon{verdissimo il verde}
\begin{haiku}
Nei campi di neve\\
verdissimo il verde\\
delle erbe nuove\\
\end{haiku}

\centerpoemon{La campana del tempio tace,}
\begin{haiku}
La campana del tempio tace,\\
ma il suono continua\\
ad uscire dai fiori\\
\end{haiku}

\label{dolorosoMondo}
\centerpoemon{di questo mondo doloroso.}
\begin{haiku}
Vieni, guarda\\
i veri fiori\\
di questo mondo doloroso.\\
\end{haiku}

\centerpoemon{Non importa quale via percorro,}
\begin{haiku}
Occorre veramente preoccuparsi\\
dell'illuminazione?\\
Non importa quale via percorro,\\
sto andando a casa.\\
\end{haiku}

\centerpoemon{la seconda sfiorò la fiamma con le sue ali e disse:}
\begin{vcentered}
    \begin{poem}
        Gli uomini sono come le tre farfalle\\
davanti alla fiamma di una candela\\
la prima si avvicinò e disse:\\
io conosco l'amore,\\
la seconda sfiorò la fiamma con le sue ali e disse:\\
io conosco ``la scottatura'' dell’amore.\\
la terza si gettò in mezzo alla fiamma\\
e si bruciò. \\
Solo lei conosce il vero Amore.\\
    \end{poem}
    \leavevmode\\
    \leavevmode\\
    \leavevmode\\
    \footnote{Massima sufista. \url{https://it.wikipedia.org/wiki/Bab\%27Aziz\_-\_Il\_principe\_che\_contemplava\_la\_sua\_anima}}
\end{vcentered}
