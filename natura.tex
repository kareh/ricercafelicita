\chapter{Sulla Natura}

Cosi' come l'energia ne' si crea ne' si distrugge, perche' l'energia ceduta da un corpo viene assorbita da un altro corpo, anche qualsiasi perdita piccola o grande o totale della propria vita e' riacquisita da un'altra persona o da altre persone o dalla Natura. Questo trasferimento di vita, avviene a volte volenti o a volte nolenti. In Cristo, si diventa sempre volenti di questi trasferimenti, che diventano doni, anche se non ricambiati, ma parliamo adesso in particolare dei trasferimenti verso la Natura. 

La nostra anima ama il corpo suo e quello di chi altri ama. Il corpo e' fatto di materia, e' una parte della Natura. Il corpo rispetta e soggiace alle leggi dell'Universo perche', come parte della Natura, ama tutta la Natura, e per questo i suoi atomi sono sempre fedeli agli altri atomi. Quando ricevono energia fisica, l'assorbono, quando hanno energia la cedono agli altri atomi.
Cosi' il corpo ha un peso e, senza altre forze, tende verso il centro della Terra. Il corpo acquisisce calore dal Sole e cede calore nella notte, se lontano da un fuoco. E cosi' via, il corpo interagisce con tutta la materia, cosi' come la materia interagisce con altra materia.

A volte il corpo assorbe troppa energia, ad esempio, quando riceve troppa energia meccanica diciamo che abbiamo sbattuto contro qualcosa, e sentiamo dolore. A volte, l'energia termica ricevuta e' troppa, e sentiamo caldo o ci bruciamo, e se invece e' troppa quella ceduta sentiamo freddo. A volte, non abbiamo piu' energia per i nostri muscoli e per l'organismo per la fame e ci sentiamo deboli. Nei casi estremi, dobbiamo affrontare dei drammi, perche' l'amore del corpo verso la Natura, ci priva seriamente della nostra vita. In ogni caso, la nostra anima puo' pensare alla Pace, se si rende consapevole che Lei ama il corpo. Ogni cosa che le accade o deve affrontare nella Natura, e' dettata dal suo amore verso il corpo. Non e' quindi lei in castigo esigliata nella terra, piuttosto lei vive nella Natura e con la Natura, proprio perche' vive il suo forte desiderio di amare il corpo e quindi anche la Natura. 

Il corpo, anche se spiritualmente non e' il fine dell'esistenza, e' il punto di partenza fondamentale. Non possiamo esprimerci senza parlare, senza usare le corde vocali, o senza fare dei segni muovendo il corpo. Non possiamo voler bene ad un'altra anima senza pensare al suo corpo, pensando a dove si trova, se sente freddo o caldo, se e' stanca o riposata, se e' vicina ai corpi di chi vuole bene o no. Ne' possiamo essere vicini a colei che soffre rimanendo nelle nostre stanze, ma muovendoci nello spazio per raggiungerla, attrezzandoci per affrontare le distanze, il freddo, la fame, il tempo, e una volta raggiunta, abbracciarla, ascoltare la sua voce e vedere cio' che vede.
Solo con il corpo possiamo pregare o lodare Dio. Solo con il corpo possiamo vivere la pace e la gioia di Dio, e possiamo condividerle agli altri. E' vero anche che a volte la fede chiede di andare oltre il corpo, ma cio' non significa abbandonarlo, ma accentando i suoi limiti, continuare ad amare col nostro cuore.

Dal punto di vista psicologico il corpo, e di conseguenza tutta la Natura, e' lo strumento fondamentale tramite cui l'anima ama, e' cosi' fondamentale che l'anima e' anche corpo e Natura: l'anima vivente non e' come pensavano gli antichi distaccata dal corpo e dalla Natura, l'anima ama cosi' tanto il corpo che e' un tutt'uno. Ella e' influenzata dal corpo e il corpo dall'anima, in un intreccio forte e stretto, \footnote{
    \url{https://it.wikipedia.org/w/index.php?title=Unit\%C3\%A0\_psicofisica&oldid=121651506}
    La psicoanalisi nella sua globalità è l'insieme degli studi che analizzano le relazioni tra la dinamica ormonica e la costituzione della personalità psichica. 
}, il corpo e' influenzato dalla Natura e la Natura dal corpo.


\section{Perche' questa Natura?}
La materia e' necessaria per dar forma alla vita sociale ed e' sufficiente per creare e mantenere la vita che il nostro corpo desidera.

E' possibile scrivere senza una penna e una carta, o senza una macchina da scrivere e un supporto che memorizza cio' che e' scritto? 
La materia serve a noi per dare forma ai nostri sentimenti, per mantenere nel tempo cio' che e' importante. Potresti matematicamente obbiettare cosi': ``se io fossi un punto, ed ogni mio desiderio e bisogno fosse appagato, non avrei bisogno della materia, ne' della Natura''. E' vero, ma se tu desiderassi amare anche un solo altro essere simile a te, ci dovrebbe essere almeno un altro punto. Inoltre, non esiste amore se non c'e' un bisogno, un qualcosa da soddisfare, da colmare. E quindi questi punti dovrebbero avere un loro stato, ad esempio, contenere una certa quantita' di energia, variabile. Ecco che un punto potrebbe trasferire la sua energia ad un altro punto per amare. Ed ecco nata una Natura, molto astratta, ma pur sempre Natura: uno spazio che contiene punti, un'energia trasferibile, e un'energia fissabile nei punti (materia).

Ora, perche' non possiamo vivere in questo Universo astratto, che sembra molto piu' semplice da vivere? In fondo, e' da ammettere che la materia e' costosa. Trasportare anche pochi litri d'acqua e' faticoso. Il nostro corpo e' delicato, ha sempre bisogno di cure e puo' poco fisicamente rispetto a quanto noi a volte desideriamo. Tuttavia, un Universo piu' complesso consente la creazione di oggetti e corpi meravigliosi e belli, e consente di vivere esperienze piu' raffinate e grandi. Viceversa, un Universo piu' semplice diventerebbe banale: se non esistesse il peso della gravita', non esisterebbe il camminare, staremmo tutti fluttanti nello spazio come dei pesci. Se non esistesse la complessita' della chimica, e tutto fosse di una stessa sostanza, sarebbe tutto un mondo in bianco e nero, insapore. Se tutta la materia fosse commestibile, non esisterebbe l'evoluzione, e saremmo rimasti tutti delle cellule o al piu' delle amebe. 

E' chiaro che magari uno si puo' mettere alla ricerca di universi migliori, dove la forza di gravita' e' un po' meno faticosa, dove si ci puo' trasferire da un posto all'altro piu' facilmente (magari con un teletrasporto), anche se e' difficile\footnote{
E' difficile progettare un universo che possa accogliere la vita come la conosciamo. Ad esempio se non esistesse la gravita' non esisterebbe forse la vita, perche' non esisterebbe un pianeta Terra che raccolga in se l'acqua e tanti altri elementi indispensabili alla vita. Ancora, se non esistesse la massa, ad ogni piccola spinta, si viaggerebbe alla velocita' massima (della luce), e quindi, ancora, non potrebbero esistere dei corpi adatti alla vita. Se non esistessero le forze elettriche che consentono a particelle con la stessa carica di respingersi, tutto sarebbe fuso in una massa informe. Continuando ad analizzare la situazione, si vede che una piccolissima modifica delle costanti e delle forze fisiche dell'universo, risulta in un universo inadatto alla vita, vedi Principio Antropico per approfondire \url{https://en.wikipedia.org/wiki/Anthropic\_principle}.
}.
In questo Universo, ci sono situazioni, che conducono, alcune in maniera piu' rempentina e violenta, come incidenti, altre in maniera piu' lenta e dolce, come raggiungere tranquillamente gli ultimi anni della vecchiaia, alla morte. Per il corpo la morte non e' una maledizione ordinata da divinita' crudeli, e', come diceva San Francesco, ``sorella morte''. Amare anche se non si e' amati e' pure una forma di morte, che pero' per il corpo e' pienamente vivibile in maniera serena e gioiosa. I nostri problemi psicologici, la nostra immaturita' spirituale, ci allontana dal vivere con serenita' queste dimensioni, e quando ci troviamo costretti ad un faccia a faccia con alcune di loro, soffriamo e ci allontaniamo dal vivere naturalmente nel nostro universo\footnote{Vedi Alexander Lowen, cap. \ref{chapRiferimenti} pag. \pageref{chapRiferimenti}}.
Ma allora, solo risolvendo i nostri problemi psicologici e comprendendo i problemi di chi e' accanto a noi, crescendo tramite il percorso spirituale in cui abbiamo fede, e ritornando cosi' alla naturale unita' con l'anima e con il corpo potremo sentire di nuovo di far parte di questo universo e di non avere bisogno di altri, piu' allettanti, ma lontani, universi. E potremo dire che il nostro universo e' la realta' che la nostra anima ha desiderato fin dal principio vedere, sentire, toccare, capire e modellare con il suo corpo.

Se comunque l'idea di universi migliori alletta, continuiamo il discorso dicendo che per quanto migliore sia l'universo che possiamo progettare e costruire o, in qualche modo, scoprire e colonizzare, cio' non basterebbe a raggiungere la felicita'. Cosi' come le cardinalita' di Cantor, in cui esiste sempre un infinito piu' grande, in ogni universo in cui ci troveremo, desidereremo un universo migliore, piu' performante. Questo perche', fino a quando non risolveremo i problemi della nostra anima, la causa della nostra inquietudine, non bastera' tutta l'energia dell'universo ne' tutta la profondita' di una super intelligenza artificiale per colmare i nostri vuoti. 

Non e' svalutando il proprio universo e ricercandone uno ``migliore'' che si raggiunge la felicita'. Si puo' fare molto nel proprio universo, ambiente e cultura, qualsiasi essa sia. Si puo' diventare molto bravi. E, solo diventando bravi nel proprio universo, poi nel tempo, con la ricerca si possono trovare universi migliori.
Per confermare cio', consideriamo che molte scoperte e innovazioni scientifiche sono state dettate da innovazioni nella tecnica. Se il commercio dei materiali non fosse stato maturo, e gli artigiani non avessero raggiunto un buon livello di abilita' nella lavorazione di quei materiali, non ci sarebbero stati i salti tecnologici e senza questi non ci sarebbero stati i salti nella scienza\footnote{ad esempio, con l'avvento dei telescopi e dei microscopi. \url{https://www.google.com/search?channel=fs&client=ubuntu\&q=How+has+technology+helped+in+the+advancement+of+scientific+discoveries\%3F}}
Cio' vale anche nel proprio piccolo. Ad esempio, e' solo diventando bravi nel proprio lavoro o ruolo che si puo' poi ambire a lavori ``migliori''. E' solo apprezzando il meglio della propria terra, che si possono apprezzare altri paesi. E' solo amando i propri genitori che si puo' essere genitori migliori (o al pari) di loro, anche se loro hanno avuto carenze nell'esserlo.

%Una nota scientifica: perche' l'universo in cui stiamo rispetta esattamente delle regolarita' e queste e non altre regolarita'? Una prima risposta e' il ``principio antropico'' (vedi wikipedia). La domanda a seguire e', anche se il principio antropico e' da rispettare, perche' ci sono queste regolarita' e non altre, che tuttavia consentirebbero la vita? In realta', sembrerebbe che le leggi dell'universo non sono costanti nello spazio, vedi \footnote{Variazione nello spazio della costante alpha di fine struttura\url{https://en.wikipedia.org/wiki/Fine-structure\_constant\#Spatial\_variation\_\%E2\%80\%93\_Australian\_dipole}}. Ad ogni modo, la materia esiste, ed esiste come le pare e piace, fino a quando vuole\footnote{questa affermazione e' coerente con l'approccio scientifico basato sulle osservazioni: anche se una legge scientifica stabilisce che anche domani il Sole sorgera', nulla proibisce alla Natura di non far accadere cio'. Semplicemente, gli scienziati dovranno poi prendere atto che il Sole non e' sorto (e' diventato all'istante di pietra?)}.

L'uomo, e' materia che si evolve secondo le leggi fisiche. Tuttavia, anche se tutti i nostri pensieri, sensazioni ed emozioni, sono il risultato di una macchina che e' il nostro corpo, non siamo una briciola nell'universo. Siamo cio' che da' senso ad ogni cosa. Anche perche', gli atomi non hanno il concetto di grande o piccolo. Il Sole non e' per un atomo tanto piu' grande del suo naso, ad esempio. La numerosita' delle stelle, non e' per la pietra ferma-carte sopra la scrivania, tanto piu' sorprendente del numero dei suoi spigoli. Che un vulcano esploda o crolli, non e' per le rocce che lo componevano e che vengono distrutte tanto piu' significativo di essere rimaste a formare il cono del vulcano per centinaia di migliaia di anni.

In conclusione, la materia e' l'inchiostro e la carta con cui Dio scrive il libro della Natura, per dire a chi a bisogno e desidera: ``Ti amo, sei una parte preziosa di me''. A volte, dirlo e' difficile e doloroso, a volte e' piu' piacevole ed entusiasmante. Ma solo essendo la parola che Lui vuole, nelle Sue frasi, la natura, gli altri e noi stessi non saranno ostili, incomprensibili e aridi e l'universo diventera' una armoniosa danza della materia.



