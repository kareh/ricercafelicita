\chapter{Riferimenti}
\label{chapRiferimenti}

  \url{https://github.com/kareh/calabella}\\
  ``Calabella''\\
  Calabella e' un racconto ispirato alla storia vera di mio nonno paterno, che da bambino, orfano di padre, di famiglia povera, ha dovuto affrontare tante difficolta'. Da quando il protagonista incontra l'alchimista, la storia diventa di fantasia, e racconta come sarebbe potuto andare se mio nonno avesse incontrato una guida spirituale.\\

  ``Il treno dei bambini, Viola Ardone''\\
  Nell'essere poveri sembra che e' il non avere che non consente l'amare. Non si ci rende conto di avere gia' tutto cio' che e' divino, e non amando si perde cio' che veramente conta. Cosi', nella poverta' si diventa duri. E le maniera dure generano delle rotture, che la Vita impieghera' molto tempo per riparare. \\

  ``Alexander Lowen, Il piacere''\\
  Il piacere e' la sensazione dello stare bene. Il mondo moderno decadente, illude le persone
  con falsi piaceri, ben lontani da quelli primordiarli e autentici, come quello, ad esempio,
  di respirare. Un libro dello psico-terapeuta, Alexander Lowen.\\
  
  ``Eric Berne, Ciao... e poi''\\
  L'approccio dello psico-terapeuta Eric Berne per la risoluzione dei conflitti interiori.\\
  
  ``Luigi Maria Epicoco. Sale, Non Miele. Per una fede che brucia.``\\
  Luigi Maria Epicoco e' un presbitero, teologo, filosofo e scrittore italiano. Questo libro e' un'introduzione sia semplice, sia teologica al mettersi in cammino in un percorso di fede.\\

  ``Ilahi Kitabi, A book of Ilahis''\\
  \url{http://www.rifai.org/sufism/wp-content/uploads/2012/02/a-book-of-ilahis.pdf}\\
  \url{https://en.wikipedia.org/wiki/Sufism}\\
  I veri dervisci sono dediti alla ricerca dell'Amore puro e universale, e sono pronti a
  dedicare la loro vita per difenderlo e manifestarlo.\\

  ``Richard David Precht, Ma io chi sono? (ed eventualmente, quanti sono?)''\\
  Viaggio nella filosofia tramite un approccio moderno: neurofisiologia, accenni a The Matrix, psicologia ed altro.\\
   
