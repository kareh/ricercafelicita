\chapter{Versi}

Poesie concise, ispirate agli Haiku zen.\\

\vfill

\section{Tracce sul cammino}

\centerpoemon{L'universo che ama,}
\begin{haiku}
L'universo,\\
infinito, freddo.\\
\leavevmode\\
L'amore,\\
caldo, limitato.\\
\leavevmode\\
Cosa altro cercare?\\
\end{haiku}


\begin{haiku}
Adattarsi \\
vuol dire\\
non forzare.\\
\end{haiku}

\centerpoemon{le macchine gia' giunte alla Meta}
\begin{haiku}
Nelle strade larghe,\\
le macchine superano.\\
\leavevmode\\
Nelle strade strette,\\
chi non vede chi veramente lo rallenta,\\
suona,\\
chi Lo vede, soffre,\\
e soffrendo per contempla la Sua pace.\\
\end{haiku}

\centerpoemon{e piano, a volte inaspettatemente,}
\begin{haiku}
Dal lunedi' al venerdi',\\
in miniera lavoro.\\
\leavevmode\\
Sabato mattina,\\
il Sole acceca.\\
\leavevmode\\
Poi, il vento\\
mi riconduce al Suo amore,\\
e piano, a volte inaspettatamente,\\
un incantevole fiore sboccia.\\
\leavevmode\\
Con nuova forza,\\
lunedi' al lavoro!\\
\end{haiku}

\centerpoemon{per accogliere, accettare e non chiedere oltre}
\begin{haiku}
Il vero Guerriero\\
combatte solo con se stesso,\\
per annientare il suo ego.\\
\leavevmode\\
Per dare senza aspettare un grazie.\\
Per accogliere e ricevere solo\\
quello chi gli viene donato.\\
\leavevmode\\
Contento di cio' che esiste,\\
esalta le richezze delle terre piu' povere,\\
si rallegra delle virtu', piccole o grandi,\\
di chi incontra.\\
\end{haiku}

\centerpoemon{Con quattro giochi,}
\begin{haiku}
Con quattro giochi,\\
semplici e poveri,\\
felici.\\
\end{haiku}

\centerpoemon{
nuotando piano piano,
}
\begin{haiku}
La boa lontana,\\
nuotando piano piano,\\
si avvicina.\\
\end{haiku}

\centerpoemon{
la nostra vela viene facilmente spinta da
}
\begin{haiku}
Non perche' il sole e' molto grande,\\
noi siamo molto piccoli.\\
\leavevmode\\
Infatti, proseguendo nella grande rotta,\\
la nostra vela viene facilmente spinta da\\
un forte e sicuro vento,\\
generato da un grande sole.\\
\end{haiku}

\centerpoemon{di criptografia, implementano server e programmi}
\begin{haiku}
Il vero matematico e' colui che\\
crede che l'Amore esiste\\
e che vive per dimostrarlo.\\
\leavevmode\\
Nota: a volte, Egli (o Ella) e' un
matematico dei numeri, delle forme e delle regole\\
e dimostra teoremi per arrivare alla\\
stessa conclusione.\\
Ad esempio, Eulero con il suo teorema,\\
ha permesso ai matematici di capire meglio\\
la teoria dei numeri, e agli informatici\\
di ideare la criptografia R.S.A.\\
I programmatori, avvalendosi degli algoritmi\\
di criptografia, implementano server e programmi\\
sicuri, un esempio e' https, che protegge\\
il traffico web wifi, cosi' quando\\
noi ci logghiamo su un sito come Gmail o poste.it,\\
nessun ladro puo' captare la nostra password\\
con un antenna.\\
\end{haiku}

\centerpoemon{Cambiare allontandosi dal non stare bene,}
\begin{haiku}
    Amore:\\
    stare bene.\\
    \leavevmode\\
    Cosi' come si e',\\
    per cosi' come gli altri sono,\\
    per cosi' come l'ambiente e'.\\
    \leavevmode\\
    Se non si sta' bene,\\
    cambiare, esortare, lavorare,\\
    camminando verso l'infinito bene.\\
    \leavevmode\\
    Pregare affinche' anche gli altri\\
    stiano bene,\\
    e sempre meglio provino\\
    questo completo,\\
    autentico piacere.\\
    \leavevmode\\
    Per quanto stare e cambiare,\\
    sia difficile.\\
\end{haiku}

\centerpoemon{affinche' non smetta mai di esserlo.}
\begin{haiku}
    Sia Dio\\
    la personificazione\\
    dell'Amore.\\
    \leavevmode\\
    E la religione,\\
    l'emulare l'Amore\\
    tramite regole, pensieri, parole,\\
    fino a che il proprio respiro\\
    diventi Amore,\\
    affinche' non smetta mai di esserlo.\\
\end{haiku}

%\centerpoemon{Quando l'avrai fatto abbondantemente,}
%\begin{haiku}
%    Non lamentarti \\
%    per coloro che non vogliono cambiare. \\
%    \leavevmode \\
%    Fatica \\
%    per chi lo desidera. \\
%    \leavevmode \\
%    Quando l'avrai fatto abbondantemente,\\
%    chiediti: se cambiare, \\
%    vuol dire avvicinarsi a Lui,\\
%    chi non lo desidera ?\\
%    \leavevmode\\
%    E il tuo dolore, scomparso,\\
%    sara' comprensione e carita'.\\
%\end{haiku}



\centerpoemon{ Gli animali piu' semplici }
\begin{haiku}
    Gli animali piu' semplici\\
    capiscono solo\\
    se stanno bene o male.\\
    E questo li rende felici.\\
\end{haiku}

\centerpoemon{Non c'e' niente delle meraviglie della scienza}
\begin{haiku}
    Non c'e' niente delle meraviglie della scienza,\\
    sia fisiche, microscopiche o cosmische,\\
    sia psicologiche, emotive, individuali, collettive o globali,\\
    che eguagli la bellezza della religione,\\
    del consumare la vita,\\
    per il bene a cui la vita crede.\\
    \leavevmode\\
    Ma ogni meraviglia della scienza,\\
    ricorda la Verita' e indirizza a Lei:\\
    Dio vuole la vita,\\
    non la morte.\\
    \leavevmode\\
    Dio non ammette uno spreco\\
    di neanche un infinitesimo di vita.\\
    E se sulla terra il corpo si logora nel tempo,\\
    e l'animo e' messo a continue prove,\\
    e' perche' il nostro corpo, ama la Natura,\\
    con ogni suo atomo.\\
    Quindi, se l'Io si ostina a volere di piu'\\
    di quello che puo' prendere dalla Natura,\\
    se testardamente non sacrifica niente di se'\\
    per voler vincere la sorte naturale,\\
    viviamo fatica e paura.\\
    Allo stesso modo, il nostro animo, \\
    limitato fisicamente nei suoi desideri,\\
    soffre a volte molto nel non realizzarli.\\
    ...\\
\end{haiku}

\centerpoemon{in queste Terre con un Sole ed una Luna.}
\begin{haiku}
    ... Solo amando il nostro corpo,\\
    gioiendo dei suoi doni,\\
    apprezzando i suoi limiti,
    e rendendolo Suo tempio,\\
    ritroveremo il senso della vita fisica,\\
    in queste Terre con un Sole ed una Luna.\\
\end{haiku}


\centerpoemon{e che hai suoi occhi ogni fuocherello di ogni cellula,}
\begin{haiku}
    Siamo un atomo, tra gli atomi,\\
    mossi dalle maree e dai venti \\
    della Natura e della Societa'.\\
    A stento riusciamo a muovere noi stessi,\\
    e tutto ci si oppone.\\
    \leavevmode\\
    Ma come legna, bruciamo!\\
    Che sia poco il nostro peso,\\
    o difettoso ed umido il nostro coraggio,\\
    che ci siano inteperie e pioggie\\
    che attutiscano il nostro fuoco,\\
    bruciamo.\\
    \leavevmode\\
    Bruciamo, per il nostro Signore,\\
    che ha stabilito sacre le nostre cellule,\\
    e che ai suoi occhi ogni fuocherello di ognuna di esse,\\
    illumina il profondo spazio,\\
    piu' intensamente di tutte le galassie.\\
\end{haiku}

\centerpoemon{senza operazioni.}
\begin{haiku}
    Computare\\
    un numero\\
    senza operazioni.\\
\end{haiku}

\centerpoemon{Io, per l'Altro, sono diverso o uguale a lui,}
\begin{haiku}
    L'Altro e' altro da me,\\
    al pari di me.\\
    Io, per l'Altro, sono diverso o uguale a lui,\\
    e piu', ugualmente o meno importante di lui.\\
    Perche' voglio vivere solo di carita',\\
    l'amore del Signore, che e' buono,\\
    ed ama senza chiedere,\\
    ed ama spontaneamente e di Sua volonta'.\\
\end{haiku}

\centerpoemon{della complessita' e della Natura}
\begin{haiku}
    Dio e' morto?\\
    Si, perche', forti della scienza,\\
    lo riduciamo a noi stessi,\\
    e miseri di fronte all'infinito\\
    della complessita', della Natura\\
    e del male generato dall'uomo,\\
    ci duoliamo della nostra sorte.\\
    \leavevmode\\
    Dio e' in Pace\\
    ed e' Pace\\
    per ogni uomo.\\
    Anche se Lui\\
    e' denudato\\
    di ogni gloria,\\
    di ogni affetto e riguardo umano,\\
    eternamente e' Pace.\\
    Anche se Lui\\
    e' emarginato e rifiutato,\\
    stabilmente, incessabilmente,\\
    e' in Pace\\
    ed e' Pace.\\
    \leavevmode\\
    Dio e' vero e presente,\\
    e lo ha dimostrato\\
    Gesu' Cristo,\\
    e lo dimostra\\
    chi lo ama,\\
    con la propria vita.\\
\end{haiku}

\centerpoemon{dell'Anima}
\begin{haiku}
    Chiesa\\
    palestra\\
    dell'Anima\\
\end{haiku}
