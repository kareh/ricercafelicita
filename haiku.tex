\chapter{Tracce sul cammino}

Poesie, inizialmente ispirate agli Haiku zen, ognuna delle quali segna un punto del mio personale cammino, e del cammino che credo ognuno di noi compira', similmente.

\vfill

\centerpoemon{L'universo che ama,}
\begin{haiku}
L'universo,\\
infinito, freddo.\\
\leavevmode\\
L'amore,\\
caldo, limitato.\\
\leavevmode\\
Cosa altro cercare?\\
\end{haiku}


\begin{haiku}
Adattarsi \\
vuol dire\\
non forzare.\\
\end{haiku}

\centerpoemon{e piano, a volte inaspettatemente,}
\begin{haiku}
Dal lunedi' al venerdi',\\
in miniera lavoro.\\
\leavevmode\\
Sabato mattina,\\
il Sole acceca.\\
\leavevmode\\
Poi, il vento\\
mi riconduce al Suo amore,\\
e piano, a volte inaspettatamente,\\
un incantevole fiore sboccia.\\
\leavevmode\\
Con nuova forza,\\
lunedi' al lavoro!\\
\end{haiku}

\centerpoemon{per accogliere, accettare e non chiedere oltre}
\begin{haiku}
Il vero Guerriero\\
combatte solo con se stesso,\\
per annientare il suo ego.\\
\leavevmode\\
Per dare senza aspettare un grazie.\\
Per accogliere e ricevere solo\\
quello chi gli viene donato.\\
\leavevmode\\
Contento di cio' che esiste,\\
esalta le richezze delle terre piu' povere,\\
si rallegra delle virtu', piccole o grandi,\\
di chi incontra.\\
\end{haiku}

\centerpoemon{Con quattro giochi,}
\begin{haiku}
Con quattro giochi,\\
semplici e poveri,\\
felici.\\
\end{haiku}

\centerpoemon{
nuotando piano piano,
}
\begin{haiku}
La boa lontana,\\
nuotando piano piano,\\
si avvicina.\\
\end{haiku}

\centerpoemon{
la nostra vela viene facilmente spinta da
}
\begin{haiku}
Non perche' il sole e' molto grande,\\
noi siamo molto piccoli.\\
\leavevmode\\
Infatti, proseguendo nella grande rotta,\\
la nostra vela viene facilmente spinta da\\
un forte e sicuro vento,\\
generato da un grande sole.\\
\end{haiku}

\centerpoemon{di criptografia, implementano server e programmi}
\begin{haiku}
Il vero matematico e' colui che\\
crede che l'Amore esiste\\
e che vive per dimostrarlo.\\
\leavevmode\\
Nota: a volte, Egli (o Ella) e' un
matematico dei numeri, delle forme e delle regole\\
e dimostra teoremi per arrivare alla\\
stessa conclusione.\\
Ad esempio, Eulero con il suo teorema,\\
ha permesso ai matematici di capire meglio\\
la teoria dei numeri, e agli informatici\\
di ideare la criptografia R.S.A.\\
I programmatori, avvalendosi degli algoritmi\\
di criptografia, implementano server e programmi\\
sicuri, un esempio e' https, che protegge\\
il traffico web wifi, cosi' quando\\
noi ci logghiamo su un sito come Gmail o poste.it,\\
nessun ladro puo' captare la nostra password\\
con un antenna.\\
\end{haiku}

\centerpoemon{Cambiare allontandosi dal non stare bene,}
\begin{haiku}
    Amore: \\
    stare bene.\\
    Partendo da come si e',\\
    da come gli altri sono,\\
    da come l'ambiente e',\\
    per stare meglio.\\
    \leavevmode\\
    Da soli,\\
    con altri.\\
    Nel deserto,\\
    nella valle in fiore.\\
    \leavevmode\\
    Se non si sta' bene,\\
    cambiare, lavorare,\\
    camminando verso l'infinito bene.\\
    \leavevmode\\
    Volgere il proprio stare,\\
    affinche' anche gli altri\\
    stiano bene,\\
    ma ancora di piu',\\
    anche loro cerchino \\
    questo completo,\\
    autentico piacere.\\
    \leavevmode\\
    Per quanto stare e cambiare,\\
    sia difficile.\\
\end{haiku}

\centerpoemon{affinche' non smetta mai di esserlo.}
\begin{haiku}
    Sia Dio\\
    la personificazione\\
    dell'Amore.\\
    \leavevmode\\
    E la religione,\\
    l'emulare l'Amore\\
    tramite regole, pensieri, parole,\\
    fino a che il proprio respiro\\
    diventi Amore,\\
    affinche' non smetta mai di esserlo.\\
\end{haiku}

%\centerpoemon{Quando l'avrai fatto abbondantemente,}
%\begin{haiku}
%    Non lamentarti \\
%    per coloro che non vogliono cambiare. \\
%    \leavevmode \\
%    Fatica \\
%    per chi lo desidera. \\
%    \leavevmode \\
%    Quando l'avrai fatto abbondantemente,\\
%    chiediti: se cambiare, \\
%    vuol dire avvicinarsi a Lui,\\
%    chi non lo desidera ?\\
%    \leavevmode\\
%    E il tuo dolore, scomparso,\\
%    sara' comprensione e carita'.\\
%\end{haiku}



\centerpoemon{ Gli animali piu' semplici }
\begin{haiku}
    Gli animali\\
    piu' semplici,\\
    capiscono solo,\\
    se stanno bene o male.\\
    E questo li rende felici.\\
\end{haiku}

\centerpoemon{Non c'e' niente delle meraviglie della scienza}
\begin{haiku}
    Non c'e' niente delle meraviglie della scienza,\\
    sia fisiche, microscopiche o cosmische,\\
    sia psicologiche, emotive, individuali, collettive o globali,\\
    che eguagli la bellezza della religione,\\
    del consumare la vita,\\
    per il bene a cui la vita crede.\\
    \leavevmode\\
    Ma ogni meraviglia della scienza,\\
    ricorda la Verita' e indirizza a Lei:\\
    Dio vuole la vita,\\
    non la morte.\\
    \leavevmode\\
    Dio non ammette uno spreco\\
    di neanche un infinitesimo di vita.\\
    E se sulla terra il corpo si logora nel tempo,\\
    e l'animo e' messo a continue prove,\\
    e' perche' il nostro corpo, ama la Natura,\\
    con ogni suo atomo.\\
    Quindi, se l'Io si ostina a volere di piu'\\
    di quello che puo' prendere dalla Natura,\\
    se testardamente non sacrifica niente di se'\\
    per voler vincere la sorte naturale,\\
    viviamo fatica e paura.\\
    Allo stesso modo, il nostro animo, \\
    limitato fisicamente nei suoi desideri,\\
    soffre a volte molto nel non realizzarli.\\
    ...\\
\end{haiku}

\centerpoemon{in queste Terre con un Sole ed una Luna.}
\begin{haiku}
    ... Solo amando il nostro corpo,\\
    gioiendo dei suoi doni,\\
    apprezzando i suoi limiti,
    e rendendolo Suo tempio,\\
    ritroveremo il senso della vita fisica,\\
    in queste Terre con un Sole ed una Luna.\\
\end{haiku}


\centerpoemon{e che hai suoi occhi ogni fuocherello di ogni cellula,}
\begin{haiku}
    Siamo un atomo, tra gli atomi,\\
    mossi dalle maree e dai venti \\
    della Natura e della Societa'.\\
    A stento riusciamo a muovere noi stessi,\\
    e tutto ci si oppone.\\
    \leavevmode\\
    Ma come legna, bruciamo!\\
    Che sia poco il nostro peso,\\
    o difettoso ed umido il nostro coraggio,\\
    che ci siano inteperie e pioggie\\
    che attutiscano il nostro fuoco,\\
    bruciamo.\\
    \leavevmode\\
    Bruciamo, per il nostro Signore,\\
    che ha stabilito sacre le nostre cellule,\\
    e che ai suoi occhi ogni fuocherello di ognuna di esse,\\
    illumina il profondo spazio,\\
    piu' intensamente di tutte le galassie.\\
\end{haiku}

\centerpoemon{senza operazioni.}
\begin{haiku}
    Computare\\
    un numero\\
    senza operazioni.\\
\end{haiku}

\centerpoemon{Io, per l'Altro, posso anche essere rimpicciolito}
\begin{haiku}
    L'Altro e' altro da me,\\
    al pari di me.\\
    Io, per l'Altro, sono cio' che sono per lui,\\
    che io sia io,\\
    che io sia niente.\\
    Perche' il Signore e' grande,\\
    e' buono,\\
    ed ama senza aver bisogno di essere amato,\\
    tutti.\\
\end{haiku}

\centerpoemon{della complessita' e della Natura}
\begin{haiku}
    Dio e' morto?\\
    Si, perche', forti della scienza,\\
    lo riduciamo a noi stessi,\\
    e miseri di fronte all'infinito\\
    della complessita', della Natura\\
    e del male generato dall'uomo,\\
    ci duoliamo della nostra sorte.\\
    \leavevmode\\
    Dio e' in Pace\\
    ed e' Pace\\
    per ogni uomo.\\
    Anche se Lui\\
    e' denudato\\
    di ogni gloria,\\
    di ogni affetto e riguardo umano,\\
    eternamente e' Pace.\\
    Anche se Lui\\
    e' emarginato e rifiutato,\\
    stabilmente, incessabilmente,\\
    e' in Pace\\
    ed e' Pace.\\
    \leavevmode\\
    Dio e' vero e presente,\\
    e lo ha dimostrato\\
    Gesu' Cristo,\\
    e lo dimostra\\
    chi lo ama,\\
    con la propria vita.\\
\end{haiku}

\centerpoemon{dell'Anima}
\begin{haiku}
    Chiesa\\
    palestra\\
    dell'Anima\\
\end{haiku}

\centerpoemon{L'Io sopra tutti ed ogni cosa,}
\begin{haiku}
    L'ignoranza\\
    priva dell'infinita energia\\
    dell'Universo.\\
    \leavevmode\\
    L'Io, sopra tutti ed ogni cosa,\\
    dell'infinito amore\\
    di Dio.\\
\end{haiku}

\centerpoemon{supereremo anche i casi piu' difficili,}
\begin{haiku}
Dio\\
sacra persona\\
di infinita bonta'.\\
\leavevmode\\
Tu,\\
non piu' ideale del sognatore\\
ne' promessa del profeta,\\
esisti se io ti amo,\\
amando la vita\\
tramite solo la carita'.\\
\leavevmode\\
E la tua esistenza\\
e' fonte di Pace,\\
di momenti di vera gioia,\\
piu' preziosi\\
del metallo piu' nobile,\\
piu' meravigliosi\\
dello splendore del cosmo.\\
\leavevmode\\
Gesu',\\
dono Tuo esemplare,\\
dimostra che Tu esisti\\
sempre,\\
e che amandoti,\\
supereremo anche i casi piu' difficili,\\
ed anche nelle condizioni piu' dure\\
avremo coraggio.\\
\end{haiku}

\begin{haiku}
    Ah Dio,\\
    sei cosi' vicino a me,\\
    cosi' talmente vicino,\\
    che per cosi' tanti anni\\
    ti ho passato, guardando avanti,\\
    cercato nei posti piu' remoti\\
    e arcani\\
    e non ti vedevo.\\
    \leavevmode\\
    E non era per segreto\\
    che gli antichi \\
    non ti enunciavano a gran voce,\\
    in parole piu' semplici\\
    piu' dirette.\\
    \leavevmode\\
    Era il mio ego,\\
    il mio orgoglio,\\
    la mia vanita',\\
    le mie passioni per gli argenti,\\
    i miei inquieti amori\\
    per le tue creature,\\
    che oscuravano\\
    il Tuo amore.\\
\end{haiku}
