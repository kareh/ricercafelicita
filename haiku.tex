\chapter{Versi}

Poesie concise, ispirate agli Haiku zen.\\

\vfill

\section{Tracce sul cammino}

\centerpoemon{L'universo che ama,}
\begin{haiku}
L'universo,\\
infinito, freddo.\\
\leavevmode\\
L'amore,\\
caldo, limitato.\\
\leavevmode\\
Cosa altro cercare?\\
\end{haiku}


\begin{haiku}
Adattarsi \\
vuol dire\\
non forzare.\\
\end{haiku}

\centerpoemon{le macchine gia' giunte alla Meta}
\begin{haiku}
Nelle strade larghe,\\
le macchine superano,\\
non suonano.\\
\leavevmode\\
Nelle strade strette,\\
chi alla Meta e' gia' arrivato,\\
non suona.\\
\end{haiku}

\centerpoemon{e piano, a volte inaspettatemente,}
\begin{haiku}
Dal lunedi' al venerdi',\\
in miniera lavoro.\\
\leavevmode\\
Sabato mattina,\\
il Sole acceca.\\
\leavevmode\\
Poi, il vento\\
mi riconduce al Suo amore,\\
e piano, a volte inaspettatemente,\\
un incantevole fiore sboccia.\\
\leavevmode\\
Con nuova forza,\\
lunedi' al lavoro!\\
\end{haiku}

\centerpoemon{per accogliere, accettare e non chiedere oltre}
\begin{haiku}
Il vero Guerriero\\
combatte solo con se stesso,\\
per annientare il suo ego.\\
\leavevmode\\
Per dare senza aspettare un grazie.\\
Per accogliere e ricevere solo\\
quello chi gli viene donato.\\
\leavevmode\\
Contento di cio' che esiste,\\
esalta le richezze delle terre piu' povere,\\
si rallegra delle virtu', piccole o grandi,\\
di chi incontra.\\
\end{haiku}

\centerpoemon{Con quattro giochi,}
\begin{haiku}
Con quattro giochi,\\
semplici e poveri,\\
felici.\\
\end{haiku}

\centerpoemon{
nuotando piano piano,
}
\begin{haiku}
La boa lontana,\\
nuotando piano piano,\\
si avvicina.\\
\end{haiku}

\centerpoemon{
la nostra vela viene facilmente spinta da
}
\begin{haiku}
Non perche' il sole e' molto grande,\\
noi siamo molto piccoli.\\
\leavevmode\\
Infatti, proseguendo nella grande rotta,\\
la nostra vela viene facilmente spinta da\\
un forte e sicuro vento,\\
generato da un grande sole.\\
\end{haiku}

\centerpoemon{di criptografia, implementano server e programmi}
\begin{haiku}
Il vero matematico e' colui che\\
crede che l'Amore esiste\\
e che vive per dimostrarlo.\\
\leavevmode\\
Nota: a volte, Egli (o Ella) e' un
matematico dei numeri, delle forme e delle regole\\
e dimostra teoremi per arrivare alla\\
stessa conclusione.\\
Ad esempio, Eulero con il suo teorema,\\
ha permesso ai matematici di capire meglio\\
la teoria dei numeri, e agli informatici\\
di ideare la criptografia R.S.A.\\
I programmatori, avvalendosi degli algoritmi\\
di criptografia, implementano server e programmi\\
sicuri, un esempio e' https, che protegge\\
il traffico web wifi, cosi' quando\\
noi ci logghiamo su un sito come Gmail o poste.it,\\
nessun ladro puo' captare la nostra password\\
con un antenna.\\
\end{haiku}

\centerpoemon{Cambiare allontandosi dal non stare bene,}
\begin{haiku}
    Amore:\\
    stare bene.\\
    \leavevmode\\
    Cosi' come si e',\\
    per cosi' come gli altri sono,\\
    per cosi' come l'ambiente e'.\\
    \leavevmode\\
    Se non si sta' bene,\\
    cambiare, esortare, lavorare,\\
    camminando verso l'infinito bene.\\
    \leavevmode\\
    Pregare affinche' anche gli altri\\
    stiano bene,\\
    e sempre meglio provino\\
    questo completo,\\
    autentico piacere.\\
    \leavevmode\\
    Per quanto stare e cambiare,\\
    sia difficile.\\
\end{haiku}

\centerpoemon{affinche' non smetta mai di esserlo.}
\begin{haiku}
    Sia Dio\\
    la personificazione\\
    dell'Amore.\\
    \leavevmode\\
    E la religione,\\
    l'emulare l'Amore\\
    tramite regole, pensieri, parole,\\
    fino a che il proprio respiro\\
    diventi Amore,\\
    affinche' non smetta mai di esserlo.\\
\end{haiku}

\small{Nota: queste definizioni di Dio, Amore, Religione sono intese nel loro senso puro, slegate da dottrine, chiese, partiti e altre organizzazioni.}

%\begin{haiku}
%    La Natura e' bellissima, fredda o troppo calda, e senza intelletto.\\
%    Ma e' fedele a Lui, e tutta la sua abbondanza e' per chi la comprende, con pazienza, resistenza e audacia.
%\end{haiku}

\centerpoemon{per coloro che non vogliono cambiare.}
\begin{haiku}
    Non lamentarti\\
    per coloro che non vogliono cambiare.\\
    Fatica\\
    per chi desidera essere vicino a Lui.\\
    \leavevmode\\
    E, in verita',\\
    chi non lo desidera\footnote{Forse un essere incosciente \emph{vuole} essere dove sta', lontano poco o molto dal Signore, ma tutti \emph{desiderano} stare vicino a Lui, perche' Lui e' l'appagamento dei loro desideri piu' sani, belli e potenti}?\\
\end{haiku}

\centerpoemon{ Gli animali piu' semplici }
\begin{haiku}
    Gli animali piu' semplici\\
    capiscono solo\\
    se stanno bene o male.\\
    E questo li rende felici.\\
\end{haiku}

\centerpoemon{Non c'e' niente delle meraviglie della scienza,}
\begin{haiku}
    Non c'e' niente delle meraviglie della scienza,\\
    sia fisiche, microscopiche o cosmische,\\
    sia psicologiche, emotive, individuali, collettive o globali,\\
    che eguagli la bellezza della religione,\\
    del sacrificio della propria vita,\\
    per il bene a cui crediamo.\\
    \leavevmode\\
    Ma ogni meraviglia della scienza,\\
    ricorda la Verita' e indirizza a Lei:\\
    Dio vuole la vita,\\
    non la morte.\\
    Dio non ammette uno spreco di neanche un infinitesimo della vita,\\
    e se sulla terra il corpo si logora nel tempo,\\
    e l'animo e' messo a continue prove,\\
    e' perche Lui ama la nostra madre Natura,\\
    cosi' come il nostro corpo, con i suoi atomi, la ama.\\
\end{haiku}
