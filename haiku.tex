\chapter{Tracce sul cammino}

\centerpoemon{L'universo che ama,}
\begin{haiku}
L'universo,\\
infinito, freddo.\\
\leavevmode\\
L'amore umano,\\
caldo, limitato.\\
\leavevmode\\
Cos'altro cercare?\\
\end{haiku}

\centerpoemon{Dimostra che l'amore esiste,}
\begin{haiku}
    Dimostra che\\
    l'amore esiste,\\
    vero matematico!\\
\end{haiku}

\centerpoemon{Cambiare allontandosi dal non stare bene,}
\begin{haiku}
    Vivo,\\
    quando sto' bene.\\
    Amo,\\
    quando stai bene.\\
\end{haiku}

\begin{haiku}
Adattarsi \\
vuol dire\\
non forzare.\\
\end{haiku}

\centerpoemon{e piano, a volte inaspettatemente,}
\begin{haiku}
Dal lunedi' al venerdi',\\
in miniera lavoro.\\
\leavevmode\\
Sabato mattina,\\
il Sole acceca.\\
\leavevmode\\
Poi, il vento\\
mi riconduce al Suo amore,\\
e piano, a volte inaspettatamente,\\
un incantevole fiore sboccia.\\
\leavevmode\\
Con nuova forza,\\
lunedi' al lavoro!\\
\end{haiku}

\centerpoemon{per accogliere, accettare e non chiedere oltre}
\begin{haiku}
Il vero Guerriero\\
con se stesso\\
combatte.\\
\leavevmode\\
Per annientare \\
    il suo ego.\\
Per dare e \\
    non aspettare \\
    un grazie.\\
Per farsi carico\\
e proteggere\\
cio' che e' a lui\\
donato e\\
nient'altro \\
ricevere.\\
\leavevmode\\
Contento di cio' che esiste,\\
le richezze delle terre\\
povere esalta,\\
le miserie di quelle ricche\\
non trascura.\\
\end{haiku}

\centerpoemon{Con quattro giochi,}
\begin{haiku}
Noi che siamo\\
ancora bambini,\\
con quattro giochi,\\
semplici e poveri,\\
felici.\\
\end{haiku}

\centerpoemon{affinche' non smetta mai di esserlo.}
\begin{haiku}
    Sia Dio\\
    la personificazione\\
    dell'Amore.\\
    \leavevmode\\
    E la religione\\
    l'imitare,\\
    il ricercare,\\
    il praticare,\\
    l'Amore.\\
    \leavevmode\\
    Tramite regole, \\
    pensieri,\\
    parole.\\
    \leavevmode\\
    Perche' il mio respiro\\
    diventi Amore,\\
    e non smetta mai\\
    di esserlo,\\
    per me,\\
    per chi bene mi vuole,\\
    per chi bene mi vorrebbe,\\
    per chi bene mai vorrebbe.\\\
\end{haiku}

%\centerpoemon{Quando l'avrai fatto abbondantemente,}
%\begin{haiku}
%    Non lamentarti \\
%    per coloro che non vogliono cambiare. \\
%    \leavevmode \\
%    Fatica \\
%    per chi lo desidera. \\
%    \leavevmode \\
%    Quando l'avrai fatto abbondantemente,\\
%    chiediti: se cambiare, \\
%    vuol dire avvicinarsi a Lui,\\
%    chi non lo desidera ?\\
%    \leavevmode\\
%    E il tuo dolore, scomparso,\\
%    sara' comprensione e carita'.\\
%\end{haiku}



\centerpoemon{ Gli animali piu' semplici }
\begin{haiku}
    Gli animali\\
    piu' semplici,\\
    capiscono solo,\\
    se stanno bene o male.\\
    E questo li rende felici.\\
\end{haiku}

\centerpoemon{ma per vivere piu' nitidamente, semplicemente,}
\begin{haiku}
    Nessuna meraviglia\\
    e potenza della scienza,\\
    sia fisica, microscopica o cosmisca,\\
    sia psicologica, fisiologica o biologica,\\
    sia rivolta all'individuo, \\
    o rivolta alla societa',\\
    eguaglia la bellezza e\\
    onnipotenza\\
    della santita',\\
    del consumare la vita,\\
    per la vita.\\
    \leavevmode\\
    La santita',\\
    non la vanita'.\\
    Dio vuole la vita,\\
    non la morte.\\
    Non peccare,\\
    non per reprimere,\\
    ma per amare.\\
    Non sporcarsi,\\
    non per mostrarsi,\\
    ma per vedere nitidamente,\\
    per vestirsi con\\
    quanto di poco\\
    e prezioso solo\\
    si possiede,\\
    per gioire profondamente.\\
    Limitarsi, \\
    non per essere schiavi
    di regole,\\
    ma per accogliere \\
    i confini dell'Altro.\\
\end{haiku}


\centerpoemon{e che hai suoi occhi ogni fuocherello di ogni cellula,}
\begin{haiku}
    Siamo un atomo, tra gli atomi,\\
    mossi dalle maree e dai venti \\
    della Natura e della Societa'.\\
    A stento riusciamo a muovere noi stessi,\\
    e tutto ci si oppone.\\
    \leavevmode\\
    Ma come legna, bruciamo!\\
    Che sia poco il nostro peso,\\
    o difettoso ed umido il nostro coraggio,\\
    che ci siano inteperie e pioggie\\
    che attutiscano il nostro fuoco,\\
    bruciamo.\\
    \leavevmode\\
    Bruciamo, per il nostro Signore,\\
    che ha stabilito sacre le nostre cellule,\\
    e che ai suoi occhi ogni fuocherello di ognuna di esse,\\
    illumina il profondo spazio,\\
    piu' intensamente di tutte le galassie.\\
\end{haiku}

\centerpoemon{Io, per l'Altro, posso anche essere rimpicciolito}
\begin{haiku}
    L'Altro e' altro da me,\\
    al pari di me.\\
    Io, per l'Altro,\\
    sono cio' che sono per lui,\\
    che io sia io,\\
    che io sia niente.\\
    Perche' il Signore e' grande,\\
    e' buono,\\
    ed ama senza aver bisogno di essere amato,\\
    tutti.\\
\end{haiku}

\centerpoemon{L'altro e' quella infinita parte di me}
\begin{haiku}
	L'altro e' quella infinita parte di me\\
	che non posso esperire con i sensi,\\
	ma con l'amore;\\
	che non vedo, non sento, non penso,\\
	ma credo, colgo, ascolto.\\
\end{haiku}

\centerpoemon{della complessita' e della Natura}
\begin{haiku}
    Dio muore\\
    quando forti della scienza,\\
    lo riduciamo a noi stessi,\\
    e miseri di fronte all'infinito\\
    della complessita', della Natura\\
    e del male generato dall'uomo,\\
    ci duoliamo della nostra sorte.\\
    Se solo vedessimo la Sua via,\\
    non ci fermeremmo onnipotenti\\
    sulla materia morta,\\
    ma cammineremmo piccoli,\\
    deboli e feriti dai nostri\\
    peccati,\\
    verso il Suo regno di vera pace.\\
\end{haiku}

\centerpoemon{dell'Anima}
\begin{haiku}
    Chiesa\\
    palestra\\
    dell'Anima.\\
\end{haiku}

\centerpoemon{La volonta' esclusivamente propria}
\begin{haiku}
    L'ignoranza\\
    priva dell'infinita energia\\
    dell'Universo.\\
    \leavevmode\\
    La durezza del cuore\\
    dell'infinito amore\\
    di Dio.\\
\end{haiku}

\centerpoemon{supereremo anche i casi piu' difficili,}
\begin{haiku}
Dio\\
sacra persona\\
di infinita bonta'.\\
\leavevmode\\
Tu,\\
non piu' ideale del sognatore\\
ne' promessa del profeta,\\
esisti se io ti amo,\\
amando la vita\\
tramite solo la carita'.\\
\leavevmode\\
E la tua esistenza\\
e' fonte di Pace,\\
di momenti di vera gioia,\\
piu' preziosi\\
del metallo piu' nobile,\\
piu' meravigliosi\\
dello splendore del cosmo.\\
\leavevmode\\
Gesu',\\
dono Tuo esemplare,\\
dimostra che Tu esisti\\
sempre,\\
e che amandoti,\\
supereremo anche i casi piu' difficili,\\
ed anche nelle condizioni piu' dure\\
avremo coraggio.\\
\end{haiku}

% ancora non ho raggiunto l'illuminazione, qui l'avevo intravista. Non sono ancora degno di pronunciare queste parole.
%\begin{haiku}
%    Ah Dio,\\
%    sei cosi' vicino a me,\\
%    cosi' talmente vicino,\\
%    che per cosi' tanti anni\\
%    ti ho passato, guardando avanti,\\
%    cercato nei posti piu' remoti\\
%    e arcani\\
%    e non ti vedevo.\\
%    \leavevmode\\
%    E non era per segreto\\
%    che gli antichi \\
%    non ti enunciavano a gran voce,\\
%    in parole piu' semplici\\
%    piu' dirette.\\
%    \leavevmode\\
%    Era il mio ego,\\
%    il mio orgoglio,\\
%    la mia vanita',\\
%    le mie passioni per gli argenti,\\
%    i miei inquieti amori\\
%    per le tue creature,\\
%    che oscuravano\\
%    il Tuo amore.\\
%\end{haiku}

\centerpoemon{tu mi ricompensi per mille volte,}
\begin{haiku}
    Nel grazie,\\
    pur nelle avversita',\\
    ti trovo,\\
    Padre.\\
\end{haiku}

\centerpoemon{la fine della mia piccola identita' presente,}
\begin{haiku}
    Dio mio,\\
    non sei forse sempre con me?\\
    Ti trovo negli altri,\\
    nella loro bonta', giustizia e verita',\\
    e quando in loro non ti vedo,\\
    e non ti vedo piu',\\
    ti cerco.\\
    Rinnego me stesso,\\
    faccio del mio cuore cenere,\\
    fertilizzante per i Fiori\\
    amati del tuo regno.\\
    Faccio la tua volonta',\\
    anche se non e' ancora mia.\\
    E non manchera' molto,\\
    d'incontrare la morte,\\
    la fine della mia piccola\\
    identita' presente,\\
    per far spazio alla Tua,\\
    per essere umilmente,\\
    Tua dimora.\\
    E tutto brilla,\\
    e ti rivedo in chi e' vicino a Te,\\
    e chi non ha luce,\\
    la rivede in Te.\\
\end{haiku}


\begin{vcentered}

    \begin{align*}
        &\lim_{\textrm{amore}\to\infty} \textrm{EssereUmano} = \textrm{Dio}
    \end{align*}

\centerpoemon{abbiamo i tuoi stessi sentimenti.}
    \begin{poem}
        Limite infinito \\
        dell'Io mio e altrui,\\
        sfiorabile amando \\
        e soltando amando,\\
        oltre la paura, la sofferenza,\\
        le difficolta', la propria vita.\\
        Limite del vivere\\
        senza alcuna declinazione\\
        di morte,\\
        pronunciata per noi stessi\\
        o per gli altri.\\
        Punto all'infinito,\\
        del sentiero della verita' e\\
        della luce.\\
        Infinita realizzazione,\\
        di ogni essere umano,\\
        che vive per la vita,\\
        di ogni essere umano,\\
        tramite la vita e solo\\
        la vita.\\
        \leavevmode\\
        Come un seme,\\
        dimori in noi.\\
        E tanto piu' noi\\
        viviamo, e consumiamo tutta,\\
        la nostra vita,\\
        nella tua direzione,\\
        tanto piu' i nostri sentimenti\\
        sono i Tuoi, e Tu\\
        diventi vivo e presente,\\
        cresci nella terra,\\
        come un albero in cui trovare\\
        frutti, e riparo.\\
    \end{poem}

\end{vcentered}

\centerpoemon{produciamo il calcolo di quanto girare una vite,}
\begin{haiku}
Siamo ingranaggi\\
di una macchina.\\
Ogni giorno,\\
nolenti o volenti,\\
diciamo di si\\
ad essa.\\
Il peso della fatica,\\
col metallo fragile\\
del nostro corpo\\
sopportiamo.\\
Per non essere,\\
scartati, gettati e\\
sostituiti,\\
giriamo,\\
ciechi giriamo\\
intorno al nostro asse.\\
Che velocita',\\
che determinazione,\\
che durezza,\\
``ottimi ingranaggi\\
siamo'',\\
``quanto scarsi quelli\\
che si son fermati'',\\
``quanto bella la rotella\\
di me piu' grande,\\
di piu' giri capace''.\\
Con indifferenza,\\
giriamo,\\
a volte, limando \\
le rondelle vicine,\\
nella nostra autocrata corsa.\\
Con insofferenza,\\
giriamo,\\
non tollerando, i limiti,\\
gli sbagli, la piccolezza\\
delle rondelle vicine,\\
nella nostra autocrata corsa.\\
Gli scontri,\\
metallo con metallo,\\
producono scintille,\\
e graffiano, \\
si incidono\\
nella nostra pelle,\\
nelle nostre ossa.\\
Ma tutto questo,\\
alla macchina non importa.\\
\leavevmode\\
Solo in Te\\
riposa l'anima mia,\\
in Te che la grandezza\\
delle mie piccole forze\\
riconosci,\\
proporzionate alla resistenza\\
dei miei materiali.\\
E non per finta lode,\\
mi incoraggi,\\
e degno di stima\\
mi ritieni:\\
le stelle, \\
le montagne e i mari,\\
grazie ad ogni loro atomo,\\
granello di terra, e goccia,\\
grazie ad ogni loro piccolo\\
bagliore, resistenza e spinta,\\
manifestano la Tua potenza,\\
e cosi' noi.\\
In Te,\\
trovo la pieta' paterna,\\
ogni mio piccolo scricchiolio\\
e affanno ascolti,\\
con dolcezza, tendi la mano,\\
ogni volta che,\\
resomi conto di \\
non essere Te,\\
mi smarrisco,\\
e chino\\
il capo cade.\\
\leavevmode\\
Nei momenti alti,\\
con Te, dico:\\
a tutta la Terra\\
contribuisco\\
e di beneficiarne da tutta,\\
sono in diritto.\\
E per quanto per Te,\\
in diritto molto maggiore\\
di quello dato dall'uomo,\\
io sono,\\
e indurendo il cuore,\\
mille volte potrei avere\\
cio' che ricevo,\\
perdono tutto,\\
perdono tutti.\\
Amando \\
il tuo unico figlio,\\
mite agnello,\\
re dei cieli,\\
amo tutti,\\
in tutto.\\
\leavevmode\\
E cosi',\\
per il vestitino nuovo\\
del bambino,\\
produciamo un bottone,\\
ricoprendoci di polvere di plastica\\
e sudore.\\
Per la famiglia stanca a cena,\\
produciamo quattro arance,\\
spostando 25Kg di casse\\
tutto il giorno.\\
Per il treno che il pendolare \\
ritrova ogni mattina,\\
produciamo il calcolo di quanto girare una vite,\\
risultato di mille altri calcoli,\\
un decenno di studi, \\
tra i libri ed esami,\\
un decenno di tirocini,\\
lontani da casa.\\
\leavevmode\\
\leavevmode\\
Ah, sapessero il bambino,\\
la famiglia, e il pendolare,\\
quanto da Te sono amati.\\
Sono il tuo tutto,\\
ogni loro sorriso\\
e' festa nei cieli,\\
ogni loro pianto\\
muove il Tuo spirito\\
ad avvolgere l'Universo,\\
a zittire ogni voce e rumore,\\
e nel silenzio del dolore,\\
far riaffiorare la tua voce.\\
Per loro,\\
il tuo stesso sangue sacrifichi, \\
da 2000 anni,\\
con noi,\\
gradualmente nel tempo,\\
fino all'anzianita'.\\
Per Te,\\
bambino,\\
per Voi,\\
famiglia,\\
per Te,\\
uomo o donna,\\
per Te,\\
di cui nessuno\\
si ricorda piu'\\
di essere tuo amico,\\
questa mattina\\
mi vesto,\\
chiudo nella cassaforte\\
i miei sogni,\\
il sole, e il cielo azzuro,\\
e con Lui,\\
ingranaggio,\\
muovo il cielo.\\
\end{haiku}

\centerpoemon{tra settantamila miliardi di miliardi}
\begin{haiku}
    L'Universo un punto\\
    nello spazio di fase.\\
    Possibilita',\\
    tra infinite possibilita',\\
    che noi vivendo,\\
    chiamiamo realta'.\\
    E noi, parte dell'universo,\\
    siamo una possibilita'\\
    che vive una possibilita'.\\
    E' tutto cosi' \\
    una breve fluttuazione \\
    del nulla.\\
    Ma Tu,\\
    o mio Dio,\\
    ti sei fatto atomo,\\
    disprezzato dagli uomini\\
    che amavano il nulla,\\
    amato dagli uomini\\
    poveri di spirito.\\
    Divenuto re della storia\\
    e dei cieli,\\
    hai proclamato la vita \\
    come tua grande e sacra opera.\\
    Cosi' l'Universo lo hai reso\\
    servo della vita,\\
    il tempo e lo spazio\\
    teatro della sua storia epica.\\
    Il sorriso dei bambini,\\
    l'affetto e la cura dei grandi,\\
    l'amore tra i fratelli e le sorelle,\\
    la meraviglia piu' grande\\
    tra settantamila miliardi di miliardi\\
    di stelle.\\
\end{haiku}

\begin{vcentered}
Dio e' quell'unico essere $D$ tale che
    \begin{align*}
        &\forall a \in \textrm{Anima}\;\;\forall b \in \textrm{Desiderio}(a)\;\;\\
        &\quad\quad b\textrm{ e' realizzato o sublimato da }D
    \end{align*}
    Vedi \ref{FormulazionePuntualeDiDio} pag. \pageref{FormulazionePuntualeDiDio}
\end{vcentered}

\begin{vcentered}
    Se un uomo fosse contento della sua vita, qualunque essa sia, a rigor di logica (egoistica), potrebbe alzarsi e dire: ``sono il Signore, ogni atomo dell'universo mi appartiene''. 

    Chi vive secondo la carita', invece, contento della sua vita, dice: ``Gesu' e' il Signore, ed e' risorto in me, per quanto con i miei peccati deturpi la sua immagine''. E' il Signore, perche' lui ha fatto degli ultimi i primi, e li ha posti accanto a lui nel regno dei cieli. E' il Signore perche' senza macchia, ha guarito ed insegnato a molti, ed ha preferito la morte della sua persona per donare il suo Spirito a tutti coloro che vivono secondo carita'. Tutti coloro che vivono secondo la carita', diventando santi, in coro, all'unisono, sono un Essere solo, e coloro che amano Gesu' riconoscono l'Essere Santo proprio come Gesu'. Cosi', Gesu' e' il Signore, risorto.
\end{vcentered}


\centerpoemon{Reale non proiezione, ma manifestazione,}
\begin{haiku}
Padre buono,\\
saggio e forte,\\
se dovessi vedere tutto \\
come mera materia,\\
allora, ancora, \\
    scientificamente eccoti:\\
    Gesu'.\\
    \leavevmode\\
Reale,\\
    non proiezione, \\
    ma manifestazione \\
dell'infinita anima \\
che abita l'uomo,\\
e del bruciante, immutabile, \\
ed eternamente fedele spirito\\
che e' in potenza nell'uomo\\
e che oltre i limiti \\
    posti dall'uomo stesso,\\
ama,\\
oltre la sua stessa morte,\\
abbandonato alla sua passione,\\
dona.\\
    \leavevmode\\
Gesu',\\
figlio dell'uomo,\\
e' Te, Padre,\\
che sei nell'uomo,\\
e che l'uomo conosce solo\\
tramite proiezione della sua parte santa,\\
solo nei momenti e nei luoghi santi,\\
solo in compagnia santa.\\
Beati i Santi,\\
che ti conoscono nell'interezza,\\
in ogni momento,\\
in ogni luogo,\\
in ogni condizione,\\
fino alla fine.\\
I Santi che si sono uniti a Gesu'\\
e amano ancora,\\
parimenti a Lui,\\
nell'umilta' della loro piccolezza,\\
per la grandezza della vita.\\
Non e' questa infinita potenza?\\
Manifestazione delle altezze dell'anima,\\
non ideali e parole vuote,\\
ma spirito che muove sentimenti, \\
pensieri e azioni,\\
fino all'ultimo pagate\\
e brucianti d'amore.\\
Non mi spaventa \\
    questa materiale Tua visione.\\
Senza sovvertire \\
    un solo atomo della materia,\\
ma in armonia con la Natura,\\
    puoi dare gioia \\
    e, quando non gioia, pace\\
ad un mondo intero, \\
    ad ogni anima\\
in qualsiasi condizione si trovi.\\
\leavevmode\\
\leavevmode\\
\leavevmode\\
Per questo ti amo,\\
per questo sei veramente \\
    onnipotente.\\
\end{haiku}

\centerpoemon{piuttosto che}
\begin{haiku}
    Essere,\\
    piuttosto che\\
    essere.\\
    Cosi',\\
    un Dio.\\
    \leavevmode\\
    Amare,\\
    piuttosto che\\
    essere.\\
    Cosi',\\
    il Padre.\\

\end{haiku}

\centerpoemon{con la sua abbondanza di misericordia,}
\begin{haiku}
    Tu\\
    sei chi e' buono, bello,\\
    grande e doveroso essere.\\
    \leavevmode\\
    Come un padre,\\
    tu sei chi sono,\\
    ma ancora non ho il coraggio,\\
    ne' la grazia, ne' la purezza,\\
    di essere.\\
    \leavevmode\\
    Tu sei l'Io, l'Altro, il Noi,\\
    che e' la via,\\
    la verità,\\ 
    la vita.\\
    \leavevmode\\
    In embrione in noi, \\
    figli tuoi.\\
    Manifesto nell'interezza,\\
    nel Tuo primogenito.\\
    \leavevmode\\
    Rimani con me,\\
    per tua abbondanza \\
    di misericordia,\\
    fino a quando,\\
    io raggiunga il Tuo regno,\\
    la mia identita',\\
    figlio Tuo.\\
\end{haiku}


\centerpoemon{con la sua stessa vita,}
\begin{haiku}
\emph{Festival dei Giovani di Medjugorje 2024}\\
\leavevmode\\
Che tutti siano\\
un solo essere.\\
\leavevmode\\
Ognuno in Lui,\\
Lui un giorno,\\
pienamente\\
in ognuno.\\
\leavevmode\\
Lui che ama,\\
ed ama pur chi sbaglia,\\
e perdona,\\
con la sua stessa vita,\\
anche chi tradisce.\\
\leavevmode\\
Gesù.\\
\end{haiku}

\centerpoemon{tra le onde che colpiscono violente}
\begin{haiku}
% Cristo,\\
% tu sei l’unico vero Io,\\
% presente,\\
%     almeno in potenza,\\
%     in ciascuno di noi.\\
% Io d'amore,\\
%     centrato sull’Anima,\\
% unione delle anime\\
% dell’umanita’,\\
% sui suoi bisogni e\\
% desideri necessari.\\
% \leavevmode\\
% Reale, ancora oggi,\\
% quando in due o tre,\\
% riuniti nel tuo nome,\\
% gioiamo,\\
% quando soli,\\
% abbracciati dal tuo amore,\\
% riposiamo.\\
% \leavevmode\\
Mio Dio,\\
azzura limpida infinita acqua,\\
quando ne' gioiamo,\\
ne' riposiamo,\\
dove sei in queste terre?\\
Dove durante la tempesta,\\
tra le onde che colpiscono violente\\
le case e i raccolti?\\
Acqua tranquilla,\\
ricordo, speranza dell'Anima,\\
pace desiderata.\\
\leavevmode\\
Ma eccoTi,\\
ti aspettavo ora da tanto,\\
ora da poco.\\
Nella barchetta,\\
solo, abbandonato, tradito,\\
nel mare in tempesta,\\
vieni verso noi,\\
dicendo ad ognuno:\\
non temere,\\
sono con te,\\
figlio mio.\\
La Sua potenza e' grande,\\
l'hai usata male,\\
e tutto e' agitato.\\
Ora, rialzati,\\
lascia tutto,\\
seguimi.\\
\leavevmode\\
Ed eccoti ancora,\\
bianca colomba,\\
che sorvoli Lui,\\
infinito e maestoso,\\
coronato dall'arcobaleno\\
della vita.\\
\end{haiku}

\centerpoemon{risplende maggiormente il volto suo}
\begin{haiku}
Gesu' un corpo \\
dell'Universo,\\
ma non l'Universo.\\
Gesu' un'anima\\
dell'umanita',\\
ma non l'umanita'.\\
Gesu' figlio del Padre,\\
ma non il Padre.\\
Ma lo Spirito del Padre,\\
in lui,\\
e' in armonia \\
con l'Universo tutto.\\
Ogni forza gli appartiene,\\
e ne fa scienza e arte\\
per la vita.\\
La sua anima,\\
dice si allo Spirito,\\
e cosi' accoglie ogni altra,\\
nei dolori e nelle gioie.\\
Gesu' e' nel Tutto \\
e il Tutto e' in Lui.\\
Cio' che vuole il Padre\\
e' cio' che Lui vuole,\\
cio' che Lui guarda,\\
e' cio' che il Padre ama.\\
Cosi', lui stesso \\
e' Dio,\\
uno con il Padre.\\
Noi uomini come lui,\\
ma non lui.\\
Ma lui si e' donato a noi,\\
il Suo spirito ha risuonato\\
nei suoi e in chi ha creduto.\\
E ogni volta che risolviamo \\
i conflitti interiori nostri,\\
nella pace e solo con la pace,\\
risplende maggiormente il volto suo\\
nel nostro,\\
ci facciamo uni con Lui,\\
e con il Padre.\\
In terra vive Dio,\\
con noi, in noi, per noi.\\
\end{haiku}

\centerpoemon{trasformi l'h2o}
\begin{haiku}
Solo Tu\\
trasformi l'h2o\\
in acqua.\\
\end{haiku}

\centerpoemon{per amore di Colui che Ama}
\begin{haiku}
    Dire grazie\\
all'impossibilità,\\
alla sofferenza,\\
alla malattia,\\
    alla morte,\\
per amore \\
di Colui che Ama.\\
\end{haiku}

\centerpoemon{cha ha bisogno}
\begin{haiku}
    Dio e' colui\\
che ama, pur non\\
essendo amato.\\
\leavevmode\\
L'uomo e' colui\\
cha ha bisogno\\
del Tuo amore,\\
per amare.\\
\end{haiku}

\centerpoemon{Dio che e' lo Spirito di Gesu',}
\begin{haiku}
Nell'Universo materia morta\\
ed esseri,\\
il resto illusioni ed idoli.\\
\leavevmode\\
Solo l'Io, l'Altro, il Noi,\\
che e' pace e gioia,\\
solo nella santita',\\
solo nel vero bene,\\
e' Dio in terra.\\
\leavevmode\\
Dio Spirito, che Gesu',\\
lascio' in eredita'\\
a noi, all'altro, a me.\\
\end{haiku}


\centerpoemon{dell'Io, dell'Altro, del Noi,}
\begin{vcentered}
\begin{poem}
Tu Padre,\\
centro dell'esistenza,\\
Tu perfetta, infinita, \\
    realizzazione\\
dell'Io, dell'Altro, del Noi,\\
Tu pace e gioia\\
dell'anima.\\
\end{poem}
\leavevmode\\
\leavevmode\\
L'uomo e' come una lente d'ingrandimento che convoglia pensieri, sentimenti, speranze verso un'unico centro. Questo centro e' il Dio che adoriamo intimamente nei nostri cuori. Che noi diciamo di credere a un Dio o meno, non importa, infatti, crediamo sempre a qualcosa nella vita, anzi, crediamo sempre a Qualcuno. Abbiamo sempre un'immagine di chi vorremmo e dovremmo essere, di chi al momento non riusciamo ad essere e di chi in parte riusciamo ad essere. Gesu' ha creduto al Dio che ama, e che ama tutti, anche chi non avvantaggia di alcunche' la propria vita, anche chi e' di piccolo o grande ostacolo per la propria vita, anche chi pone a rischio la propria vita. \\
Dio e' il centro dell'esistenza. Un Io e' alla costante ricerca del suo centro. E' quella posizione esistenziale, istantanea ed eterna, dove l'Io dice: ``ecco, cosi', solo cosi' sono''. \\
Tale centro non e' statico, puo' variare nel corso della vita, e l'Io deve adattarsi, come farebbe un pianeta con il suo Sole.\\
Di norma l'Io raggiunge solo approssimazioni di tale centro, a causa di errori dovuti alla propria immaturita' o a traumi, piccoli o grandi, della propria psiche, o di forze centrifughe come possono essere varie sfide nella propria vita. Tuttavia, il centro, l'ottimo dell'esistenza dell'Io, esiste. \\
\leavevmode\\
Un'immagine duale alla precedente, dove non e' l'uomo che agisce (esterno-interno), ma e' Dio che agisce (interno-esterno), e' vedere l'anima come il guscio di una sfera. Il centro della sfera e' il Padre. L'Io umano, non santo, e' un punto interno della sfera, che approssima il centro, ma non e' il centro. L'Io umano santo, e' il centro, e' tutto Spirito Santo, e coincide con il Padre. L'Io illumina, riscalda, alimenta l'anima, e tanto piu' coincide col centro, tanto meglio riesce a farlo.\\
\end{vcentered}

\centerpoemon{Colui che e' il bene,}
\begin{haiku}
    Colui che
    ama,\\
    solo ama.\\
    \leavevmode\\
    Colui che e' il bene,\\
    solo nel bene,\\
    per il bene.\\
\end{haiku}

\centerpoemon{ha dato per sempre risposta.}
\begin{haiku}
    Colui che l'anima\\
    chiama, desidera, invoca,\\
    spera, prega.\\
    Colui che,\\
    in parole e in carne,\\
    per sempre ha dato risposta.\\
    Colui che,\\
    nella nostra piccolezza,\\
    e' grande risposta.\\
\end{haiku}

\centerpoemon{E l'Universo fa vibrare}
\begin{haiku}
    Nella Tua lode,\\
    l'arpa mia\\
    accordi.\\
    Nella piccolezza\\
    per chi prego\\
    suona.\\
    L'Universo fa vibrare\\
    e nuova antica vita\\
    canta.\\
\end{haiku}

\centerpoemon{Che tormento allontanarmi}
\begin{haiku}
    Seppur fragile,\\
    limitato,\\
    umano,\\
    la mia anima,\\
    unita a Te\\
    con candida veste\footnote{la santita'},\\
    e' magnifica,\\
    e infinito e' lo Spirito\\
    che l'alimenta.\\
    Io sono, \\
    ai Tuoi occhi,\\
    una meraviglia stupenda,\\
    e ti rendo grazie.\\
    E che peccato,\\
    sprecar tutto questo,\\
    d'una sola macchia\\
    violarlo.\\
    Che tormento allontanarmi\\
    da Te,\\
    non vederti piu' \\
    un solo istante.\\
\end{haiku}

\centerpoemon{Chi, in parte, nel bene}
\begin{haiku}
    Chi non siamo,\\
    ma amiamo essere\\
    nel bene.\\
    \leavevmode\\
    Chi nel bene\\
    e' veramente stato\\
    ed amiamo.\\
    \leavevmode\\
    Chi, in parte, nel bene\\
    siamo,\\
    e un domani in tutto\\
    saremo.\\
\end{haiku}

\centerpoemon{dalla dura roccia a cinta del cuore}
\begin{haiku}
\leavevmode\\
Dice il Signore: \\\enquote{
Caro figlio, cara figlia,\\
sebbene le forze tue \\
e dell'uomo sono grandi \\
e ammirabili, \\
per bene stare in questo mondo \\
non bastano.\\
Solo l'acqua della mia carita', \\
versata in abbondanza dai cieli\\
sulla terra, dal mio costato,\\
brilla tra le tenebre, \\
ha sapore e nutre.\\
Chi non mi conosce,\\
dalla dura roccia \\
a cinta del cuore\\
trova rugiada, ma\\
al mattino scompare.\\
Nelle cose, nelle imprese,\\
nei favori, nelle lusinghe,\\
nel potere, nel piacere, \\
invano cerca ancora.\\
Chi mi cerca, teme ed ama,\\
con lanterna accesa\\
dal mio Spirito,\\
visita le oscure\\
fredde, tenebrose, \\
sue profondita', \\
nascondigli  di peccati \\
passati  e futuri,\\
gioie proprie e altrui,\\
che furono perse e\\
mai piu' saranno.\\
Le visita, e scruta,\\
scava, filtra e lava,\\
di volonta', pazienza e carita'\\
vestendosi,\\
di psicologia, biologia,\\
fisica e matematica\\
armandosi,\\
non per dominarle, \\
perche' niente di contrario\\
a me, ho posto nella terra,\\
ma per ascoltarle, servirle,\\
e consolarle.\\
A tempo debito,\\
gli inferi di chi mi cerca,\\
per la prima volta \\
il mio volto vedranno.\\
Con liberta', con desiderio,\\
e con coraggio si convertiranno,\\
mi ascolteranno, mi serviranno\\
e mi consoleranno.\\
Cosi', chi mi cerca,\\
dalle ceneri del suo cuore,\\
le terre, rese fertili,\\
in un giardino rigoglioso,\\
eterno, germoglieranno.\\
Figlio, figlia, \\
non seguire l'esempio del mondo,\\
che per paura di vedere \\
la propria miseria, \\
debolezza e infermita', \\
per non rischiare di cadere, \\
e faticare nel rialzarsi,\\
cerca vane vie!\\
Si fa onnipotente \\
nel dominare la materia morta,\\
si fa libero nel perdersi,\\
dimenticando se stesso\\
e gli altri.\\
Si proclama giusto e intelligente,\\
ma il suo cuore e' duro\\
come la pietra,\\
e la sua parola stride \\
come il ferro sul ferro.\\
Guardami figlio, guardami figlia, \\
guarda la vera potenza e gloria,\\
manifesta in ogni tuo respiro.\\
Tutto l'Universo mi serve,\\
splende e si muove,\\
per quel tuo respiro.\\
Da quando nel grembo \\
di tua madre eri,\\
tra le mie braccia\\
ti ho custodito.\\
Sei la pupilla dei miei occhi.\\
Desidero la tua liberazione\\
dai lacci che ti fanno soffrire.\\
Ama, fa cio che vuoi ed ama!\\
Col mio amore,\\
che non giudica,\\
ne scusa superficialmente,\\
il cuore comprenderai,\\
e non rimarra' risposta occulta\\
ai tuoi occhi.\\
L'anima nutrirai e consolerai,\\
e gioia le daro' in rincompensa\\
della sua fedelta'.\\
I pesci e i pani,\\
che strariperanno dalla cesta,\\
condividerai a tutti.\\
Ricambieranno in pochi,\\
con poco pane, ma di quei pezzi\\
avanzati, \\
voi soli mai piu' sarete,\\
e insieme, \\
in canti festosi,\\
in sereni dolori,\\
in abbracci,\\
vivrete.\\
}
\end{haiku}



\centerpoemon{Nihil ex nihilo}
\begin{haiku}
	Nihil ex nihilo\\
	gaudium ex Deo\\
\end{haiku}

\centerpoemon{il cuore dell'Universo,}
\begin{haiku}
	Tu sei il cuore \\
    dell'Universo,\\
	l'anima \\
    dell'umanita'.\\
\end{haiku}
