\centerpoemon{E non posso essere quell'$A$ che e' tutto,}
\begin{haiku}
    O corpi cibernetici,\\
    o corpi dal dna programmato,\\
    con cellule di nanomateriali,\\
    alimentate da fusioni nucleari,\\
    non potro' sfuggirTi,\\
    la morte esistera' sempre.\\
    L'insoddisfazione della mia anima\\
    che reclama l'infinito,\\
    dice Cantor,\\
    mai sara' appagata.\\
    Un'infinito piu' grande,\\
    sara' sempre successore.\\
    In fondo,\\
    se esisto, e sono $A$,\\
    non sono $\lnot A$. \hspace{2cm} {\footnotesize nota}\footnote{$\lnot A$ si legge ``not A'', ovvero, ``non A''}.\\
    E non posso essere quell'$A$ che e' tutto, \\
    compreso cio' che non e'.\\
    Ah, che liberazione,\\
    e' giusto non essere Te,\\
    finito, fragile, mortale.\\
    Cosi',\\
    sono.\\
\end{haiku}

\centerpoemon{}
\begin{haiku}
    Tu,\\
    l'origine dello spazio \\
    e del tempo,\\
    dell'universo\\
    dove regna amore.\\
\end{haiku}
