\chapter{Prima dell'inizio del tempo}
Cio' che esiste ha sempre un fine, o a favore degli uomini e delle donne, o contro essi, oppure a volte a favore ed altre volte a sfavore. Ad esempio, la montagna esiste ed e' grande. E' difficile da scalare, ma raggiunte le alture, nasconde in se ruscelli e sorgenti d'acqua pura. E allora la sua vista e' sia sfida che ammirazione. Ora, dal punto di vista strettamente fisico-matematico, nulla esiste, perche' nulla ha un fine, ma tutto e' solo un susseguirsi meccanico di trasformazioni di materia morta.
Ad esempio, la terra che compone la montagna non ha un cuore per curarsi di quanto sia difficile scalarla, ne' di quanto sia gioioso trovare in se' acqua che sgorga dagli anfratti puliti delle sue rocce. Infatti, a rigor di scienza non ha fibre nervose per concepire alcun che'. Allora, al di la' delle nostre sfide e delle nostre gioie, la materia e' niente, e' un niente che sotto alcune condizioni e' a noi utile e sotto altre e' a noi d'ostacolo. Ma finche' ci sara' vita, e finche' la vita avra' speranza nella vita, sommando tutti i pro e i contro, ci sara' almeno un pro, seppur piccolo, che fara' vedere la terra ammucchiata in gran quantita' come una bella montagna, che reca nei suoi segreti doni preziosi.

Cosi', avendo fede nella vita che affronta paziente e con coraggio gli ostacoli d'innanzi a se, non tramite il potere e il dominio su la natura, su le cose, su gli uomini e le donne, ma solo tramite il suo amore per tutta la vita, posso parlare della generazione dell'universo, come facevano gli antichi, che come noi avevano affrontato secoli di pericoli, e che tenendo cara la vita di ognuno, avevano intuito i segreti delle nostre anime, le nostre debolezze, le nostre virtu'. Quindi, sono autorizzato ad andare oltre un discorso prettamente scientifico, e ad estendere la visione dell'Universo includendo cio' che nutre l'anima, oltre la mente.

\enquote{Nell'eternita', prima dell'inizio del tempo, vi era un'unica Anima, infinita e bella, quella di Colui che Unicamente e Sempre Ama. Non possedeva ne' un corpo, ne' memoria, ne' alcun che' di cio' che ci piace oggi, ma della sua stessa pace era appagata, e tale era il suo appagamento che non provava interesse per spazi, tempi, forme e colori. Anche se, di queste cose se ne intendeva, e si compiaceva nel crearle, vederle, studiarle e poi riporle nel libro della Conoscenza. Nel far  cio', non impiegava tutto se stessa, gli bastava impiegare una piccola ed infinita parte di se', che chiamiamo Natura. Oltre a Natura, vi erano altre Sue parti, che chiamiamo Angeli, che, seppure una miriade in numero, erano tutte in armonia ed unite, cosi' che Lui era sempre Uno.

Cosi', nell'eternita', prima dell'inizio del tempo, Natura, bambina, giocava. Era intelligente, e di matematica ragionava. Con l'abilita' di un'artista, sorprendeva tutti creando piccole piante, facendo zampillare l'acqua dal terreno, sfornando pietre e cristalli di vari colori e forme, alcune semplicemente belli alla vista, altri con sorprendenti proprieta' di resistenza o di essenza.
Gli Angeli si meravigliavano del suo talento, si divertivano a giocare con lei con tutte le cose curiose e interessanti che portava loro, ed amavano coccolarla e riempirla di attenzioni.

Quando divenne grande, rivelo' a tutti la sua opera piu' bella a cui lavorava da tempo. Parlo' di una sfera luminosa di vari colori che sprigionava un forte calore. Parlo' di particelle che cariche di energia andavano in tutte le direzioni, sbattevano tra loro, emettevano luce, generavano altre particelle. Poi parlo' di stelle e di galassie.
Tutti le chiesero cosa fosse.
Lei rispose: \enquote{Questa sfera e' l'Universo. In esso e' racchiusa luce e materia. Esse si trasformano perpetuamente, conservando l'energia, cosi' anche se tutto al suo interno muta, in realta' nulla cambia. Questo e' l'Universo e lo progetto secondo l'amore dell'Altissimo.}
Gli Angeli rimasero stupiti guardando quella sfera che emanava una luce intensa di molti colori, anche se non capivano bene quello che diceva la Natura.
Poi, un angelo bambino chiese: \enquote{Cosa vuol dire Amore? Che cosa vuole il Signore fare?}
La Natura chiamo' Sapienza, e lei rispose: \enquote{Lui, e noi in Lui, non abbiamo bisogno ne' di muoverci ne' di muovere per ottenere cio' che ci serve, perche' gia' abbiamo ogni cosa. Quindi, lo spazio non lo concepiamo. Non dobbiamo aspettare per diventare diversi o per incontrare qualcuno che ancora non conosciamo, perche' Lui e' colui che e', da sempre e per sempre, e non esiste tempo. Tuttavia, ancora non conosciamo la piena gioia a cui lui ambisce. Lui creera' la meraviglia piu' grande di tutte e di essa gioira', e noi tutti gioieremo, nell'eternita'.}
Poi continuo' Natura: \enquote{Per questo suo progetto, mi rivolsi al Nulla, e pregai il Signore per lo spazio ed il tempo, ed il Nulla divenne infinito e vasto. Poi riempii il Nulla di altissima energia, ma neutra, ne' cattiva ne' buona, e secondo Sapienza, stabilii di aspettare moltissimo tempo, per ricamare e tessere con finezza la materia, e cosi' accumulare e produrrre nelle Stelle gli elementi che sarebbero serviti per la Meraviglia.}.
  Gli Angeli, ascoltato il discorso di Natura, sentirono anche loro il desiderio di creare la Meraviglia. Compresero che sarebbe stato qualcosa di veramente grande e speciale.
  L'angelo bambino chiese: \enquote{Voglio anch'io amare secondo nostro Signore}. Un angelo bambino disse: \enquote{Voglio aiutare e vedere il momento della nascita della Meraviglia}. E mille altri angeli ancora, esclamarano la loro eccitazione.
  Sapienza allora proferi': \enquote{La Meraviglia e' la Vita.}
\enquote{La Vita sara' simile a Lui tutto, padroneggiera' le forze dell'Universo per prosperare, dalla materia fredda si ergera' e la rendera' simile a quella calda delle stelle. Ma piu' di tutti i doni, avra' il Libero Arbitrio, avra' volonta' propria, e potra' scegliere di seguire le Sue parole o perfino di ribellarsi a Lui, perche' Lui vuole dare tutto di se stesso a Lei, anche la Sua infinita liberta', la Sua incondizionata volonta'.}
  Sapienza aggiunse:
\enquote{Una miriade\footnote{Dalla nascita dell'uomo, ad oggi, 117 miliardi di esseri umani hanno abitato la Terra. \url{https://www.prb.org/articles/how-many-people-have-ever-lived-on-earth/}} di voi angeli si incarnera' nell'Universo. 
Il Signore formera' il vostro corpo nel grembo di vostra madre. Fara' di voi una meraviglia stupenda\footnote{Frase presa dal Salmo 139 della Bibbia \url{https://www.bibbiaedu.it/CEI2008/at/Sal/139/}}.
Avrete con voi la potenza dello Spirito, e li' dove ci saranno pericoli e difficolta', se con coraggio Lo ascolterete, li supererete nella Sua gloria, e li' dove il male regnera', se con la preghiera ascolterete la Sua voce, sarete dei semi che da cio' che marcisce faranno sorgere fiori e campi. 
Nella pace, insieme gioirete delle forme, dei colori, dei suoni, e vi potrete liberamente muovere e danzare nello spazio.  
Cosi', scriverete nella materia, per l'eternita', la Sua storia. Ogni passo fatto da voi per amarlo, sara' a Lui gradito, lodato e favorito.}

Diversi angeli, non avevano pienamente compreso il discorso di Natura e Sapienza, e rimasero a guardare l'universo per qualche tempo. 
Vedevano in esso lo spazio: la possibilita' di scegliere di stare nel punto che piu' si gradisce. Vedevano la luce: la possibilita' di vedere le forme che piu' piacciono. Vedevano la materia, che permetteva di mantenere una forma stabile nel tempo, in un punto dello spazio.
  Ancora le condizioni iniziali non erano scelte, per questo l'universo non era come lo conosciamo, era instabile. Ma molti Angeli erano incantati, vedevano luci e forme che si creavano e si distruggevano. Danze di materia ed energia che dipingevano sulla tela dello spazio. Poi, capirono che incarnandosi in un Universo stabile, avrebbero anche loro avuto una forma. E questo fra tutte le meraviglie li stupi'. Avrebbero potuto guardarsi, ascoltarsi e parlare, cantare e correre.
Cosi', cominciarano a pensare solo a come sarebbe stato bello tutto questo e solo a questo. Crebbe in loro un desiderio grandissimo ed indistruttibile di avere un corpo e vivere.
Poi, Natura, tra tutti i possibili universi concepi' quello dove era favorevole la nascita della vita. Cosi' scelse le condizioni iniziali, le costanti che avrebbero determinato la storia dell'Universo. E il tempo ebbe inizio.

All'inizio del tempo, lo spazio nacque e crebbe vertiginosamente ad ogni istante. Assieme allo spazio, un incommensurabile fuoco scaturi' dal centro dell'universo. Il fuoco era composto da un mare di particelle, che cariche di energia, andavano in tutte le direzioni, sbattevano tra loro, emettevano fotoni, generavano altre particelle.
Gradualmente, sotto l'azione della gravita', si formarono le nebulose, nuvole di particelle variopinte che si estendevano per milioni di miliardi di chilometri.
Nelle nebulose, le stelle si formavano e brillavano, roteavano su se stesse, e quest'ultime, in gruppo, formavano le galassie.
  Cosi' le cose continuarono per un miliardo di anni, fino a che' il fuoco iniziale, ormai raffreddato, veniva mantenuto acceso dalla miriade di stelle rimaste. Per l'esattezza, mille miliardi di miliardi di stelle\footnote{\url{http://scienceline.ucsb.edu/getkey.php?key=3775}}. 
  Le stelle brillarono nel vuoto dello spazio per altri 9 miliardi di anni\footnote{Cerca su google ``age of life on earth'' e ``age of the universe''}. Poi in un punto dell'universo, in un pianeta blu, gli atomi e le molecole si erano mescolate e ricombinate molte volte, cosi' molte volte che la vita si sprigiono' come un'ulteriore esplosione. La vita, inizialmente mono-cellulare, era sorprendente, perche' mentre la materia era alla merce' delle forze della natura, la vita poteva fare delle piccole azioni meccaniche e chimiche che le permettevano di perpetuarsi nel tempo.
 4 miliardi di anni dopo\footnote{``Intelligent life emerged on a timescale similar to that of Earth's lifetime. It took 4 Ga for intelligent life to emerge, ...''
    \url{https://www.liebertpub.com/doi/full/10.1089/ast.2019.2149}  See also, Franck S, Bounama C, and Von Bloh W (2006) Causes and timing of future biosphere extinctions. Biogeosciences 3:85–92. Crossref, Google Scholar}, i primi esseri senzienti aprirono gli occhi. Inizialmente, gli animali erano delle strane creature, deboli, istintive e sentivano due sole emozioni: paura e piacere. Nel tempo pero', grazie all'abbondanza di energia e risorse disponibili nella natura, la vita accumulo' doni e richezze. Il loro corpo si modello' per meglio sopportare le fatiche e vincere gli ostacoli, sviluppo' nuovi organi di senso come quello dell'udito, e poi quello della vista (inizialmente senza colori). Tutto questo avvenne tramite l'ereditarieta' della specie e tramite il rinnovamento e il cambiamento dei geni nei figli.
Il termine scientifico e' "selezione naturale". Tuttavia, con questo termine sembra quasi che la Natura sceglie chi vive e chi muore. Vista in ottica diversa, e' semplicemente che i tratti buoni si tramandano ai figli e a volte i figli hanno tratti migliori, che li avvantaggeranno, e che tramanderanno ai loro figli. E' chiaro che gli esseri piu' avvantaggiati permaneranno e popoleranno di piu' la terra. Ma cio' non significa che i piu' popolosi saranno migliori di quelli che poi, pian piano, non avranno piu' discendenza. Infatti, e' un uomo vivo migliore di un uomo buono e che ha fatto sempre del bene e che e' morto ? Agli occhi di Dio, tutte le creature sono importanti, e cosi' anche agli occhi della Natura.

125 mila anni fa', nacque l'uomo come lo conosciamo\footnote{H. s. sapiens is thought to have evolved sometime between 160,000 and 90,000 years ago in Africa before migrating first to the Middle East and Europe and later to Asia, Australia, and the Americas. \url{https://www.britannica.com/topic/Homo-sapiens-sapiens}}. L'uomo, come i mammiferi, non provava solo paura e piacere, ma anche amore. A differenza pero' degli altri mammiferi, eccelleva in intelligenza. Cosi' che mentre tutti gli altri animali erano sempre cio' che erano, l'uomo poteva scegliere chi era. Le cose e gli animali sono sempre cio' che sono. Esse e loro appartengono armoniosamente  all'opera di Dio. Infatti, anche quando potrebbero sembrare fuori posto, non sono odiose, ad esempio, un masso pesante che ostacola il cammino da' fastidio ma non e' odiabile, poveretto lui era li' da centinaia d'anni. Un animale pericoloso e' temibile ma non e' odiabile, lui cerca di ricavarsi il suo territorio nell'aperta natura selvaggia, per difendersi da altri animali e per assicurarsi il cibo. Un uccello che non si avvicina per essere ammirato meglio non da' frustrazione, piccolino cerca di ripararsi da pericoli piu' grandi di lui.
  L'uomo, invece, ha la capacita' di scegliere chi e'. Ha la capacita' di dirigere la sua vita in una di molteplici direzioni. Alcune di queste sono fruttuose, danno pace e incarnano in lui il Suo spirito, lo rendono buono e forte. Altre rendono la sua vita piu' pesante e difficile di quella degli animali e lo rendono piu' brutto e crudele degli insetti.
  12 mila anni fa', dopo l'ultima era glaciale, nacque l'agricoltura. \footnote{\url{https://en.wikipedia.org/wiki/History\_of\_agriculture}} L'uomo divento' sedentario e costrui' case e villaggi. I villaggi divennero citta'. Nelle citta' pero' l'uomo sceglieva quasi sempre una direzione che lo allontanava da se stesso. E cio' provocava avidita', tristezza, infedelta'. E la vita in citta' era forse piu' pericolosa della vita nelle foreste.
  I capi, per mantenere l'ordine, imposero leggi dure e severe\footnote{Un esempio e' il codice di Hammurabi, \url{https://it.wikipedia.org/wiki/Codice\_di\_Hammurabi}}. Allora, i saggi studiarono e pregherano a lungo e per molte generazioni, per capire chi era veramente l'uomo e quali erano le migliori direzioni. 
  Molti saggi e profeti avevano parlato e insegnato, in tanti popoli, in molte culture. Uno di questi, saggio e santo, fu' Gesu'. Gesu' studio' e prego' fin da quando era bambino. Quando fu' adulto, divenne povero come i poveri che amava, divenne santo e perfetto ma buono con i peccatori, insegno' la legge di Dio anche quando gli uomini ingiusti che detenevano il potere avevano fastidio di lui. Diceva apertamente cio' che tutti volevano dire, ma non osavano dire. Lo poteva fare perche' lo diceva per amore e con amore, e non accusava personalmente neanche le persone piu' corrotte col potere. Predicava comportamenti e mete per indirizzare la propria vita in maniera semplice e fruttuosa, predicava un cammino di vita che il Padre nostro, l'Altissimo stesso, amava. Ma poiche' i frutti che predicava non erano mangiabili con la propria bocca, ne' facilmente visibili ad occhio, adoperava parabole. Infine, con i piu' deboli e poveri, dimenticati dal resto degli uomini, parlava in maniera diretta, e loro lo capivano subito e, finalmente, avendo trovato loro di nuovo scandita a voce alta la parola di Dio, che non udivano da molto tempo, ritornavano ad alzarsi e ad amare la vita.
  Gesu' sapeva che cosi' facendo seminava semi piu' nutrienti del grano per tutto il popolo e che al contempo, i prepotenti, e i servi loro, accumulavano odio verso lui. Loro volevano il potere della paura, lui predicava l'arrendersi all'amore. Loro esaltavano solo se stessi, lui esaltava l'essere umano, gli uomini e le donne, povere e ricche, gli esseri emarginati e quelli integrati. Lui parlava di liberta' ai cuori degli uomini ormai rinchiusi in gusci di ferro, costruiti per difendersi gli uni dagli altri. Loro, invece, con una mano sorridevano, con quella dietro si approfittavano delle debolezze degli altri.
  Cosi', dopo non molti anni, successe cio' che era stato predetto, molti lo lodavano e ascoltavano le sue parole, molti altri lo odiavano, e fu messo a morte.
  Tuttavia, lui aveva amato l'uomo, e molti con i suoi insegnamenti e il suo esempio rinacquero. Molti uomini capirono che la loro vera direzione e' quella dell'Amore. Che l'Amore e' la piu' difficile delle direzioni, ma la piu' fruttuosa e bella. Con questa nuova forza Gesu' risorse e fu' sempre accanto a loro, in ogni gesto d'amore, piccolo o grande, che si scambiavano, e loro affrontarono e amarono la vita.
  Da allora, i saggi di molte culture non dissero piu' di essere saggi. Il piu' saggio per loro era stato Gesu'. E tutti questi, seguendo il suo esempio, ricercavano l'Amore e la Pace. E cio' che insegnavano lo insegnavano a nome suo, per continuare la sua Opera. Costoro erano i Santi. Ma anche i saggi di ancora altre molte culture, ammirarono la condotta e le opere di Gesu', e inspirati da lui, fecere al pari cose grandi.
  Oggi, sono trascorsi 2022 anni dopo la nascita di Gesu', questo e' il 17% del tempo passato dalla nascita dell'agricultura e solo l'1.6% dalla nascita dell'uomo.
  L'uomo e' sempre l'uomo, ed e' sempre posto nella difficile impresa di gestire il suo potere di scegliere per la sua vita, nel lavoro, nell'amore, per la vita di se stesso e degli altri che lo amano, o necessitano di rispettarlo. 
E' vero che e' molto piu' agevole e sicuro camminare in Europa. Esistono leggi che sono concepite per tutelare i diritti umani, ospedali che, anche se a volte non sono efficienti, salvano vite umane. E molti possono lavorare in maniera onesta e a volte utile per gli altri, e molti ancora hanno avuto la possibilita' di istruirsi e di comprendere le dinamiche del mondo.
Tuttavia, fino a quando il mondo sara' sbilanciato e ci saranno paesi superiori e paesi inferiori, uomini che stanno bene e uomini che stanno male, non sara' cambiato molto da allora. La guerra intestina fra gli uomini continuera'. Anche se oggi e' piu' difficile vederla, c'e' ancora: competizione per il lavoro, stress letale per non perdere il posto del lavoro, alienazione e distanza dalla propria umanita'.
  Tuttavia, per chi ha orecchie per sentire, la Parola e' gia' stata rivelata, e chi dedica la propria vita per incarnarLa in se', non solo conoscera' momenti di Pace e di gioia, e l'umilta' e potenza dell'Amore, ma ricordera' a tutti il fine del Suo amore: la vita. E dimostrera' la potenza del Suo Amore, che genera vita, tramite la vita, solo per la vita e per tutta la vita.}



