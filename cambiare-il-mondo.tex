\chapter{Cambiare il mondo}

Se si vuole che le cose nel mondo cambino, bisogna cambiare se stessi. Questo basta e, anzi, non bisogna \emph{pretendere} il cambiamento negli altri.\\

Quello che si capisce e' che, inizialmente, molte delle proprie idee che cambierebbero le cose nel mondo sono un nostro chiedere e pretendere dei cambiamenti nelle persone. Questi cambiamenti non sono semplici da attuare e richiedono una buona dose di sacrificio. Un esempio banale e' il ritenere giusto usare Linux e stupido usare Windows. Per chi non e' appassionato di informatica, non sarebbe semplice imparare ad usare un nuovo sistema operativo (anche se oggi giorno Linux e' diventato molto piu' usabile).

Non e' giusto pretendere. Al massimo, si puo' chiedere e far vedere il proprio punto di vista. L'altra persona sara' poi libera di scegliere. Si può spiegare perche' e' meglio usare Linux, senza pero' prendersela se lei non lo usera' e, comprendendo i suoi motivi, continuare ad essere contenti di lei.

Ma allora, come puo' cambiare il mondo?

Affrontando un percorso interiore innanzitutto si diventa sempre piu' contenti della propria vita, e piu' forti e coraggiosi nell'affrontare le difficolta'. Cosi', cio' che e' esterno, diventa una semplice condizione naturale delle cose. Ce la prenderemmo mai con un fiume che ha smesso di scorrere? Cercheremmo acqua in altri posti, magari in punti piu' remoti e difficili da raggiungere, ma cercheremmo di trarre il massimo dalla vita lo stesso, e cosi' saremmo ugualmente felici. 
Inoltre, ognuno nel mondo sta' facendo cio' che crede sia meglio fare. Anche quando sceglie il male. Semplicemente perche' non ha mai vissuto il bene, ne' nessuno gliene ha mai dato prova. 
Se noi raggiungiamo il bene, per migliorare il mondo, e' necessario condividerlo a chi non lo conosce.

Sempre pensando alla societa', intanto dovremmo essere in grado di amministrare il nostro potere microscopico allo stesso modo di come vorremmo che venga amministrato il potere macroscopico dagli stati, dalle istituzioni e dai nostri capi, ed anche dai nostri genitori.
Ad esempio, se non ci piace l'uso consumistico ed individuale delle automobili, noi dovremmo prediligire il trasporto pubblico, la bicicletta, e l'aggregarsi con altri per gli spostamenti.
Se a lavoro vogliamo che dall'alto vengano riconosciuti i meriti di chi sta' in basso, dobbiamo fare cio' nel nostro piccolo, ad esempio, cedendo un progetto a cui teniamo ad un nuovo arrivato, ma capace, per dargli uno spazio che altrimenti non avrebbe, o avrebbe piu' difficilmente.

Infine, e' chiaro che se tutti apportassero un contributo, la somma totale sarebbe notevole e il mondo cambierebbe davvero. Ma per arrivare a questo punto, ci vorranno molte generazioni, vissute senza traumi di guerre, emarginazioni e discriminazioni. Fino ad allora, il mondo ha bisogno di piccoli eroi che si mettono il cammino per far sgorgare e crescere il bene in se stessi e metterlo a disposizione degli altri. Sara' dura, ma il bene vincera'.

