\chapter{Cambiare il mondo}

Se si vuole che le cose nel mondo cambino, bisogna cambiare se stessi. Questo basta e, anzi, non bisogna \emph{pretendere} il cambiamento negli altri.\\

Segue un testo tratto da ``Hacking interiore''\\
\url{https://github.com/opendatahacklab/aaronwinstonsmith/blob/master/hacking-interiore/hacking-interiore.md}\\

``Quello che si capisce e' che, inizialmente, molte delle proprie idee che cambierebbero le cose nel mondo sono un nostro chiedere e pretendere dei cambiamenti nelle persone. Questi cambiamenti non sono semplici da attuare e richiedono una buona dose di sacrificio. Un esempio banale e' il ritenere giusto usare Linux e stupido usare Windows. Per chi non e' appassionato di informatica, non sarebbe semplice imparare ad usare un nuovo sistema operativo (anche se oggi giorno Linux e' diventato molto piu' usabile).

Non e' giusto pretendere. Al massimo, si puo' chiedere e far vedere il proprio punto di vista. L'altra persona sara' poi libera di scegliere. Si può spiegare perche' e' meglio usare Linux, senza pero' prendersela se lei non lo usera' e, comprendendo i suoi motivi, continuare ad essere contenti di lei.

Ma allora, come puo' cambiare il mondo?

Affrontando un percorso interiore ognuno arriva alla sua risposta. La mia e' questa: sono sempre piu' contento delle cose per come sono, perche', in un certo senso, ognuno nel mondo sta' facendo del suo meglio.''

E, per continuare il discorso, peggio per chi non sta' facendo del suo meglio. Perde l'opportunita' di fare qualcosa che ha un senso.\\
Infine, e' chiaro che ``se tutti apportassero un contributo, la somma totale sarebbe notevole e il mondo cambierebbe davvero''. Tuttavia, forse, non e' neanche questo il punto: il punto e' stare in pace con se stessi: non compro la coca-cola perche' danneggia me e perche' dare piu' potere economico ad una azienda che non investe realmente sul benessere del mondo danneggia, indirettamente e a lungo termine, anche altri nel mondo.
