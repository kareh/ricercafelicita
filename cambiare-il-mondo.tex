\chapter{Cambiare il mondo}

Se si vuole che le cose nel mondo cambino, bisogna cambiare se stessi. Questo basta e, anzi, non bisogna \emph{pretendere} il cambiamento negli altri.\\

Quello che si capisce e' che, inizialmente, molte delle proprie idee che cambierebbero le cose nel mondo sono un nostro chiedere e pretendere dei cambiamenti nelle persone. Questi cambiamenti non sono semplici da attuare e richiedono una buona dose di sacrificio. Un esempio banale e' il ritenere giusto usare Linux e stupido usare Windows. Per chi non e' appassionato di informatica, non sarebbe semplice imparare ad usare un nuovo sistema operativo (anche se oggi giorno Linux e' diventato molto piu' usabile).

Non e' giusto pretendere. Al massimo, si puo' chiedere e far vedere il proprio punto di vista. L'altra persona sara' poi libera di scegliere. Si può spiegare perche' e' meglio usare Linux, senza pero' prendersela se lei non lo usera' e, comprendendo i suoi motivi, continuare ad essere contenti di lei.

Ma allora, come puo' cambiare il mondo?

Affrontando un percorso interiore innanzitutto si diventa sempre piu' contenti della propria vita, e piu' forti e coraggiosi nell'affrontare le difficolta'. Cosi', cio' che e' esterno, diventa una semplice condizione naturale delle cose. Ce la prenderemmo mai con un fiume che ha smesso di scorrere? Cercheremmo acqua in altri posti, magari in punti piu' remoti e difficili da raggiungere, ma cercheremmo di trarre il massimo dalla vita lo stesso, e cosi' saremmo ugualmente felici. 
Inoltre, ognuno nel mondo sta' facendo cio' che crede sia meglio fare. Anche quando sceglie il male. Semplicemente perche' non ha mai vissuto il bene, ne' nessuno gliene ha mai dato prova. 
Se noi raggiungiamo il bene, per migliorare il mondo, e' necessario condividerlo a chi non lo conosce.

Sempre pensando alla societa', intanto dovremmo essere in grado di amministrare il nostro potere microscopico allo stesso modo di come vorremmo che venga amministrato il potere macroscopico dagli stati, dalle istituzioni e dai nostri capi, ed anche dai nostri genitori.
Ad esempio, se non ci piace la corruzione, noi per primi non sporcizia di carte e plastiche nella citta', noi per primi non dovremmo buttare rifiuti piccoli o grandi in giro. Oppure, se crediamo nel lavoro come necessita' per vivere e come servizio dato alla societa', noi per primi dovremmo essere contenti del proprio lavoro se permette il vivere degnamente e se e' anche in una minima percentuale utile agli altri. Dovremmo non curarci del fatto che non e' una figura lavorativa di spicco, o se ci sono lavori che pagano di piu' ma che sono meno utili agli altri.

Infine, e' chiaro che ``se tutti apportassero un contributo, la somma totale sarebbe notevole e il mondo cambierebbe davvero''. Tuttavia, non e' proprio questo il punto. Non serve veramente che ogni persona del mondo faccia le cose come noi le vorremmo. Il punto e' stare in pace con se stessi: non compro la coca-cola perche' danneggia me e perche' darei piu' potere economico ad una azienda che non investe sul benessere di ognuno. Se anche gli altri lo facessero, molto meglio! Gli stessi euro risparmiati per una bevanda zuccherata che porta obesita', potrebbero essere spesi per matite prodotte in un paese in via di sviluppo da una azienda che rispetta i suoi lavoratori.

Se io stesso faccio cio' che il mio cuore desidera posso stare in pace, se anche altri due lo fanno, posso gioire.
