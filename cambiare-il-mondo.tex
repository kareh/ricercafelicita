\chapter{Cambiare il mondo}

Il male nel mondo e' il risultato della somma delle paure, egoismi ed insensibilita' di ciascuno. Queste, sono la conseguenza naturale del male nel mondo in cui ciascuno e' cresciuto nella sua infanzia. Se si cresce da bambini in un ambiente aggressivo, e' facile diventare aggressivi. Se da bambini non ci siamo potuti fidare degli estranei, a maggior ragione da grandi non ci possiamo fidare degli sconosciuti, ed appena qualcuno si discosta dalla nostra normalita', lo giudichiamo e ci sembra pericoloso, od inferiore e noi superiori. Per questo meccanismo, chi viene da una classe sociale svantaggiata, difficilmente riesce a inserirsi in una comunita' piu' privilegiata. Continuando a vivere in una classe svantaggiata, sara' piu' incline a discostarsi dal rispetto delle regole ``normali''. Cosi' sara' portato a commettere crimini piccoli, a non rispettare ed avere fiducia nelle istituzioni, o a commettere crimini piu' gravi. Questi sono esempi di come si crea il circolo vizioso che se non interrotto porta alla distruzione della societa'.

Non ci sono altri mali, o poteri piu' forti della somma degli infinitesimi maligni a cui ognuno contribuisce ogni giorno con la sua identita' ferita. Ad esempio, ci lamentiamo delle multinazionali, che chissa' quale potere pare abbiano. Tale potere economico, politico e sociale, pero', siamo noi a darglielo! E' piu' vantaggioso comprare prodotti a minor prezzo, e con un click, che pagare un prodotto fatto da una azienda locale. Questo muove una massa di persone e di soldi verso il monopolio, iper-industrializzato, di multinazionali. Ci lamentiamo che tali multinazionali sfruttano i lavoratori, nel nostro paese, e ancor di piu' nei paesi sottosviluppati. Ma siamo disposti a pagare di piu', ad avere meno tempo, a crearci nuove difficolta' che nessuno ha?  E' vero, tutto questo e' difficile permetterselo, ma qui sta' il punto. Il nostro non-sacrificio, sacrifica il resto del mondo, e di riflesso, anche noi stessi.

La disoccupazione, l'alta competizione, lo sfruttamento e lo stress nel lavoro, i problemi sociali, le guerre, sono tutte descrivibili allo stesso modo. Non c'e' un grande cattivo che architetta tutto cio'. Possono esserci sciacalli che ci bagnano il pane nei problemi del mondo, come chi specula in borsa, o chi progetta truffe, ma cio' e' una percentuale irrisoria rispetto al vero problema. Se cosi' non fosse, basterebbe destituire il dittatore di turno, e il mondo sarebbe felice e sorridente. Ma e' mai stato cosi'? Ne' ci sara' mai un grande leader che risolvera' per piu' di un anno, anche solo uno di un nostro problema personale. Fino a quando tutti saremo la causa del malessere collettivo, il mondo rimarra' lo stesso, cambieranno i nomi, ``monarchia'' e ``sudditanza'' si chiamera' ``capitalismo'' e ``proletariato'', ``teocrazia'' si chiamera' ``tecnocrazia'', ma il mondo rimarra' sempre lo stesso.

Se si vuole che le cose nel mondo cambino, non c'e' altro modo che cambiare quell'infinitesimo di noi stessi, senza pretendere un minimo di cambiamento negli altri. Questo, varra' esponenzialmente tanto, e nei secoli, nei millenni convergera' verso un mondo migliore. E quand'anche il mondo trovera' la sua fine, noi lo avremo amato veramente, e ogni abuso subito in vita, non ci sara' pesato, perche' sara' servito, almeno un po', ad almeno una persona.

Per cambiare noi stessi, e' lodevole privilegiare prodotti ``green'', discostarsi dal consumismo, scegliere modi di vita alternativi. Tuttavia, cio' che Gesu' dice e' ancora piu' forte: dobbiamo cambiare la nostra identita', i nostri stessi sentimenti, il nostro spirito. Questo, nel nostro piccolo, al di la' di cosa compriamo o meno, ha profonde conseguenze. Ha conseguenze sul rapporto con i nostri colleghi, con i clienti, col valutare e giudicare le scelte dei nostri superiori, su come trattiamo chi ha un ruolo subordinato a noi. Gli effetti sono invisibili e infinitesimi, ma cosi' come le ottimizzazioni ingegneristiche sui motori, che migliorano di una piccola percentuale l'efficienza di un ingranaggio, si traducono in una molto piu' elevata prestazione, perche' quell'ingranaggio gira migliaia e migliaia di volte, e quindi la piccola percentuale si cumula, allo stesso modo nella nostra vita, le migliorie spirituali che esaltano la vita propria ed altrui, si cumulano nel tempo, e poi, in momenti opportuni, aprono le porte a nuovi orizzonti. E' solo cosi' che possono avvenire piccoli grandi miracoli, come un impiegato che discute criticamente l'eticita' di una scelta aziendale, mettendo in gioco il suo posto. O come un manager che difende i suoi lavoratori, mettendo in gioco la produttivita' capitalistica dell'azienda e, di rimando, il suo posto. O come di un partner, che accetta gli ideali del suo partner di vita equosolidale, al costo di vivere con ristrettezze economiche e materiali.

Crescendo e maturando, si capisce che inizialmente, molte delle proprie idee che cambierebbero le cose nel mondo sono un nostro chiedere e pretendere dei cambiamenti nelle persone. Questi cambiamenti non sono semplici da attuare e richiedono una buona dose di sacrificio. Un esempio banale e' il ritenere superiore usare Linux ed inferiore usare Windows. Per chi non e' appassionato di informatica, non e' semplice imparare ad usare un nuovo sistema operativo\footnote{anche se oggi giorno Linux e' diventato molto piu' usabile}.
Pretendere non e' amare, anche se si pretende il giusto. Piuttosto, si puo' annunciare e spiegare il proprio ideale, testimoniando con la propria vita la sua applicazione. Cosi', l'altra persona sara' libera di scegliere. Si può nella propria quotidianeta' usare Linux, affrontando tutte le piccole, e a volte grandi, difficolta' connesse, e spiegare agli altri perche' e' piu' etico, e nel lungo tempo, piu' efficiente usare Linux, senza pero' prendersela con chi non lo usa, comprendendo i suoi motivi, e continuando ad armarlo.

Affrontando un percorso interiore innanzitutto si diventa sempre piu' contenti della propria vita, e piu' forti e coraggiosi nell'affrontare le difficolta'. Cosi', cio' che e' esterno, diventa una semplice condizione naturale delle cose. Ce la prenderemmo mai con un fiume che ha smesso di scorrere? Cercheremmo acqua in altri posti, magari in punti piu' remoti e difficili da raggiungere, ma cercheremmo di trarre il massimo dalla vita lo stesso, e cosi' saremmo ugualmente felici. 

Inoltre, ognuno nel mondo sta' facendo cio' che crede sia meglio fare. Anche quando sceglie il male, sta' scegliendo cio' che crede essere bene. Se e' vero che il bene che noi viviamo e' superiore a quello scelto dall'altro, allora facendolo sperimentare e conoscere, l'altro lo accogliera'.
Quindi, per curare il male negli altri, prima dobbiamo raggiungere il bene in noi stessi, e dopo, metterci nella difficile ma nobile impresa di comunicarlo e farlo conoscere a chi non lo conosce.

Questo modo di pensare e agire, e' molto diverso dallo stile imperialista a cui siamo abituati, in cui c'e' una verita' piu' vera delle altre, o una giustizia piu' giusta. Questo stile imperialista ha portato a molte guerre e divisioni, e continuera' a farlo. Solo l'amore puo' portare ad un bene superiore, e l'amore non con-vince, l'amore disseta.

Riguardo al potere, l'errore piu' grande e' il non avere un potere sano sulla propria anima. E' quando non siamo in grado di portare noi stessi alla pace interiore, a buone relazioni, al vivere sane emozioni, che proiettiamo la nostra incapacita' nei leader mondiali, nei grandi sistemi, nelle situazioni di crisi macroscopiche. Un santo, invece, riesce a sorridere anche nelle situazioni piu' difficili ed inumane, perche' nel suo cuore e' incrollabile l'amore che prova Dio per lui.

Chi scegliera' di amare il mondo, cosi' come sopra descritto, nella sua vita potra' giungere a questa fede: ``l'umanita' continuera' a vivere, a rinnovarsi. Non esistera' mai un armageddon totale. L'umanita' si rialzera' sempre dalle proprie ceneri. Il bene vincera' sempre, e chi avra' perseguito il male, sara' dimenticato come polvere al vento. Dopo molte generazioni, dopo aver superato molti traumi di guerre, emarginazioni e discriminazioni, sfide e mutamenti tecnologici di stili di vita, e mutamenti di DNA, dopo tutto questo, conoscera' un periodo di pace e prosperita', che durera' fino alla fine dell'universo''. Fino ad allora, l'umanita' ha bisogno di piccoli eroi che per lei si mettono in cammino per far sgorgare e crescere il bene in se stessi e metterlo a disposizione degli altri e di tutti.

