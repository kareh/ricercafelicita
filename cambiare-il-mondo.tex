\chapter{Cambiare il mondo}

Il male nel mondo e' il risultato della somma delle paure, egoismi ed insensibilita' di ciascuno. Queste, sono la conseguenza naturale del male nel mondo in cui ciascuno e' cresciuto nella sua infanzia. Se si cresce da bambini in un ambiente aggressivo, e' facile diventare aggressivi. Se da bambini non ci siamo potuti fidare degli estranei, a maggior ragione da grandi non ci possiamo fidare degli sconosciuti, ed appena qualcuno si discosta dalla nostra normalita', lo giudichiamo e ci sembra pericoloso, od inferiore e noi superiori. Per questo meccanismo, chi viene da una classe sociale svantaggiata, difficilmente riesce a inserirsi in una comunita' piu' privilegiata. Continuando a vivere in una classe svantaggiata, sara' piu' incline a discostarsi dal rispetto delle regole ``normali''. Cosi' sara' portato a commettere crimini piccoli, a non rispettare ed avere fiducia nelle istituzioni, o a commettere crimini piu' gravi. Questi sono esempi di come si crea il circolo vizioso che se non interrotto porta alla distruzione della societa'.

Non ci sono altri mali, o poteri piu' forti della somma degli infinitesimi maligni a cui ognuno contribuisce ogni giorno con la sua identita' ferita. Ad esempio, ci lamentiamo delle multinazionali, che chissa' quale potere pare abbiano. Tale potere economico, politico e sociale, pero', siamo noi a darglielo! E' piu' vantaggioso comprare prodotti a minor prezzo, e con un click, che pagare un prodotto fatto da una azienda locale. Questo muove una massa di persone e di soldi verso il monopolio, iper-industrializzato, di multinazionali. Ci lamentiamo che tali multinazionali sfruttano i lavoratori, nel nostro paese, e ancor di piu' nei paesi sottosviluppati. Ma siamo disposti a pagare di piu', ed anche spendere del tempo per scegliere delle alternative, e fare piu' tentativi fino a trovarne una migliore? E' vero, tutto questo e' difficile permetterselo, ma qui sta' il punto. Il nostro non-sacrificio, sacrifica il resto del mondo, e di riflesso, anche noi stessi.

La disoccupazione, l'alta competizione, lo sfruttamento e lo stress nel lavoro, i problemi sociali, le guerre, sono tutte descrivibili allo stesso modo. Non c'e' un grande cattivo che architetta tutto cio'. Possono esserci sciacalli che ci bagnano il pane nei problemi del mondo, come chi specula in borsa, o chi progetta truffe, ma cio' e' una percentuale irrisoria rispetto al vero problema. Se cosi' non fosse, basterebbe destituire il dittatore di turno, e il mondo sarebbe felice e sorridente. Ma e' mai stato cosi'? Ne' ci sara' mai un grande leader che risolvera' per piu' di un anno, anche solo uno di un nostro problema personale. Fino a quando tutti saremo la causa del malessere collettivo, il mondo rimarra' lo stesso, cambieranno i nomi, ``monarchia'' e ``sudditanza'' si chiamera' ``capitalismo'' e ``proletariato'', ``teocrazia'' si chiamera' ``tecnocrazia'', ma il mondo rimarra' sempre lo stesso.

Se si vuole che le cose nel mondo cambino, non c'e' altro modo che cambiare quell'infinitesimo di noi stessi, senza pretendere un minimo di cambiamento negli altri. Questo, varra' esponenzialmente tanto, e nei secoli, nei millenni convergera' verso un mondo migliore. E quand'anche il mondo trovera' la sua fine, noi lo avremo amato veramente, e ogni abuso subito in vita, non sara' pesato, perche' sara' servito, almeno un po', ad almeno una persona.\\

In dettaglio, quello che si capisce e' che, inizialmente, molte delle proprie idee che cambierebbero le cose nel mondo sono un nostro chiedere e pretendere dei cambiamenti nelle persone. Questi cambiamenti non sono semplici da attuare e richiedono una buona dose di sacrificio. Un esempio banale e' il ritenere giusto usare Linux e stupido usare Windows. Per chi non e' appassionato di informatica, non sarebbe semplice imparare ad usare un nuovo sistema operativo (anche se oggi giorno Linux e' diventato molto piu' usabile).

Non e' giusto pretendere. Al massimo, si puo' chiedere e far vedere il proprio punto di vista. L'altra persona sara' poi libera di scegliere. Si può spiegare perche' e' meglio usare Linux, senza pero' prendersela se lei non lo usera' e, comprendendo i suoi motivi, continuare ad essere contenti di lei.

Ma allora, come puo' cambiare il mondo?

Affrontando un percorso interiore innanzitutto si diventa sempre piu' contenti della propria vita, e piu' forti e coraggiosi nell'affrontare le difficolta'. Cosi', cio' che e' esterno, diventa una semplice condizione naturale delle cose. Ce la prenderemmo mai con un fiume che ha smesso di scorrere? Cercheremmo acqua in altri posti, magari in punti piu' remoti e difficili da raggiungere, ma cercheremmo di trarre il massimo dalla vita lo stesso, e cosi' saremmo ugualmente felici. 
Inoltre, ognuno nel mondo sta' facendo cio' che crede sia meglio fare. Anche quando sceglie il male. Semplicemente perche' non ha mai vissuto il bene, ne' nessuno gliene ha mai dato prova. 
Se noi raggiungiamo il bene, per migliorare il mondo, e' necessario condividerlo a chi non lo conosce.

Sempre pensando alla societa', intanto dovremmo essere in grado di amministrare il nostro potere microscopico allo stesso modo di come vorremmo che venga amministrato il potere macroscopico dagli stati, dalle istituzioni e dai nostri capi, ed anche dai nostri genitori.
Ad esempio, se non ci piace l'uso consumistico ed individuale delle automobili, noi dovremmo prediligire il trasporto pubblico, la bicicletta, e l'aggregarsi con altri per gli spostamenti.
Se a lavoro vogliamo che dall'alto vengano riconosciuti i meriti di chi sta' in basso, dobbiamo fare cio' nel nostro piccolo, ad esempio, cedendo un progetto a cui teniamo ad un nuovo arrivato, ma capace, per dargli uno spazio che altrimenti non avrebbe, o avrebbe piu' difficilmente.

Infine, e' chiaro che se tutti apportassero un contributo, la somma totale sarebbe notevole e il mondo cambierebbe davvero. Ma per arrivare a questo punto, ci vorranno molte generazioni, vissute senza traumi di guerre, emarginazioni e discriminazioni. Fino ad allora, il mondo ha bisogno di piccoli eroi che si mettono in cammino per far sgorgare e crescere il bene in se stessi e metterlo a disposizione degli altri. Sara' dura, ma il bene vincera'.

