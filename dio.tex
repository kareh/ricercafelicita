\chapter{La ricerca di Dio con approccio scientifico}
\label{chapterDio}

Oggi giorno, la fede verso Dio non esiste piu', se non tra pochi. Si trascura la tematica di Dio, perche' e' la scienza che da' ogni risposta e si crede che Dio e' solo una superstizione per chi non sa' ragionare.

Tuttavia, cosi' come la ragione e' una dote naturale che l'uomo puo' adoperare per migliorare la sua vita, anche la fede nell'Essere che sempre ed infinitamente ama, e' una vocazione naturale dell'uomo che puo' essere un valido strumento per la ricerca della felicita'. In questo capitolo si discute in termini moderni, lucidi e razionali riguardo a Dio.

E' piu' importante viaggiare che arrivare alla meta, perche' e' nel viaggio che si cresce, mentre raggiunta la destinazione si puo' solo stare fermi o iniziare un nuovo viaggio. Allo stesso modo, e' piu' importante la ricerca di Dio, del vero piacere, della vera essenza dell'amore disinteressato e della vita, piuttosto che avere una risposta pronta alla domanda: Dio esiste? Quindi, non si cerchera' di convincere dell'esistenza di Dio, ma piuttosto si daranno indicazioni su come mettersi in cammino per amare Lui.

La fede in Dio non e' il rispetto formale e repressivo di leggi imposte dall'alto da persone potenti. Se si e' arrivati a credere cio' e' perche' nel passato con la religione si esagerava: o la si mischiava col potere temporale e sfociava cosi' nella corruzione, sia a causa dei cittadini che lo accettavano e permettevano, sia a causa dei governanti che si nascondevano e si approfittavano del loro ruolo. Oppure, la si adoperava in maniera opprimente e repressiva, per non affrontare i problemi veri della vita, per rispettare la ``legge del gregge'' e non essere emarginati dai pettegolezzi del paese o per farsi belli e contare qualcosa. Cosi', veniva dato a Dio il contentino di un rispetto formale della Legge.

Nel '900, con il progresso scientifico e tecnologico, le persone hanno avuto il coraggio di dire che cosi' le cose non andavano. Il nuovo benessere e le vittorie sulle malattie che si facevano con la tecnologia, spingevano con forza la societa' all'abbandono delle sue superstizioni e leggi repressive. Al contempo, pero', hanno portato con se' anche l'illusione del dominio dell'uomo sulla Natura e della superiorita' della ragione sui sentimenti e sulle altre persone meno istruite. Allora, paradossalmente, e' diventato insensato credere che la propria vita e' per se stessi e per gli altri, e credere che la vita non e' una lotta contro la Natura. E' piu' facile oggi credere che la vita e' solo un fatto meccanico, di cui, quando ci va bene, noi siamo padroni e di cui, nella nostra onnipotenza possiamo disporre arbitrariamente, oppure, quando ci va male, di cui noi non possiamo fare nulla perche' schiacciati dalle forze della natura e della societa'.

Cosi', si e' perso il dare ufficiale valore ai temi umani, alla sacralita' della vita, alla purezza dell'amore. Solo in pochi, i piu' sensibili e ``deboli'' o i piu' provati dalla vita, si consolano pensando a queste cose con belle poesie, vergognandosi o nascondendosi dagli altri piu' ``forti''.

Si ricerca il piacere, ma non si sa' dove trovarlo, non si sa' cos'e'. Viene definito solamente tramite i sensi. Ma il ``senso dei sensi'' non viene definito. Si cercano scappatoie, nuove realta' virtuali, aumentate, tour ed esperienze trascendenti, ma al ritorno si e' sempre gli stessi. Si trascura il fatto che l'anima desidera ed ha bisogno dell'infinito, e qualsiasi raggiungimento, traguardo, sensazione, e' sempre superata da un desiderio e bisogno piu' grande.\\
Cio' che rimane nel mondo scientifico e' deludente, ne' piu' deludente del mondo magico medievale, ne' meno. Oggi, ci sono solo effetti speciali, tecnologie all'ultimo grido, immagini ed esperienze mozza fiato, una dietro l'altra, che adoperiamo, incosciamente e cosciamente, per convincerci sempre che va bene rimanere tali e quali a come si e', che siamo perfetti, che al piu' sono gli altri a sbagliare a non allinearsi al nostro modo di fare, che e' il mondo che ancora non ci capisce.\\
Tutte queste tecnologie, sono una piccola cosa di fronte alla vera forza che e' quella Sua, forza che e' comprensione, consolazione ed estatico nutrimento dell'anima, forza che Lui ha per tutti e che muove ogni atomo.\\

Nei seguenti paragrafi si parlera' un po' delle ``solite cose'': amore, gli altri, il se'. Anche se nel mondo, in ogni pubblicita' e tazzina di caffe' si parla sempre di amore, la ricerca dell'Amore e' una cosa profonda, vera e difficile e che richiede tutta la propria vita per conoscerlo e trovarlo. I discorsi che seguono sono i miei piccoli appunti a riguardo, scritti con un occhio scientifico.

Nel paragrafo \ref{extInt} pag. \pageref{extInt}, si discute sulla visione atomistica e meccanicistica del Tutto, e nel sottoparagrafo \ref{ossScientificheFilosofiche}, si discute del significato di Dio in tale contesto.

Nell'ultimo paragrafo \ref{procAx} pag. \pageref{procAx}, si fara' una esposizione formale, in stile matematico di quanto detto in questo capitolo. Questo paragrafo e' per chi ha imparato a ragionare in maniera rigorosa o matematica/scientifica. In esso si cerca di condensare tutto il pensiero su Dio, in definizioni precise e logiche.


\section{Amare}
\label{amareSe}


\subsubsection{Se stessi}
Amare se stessi e' fondamentale, cosi' come amare gli altri.

Per alcune persone amare se stessi e' facile, e anzi esagerano in cio', per altre e' difficile e poco si amano.

Amare se stessi e' 1. sentire come stiamo 2. essere, pensare e fare cio' che ci fa' sentire bene 3. migliorare, tendendo ad un bene profondo.\\

Come spiegano Eric Berne ed Alexander Lowen (vedi capitolo \ref{chapRiferimenti} pag. \pageref{chapRiferimenti}), in questo mondo difficile, chi piu', chi meno, riceve degli insegnamenti inconsci distorti dai suoi genitori, che a loro volta hanno ricevuto dai loro genitori, e cosi' via. Questi insegnamenti, inizialmente, sono stati scatenati nella storia da traumi veri naturali, come la morte di un genitore o di un figlio, o anche da traumi con il resto della societa', ad esempio, un matrimonio forzato da parte della famiglia.

Questi insegnamenti, ci allontanano dalla nostra vera natura. Nei casi piu' estremi, degenerano in psicopatologie. Nei casi comuni, sono insegnamenti accettati dalla societa' in cui si vive, e quindi, nessuno, se non scruta e scava dentro se stesso si accorge che non sono veri. Un esempio, e' la cultura del divertimento del modello Americano, dove chi non si diverte e' marchiato come strano ed emerginato dal gruppo, e dove il sesso o altre cose serie e che hanno un profondo impatto nella persona, sono considerati come piaceri superficiali.

Infine, l'istinto dell'Io che e' quello di dire ``io ho ragione'', ``io ho bisogno'', ``io sono''. Se l'io si allontana dal sentire il se' e gli altri, allora dira' di avere ragione, di avere bisogno e di essere qualcuno anche quando questo non sara' cosi'. Se percio' l'Io non viene compreso e allenato, costantemente e con perserveranza nel tempo, l'Io portera' la persona piu' verso la tristezza e la morte che verso la gioia e la vita. Amera' immagini e poteri, personaggi e cose che non saranno al servizio delle sensazioni e emozioni vere della persona.

Amare se stessi e' quindi, intanto capire chi siamo e cosa proviamo, veramente, e questo, anche se il nostro io dice di sapere benissimo chi siamo, cosa vogliamo e se stiamo bene o male, non e' semplice, richiede tempo e tenacia. Riusciti in questa impresa, si potra' conoscere il vero piacere, la vera gioia (ma anche mettersi in guardia dai veri pericoli). Questo piacere e questi pericoli sono quelli che i nostri sogni ci ricordano instancabilmente ogni notte, e che sono troppo difficili da esprimere a parole, e che a fatica il nostro io durante la veglia riesce a comprendere se non li ha conosciuti, studiati ed attenzionati negli anni. Questo piacere e questi pericoli, sono quelli veri, che vanno al di la' di tutti i desideri e sogni che il nostro io proietta e promette durante la veglia. Promesse come la posizione sociale o lavorativa, economica o famigliare. Noi, in verita', valiamo molti ordini di grandezza in piu' rispetto a tutte queste cose.

Quando riconosciamo che con nessun nostro impegno, che con nessuna richezza o potenza, per quanto grande, riusciremo mai a soddisfare la nostra anima, perche' limitati di fronte a qualcosa di illimitato, e quando al contempo riconosceremo il valore di ogni cosa, per quanto piccola, che ci viene incontro o da noi stessi, dagli altri o dallo stesso mondo che sempre disprezziamo, allora comincieremo a metterci in cammino verso il veramente amare noi stessi.\\

Infine, amare se stessi e' legato ad amare gli altri, in quanto, cosi' come il se' e' parte del Tutto, anche gli altri lo sono, e non si puo' veramente gioire se si ama una sola sua parte. Di questo parleremo in seguito.

\subsubsection{L'indipendenza}

Per quanto un'altra persona possa essere importante e influenzare la nostra vita, siamo noi ad innamorarci nell'amore o nell'odio di lei, e dire ``mi sta' succendendo X a causa sua'', ``lei mi sta' facendo X''.\\
Se X e' piacevole, allora va tutto bene, ad esempio: ``lui/lei mi sta' facendo fare una bella passeggiata''. Il problema e' quando X e' fastidioso o doloroso e diciamo ``lui/lei mi sta' facendo soffrire''.\\
Solo noi stessi possiamo desiderare e volere cio' che noi desideriamo e vogliamo. A volte preferiremmo che lo facesse qualcun'altro al posto nostro, qualcuno che ci piace, o qualcuno che non ci piace. Nel primo caso vorremmo che lui desiderasse cio' che ci piace, nel secondo che desiderasse la distruzione di cio' che non ci piace (e siccome non lo desidera, poi ci appare antipatico / odioso).\\
Se non dessimo alcuna delega di desiderare cio' che desideriamo ad altri o anche a volte a tecnologie, a cose costose o ad ideali, a religioni, a modi di fare, a maschere o qualsiasi altra cosa esterna che ci appare potrebbe ``risolvere il problema'' o portare ad un piacere piu' grande,  diremmo piuttosto: ``Io desidero X. Lui/lei desidera Y. Sto' vivendo Z. Z e' piacevole (oppure non e' piacevole, e' doloroso).''.\\
Questa e' una gran bella differenza. $Y$ e' distinto da $X$. Solo se io non desidero, rimane solo il desiderio dell'altro $Y$, e $Z$ e' generato da $Y$ esclusivamente, e cosi' mi sembra essere preda dell'altro, l'altro appare nemico, l'altro appare avere poteri superiori, o avere una cattiveria che influisce sul mondo.\\
Facciamo un esempio: invece di dire ``mio padre mi fa passare i pomeriggi a studiare duramente'', si potrebbe dire ``sto' studiando ogni pomeriggio, intensamente. Mio padre vuole che io studio. Io voglio essere libero e studiare senza soffrire.''. \\
Capito, questo, potrei dire a mio padre: ``io soffro studiando troppo''. Se questo non bastasse per un cambiamento in mio padre, questo sarebbe sufficiente per un cambiamento in me. Io saprei che non e' giusto, e saprei che, allo stesso modo, mio padre e' convinto che e' giusto. Soppesando la mia vita, potrei nel complesso essere comunque contento di mio padre. Poi la situazione puo' evolvere in mille modi, dal migliorare e sviluppare tecniche per fare piu' compiti in maniera piu' efficiente e meno dolorosa, dal collaborare con compagni nello svolgimento dei compiti in eccesso, o dal saltare, qualche volta, qualche compito dicendo di stare male perche', in fondo, e' veramente cosi'. \\
Ad ogni modo, la mia autoconsapevolezza sarebbe sufficiente per ritornare a vivere, per smettere di covare solo risentimenti e per trovare vere soluzioni.\\

Riassumendo, nessuna cosa estranea, che sia interna od esterna, che sia un dittatore o un ammaliatore, un demone o un impulso, puo' obbligare un essere a volere una cosa che non vuole. Se succede e' lo stesso essere che, alla fine, sta' cambiando la sua volonta', annullando il suo desiderio originale.\\

Nota per i lettori padri: anche utilizzando tutta la conoscenza della psicologia, della neurofisiologia cerebrale e del vivere nel mondo, non si amerebbero i propri figli se non avessero liberta'. Come dice Richard David Precht, ``posso volere cio' che voglio? E' vero che la psicologia-positiva [e' un insieme di regole efficaci per raggiungere la felicita'], ma [comunque non risolve il punto fondamentale della liberta']. A cosa mi servono [le direttive e le regole] piu' intelligenti [e piu' giuste] se non sono libero di metterle in pratica?'' \footnote{Libro di Richard David Precht, ``Ma io chi sono? (ed eventualmente, quanti sono?)''}

\subsubsection{L'inconscio}

A volte le cose ci succedono senza sapere perche', ci sembra che sbagliamo o ci sembra che siamo piu' fortunati di quello che crediamo di essere.\\ 
E' il nostro inconscio (vedi ``Eric Berne, Ciao e poi...''). Siamo noi stessi, anche se crediamo di non esserlo e non ci rispecchiamo in quello che in realta' siamo.\\
Non e' niente di oscuro il nostro inconscio: e' la nostra volonta', che non conosciamo. Se ci immergessimo nella nostra anima e vi ci vivessimo a lungo, vista cosi' alla luce del giorno, per alcuni versi ci piacerebbe da impazzire, altre volte ci spaventerebbe o ripugnerebbe.\\
Il piu' delle volte rifiutiamo le nostre emozioni inconscie perche' dobbiamo vivere nella societa' e vogliamo farci rispettare dagli altri e, soprattutto, per seguire le direttive parentali che abbiamo ricevuto da piccoli e che, a loro volta, i nostri genitori hanno ricevuto da piccoli\footnote{Vedi il concetto del ``copione'', di Eric Berne in ``Ciao e poi...''}. Cosi', molte volte andiamo, inconsapevolmente, contro la nostra stessa natura per uniformarci alla morale, alle leggi, al gruppo, con la speranza cosi' di sopravvivere e vivere serenamente.\\
Ad esempio, se dei ragazzacci per strada mi insultano perche' ho inciampato e, tenendo a freno la rabbia dico ``e' stato uno sbaglio, loro sono ignoranti, io sono civile, e vado avanti nella mia vita''. E' vero quello che dico? Se indagassi, invece, forse vedrei che preferirei una bella lotta all'ultimo colpo, per difendere il mio io. E forse, se indagassi di piu', vedrei paura, verso quei ragazzi poco raccomandabili, che svalutano e disprezzano cio' che amo: me stesso. E se finalmente, guardassi me stesso, vedrei la verita': paura e rabbia verso me stesso proiettata in loro. Si, verso me stesso che, in realta', concordo con quei ragazzini, e che, se fossi stato al posto loro, dentro di me avrei deriso allo stesso modo un'altra persona nella mia situazione.

Non c'e' una volonta' (conscia/inconscia) brutta o bella, migliore o peggiore. La volonta' e' cio' che e', cio' che siamo. Disprezzarci, o obbligarci, non ci rendera' migliori.\\
E' solo accompagnando il nostro io con gentilezza, intelligenza e pazienza, nel tempo e negli anni, che si cambia.\\
Si puo' cambiare a tal punto, si puo' raggiungere una consapevolezza di se stessi tale, fino a poter dire: ``ho sbagliato, ero distratto e sono inciapato. Saro' apparso goffo a quei ragazzi. E a loro e' sembrato un piccolo numero di circo. Peccato, avevo bisogno della loro stima, poiche' da piccolo non ne ho ricevuta abbastanza. Non me la prendo con loro. Non sanno ancora cos'e' la vita, quanto e' delicata e quanto e' piacevole e difficile amarla. Arrivato a casa, questa piccola ferita d'orgoglio passera'.'' E cosi' continuerei senza troppi turbamenti.\\

\subsubsection{L'Altro}
\label{altrui}

Di norma \emph{siamo} solo per noi stessi. Quando stiamo con gli altri, pensiamo a loro, ci dedichiamo a loro, \emph{siamo} anche per gli altri. Le forme, i suoni, le sensazioni e i concetti acquistano un senso per noi e per l'altro. Non sono piu' solo sensazioni proprie, sono sensazioni che sono anche dell'altro. Non c'e' piu' solo il nostro corpo, il nostro se', c'e' anche il corpo dell'altro, cio' che lui sente, il suo se'. La nostra anima da' sostanza al nostro se' e al se' altrui, tramite le sensazioni fisiche e le emozioni. \\
Allo stesso modo di come noi consideriamo noi stessi in un particolare modo, acquisiamo un particolare significato per l'altro, un significato che puo' non essere lo stesso che diamo a noi stessi. A volte ci da' piu' importanza (ad esempio, una madre), a volte meno importanza. Tuttavia, se veramente \emph{siamo} anche per l'altro, comprenderemo il suo punto di vista. E il suo punto di vista sara' la Sua verita'. Se quel suono e' per lui dolce, sara' dolce. Non diremo ``lui sente quel suono come dolce, ma in realta' e' per me un normale suono''. Perche', se lo amiamo, cio' che e' suo e' nostro.\\
Viceversa, l'altro acquista un significato per noi. Questo e' scontato, tutti diamo un significato a tutto e a tutti. Alcune persone sono per noi importanti altre meno, alcune simpatiche, altre antipatiche. L'armonia si raggiunge quando il significato che diamo non e' una imposizione, una forzatura per l'altro. Cioe' quando quello che vogliamo dall'altro non diventa un obbligo innaturale per lui e lei, e cioe' una violenza piccola o grande. L'amore si raggiunge, quando cio' che desideriamo e' cio' che l'altro desidera. \\
Tutto quanto detto avviene sia a livello conscio che inconscio.\\

Amare l'altro vuol dire stare bene e provare piacere nell'essere noi stessi e, cosi' essendo, essere causa di poco o molto benessere e piacere per l'altro.\\
La quantita' di benessere e piacere determinata e' derivante da quanto l'altro sta' desiderando e da quanto il nostro amore porta l'altro nel compimento del suo desiderio e non nel compimento di altre cose che non c'entrano nulla, come 1. il soddisfacimento di altri nostri desideri, 2. il compiacimento di suoi desideri superflui o non autentici. \\
La coppia desiderio-amore, trascende limitazioni fisiche. Un essere potrebbe essere felice anche solo di raccogliere un fiore, e allo stesso modo, anche solo del fatto che noi desideriamo cio' che lui desidera. Viceversa, un essere potrebbe non essere mai contento di nulla. \\
Non c'e' limite a cosa possiamo essere, dare e fare per l'altro. L'unico limite e' l'amore per noi stessi. Ad esempio, se sentiamo troppa fatica, vorremmo interrompere il lavoro che stavamo facendo per l'altro. \\
L'altro non deve alcunche' a noi per qualsiasi cosa abbiamo fatto e faremo per lui. Se lo facciamo, e' perche' il farlo ci gratifica, non per secondi fini. Se e' vero che facciamo qualcosa per l'altro e non per noi stessi, non ricercheremo premi o ricompense.\\
Ne' nulla e nessuno puo' imporci o allettarci falsamente di amare. Infatti, dovremmo amare solo se veramente stiamo bene\footnote{a meno che non ci siano emergenze o urgenze, e bisogna intervenire anche se non ci sentiamo di farlo}, e se veramente capendo che non ne avremo alcun guadagno (anche emotivo o sensoriale), comunque per nostra natura ci sentiamo bene e proviamo serenita' o piacere nel farlo.\\

Possiamo desiderare qualcosa dall'altro. Ma questo desiderio sara' sano e grande e potra' realizzarsi solo se coincidera' con l'amore dell'altro. Se desidero un dolce da un pasticciere che incontro, solo se lui ha piacere di fare un dolce per me potro' gustare un vero dolce. Se non lo ha, per qualsiasi motivo, non sara' la stessa cosa. Potrei 1. proporgli del denaro o allettarlo con altri premi in cambio, 2. andare da un altro pasticciere\footnote{oppure armarmi di forza di volonta' e fare il dolce da me}, 3. dire che il mio desiderio e' fuori luogo. Tuttavia, nessuna delle tre soluzioni e' in realta' una soluzione. La vera soluzione e' dire: io ho questo desiderio, ma ho bisogno che anche lui desideri amarmi, fino ad allora il mio sara' un desiderio, non una realta'. Solo cosi', godro' della vita presente e viva, e conoscero' tutti i desideri miei che, inconsapevolmente, io, gli altri e la natura stanno gia' soddisfacendo.\\

Noi siamo potenti non quando decidiamo del destino dell'altro, ma quando l'altro cercando amore in noi, lo trova perche' noi abbiamo desiderato di soddisfare il suo desiderio e, in parte o del tutto, e' stato soddisfatto. Tutto cio' che avremo fatto sara' di proprieta' e sara' ad uso e consumo dell'altro, e l'avremo fatto pacificamente o piacevolmente. Questo e' potente perche' noi avremo vissuto un momento carico di significato e l'altro avra' avuto qualcosa di veramente suo.\\

Se l'altro riuscira' ad apprezzare quanto di vero c'e' nel nostro amore, apprezzera' anche solo una parola nostra, un fiore, e anche avendo ricevuto da noi centomila doni, non dara' troppa importanza alle cose, e continuera' a dare valore solo al nostro amore. \\
Se l'altro non riuscira' ad apprezzare, non possiamo e non dobbiamo fare nulla per rimediare. Il giudizio finale dell'altro dipendera' da quanto lui e' in pace con se stesso. Se e' un santo, ringraziera' anche se avra' ricevuto nel concreto una cosa che comunemente ha poco valore, se, invece, e' in preda alle tempeste della vita disprezzera', anche se chi lo ha amato avra' fatto il suo servo per decenni. Ad ogni modo, tutto cio' non importi per chi ama. Chi ama sia gia' appagato del fatto che sta' amando. E se chi ama non tradira' se stesso, stara' sempre bene, anche se l'amato sara' avverso a lui, e cio' sara' molto difficile da digerire. \\

\section{La ricerca di Dio}
Se noi crediamo che Dio e' amore puro, totale e incondizionato, per ogni essere vivente, e quindi per noi stessi e per gli altri, allora ora rimane da capire come amare e come amare meglio.\\

Essere amore per se stessi e gli altri e' difficile. In principio e' semplicissimo: basta mettere da parte il proprio io, noi stessi, ed impegnarsi, sforzandosi, per se stessi e per l'altro. Tuttavia, cio' e' barare tanto quanto prendersi delle pillole per non sentire lo sforzo in una gara agonistica.\\
Il vero amore e' la carita', e la carita' non nasce da un obbligo esterno imposto da noi stessi o da altri. Nasce da un profondo soddisfacimento, da una piena propria realizzazione.\\
Piuttosto che forzare l'ego, anche se a fin di bene, e' piu' piacevole e proficuo ascoltare, essere aperti e chiari, voler bene, piuttosto che forzare, imporre, svalutare, provocare. Non importa se si e' nel giusto. Le maniere forti allontanano.

Le maniere forti sono piu' facili. Quelli gentili sono piu' difficili, richiedono piu' tempo, ma sono, nel lungo andare, piu' proficue e soddisfacenti. Solo i deboli dicono che le maniere gentili e comprensive non danno risultati. Sopportare lo stress, non sfociando nell'aggressivita', mantenendo la comprensione dell'altro e' molto piu' difficile. 

La vita e' la ricerca costante del piacere e la minimizzazione del dolore. Quindi, chi parla il linguaggio del piacere, parlera' il linguaggio della vita. Non c'e' pericolo di sedurre falsamente con belle parole, perche' come dice Oscar Wilde nella favola dell'usignolo e la rosa\footnote{Oscar Wilde, ``Il principe felice e altri racconti''}, ``il vero amore e' silenzioso''. Bastano poche parole, in un lungo e piacevole silenzio.

Le cose belle, sono fiori in un campo verde e vastissimo. Questi fiori sono molto pochi, quasi non ce ne sono. Ma sono molto belli. Se, invece, ce la prendiamo con noi stessi o con gli altri o con il campo, e diciamo: ``qui non c'e' niente, corri, affaticati per trovare fiori'', si allora correndo e affaticandoci troveremo piu' prontamente altri fiori, ma avremo perso nel frattempo molto, il campo sara' diventato tutto giallognolo e la bellezza dei fiori non splendera' piu'.\\

La ricerca di Dio, consiste nel migliorare con pazienza e senza sforzo nel tempo, sia materialmente (lavoro, salute, ...), sia psicologicamente. I Sufisti affermano che si puo' raggiungere il paradiso in terra, che l'anima puo' avvicinarsi a Dio. Si puo' fare, lavorando incessantemente sul proprio Io, rimpicciolendolo dove e' sovrabbondate e portandolo ai suoi limiti dove e' carente, purificandolo da desideri e tendenze che lo allontanano dalla meta, che in realta', soffermandosi, puo' riconoscere di essere superflue. Tutto questo, al fine di amare Dio. \\

Chi fa da se', utilizza libri, pratica cio' che impara o pensa o crede, nel rispetto degli altri, nella continua ricerca, auto-critica. Fa' tesoro dei consigli, comodi o scomodi, di chi lo vuole bene.

Coglie spunti dai commenti, anche impliciti, degli altri. In ogni battuta, critica, complimento implicito o esplicito, c'e' qualcosa che l'altro comunica di come lui percepisce noi, di come sta' e di cosa sta' desiderando da noi. Molte volte, riflettendoci, obbiettivamente, si trovano ottimi spunti per migliorare nell'amare noi stessi, lui e gli altri.

Infine, e' importante ma non sufficiente, frequentare una o due\footnote{anche di piu', ma non troppe, altrimenti non credo si possa veramente frequentarle tutte} comunita': associazioni culturali, sportive, centri sociali, parrocchie. Un luogo dove si ci trova a proprio agio e dove si ci puo' slanciare con nuove sfide, e dove le attivita' sono la moneta con cui si interagisce con gli altri, dove cosi' nascono naturalmente relazioni interpersonali e si ha il privilegio e l'opportunita' di stare con propri simili. Solo amando anche gli altri si puo' migliorare nell'amare se stessi (e viceversa).\\

Oltre a tutto questo, per superare certi ostacoli, e' valido ricorrere sporadicamente o frequentemente a chi e' piu' esperto di noi: genitori, zii, nonni, filosofi, ministri di Dio (preti) o psichiatri/psicologi di professione.

Adesso parlero' di quest'ultimi. La ``Psicoterapia'' vuol dire, cura dell'anima, e riguarda terapie realizzate con strumenti psicologici quali il colloquio, l'analisi interiore, il gruppo, ecc., per cambiare quei processi psicologici che sono causa di un malessere o di uno stile di vita controproducente, e connotato spesso da sintomi come ansia, depressione, fobie, ecc. (tratto da Wikipedia).

La psicoterapia, non e' solo per chi con fatica raggiunge un livello di vita soddisfacente. E' strumento efficace anche se si vogliono raggiungere livelli di realizzazione superiori, se si vuole una vita piu' autentica e piena: il limite piu' difficile da superare per l'uomo e' lui stesso. E l'uomo puo' superarsi solo conoscendosi e amandosi. Ma farlo da soli richiede una vita intera. Accompagnati, forse la meta'.

Negli sport, atleti professionisti fanno un percorso di psicoterapia per superarsi. Ad esempio, ``Il tiro a volo è uno sport dove l’errore e' fatale e si entra in finale per un piattello in piu' o in meno. Anche un semplice battito di ciglia imprevisto, un pensiero che sfugge, l’emozione di un momento, possono rovinare una prestazione che sembrava perfetta.''\footnote{\urlOrig{https://www.igf-gestalt.it/wp-content/uploads/2013/07/Gestalt-Mental-Training-nel-Tiro-a-Volo-BERNARDI-tesi.pdf} ``Gestalt Mental Training nel Tiro a Volo. L'applicazione dei principi della Psicoterapia Gestalt nell'allenamento mentale con un atleta del tiro a volo''}.\\
Niccolo' Campriani campione di tiro a segno, dopo una delusione in un campionato in Cina e alcuni anni di empasse, ha superato dei suoi conflitti interiori con lo psicologo Edward Etzel, e raggiungendo un approccio diverso al tiro, piu' libero da suoi blocchi, ha vinto i campionati mondiali di Monaco nel 2010, le Olimpiadi nel 2012 e nel 2016.\\

Affrontare un percorso di crescita interiore, serve per qualsiasi fine. Ad esempio, migliorare e diventare piu' bravi nell'amore, nella famiglia, o nella scienza, nell'arte, nel lavoro, nella societa', nel proprio gruppo di amici. In generale, serve per migliorare. \\

Perche' migliorare? Non sono gia' abbastanza per quello che sono? \\
Se c'e' qualcosa che non mi piace della vita, il lavoro o la disoccupazione, il sesso o la mancanza di sesso, la solitudine o la vita mondana, il tradimento o l'essere legati ad un altro, il non essere riconosciuti, l'ingiustizia, ... Se c'e' qualcosa che ci fa soffrire, allora c'e' spazio di miglioramento.\\
Quando vediamo il male in qualcosa, in realta' stiamo proiettando una nostra sofferenza in una cosa esterna. Se ci prendessimo pienamente cura di noi stessi, non esisterebbe il male in niente. Anche quando vediamo altri soffrire, siamo anche noi che stiamo soffrendo, empaticamante. Se ci prendessimo cura di questa sofferenza, cercando di fare qualcosa che crediamo valida, fosse anche un preghiera, smetteremmo di soffrire, e a seconda della situazione e di come e' l'altro, anche lui guarirebbe un po' o molto.\\
Ma non siamo nati con il cervello gia' programmato\footnote{addirittura all'inizio i neonati non distiguono neanche le forme e, in pratica, e' per loro tutto un miscuglio psichedelico di cose mischiate fra loro. Poi la loro rete neurale, si evolve e comincia a creare forme, colori, etc...}. Amare se stessi e gli altri e' un arte che si impara strada facendo, e come ogni arte richiede tempo e dedizione. E' un'arte a scopo di lucro, che fa vivere meglio se stessi e gli altri. Per questo motivo ``migliorare'' e' importante.\\
I maestri orientali parlano di Pace, di Nirvana. Altri di Paradiso in Terra. Dimensioni raggiungibili tramite l'illuminazione interiore. Non c'e' niente di piu' importante che vivere pienamente, serenamente e piacevolemente, e condividere questo piacere con altri. I singoli piaceri che nella vita a volte cerchiamo con ossessione, sono come degli stuzzichini di un pranzo piu' grande e superbo.\\
Come si raggiunge? In principio, e' un gran casino. Ci possiamo fare male e romperci qualcosa, abbiamo paure, gli altri ci danno fastidio, siamo insoddisfatti e ce la prendiamo con gli altri di questo. Tuttavia, basterebbe ``camminare naturalmente e respirare''\footnote{Vedi Alexander Lowen, Il Piacere} per essere contenti e per essere in grado di rendere contenti gli altri. Essere felici della vita, qui ed ora, essere appagati di semplicemente stare respirando in buona salute, fisica e psichica, capire che questo e' il bene piu' grande e condividerlo tutto.  \\
Questo e' il percorso verso Dio. Si puo' fare da soli. Si puo' fare amando ed essendo amati dal proprio partner, amando ed essendo amati dai propri figli, etc...\\
Richiede, tempo, pazienza, tenacia e tutte le proprie capacita', anche quelle che non sappiamo di avere. E' un percorso epico, e tutte le imprese umane non sono che una metafora di questo percorso. \\

E' difficile sentirsi ``arrivati'' alla fine di un tale percorso. Per ogni passo, la strada si apre nuovamente con 1000 passi in piu' da poter percorrere, sognare, conquistare. Ma per quel poco che si e' gia' percorso, si ci sentira' molto contenti.

\subsection{Sull'istituzione religiosa}
La Chiesa e' la palestra dell'anima. Un luogo dove si parla di amore, e lo si pratica, allenandosi, in proprio e con gli altri dei vari gruppi offerti dalle parrocchie. Cosi' come le vere palestre, alcune piacciono di piu', altre piacciono di meno, ma cio' nonostante, non esiste luogo alternativo alla Chiesa dove si parla esplicitamente di Amore, inteso in senso non egoistico. 

Perche' la Chiesa cattolica e non un'altra? Le istituzioni, in generale, cosi' come ogni prodotto e servizio umano, sono imperfette. Per capirne le ragioni, basta pensare che gia' e' difficile per i ``capi di famiglia''\footnote{qui si da' una descrizione sul negativo per stressare le difficolta' dei vari ruoli} (genitori) gestire due o tre figli, e allo stesso modo e', in vari momenti, difficile per i figli sopportare i propri capi. Allora, sicuramente e' difficile per due o dieci persone gestire un'istituzione composta da centinaia di persone, che serve a sua volta migliaia di persone. Ed e' difficile per un singolo relazionarsi con la macchina che porta avanti l'istituzione, dato che per essa il singolo e' uno tra i migliaia dei suoi ``clienti'', ed ognuno ha sue preferenze, bisogni particolari, urgenze ed aspettative.

Tuttavia, le istituzioni sono necessarie, per potersi fidare di persone e ruoli ufficiali, standardizzati e certificati. Anche se questo sembra ledere la liberta' dell'individuo, fino a una certa misura cio' e' necessario per non cadere preda di truffe create da persone che, improvvisandosi, mettendosi un bel vestito e adoperando eccellenti strategie di marketing, poi, come sciacalli si avventano sulla fortuna delle persone senza curarsi della loro sorte\footnote{vedi, ad esempio, tutti i ``maghi'' che hanno truffato molte persone proponendo cure miracolose}.

Quindi, e' facile lamentarsi di un'istituzione, ma non si dovrebbe.

Quanto fin'ora detto vale per qualsiasi istituzione pubblica, e vale in particolare per la Chiesa Cattolica, o qualsiasi altra Chiesa del proprio paese d'origine.

Tutto quanto detto e' una visione macroscopica delle istituzioni, e della Chiesa. A livello microscopico, uno puo' cercare nel proprio quartiere, nei quartieri vicini, o nei paesi vicini una parrocchia abitata da una comunita' con cui uno si trova bene.

Nella mia esperienza, ho trovato nel gruppo del Rinnovamento nello Spirito Santo\footnote{\url{https://it.wikipedia.org/wiki/Rinnovamento\_nello\_Spirito\_Santo}} di una parrocchia consigliatami da una Suora amica di mia madre, un ottimo ambiente, molto al di la' dei pregiudizi comuni e personali sulla Chiesa.

Se si e' mossi da un autentico spirito di ricerca e di fratellanza con gli altri, e' molto facile mangiare e pregare come gli altri. Ogni rito, se eseguito con fede, e' espressione di cio' che l'uomo ha dentro di se', e non di cio' che l'uomo invanamente cerca fuori da se' (superstizione). E poco importano le sottigliezze come il dire se la Terra gira intorno al Sole o viceversa, quando cio' che si sta' insieme cercando e' molto piu' importante.

Inoltre, esistono persone religiose che hanno scelto di servire la Chiesa e di essere retribuite da essa per seguire una vocazione autentica. E tra esse molte sono persone in gamba, che nel mondo normale, se lo avessero seguito, avrebbero trovato successo.

Anche se non si ha modo di conoscerne molte di queste persone, una gia' basta per quantomeno rispettare un'intera istituzione.

Se poi, si riconosco i valori di un'istituzione, ma non si trovano persone valide, allora siamo noi stessi che dobbiamo scendere in campo per difenderli. Se, poi, riconosciamo che una regola o legge deve essere cambiata, allora dobbiamo prenderci la responsabilita' di promettere a noi stessi e agli altri che come pensiamo noi condurra' alla felicita' e alla serenita'. E quando le nostre promesse avranno portato danno, dobbiamo accogliere le colpe e le punizioni che ci verranno inflitte dalla vita e dagli altri. Tutto questo Gesu' l'ha fatto, e cosi' facendo ha cambiato la Chiesa dei suoi tempi. 

\section{Sulle leggi della religione}

Ogni legge di una religione ha un suo senso, se si e' disposti a comprenderlo. La cultura moderna occidentale, ha tanto criticato le leggi della religione, tuttavia se non si fa uno studio serio delle varie questioni, le critiche diventano superficiali e sfrontate, perche' ignorano il senso naturale e sano di tali leggi.

Se e' vero che le leggi hanno un senso, bisogna anche essere tolleranti e comprensivi. Infatti, rispettare anche una sola legge e' un grande traguardo, ma l'errore piu' grave e' imporre a se stessi o ad agli altri di raggiungere tale traguardo. Se si rispetta la legge con l'obbligo e non con un proprio senso di dovere, allora sara' peggio di non averla rispettata. La Pace della religione si raggiunge attraverso il libero arbitrio, la fiducia e impegnandosi, sbagliando e poi colto l'errore e le sue conseguenze, rialzandosi ed ancora con fiducia impegnandosi.

Beato chi ha mille volte peccato, ma che poi ha ascoltato la voce del Signore e ripreso con speranza e forza il cammino. Questo e' il senso di tutto il Vangelo, dove Gesu' scarta e denuncia con la sua missione la tradizione della Chiesa di Israele che era diventata sempre piu' rigida, formale e spietatamente dura, e non era piu' strumento di Dio per l'uomo, ma piuttosto strumento di schiavitu'.

\section{Amore in quantita' 0}

Una parte importante nell'arte dell'amare e' riconoscere, accontentarsi, apprezzare e, cosi', essere veramente contenti dell'``Amore in quantita' 0''.\\
\leavevmode\\
Se qualcuno ama, non significa che deve fare per forza qualcosa di buono, di utile, o di bello. Significa, prima di tutto che prova questo sentimento di amore. In formula, si potrebbe dire che ogni qualvolta ha a disposizione energie e risorse, le impieghera' per fare qualcosa di utile e piacevole. Tuttavia, siamo tutti limitati fisicamente e quindi, quello che qualcuno puo' fare concretamente e' ben poco, sia per se stesso sia per gli altri.\\
Allora, il suo sentimento di amore e i suoi sforzi e risultati valgono poco? No. E' cio' che nutre l'anima, se lo si apprezza, e lo si conserva e matura dentro di se'.\\

Se noi amiamo chi ama, e' sufficiente lasciarsi nutrire dal suo sentimento per essere veramente felici di Lui o di Lei.\\
E' chiaro che non di solo sentimento vive l'uomo. Ma, tutte le cose spicciole, come il lavoro e il cibo, derivano come conseguenza immediata di quel sentimento. E non viceversa.\\

``Dio si trova anche nel nulla, nel silenzio, nel non avere, nel non essere''.\\
Anzi, hanno scritto ``Dio si trova maggiormente nell'assenza''.\\

Si puo' venire incontro a noi stessi ed agli uomini ed alle donne che ci amano, tramite il ``grazie'' e, nei casi piu' difficili, tramire il ``sacrificio''. Ringraziare e' apprezzare quello che qualcuno, qualcuna o alcuni sono per noi, compresi noi stessi, e in questo apprezzamento, riconoscendo che cio' che viene dato e' un dono, e che non viene dato di piu' non per cattiveria!, rimanere appagati di cio' che esiste. Sacrificarsi e' la forma piu' forte e difficile del ringraziare, in quanto cio' che si riceve non e' sufficiente per soddisfare i bisogni, e nonostante cio', si rimane comunque grati per cio' che gli altri e noi siamo. 

Nel paragrafo \ref{loZero} a pagina \pageref{loZero}, si spiega matematicamente, usando il Dilemma del Prigioniero, che esistono situazioni in cui e' necessario un proprio sacrificio personale.

\subsubsection{L'infinito}

Se i nostri desideri o bisogni sono 
\begin{itemize}
    \item autentici,
    \item necessari o, se non necessari, non superflui,
\end{itemize}
allora Dio li soddisfera', anche se richiedono di spostare la Luna, perche' Dio ti ama e i tuoi bisogni e desideri sono per Lui importanti. Cio' avverra' senza richiedere sforzo, senza forzare nessuno, e senza sovvertire alcunche' nella natura. Avverra' in maniera piacevole, danzerai e tutto danzera' con gioia e sicurezza, e non ti ricorderai neanche del fatto che avevi qualcosa che non avevi, non esisteranno piu' desideri e bisogni.\\

Analizziamo nel dettaglio quanto detto.\\

In generale, per parlare di quello che vorremmo o di cui abbiamo bisogno, e' a volte piu' naturale parlarne senza considerare limiti, dettagli, considerando la cosa realizzabile o gia' realizzata. In matematica, questo e' un modo di fare molto ricorrente e importante\footnote{Ad esempio, nell'algebra si dice ``sia $x$ la soluzione'', e poi si tratta $x$ come se fosse gia' concreta e disponibile. Oppure, sia $A=\{a \;|\;P(a)\}$ l'insieme tale che valga una certa proprieta' $P(a)$, e poi, dimostrato che $A$ puo' esistere ($A\neq \emptyset$), si tratta $A$ come se si conoscessero tutti i suoi elementi. Infine, la logica usata in genere non e' costruttiva, cioe' si assume $A \lor \lnot A$ come vero, e si permettono dimostrazioni che assicurano l'esistenza ma non danno la soluzione.}.

Fatto cio', sentito e maturato a fondo il desiderio (o bisogno), se ci sentiamo, mettiamoci in gioco, apriamoci agli altri e fidiamoci, adoperiamo e apprezziamo la realta' per come e'. E, punto fondamentale, consideriamo che il ``fallimento'', il fatto che il desiderio puo' non realizzarsi, fa' parte del normale sviluppo del nostro desiderio stesso. 

Il fatto che non si realizza, non e' perche' la Natura o gli altri, conoscenti o estranei, si oppongono ad esso, piuttosto e' perche' nel nostro desiderio non abbiamo incluso i loro desideri. Non abbiamo incluso il desiderio di quel masso di stare dove e', e cosi' sembra che ostacola la strada, non abbiamo incluso il desiderio di quella persona in macchina di andare in fretta da chi ama o di fuggire dalla realta' oppressiva che sta' vivendo, e cosi' sembra che quando non ci viene facile evitare la sua macchina o sopportare la sua aggressivita' col clacson, sta' ostacolando la nostra vita.

Ancora, molte volte non si realizza perche' nella nostra volonta', non abbiamo incluso noi stessi, e quindi la cosa ci va' male, perche' non desideriamo cio' che vogliamo, perche' il nostro desiderio e' altro, o perche' la nostra volonta' non ancora comprende e copre pienamente il nostro desiderio.

Quindi, avendo sentito e maturato nell'intimo e nella sincerita' il nostro desiderio, accolto e compreso il desiderio altrui, non opponendosi, vivendo e interagendo col ``desiderio'' della materia e della Natura, allora senza forzare ne' noi stessi, ne' gli altri, il desiderio si realizzera' certamente e in maniera grande. Perche', cio' che si realizzera' e' il Suo desiderio, che non conosce ostacoli e di cui tutto l'Universo gioisce. 

A volte, realizzare il desiderio e' molto faticoso. Noi siamo esseri finiti con risorse limitate. Tuttavia, se mettiamo da parte il nostro ego, che dice ``io non ci riesco'' oppure ``io ci \emph{devo} riuscire, a \emph{tutti i costi}'', allora quello che rimane e' il desiderio. E' piacevole poi mettersi in moto, con le proprie forze per realizzare, per creare, per mantenere. Se risultera' impossibile raggiungere l'obbiettivo, sentiremo tuttavia che l'obbiettivo non e' stato trascurato, e saremo comunque sereni. E se l'obbiettivo e' forte e necessario, ci fermeremo in pausa solo quando subentrera' la stanchezza e i limiti dati dalla fatica. E poi riprenderemo, riposati.

Il rischio di un incidente che blocchera' il lavoro definitivamente e' sempre possibile, in maniera per quanto piccola, ma se il bisogno e' primario, allora e' meglio correrlo, piuttosto che certamente abbandonarlo, e morire nello spirito.

Anche se le condizioni potrebbero essere molto faticose, realizzare il Suo desiderio sara' quindi senza oppressione, sforzo, senza dolore\footnote{volendo essere realisti, ``dolore'' sarebbe ``dolore insopportabile''}.


\section{La Natura}
La pace e l'unita' con la Natura, si ottiene capendo che la materia e' amata da Dio e che noi siamo un atomo tra gli atomi.

Se non ci fossimo noi esseri umani, l'universo sarebbe piu' triste, monotono o caotico. La materia infatti non ha anima. Non prova sensazioni, emozioni, ne' sentimenti ne' empatia. Non ha un cuore, non desidera il bene, ne' il male.

A che serve allora? Credo, la questione si risolva con questa domanda: e' possibile scrivere senza una penna e una carta, o senza una macchina da scrivere e un supporto che memorizza cio' che e' scritto?

La materia serve a noi per dare forma ai nostri sentimenti, per mantenere nel tempo cio' che e' importante.

Certo, la materia e' costosa. Trasportare pochi litri d'acqua e' faticoso. Il nostro corpo e' delicato, ha sempre bisogno di cure e puo' poco fisicamente rispetto a quanto noi a volte desideriamo.

Tuttavia, se non esistesse il peso, non esisterebbe il camminare, l'avvicinarsi o l'allontanarsi da chi ci piace o non ci piace. Se non esistesse la distanza, tutto sarebbe fuso in una massa informe. Se non esistesse il tempo, non potremmo vivere momenti belli, e ricercarne di altri.

E' chiaro che magari uno si puo' mettere alla ricerca di universi migliori, dove la forza di gravita' e' un po' meno faticosa, dove si ci puo' trasferire da un posto all'altro piu' facilmente (magari con un teletrasporto).

Eppure, cio' non basterebbe a raggiungere la felicita'. Cosi' come le cardinalita' di Cantor, in cui esiste sempre un infinito piu' grande, in ogni universo in cui ci troveremo, desidereremo un universo migliore, piu' performante.

Ma non e' svalutando il proprio universo e ricercandone uno migliore che si raggiunge la felicita'.

Si puo' fare molto nel proprio universo, ambiente e cultura, qualsiasi essa sia. Si puo' diventare molto bravi.

E, solo diventando bravi nel proprio universo, poi nel tempo, con la ricerca si possono trovare universi migliori.

Per confermare cio', consideriamo che molte scoperte e innovazioni scientifiche sono state dettate da innovazioni nella tecnica. Se il commercio dei materiali non fosse stato maturo, e gli artigiani non avessero raggiunto un buon livello di abilita' nella lavorazione di quei materiali, non ci sarebbero stati i salti tecnologici e senza questi non ci sarebbero stati i salti nella scienza\footnote{ad esempio, con l'avvento dei telescopi e dei microscopi. \url{https://www.google.com/search?channel=fs&client=ubuntu\&q=How+has+technology+helped+in+the+advancement+of+scientific+discoveries\%3F}}

Cio' vale anche nel proprio piccolo. Ad esempio, e' solo diventando bravi nel proprio lavoro o ruolo che si puo' poi ambire a lavori ``migliori''. E' solo apprezzando il meglio della propria terra, che si possono apprezzare altri paesi. E' solo amando i propri genitori che si puo' essere genitori migliori (o al pari) di loro, anche se loro hanno avuto carenze nell'esserlo.

Infine, una nota molto filosofica/scientifica: perche' l'universo in cui stiamo rispetta esattamente delle regolarita' e queste e non altre regolarita'? Una prima risposta e' il ``principio antropico'' (vedi wikipedia). La domanda a seguire e', anche se il principio antropico e' da rispettare, perche' ci sono queste regolarita' e non altre, che tuttavia consentirebbero la vita? In realta', sembrerebbe che le leggi dell'universo non sono costanti nello spazio, vedi \footnote{Variazione nello spazio della costante alpha di fine struttura\url{https://en.wikipedia.org/wiki/Fine-structure\_constant\#Spatial\_variation\_\%E2\%80\%93\_Australian\_dipole}}. Ad ogni modo, la materia esiste, ed esiste come le pare e piace, fino a quando vuole\footnote{questa affermazione e' coerente con l'approccio scientifico basato sulle osservazioni: anche se una legge scientifica stabilisce che anche domani il Sole sorgera', nulla proibisce alla Natura di non far accadere cio'. Semplicemente, gli scienziati dovranno poi prendere atto che il Sole non e' sorto (e' diventato all'istante di pietra?)}.
L'uomo, e' materia che si evolve secondo le leggi fisiche. Tuttavia, anche se tutti i nostri pensieri, sensazioni ed emozioni, sono il risultato di una macchina che e' il nostro corpo, non siamo una briciola nell'universo. Siamo cio' che da' senso ad ogni cosa. Anche perche', gli atomi non hanno il concetto di grande o piccolo. Il Sole non e' per un atomo tanto piu' grande del suo naso, ad esempio. La numerosita' delle stelle, non e' per la pietra ferma-carte sopra la scrivania, tanto piu' sorprendente del numero dei suoi spigoli. Che un vulcano esploda o crolli, non e' per le rocce che lo componevano e che vengono distrutte tanto piu' significativo di essere rimaste a formare il cono del vulcano per centinaia di migliaia di anni.\\

In conclusione, la materia e' l'inchiostro e la carta con cui Dio scrive il libro della Natura. Noi siamo le parole della Sua storia. A volte, essere una parola piuttosto che un'altra e' piu' difficile o, spiacevole, a volte e' piu' piacevole e straordinario. Ma solo essendo la parola che Lui vuole, nelle Sue frasi, la natura, gli altri e noi stessi non saranno ostili, incomprensibili e aridi e l'universo diventera' una armoniosa danza della materia.


\section{L'unita'}
Come spiegato all'inizio del capitolo, se amiamo siamo una cosa nuova che, pur contenente piu' parti, rimane una. Esiste un'unica Anima.\\
Come spiegato in Appendice A \ref{menteCrea} pag. \pageref{menteCrea}, la natura, lo spazio e le cose sono parte di noi stessi.\\
Allora esiste un'unica cosa, un unico Essere.\\

Che sia di sesso maschile, femminile, che sia un'animale o una cosa, poco importa. D'altronde, Esso, Ella o Egli e' tutto cio' che e'. E se prevale il Suo desiderio di amore, tutto cio' che e' e' amore.

\section{Dall'esterno verso l'interno}
\label{extInt}

Con serenita', lasciamo per un momento ogni proposito come se fosse realizzato, e osserviamo tutto cio' che puo' essere osservato e con cui, sempre nell'immaginazione possiamo interagire.\\
Questo tutto e' una cosa sola. ``Tutto'' include ogni cosa, quindi, non c'e' niente oltre al tutto.\\
Questa cosa, nel complesso, cambia. \emph{Prima} e' in un modo, \emph{dopo} e' in un altro modo. Nella piu' assoluta generalita', possiamo immaginare come se cambiasse di colore. Ogni volta che cambia, diciamo che e' passato del \emph{tempo}.\\
Cambia da colore a colore, non a caso, ma tracciando un disegno, seguendo una regolarita'. \\

La prima regolarita' e' questa: il tutto e' divisibile, si puo' pensare composto da una molteplicita' di elementi indivisibili: gli atomi\footnote{o piu' precisamente, le \emph{particelle}}.\\
Quanti sono? Sono molti o pochi? Non c'e' motivo di dire ``pochi'' o ``molti'', quanti atomi sono  e' semplicemente un numero: $10^{80}$, 10 elevato ad 80, ovvero 1 seguito da 80 zeri, cioe' cento milioni di miliardi di miliardi di miliardi di miliardi di miliardi di miliardi di miliardi di miliardi (nota \footnote{\url{https://en.wikipedia.org/wiki/Observable\_universe} In realta', se si considerano le particelle subatomiche, il numero e' maggiore. Considerando solo per particelle con massa, se consideriamo il piu' grande atomo, l'Oganesson, che ha 118 protoni, 118 elettroni, 176 neutroni, e quindi 412 sotto-particelle, e, nel caso peggiore dicessimo che tutti gli atomi dell'universo sono atomi di Oganesson, si avrebbero $10^{80}\times 412 \simeq 10^{83}$, 10 alla 83 particelle. Quindi, il numero preciso sara' tra $10^{80}$ e $10^{83}$. I fisici sapranno sicuramente dire una stima piu' precisa.}). \\
Ora questo numero non e' ne' tanto spaventevole, ne' tanto affascinante. Infatti, la Natura non ha il concetto di limiti, di poco o molto. Siamo noi, nella nostra esperienza che diamo una proporzione alle cose, e diciamo ``quella scala ha molti gradini, sara' faticosa salirla''. \\
La Natura non ha mai pensato: voglio tantissimi atomi per fare qualcosa di bello, oppure di opprimente.\\
Concludiamo il discorso dicendo che il numero di atomi e' stato calcolato con metodi ed esperimenti molto raffinati, basati sulla traiettoria delle stelle, delle galassie, sul colore delle stelle, e tanti altri parametri.\\

La seconda regolarita' e' che ogni atomo ha una \emph{posizione} e una \emph{velocita'}. In genere pensiamo a qualcosa \emph{posto} dentro \emph{qualcosa}. Cio' in cui sono posti gli atomi, sia chiamato \emph{spazio}. Se l'atomo si trova in una posizione $P$, la velocita' e' la sua intenzione (o l'intenzione della Natura), di essere in una posizione $Q$ nel tempo successivo. Ad esempio, se $P=0m$, punto di partenza, e dopo un secondo si trova a $Q=1m$, un metro di distanza dal punto di partenza, allora la sua velocita' e' stata di un metro al secondo: $1 m/s$.

Se consideriamo le posizioni e le velocita' di tutti gli atomi come un singolo oggetto (matematico), chiamato \emph{configurazione nello spazio di fase}\footnote{\url{https://en.wikipedia.org/wiki/Phase\_space}}, e assegnamo un colore distinto ad ogni configurazione distinta, allora capiamo la prima frase ``...esiste solo una cosa. Questa cosa cambia di colore in colore...''.\\

Infine, esistono altre poche regolarita': la forza nucleare forte, debole, la gravita' e l'elettromagnetismo. Che descrivono come si muoveranno gli atomi nel tempo. Ad esempio, se due atomi carichi positivamente sono vicini, allora si allontaneranno.\\

E noi? Siamo come le stelle: formazioni naturali, spostanee. Rispetto alle stelle, eccelliamo in complessita': abbiamo milioni di cellule, di diversi tipi, che gia' di per se' complesse, formano organi ancora piu' complessi, ognuno che soddisfa una regolarita': il cuore pompa il sangue, il fegato lo purifica, i polmoni assorbono ossigeno, etc...\\

La cosa piu' difficile da accettare e' questa: non c'e' niente oltre a noi stessi che dirige o vuole (per lo piu' inconsciamente) la sintonia con cui lavorano i nostri organi e gli organi di chi amiamo. Se non amiamo, \emph{la vita e' materia che si aggrega, trasforma, disgrega continuamente, meccanicamente, senza alcuno scopo o volonta'}. Se amiamo, se facciamo nostro l'amore e lo coltiviamo, la vita diventa poesia, piacere, a volte dolore -per il forte desiderio del piacere fin'ora vissuto e che sembra svanito-, combattimento per il ristabilimento del piacere, e ancora poesia. \\

Cosi', se ami decidi di che fartene di tutti quei $10^{80}$ atomi e di tutte quelle cellule, quel sangue, muscoli e nervi che formano le anime di chi ami (compreso tu stesso / te stessa)\footnote{l'anima e' vista non come cosa incorporea, ma come un tutt'uno con il corpo. Non c'e' anima senza corpo, ma anche corpo senza anima! Il corpo e' in se' una cosa morta, e' l'anima e' cio' che lo rende vivo. Ma, in vita, l'anima, non puo' essere intesa come distaccata, separata, anche di un solo atomo, dal corpo}, e al contempo, allo stesso modo e' chi ti ama che decide per loro. Se e' ``amore'' nel senso classico, allora queste decisioni portarenno al Bene. Gli effetti di queste decisioni, saranno sottili ed invisibili, ma saranno cio' che alimentera' veramente la Tua vita.\\

In questo Universo d'amore, allora, alzandoti presto la mattina, potrai dire ``sorgi Sole, riscalda le nostre membra'', e una intensissima fusione nucleare illuminera' il cielo con raggi di luce che riscalderanno senza bruciare.\\

Infine, dov'e' Dio in tutta questa collezione di atomi? 

Lo Spirito e' quel processo bio-psico-fisico del nostro corpo e della nostra psiche, ``dentro'' ognuno di noi, che inizialmente e' piccolo e poi pian piano contagia tutto il nostro essere, e che ci dice: ``Io ti amo, ti daro' tutto cio' che e' realizzabile in questo Universo, con il tuo corpo e con il corpo di chi ti ama'', e che, inseparabilmente dice: ``Io amo gli altri, e ogni volta che mi ami, daro' loro tutto cio' che potro' realizzare con il tuo corpo e con il corpo di chi mi ama''. E ancora: ``E tutto questo lo faro' senza togliere un solo granello dalla tua riserva di felicita', per quanti sacrifici il mio amore possa comportarti''.

\subsection{Osservazioni scientifiche filosofiche}
\label{ossScientificheFilosofiche}

Se Dio e' dentro di noi, allora essendo noi il risultato di processi elettrochimici, anch'esso e' tale. Tuttavia, con questa osservazione, si rischia di perdere il punto del discorso. Infatti, 
\begin{enumerate}
    \item Nessuno si lamenta del fatto che quando parliamo, mangiamo o guidiamo, tutto quello che succede e' il risultato di ``nostri''\footnote{la parola nostri e' tra virgolette, perche' non sono solo nostri. Cio' viene spiegato tra poco.} processi elettrochimici.
    \item Continuando il ragionamento, allora, nasce una nuova osservazione: nulla e niente avrebbe valore, dato che tutto e' materia e tutto e' il risultato di processi elettrochimici. 

        Ma se cosi' e' vero, che valore ha questa osservazione? Non e' essa stessa un suono\footnote{mi riferisco alla osservazione come suono, immaginandola detta a viva voce} che viene prodotto da un ``computer'' fatto di neuroni?

        Una filosofia nichilistica, non aggiunge valore alla vita. Sembra che porti liberta' perche' libera i cuori dal peso di amare e di rispettare i valori umani, ma non e' buttando l'acqua appena presa dal pozzo che diventa piu' leggero trasportare il secchio fino a casa.

        Se veramente riconducessimo tutto ai meccanismi della materia e disconoscessimo l'esistenza metafisica del cuore che e' in noi, non dovremmo mai obbiettare o preferire nulla. 
Questo potrebbe forse renderci molto potenti, e renderci delle macchine perfette, che riescono in ogni obbiettivo, ma in verita', cosi' come un'automobile non ha un'anima che desidera dove farla andare, non avremmo obbiettivi. Non avremmo neanche obbiettivi mondani come la massima ricchezza, o il massimo piacere edonistico, o il divertimento esagerato ed incontrollato. Neanche l'obbiettivo di stare bene, non provare dolore e vedersi belli allo specchio. Saremmo materia tra la materia.

        Nessun essere e' cosi'\footnote{tranne forse chi soffre di depressione}. Quindi, se poniamo questi ragionamenti meccanicistici o nichilistici e diciamo che nulla ha senso e niente esiste, in realta', stiamo barando. Qualche obbiettivo lo abbiamo. Forse, neanche noi sappiamo di averlo. E se imparassimo a conoscere i nostri sentimenti, capiremmo di avere degli obbiettivi bellissimi (sogni), ed anche troppi per essere raggiunti in una intera vita.
\end{enumerate}

Ritornando in ottica positiva, se qualcosa ha valore nella vita, possiamo dire che nasce dentro di noi. Inoltre, se noi amiamo, i ``nostri'' processi elettrochimici, non sono solo nostri, ma sono anche di chi amiamo. E i processi elettrochimici degli altri hanno importanza per noi: Dio e' dentro \emph{ognuno} di noi.

Dato che siamo esseri imperfetti e limitati, quale parte di noi e' Dio e in che modo Dio nasca ed opera in noi e' interessante, e si potra' esplorare meglio nel paragrafo sucessivo che parla dell'anima e dell'animo (spirito).

Forse, anche se in verita' e' superfluo, e' opportuno spendere qualche parola sulla potenza di Dio e sui miracoli. Il punto fondamentale, e' capire non tanto se Dio sovverte la Natura, ma se Dio e' in grado di condurre alla vera vita, alla Pace, alla vera estasi perenne. In questa prospettiva, Dio opera veramente miracoli. Il modo, che sia fatto consciamente o senza neanche saperlo, l'ho gia' descritto nei paragrafi precedenti, per quanto ho potuto. Putroppo, cosi' come chi non ha fatto mai sport e' difficile immaginare i cambiamenti nella sua vita facendo attivita' sportiva, allo stesso modo il lettore che non ha esperienza non potra' immaginare realmente quanto ho scritto. Tuttavia, la mia e' sia una testimonianza, sia un invito a mettersi in cammino, e toccare con mano la verita' di quanto ho detto.

\section{Procedimento assiomatico}
\label{procAx}

In questo paragrafo, procederemo in maniera formale, per dare una definizione di ``Dio''.

In termini poetici e non formali, diciamo che ``Dio'' e' colui che ama te stesso, e che ama ogni altro, dando sostanza e forma a te ed agli altri, tramite le particelle della Natura, ed ama anche la Natura. E' colui che da' forza a te e ad agli altri affinche' ognuno procuri il piacere in ognuno, ed allontani il dolore, senza venire meno alle leggi dell'Universo, che Lui ha eternamente promesso alla Natura. Qualunque proprieta' positiva e che conduce alla Vita, e' Sua: saggezza, grazia, verita', grandezza, bellezza.

Per chi lo ama, non esistono forze e fenomeni che non rispettino o accadino secondo la Sua volonta', ogni cosa esistente e' manifestazione della Sua volonta'. Per chi lo ama, nulla lo turba e inquieta. Ogni cosa della Natura e', semplicemente od intelligentemente, una risorsa od una bellezza. Chi lo ama, per amare chi ha bisogno di Lui mette in moto e consuma ogni parte del suo essere.

Per finire la visione del concetto di Dio, diciamo che Lui conduce ad un soddisfacimento e serenita' autentica degli esseri che ama. Questa ``Pace'', e' slegata da cio' che erroniamente, nei nostri limiti umani, pensiamo o che gli altri, di cui ci avvaliamo, pensano che sia. I bisogni e i desideri che soddisfa sono i bisogni veri, naturali, essenziali o non superflui.\\

Adesso entriamo piu' nel dettaglio.

Cio' che definisce, consciamente o inconsciamente, cio' che e' piacere e cio' che e' non-piacere o dolore, sia chiamato ``anima''.\\
Cio' che desidera, consciamente o incosciamente, che l'anima provi piacere e non dolore, sia chiamato ``animo'' o ``spirito''.\\
Cio' che accoglie, amministra e spende l'anima per realizzare il desiderio dell'animo e che valuta i risultati di tale desiderio, sia chiamato ``Io'' (nota \footnote{o ``Noi'', se ci sono piu' di un Io che sono in unione tra loro}).\\

Assioma:
\begin{enumerate}
    \item l'anima e' infinita
\end{enumerate}

Diremo che un'anima e' ``empatica'' quando essa, come parte infinita di se stessa, include l'anima altrui. E quindi, il piacere e il dolore altrui e' piacere e dolore suo. \\
Osservazione: questa definizione non e' simmetrica! Se un'anima A e' empatica verso un'anima B, non e' detto che B sia empatica verso A. A potrebbe amare B senza essere amata da B (e cio' sarebbe comunque buono per A, se ama B). \\
Osservazione: un anima A empatica verso un'anima B, include B come sua sottoparte, tuttavia, ancor meglio si deve presupporre che A smette di essere A e che diventa una nuova entita' che comprende come sottoparti il \emph{se'} e l'\emph{altro}. Questa nuova entita' si puo' indicare con A+B. \\

Sia definito ``amore'' di un essere A verso un essere B quando l'anima di A e' empatica con B, quando l'animo di A desidera che A+B provi piacere e non dolore, e quando l'Io di $A$ spende saggiamente l'anima A+B per realizzare tale desiderio. Per la precisione, non si puo' parlare di A che ama B, ma piuttosto di A+B che ama B. 

In generale, se un essere $A$ ama molteplici esseri, scriveremo l'unione $\mathcal{A}=A_1+A_2+\cdots$, dove ogni $A_i$ e' un essere dell'unione.

\def\self{\textrm{self}}
\def\other{\textrm{other}}

Se $A$ ama $B$, ma non vicersa, si hanno due unioni, quella di $A$ e quella di $B$ e vale $\mathcal{A} \supset \mathcal{B}$ (nota\footnote{E' l'inclusione insiemistica, $\mathcal{B}$ e' incluso in $\mathcal{A}$ quando ogni elemento di $\mathcal{B}$ e' un elemento di $\mathcal{A}$}).  Ovvero, cio' che $B$ ama lo ama $A$, ma non viceversa, e quindi $\mathcal{A} \neq \mathcal{B}$. Se $B$ ama pure $A$, allora ogni unione contiene l'altra e $\mathcal{A} = \mathcal{B}$ e quindi si parlera' di un'unica anima. 

In questo caso, la loro anima e' distinta ma equivalente.  Ognuno pero' la vive dal proprio punto di vista. Dal punto di vista di $A$, l'anima la indichiamo con $\mathcal{A}_A$ e definiamo $\textrm{self}(\mathcal{A}_A)=A$ come il \emph{se'} di $\mathcal{A}$, mentre un qualsiasi essere dell'unione e' chiamato \emph{altro} per $A$. Viceversa, dal punto di vista di $B$, si ha $\textrm{self}(\mathcal{A}_B)=B$. Nota, poiche' $\mathcal{A}=\mathcal{B}$, e' indifferente se usare $\mathcal{A}_B$ oppure $\mathcal{B}_B$ per indicare l'anima dal punto di vista di $B$.

Molte volte parleremo solo di $A+B$, ma quanto detto si puo' estendere anche ad una unione qualsiasi $A_1+A_2+\cdots$. \\

\def\des{\textrm{des}}

Definiamo l'insieme $\des(A_i)$ dei \emph{desideri} di un essere $A_i$. I desideri li possiamo suddividere in desideri forti o deboli, e in superflui e non superflui. I desideri forti non superflui, li definiamo come \emph{bisogni}.

\def\anima#1{\mathcal{#1}}

\def\spirit#1{\textrm{spirit}(#1)}
\def\Animo#1{\spirit{\anima{#1}}}

L'animo $\spirit{A_i}$ e' la personificazione di $\des(A_i)$. E' un entita' che esiste fino a quando i desideri $\des(A_i)$ sono soddisfatti. L'Animo $\Animo{A}$, inteso come l'unione degli animi $\spirit{A_1}$, $\spirit{A_2}$, $\cdots$,  desidera che tutta e sola l'unione dei desideri di ogni $A_i$ sia soddisfatta, dando piu' priorita' ai bisogni ed ai desideri che lui considera non superflui che agli altri desideri.

Un Io $A_i$ ama $\anima{A}$ quando spende $\anima{A}$, affiche' questi desideri siano soddisfatti, o detto in altro modo, affinche' l'Animo $\Animo{A}$ sia soddisfatto.\\

Dobbiamo aggiungere tra gli esseri, un essere fittizio: N, la Natura.

Un essere $\anima{A}$ ama N quando 1. ogni sua anima e' in pace con la Natura, ovvero quando ogni $A_i$ percepisce tramite i sensi lo spazio, le forme, i suoni, ... e trova le sue percezioni come buone. Questo significa che $A_i$ e' in buona salute fisica e psichica! Questo non vuol dire che chi non e' in salute non e' in pace con la Natura. Amare la Natura significa tendere con tutte le forze per raggiungere il massimo della salute consentito dalla Natura. 2. quando almeno un $a$ di $\anima{A}$ si spende per $A_1+A_2+\cdots+N$, ovvero quando $a$ si muove nello spazio, adopera la sua energia per mantenere il corpo suo e di chi $\anima{A}$ ama. Nell'atto pratico, quando $a$ lavora per se' e per gli altri, sia in lavoretti quotidiani come cucinare, sia in progetti piu' grandi e specializzati come lavori per una ditta\footnote{o se $a$ fosse autosufficiente, quando $a$ cura un orto, un allevamento, etc...}, e quando $a$ mangia, beve, dorme e fa' mangiare, bere, dormire. \\


\subsection{Sulla ricorsione}

Questa e' una nota esclusivamente matematica che puo' essere saltata da chi non ne trova utilita'.\\

$\anima{A}_1=A_1+A_2$ e' un'anima che, amando se stessa, include essa stessa nell'unione. $A_1$ non e' diverso da $\anima{A}_1$, sono esattamente la stessa cosa. In matematica, cio' non e' strano, e si chiama ricorsione. 

$A_1=\anima{A}_1=A_1+A_2$, quindi $A_1=A_1+A_2$, quindi $A_1=A_1+A_2+A_2$, quindi $A_1=A_1+A_2+A_2+A_2$, etc...

Se consideriamo il $+$ come unione insiemistica $\cup$, allora $A_1=A_1+A_2+A_2+\cdots = A_1+A_2$. L'affermazione $A_1=A_1+A_2$ e' equivalente insiemisticamente all'affermazione $A_1 \supseteq A_2$.

Quindi, se diciamo che $A_1$ ama $A_2$ e quindi $A_1=A_1+A_2$, stiamo dicendo che $A_2$ e' una sua infinita parte, di lei stessa che e' infinita.

\subsection{Sull'unicita'}

L'anima e' una, anche se e' empatica ed ama altre anime. Come gia' detto, l'anima che ama altre anime e' una entita' distinta da come era prima quando non le amava. Ella e' uguale all'unione delle varie anime. 

All'apparenza se l'essere A ama B, e la sua anima diventa $A+B$, sembrerebbe che comunque $A+B$ risieda nel corpo di $A$, dato che l'anima e' il risultato di processi neurologici. Tuttavia, il corpo di A non e' piu' sufficiente per descrivere A+B. La percezione del dolore o del piacere di A+B, deriva dai sensi del corpo di A e del corpo di B. Se A+B ama B, allora, l'anima A+B desidera' il bene di B, ovvero desiderera' che B percepisca quanto piu' piacere e quanto meno dolore, che siano queste percezioni ``fisiche'' o ``psicologiche'' (es. emozioni). 

$A$ percepisce cio' che  $B$ percepisce tramite il vedere, sentire, ascoltare $B$, tramite il stare con $B$, tramite il parlare con $B$, tramite l'empatia.

Se, a fin di bene, $A+B$ ``possedesse'' il corpo di $B$ per percepire cio' che $B$ percepisce senza che $B$ lo desideri, parleremmo non di amore, ma di possesso\footnote{``possesso'' nel senso generico di violenza psicologica (es. stalking) o fisica}. Questo fa' capire che l'anima dell'essere $B$ deve essere concorde con l'anima di $A$, deve trarre quanto piu' piacere e quanto meno dolore dall'amore di $A$. Allora, $A+B$ vuol dire veramente l'unione dell'anima dell'essere $A$ e dell'anima dell'essere $B$. $B$ condivide ad $A$ le sue percezioni (e' aperto), e $A$ le elabora per $A+B$. 

Per questo, $A+B$ risiede fisicamente sia in $A$ sia in $B$. L'amore di $A+B$ verso $B$ non puo' sussistere senza $B$. Inoltre, si comincia a capire che se $B$ e' empatico e la sua unione e' ad esempio, $B+C$, allora $A+B+C$ e' l'unione risultante dall'unione $A+B$.

Matematicamente, sia $\mathcal{A}$ l'unione di $A$, cioe' $\self(\anima{A})=A$ e sia $B$ parte dell'unione. Sia $\mathcal{B}$ l'unione di $B$, ($\self(\anima{B})=B$). Se $\anima{A}$ e' amore, allora, $\mathcal{A} \supseteq \mathcal{B}$, ovvero, l'unione $\anima{A}$ deve includere almeno l'unione di $B$. Quindi, ogni essere parte di $\mathcal{B}$ e' parte di $\mathcal{A}$.

Un esempio di inclusione propria, in cui non vale l'uguaglianza, e' quando $A$ ama $B$ mentre $B$ non ama $A$, e si ha $\mathcal{A}=A+B+C+D$, mentre $\mathcal{B}=B+C$. In questo esempio, $D$ ed $A$ stesso sono amati da $A$ ma non da $B$.\\

Anche se e' necessaria un'interazione fisica, l'anima non automaticamente ama gli altri solo per il mero fatto che gli altri esistono, vivono fisicamente ``fuori'' dal suo se' e stanno bene perche' se la cavano da soli. L'anima ama quando interiorizzando gli altri, fa' diventare loro parti di se stessa e quando ama queste parti al pari di come ama se stessa.

Per questo motivo, per l'animo e' sufficente che ami la sua stessa anima e non altre anime ``fuori''. Amando la sua anima, ama lei stessa e tutte le altre parti dell'unione che crea la sua anima.

Se l'essere $\anima{A}_i=A_1+A_2+\cdots$ e' pero' innamorato in maniera sana di un essere $A_j$, con anima $\anima{A}_j$, e quindi di un anima apparentemente fuori dal suo se' ($A_j\ne A_i$), per l'animo $A_i$ e' sufficiente amare solo l'anima di cui e' innamorato ($A_j$). Infatti, se l'anima $\anima{A}_j$ ama cio' che $\anima{A}_i$ ama, allora $A_i$ amando $\anima{A}_j$ sta' amando $\anima{A}_i$.

Quindi, per l'animo $A_i$ e' sufficiente amare $A_j$ per amare tutta $\anima{A}_i$, inclusa lei stessa ($A_i$) e tutti gli altri esseri.

Per un animo, esiste quindi solo la sua anima, o quella di cui e' innamorato. Non serve, anzi e' fuorviante, considerare l'esistenza di altre anime.

In conclusione, possiamo dire che per un animo esiste solo l'\emph{Anima}.

Esempio: chi e' innamorato di Gesu', dice: Gesu' e solo Gesu' e' Dio. E se vuole fare piacere a qualcuno lo fara' perche' facendolo fara' piacere a Gesu', e non per altro fine. 

Allo stesso modo, chi e' innamorato di un altro essere, vedra' nell'anima dell'essere la sua Anima, e dira' cose simili.

Gli animi possono essere molteplici. Ma dato che un animo che ama la sua Anima si allea solo con animi che la amano, l'insieme di tutti gli animi che la amano agira' coerentemente e concordemente, ognuno nei suoi limiti, e percio' l'effetto complessivo sara' quello di un'unico cuore che pulsa, di due sole possenti braccia che spingono i remi, e cosa piu' difficile, di un'unica testa che dirige il timone. Esistera' un unico Animo. Che cio' poi sia realizzato con democrazia o gerarchia, poco importa. Se ognuno ha di mira il fine ultimo, ovvero il benessere e il vero, naturale e non superfluo piacere dell'Anima, qualsiasi organizzazione che gli animi sceglieranno sara' buona.

Un animo non puo' obbligare un'altro animo ad amare la sua Anima. Questo anche quando le azioni dell'altro animo sono potenzialmente nocive o dannose (se non vengono prese ed attuate misure di difesa).


\def\Dio{D}


\subsection{L'Io}

Per semplificare la notazione, da ora in poi, scriveremo $\anima{A}_i$ al posto di $\anima{A}_{A_i}$, ed anche $a_i$ al posto di $A_i$. Quindi, $\anima{A}_{A_i}=A_1+A_2+\cdots+N$ diventa $\anima{A}_i=a_1+a_2+\cdots+n$. %TODO: fin dall'inizio del capitolo si dovrebbe usare questa notazione

\def\Io{\textrm{Io}}

\def\esAi{\anima{A}_i}
\def\esAj{\anima{A}_j}
\def\esAE{\anima{A}_E}

    Abbiamo trattato un'anima, come se fosse una entita' astratta, tuttavia, essa e' anche volonta'. E' l'essere che agisce, che tratta di se stesso e degli altri e della Natura, in maniera consapevole ad egli stesso od inconscia. Quindi, un'anima e' anche un \emph{Io}.

    L'Io \emph{ricerca} e \emph{genera} il bene per se' e il bene per gli altri che ama. A priori, le sue verita' ed i suoi comandi non sono validi ed utili. Solo se chi e' amato, sente il bisogno o desiderio di lui, il suo operato avra' efficacia, altrimenti sara' violenza, forte o leggera, e non sara' amore.

    $\esAi$ ricerca e stabilisce qual'e' il bene, ascoltando l'Anima e osservando cio' che procura piacere e cio' che procura dolore. 

    Se apprezza la simpatia, tutti gli esseri che hanno una definizione di piacere e dolore uguale al suo se' saranno inclusi nell'unione $\esAi$ e amati da lui. Se e' caritatevole, anche tutti gli esseri che non sono uguali al suo se', ma rispettano, non contravvenendo la definizione del suo se', saranno pure inclusi nella sua Anima e amati. Se e' santo, anche gli esseri che non rispettano il suo se' o gli altri che ama, saranno amati da lui.

    L'essere genera il bene, perche' spende ogni forza degli animi suoi per adempiere il suo amore, ovvero per soddisfare i desideri di ogni essere $a_j$ della sua unione.

    Infine, l'Io, per lo piu' inconsciamente, ma ne si puo' diventare sempre piu' consapevoli, determina la realta' che il se' vive e che gli altri che lo amano vivono. L'Io da' sostanza ad ogni cosa, crea le forme, lo spazio e il tempo e da' il peso e la consistenza agli oggetti, stabilisce cosa e quando e' importante, e chi seguire e di chi fidarsi. Per approfondire il discorso sulla realta' fisica vedi appendice \ref{menteCrea} pag. \pageref{menteCrea}.

    Per questo, l'Io puo' facilmente confondersi e credersi Dio, che definiremo meglio nel prossimo paragrafo. Solo se e' mosso da autentico amore un Io e' una manifestazione di Dio in terra, e per questo anche suoi piccoli risultati saranno importanti. In genere i grandi risultati sono miraggio del potere e non dell'amore. Come capire e fare in modo che il proprio Io sia amore e non potere, e' il senso di ogni percorso di fede o di ricerca interiore. \\

    Quanto abbiamo detto per $\esAi$, sussiste per tutti gli esseri: ogni essere esistente e' un Io.

    Nonostante la molteplicita' degli Io, e' possibile pensare ad un solo Io: \emph{Noi}. Se pero' ci sono contrasti o differenze di vedute, si creeranno divisioni tra gli uomini e le donne. Ogni gruppo sara' unito internamente ed avra' un suo centro, un suo Noi. L'insieme di tutti i Noi forma il Mondo. Vediamo tutto questo piu' nel dettaglio.

    Consideriamo due Io $\esAi$ ed $E$. Ci sono tre casi:

    \begin{enumerate}

        \item Se i due esseri non si amano, saranno nemici, e ognuno tentera' di affermarsi come Dio.

        \item Se uno ama l'altro ma non viceversa, uno stara' nel silenzio e l'altro non lo notera'. Uno vivra' pace, rispettera' l'altro e offrira' il suo bene all'altro. Il secondo vivra' l'altro come nemico, e come dannoso per se e per chi ama, vivra' guerra e tentera' di affermare se stesso a scapito dell'altro.
            
            Per il primo potrebbe anche esserci la speranza che l'altro possa cambiare e ricambiare il suo amore. Nei casi peggiori, perdera' o morira' in parte, o del tutto, in nome del suo amore.

        \item Se entrambi si amano, allora la loro Anima sara' uguale e i lori animi concordi, ognuno vedra' pure nell'altro Dio, e Dio sara' uno solo.

                Nota: i due potrebbero avere opinioni e ideali inizialmente contrastanti, ma stando insieme e camminando insieme, le loro visioni convergeranno verso un'unica meta, probabilmente ognuno sacrificando una parte piccola o grande di se stesso.

                Se ``tutti'' si amano, allora esiste una sola Anima $\anima{A}$ ed un solo Dio.

                Nel mondo ci sara' sempre qualcuno estraneo e contrario al proprio gruppo, quindi ora discuteremo di questo.
    \end{enumerate}

Osservazione: il caso $2$ e' superiore al caso $1$ perche' il resterare in pace con gli altri permette piu' facilmente di restare in pace anche con se stessi. Usando le notazioni di prima: se $\esAi$ e' in pace con $E$, anche se $E$ non ama $\esAi$, $\esAi$ sara' piu' facilmente in pace con se stesso, perche' la sua natura umana sociale ne trae beneficio.
Questa affermazione e' il caposaldo della fede Cristiana, dove si afferma che essere caritatevoli anche verso il prossimo che non ci ama e' la via piu' breve per raggiungere il regno dei cieli.

Osservazione: anche se un essere $a_i$ ed un essere $a_j$ si amano, nel caso $3$, per $a_i$ non cambia molto dal caso $2.$. Infatti, comunque le situazioni della vita, i desideri di $a_j$ e proprio l'amore di $a_i$ per $a_j$ potrebbero comportare dei sacrifici per $a_i$ piccoli o grandi. Allora, si capisce di piu' che la questione per $a_i$ e' indipendente da quello che fara' un qualsiasi altro essere. $a_i$ deve solo scegliere se amare o meno il suo prossimo, indipendentemente dalla scelta dell'altro.

La scelta va' fatta da $a_i$ tenendo presente quale delle due opzioni dara' maggiormente nutrimento ed appagamento all'Anima $\anima{A}_i$. 

Ancora, se la verita' della natura dell'essere umano e' quella dell'amore disinteressato\footnote{carita', agape} allora scegliere l'amore e' la via migliore, per il benessere dell'Anima, anche a sacrificio dell'Io. \\

Infine, se consideriamo la Natura $N$ come un'essere, possiamo ripetere per $N$ gli stessi discorsi fatti per $\esAj$, che e' un qualsiasi essere amato da $\esAi$ e per E, essere che non ama $\esAi$. Quindi $\esAi$ puo' non amare $N$ e quindi dolersi della sua o altrui sorte fisica, non spendere energia fisica per migliorare la sorte, e considerare la Natura $N$ come matrigna e nemica. Oppure, $\esAi$ puo' amare $N$, accettare i limiti di essere un uomo o donna, sia per se stesso $A_i$, sia per ogni $A_j$ che ama, e puo' impegnarsi per tendere sempre a stare bene e in pace nella Natura. Viceversa, la Natura, essendo materia, ed essendo sempre la stessa da tempo immemore, non ha molto da amare $\esAi$. Piu' che altro perche' la materia non ha anima, non desidera ne' vivere ne' morire, non e' ne' viva ne' morta. Quindi, certe volte un sasso puo' risultare utile, altre volte inutile, altre un ostacolo. Percio', l'unico caso positivo che sussiste tra $\esAi$ e la Natura e' il caso $2.$. $N$ tuttavia, anche se non ama, non odia. Questo lo aveva osservato lungamente Epicuro e Lucrezio\footnote{De rerum Naturae}. La Natura non e' avversa a noi, semplicemente fa' il suo corso, e a volte e' facile raggiungere con lei un equilibrio, alle altre volte no, o perfino impossibile. \\

L'Io puo' essere visto come animo oppure come anima. Ogni aspetto da' luogo ad una definizione di Dio, che daremo di seguito.

\subsection{Prima definizione di Dio}

   Definiamo Dio, come Animo o Spirito, come qualsiasi colui o colei che soddisfa \emph{completamente} i bisogni e desideri \emph{infiniti} dell'anima. 

   E quando dei bisogni non possono essere soddisfatti, e' colui/colei che mostra, spiega e da' esempio di come questi bisogni non possono essere soddisfatti per il bene maggiore degli altri o del rispetto dei limiti del corpo, cosi' che' l'anima stessa, raggiunta la comprensione, puo' scegliere di moderare o sacrificare il suo bisogno.

   Non tanto e' la questione se Dio esiste. Ma piuttosto quanto esiste. Un qualsiasi animo che soddisfa al 10\% i bisogni di un anima, sara' per quell'anima Dio al 10\%.

   Se un Io lamenta che Dio non esiste, dovrebbe esaminare se stesso e vedere quanto puo' essere migliore, piuttosto che desiderare che qualcuno o qualche entita', fuori da lui, risolva i problemi e gli obbiettivi che sono a lui cari, oppure prendersela con gli altri con la scusa che non lo amano o non lo amano abbastanza.

    La maggior parte delle volte si puo' sempre fare di piu' o diversamente, questo perche' si e' convinti che gia' si sta' facendo tutto correttamente, ma in realta' non si sta' ascoltando propriamente l'Anima, ne' si sta' spendendo se stessi per amarla. Quando gia' si stara' facendo il massimo, l'Anima vivra' notevolmente meglio e, per il resto, sara' facile confortare l'Anima con quel che si potra'.

    Amare l'Anima non e' una questione oggettiva (anche se per piccole cose lo e'). E', piuttosto, se l'Io \emph{esiste} per l'anima amata, se la forza creatrice, generatrice e protettiva del suo spirito e' centrata nell'anima amata e non in altro. 

\subsection{Dio e' ultra-terreno}

Abbiamo detto che un animo e' per un anima Dio in una certa percentuale. La forza di un Io, od un Noi, non potra' mai essere al 100\% Dio per un'anima. Questo perche' vorrebbe dire che per la sola esistenza di questa forza, l'Anima non patirebbe mai sofferenze, difficolta' e smarrimenti nella sua vita in terra. 

Da questo, discende che Dio, come animo, come forza fisica e naturale, non e' presente al 100\% nella Terra. A volte, l'amore materiale e' presente anche in bassissima percentuale nelle vite dei piu' poveri od emarginati. Tuttavia, l'esistenza di Dio e' concreta fino a quando esiste almeno un Io che ama completamente l'Anima. In questo caso, anche una percentuale positiva di amore materiale proveniente da questo Io sara' riconosciuto come divino dall'Anima. L'Anima, si ricordera' cosi' della sua origine divina, e ritrovera' la forza, il coraggio e la gioia di vivere.

Infine, come spiegato nel paragrafo \ref{loZero} pag. \pageref{loZero}, l'Anima puo' venire incontro agli uomini ed alle donne che la amano nella terra, tramite il ``grazie'' e, nei casi piu' difficili, tramire il ``sacrificio''. Ringraziare e' apprezzare quello che qualcuno, qualcuna o alcuni sono per noi, compresi noi stessi, e in questo apprezzamento, riconoscendo che cio' che viene dato e' un dono, e che non viene dato di piu' non per cattiveria!, rimanere appagati di cio' che esiste. Sacrificarsi e' la forma piu' forte e difficile del ringraziare, in quanto cio' che si riceve non e' sufficiente per soddisfare i bisogni, e nonostante cio', si rimane comunque grati per cio' che gli altri e noi siamo. 

Cosi' facendo, l'Anima potra' vedere molto piu' nitidamente la presenza di Dio e godere piu' intensamente del dono della vita.

\subsection{La preghiera}

Definiamo la \emph{preghiera} come una libera condizione che l'Io lascia ad un anima. In questa dimensione, lei puo' fare chiarezza per se stessa ed esprimere, con speranza e fiducia, i suoi bisogni e desideri. 

La preghiera e' quello stato in cui l'anima parla a Dio che e' in un Io, od in un Noi, anche se questo essere di Dio in un Io e' solo in forma primordiale. E' anche l'Io che ascolta, non giudica e inizia a comprendere l'anima, e puo' scegliere di attivarsi e di spendersi per lei, nella sua misura, diventando cosi' uno con Dio, Sua manifestazione ed azione.


\subsection{Seconda definizione di Dio}

Un Io $a_i$, quando e' in Pace, e' una ``piccola'' incarnazione, di Dio in terra.

Un Io $a_i$, parte dell'Anima $\anima{A}=a_1+a_2+\cdots$, e' in Pace quando e' in pace con ogni parte $a_j$ dell'Anima, inclusa $n$ la Natura. Questo accade quando ogni desiderio forte o debole di $a_i$ e' concorde, o quanto meno non interferente, con ogni altro desiderio delle altre parti dell'anima. 

In formula, quando ogni $r\in\des(a_i)$ e' concorde con ogni $s\in\des(a_j)$, per ogni $j$. 

Anche il caso $i=j$ e' incluso: se $a_i$ ha problemi interiori psicologici lievi, come stress, o piu' gravi, ha almeno due desideri $d_1,d_2\in\des(a_i)$ che non sono concordi.

E' piu' facile che questa concordia sia raggiunta se l'Io rinuncia ai desideri superflui e non necessari, come insegna Epicuro.

In alcuni casi, l'Io deve sacrificare dei desideri anche necessari o non superflui, come insegna Gesu'.

La Pace dell'Anima e' raggiunta quando l'anima $a_i$ raggiunge il proprio ``centro''. Questo centro e' la parte essenziale e vitale dell'individuo, ed essendo di natura divina, e' proprio Dio. Ad esempio, il desiderio di amore verso una persona in difficolta', disposta a collaborare per risolvere il suo problema, e' un desiderio di Dio stesso.

Questo mostra che Dio, non e' una entita' fuori da noi, ma e' in noi. O, ancora meglio, il Dio delle scritture e' una descrizione della nostra autentica, centrale e divina natura.

Tanto piu' liberiamo noi stessi da pesi superflui, vani e che ci allontanano dalla nostra natura divina, tanto piu' noi saremo una incarnazione di Dio in Terra e noi stessi e gli altri ne trarranno beneficio.

Cio' non deve inorgoglire o riempire di vanita'. Tuttaltro, tanto piu' si ci avvicina al centro tanto piu' si vedono i ``fiori di questo doloroso mondo'', come recita l'haiku \ref{dolorosoMondo} pag. \pageref{dolorosoMondo}, e tanto piu' si e' disposti a caricarsi di responsabilita' e rischi per difendere se stessi e gli altri, ed anche sopportare fatica e rinunciare a parte della propria vita.

Ma se si e' mossi da Amore, tutto cio' condurra' a nuovi mondi, ad una nuova esistenza, in sintonia con se stessi, gli altri e la Natura tutta, e quindi con Dio Padre.

Un'Anima in Pace e' sempre un essere vivente, limitato nella sua psico-fisicita', quindi non potra' mai amare tutti gli esseri con ugual facilita' e profondita'. Da questo discende la definizione di \emph{Dio Padre}, come l'Anima che ama infinitamente, oltre i suoi stessi limiti. Dio Padre, pur non violando la natura terrena delle cose, e' in Pace anche con quei desideri dell'Anima, puri ed autentici, che non sono facilmente esprimibili o realizzabili, e si adopera affinche' non siano ne' dimenticati ne' abbandonati, cosi' che' se i limiti del corpo del se' e degli altri lo consentira', un giorno saranno realizzati.

Dio Padre non e' un ideale, quando la nostra Anima e' in Pace, noi siamo una piccola dimostrazione della Sua esistenza. Quindi, per trovare Dio, bisogna trovare la vera Pace. Per mostrare Dio, bisogna essere esempio autentico di Pace. Per condividere Dio, bisogna essere a disposizione ed impegnarsi per la Pace dell'Altro, affinche' venga salvato ed allietato Dio che e' nell'Altro.

\subsection{Dio come limite all'infinito}
\label{definizioneAnimo}

Un Io essendo il prodotto di processi neurologici, e' limitato ed imperfetto. Tuttavia, l'Io mosso da Amore, \emph{tende} ad un punto fisso, un punto di equilibrio, di matura e piena realizzazione. Ogni anima ha bisogno e desidera costantemente raggiungerlo. 

Per rendere l'idea di come e' tendere al Punto Fisso, l'Io vi si avvicina quando nella sua vita non c'e' piu' niente che di lui/lei turba il se' o gli altri, quando coglie e comprende ogni desiderio piu' essenziale e vitale dell'Anima, e riesce a spendersi per far procedere l'Anima verso la realizzazione di tale desideri. Che tali desideri non si compino effettivamente nella realta' fisica, esattamente come immaginati, a causa di forze maggiori e' secondario, perche' l'Anima apprezzera' il fatto di essere amata e di cio' sara' nutrita.

In altre parole, questo processo di realizzazione si puo' vedere come l'Io che, mosso dal desiderio dell'Animo di amore (Spirito Santo), tende ad essere sempre di piu', nei suoi limiti, \emph{Dio che si manifesta ed agisce sulla terra}.

Che l'Io tende ad essere Dio stesso, non vuol dire che diventa super potente, perfetto ed intoccabile, al di sopra della natura, e che non ha bisogno piu' di nessun altro, che e' al di sopra delle leggi, e che puo' fare cio' che vuole di ogni vita. 

Dire che l'Io tende a diventare Dio, vuol dire che tende ad essere speso ed indirizzato verso tutti i bisogni ed i desideri non superflui dell'Anima nella sua completezza e nell'individualita' di ogni sua parte, ovvero dell'anima di ogni essere che la compone. Ogni azione e pensiero che muove, respiro ed intenzione che alimenta, e' volto a creare la vita in ogni essere, a mantenerla ed esaltarla. E tutto questo, nel rispetto dei limiti fisici e psicologici degli esseri che sta' amando, ovvero nel rispetto della Natura fisica e della psiche degli esseri.

Possiamo adesso definire lo Spirito Santo (od Animo Santo) come l'Io mosso dal desiderio di amore. 

Facciamo un esempio. Se pensiamo ad un panetterie che fa' onestamente il suo lavoro, il suo punto all'infinito e' un Panetterie, ovvero, un uomo che non solo fa' onestamente e con cura il suo lavoro ma che lo fa' perche' ama se stesso ed ogni suo cliente. Idealmente sa' a chi piace quale tipo di pane e se avesse tempo farebbe un pane personale per ogni suo cliente, adatto in quel giorno alle circostanze che il cliente sta' vivendo. E, non solo, Lui facendo il suo lavoro sta' bene e trova la gioia e la forza di vivere. E, infine, non vive solo per il suo lavoro, ma e' anche un uomo che ama la vita per quello che offre ed ama le persone che gli stanno accanto.

Il fatto che un Io non abbia fisicamente capacita' infinite, non vuol dire che egli non sia manifestazione di Dio. Se quell'Io esiste veramente per l'Anima e direziona le sue energie per servirla e vederla contenta, allora cio' che sta' agendo non e' piu' un Io terreno ma e' Dio.

Ad esempio, se un insegnante si sveglia ogni mattina e con premura si dirige verso la scuola per il bene dei suoi alunni, allora loro, amandolo, vedranno in lui un riflesso della vera luce.\\

Che un Io muovi un solo piccolo passo verso l'infinito, non e' cosa per niente semplice da conquistare, ed ogni sforzo di se stesso, o di chi ama e lo ama, impiegato ad amare l'Anima, e' prezioso. Infatti, ogni sforzo vuol dire fisicamente e psicologicamente un consumo del corpo e della vita.

D'altra parte, pero', solo Dio puo' scegliere un Io per amare l'Anima in un certo tempo, per una certa durata e in un certo luogo e in che modo. Se un Io volesse essere bravo ma Dio non volesse amare tramite lui, allora non nascerebbe amore. E tale cosa non puo' essere forzata: nessun essere, neanche tutti gli esseri insieme, per quanto numerosi e capaci, possono amare l'Anima per loro volonta' egoistica o tramite cose frapposte tra loro stessi ed Ella, come ad esempio, tecnologie, possedimenti, beni e risultati meritevoli. 

Tutto cio' detto e premesso, ci sono due dimensioni su cui si puo' lavorare: una dell'anima che e' un costante definire, affermare e rifinire cio' che e' buono, cio' e' piacevole, cio' che e' vero. Una dell'animo che e' un costante capire, allenarsi e fare per realizzare e mantenere la salute ed il piacere dell'anima.

\subsection{Lo zero}
\label{loZero}
Un'anima puo' venire incontro agli animi, stabilendo che cio' che e' stato gia' raggiunto e' buono e non chiedendo piu' di cio' che sta' ricevendo: ``Grazie. Tutto cio' che e', e' buono, e' Sua volonta'.''.

In pratica, l'anima, paradossalmente, raggiunge Dio quando si rende conto che il tutto e' gia' buono, sacro e prezioso.

Anche se sembra facile a dire a parole, questo ``rendersi conto'' e' costoso, ed a volte richiede sacrifici. E' essere come un imprenditore che avendo tutta la potenzialita' di far accrescere il capitale della sua azienda, decide di non andare oltre, in nome di un bene piu' grande, ad esempio, perche' gia' il fatturato e' piu' che buono e non serve inquinare oltre l'ambiente, o sottoporre i lavoratori ad ulteriore stress, od impiegare nuove persone per lavori banali.

E' essere come un santo che, pur avendo una ferita che lo attanaglia, non si abbatte, ne' rinnega Dio, e continua ad avere una serenita' superiore e ad essere a disposizione degli altri.

Quindi, e' una cosa grande quando l'anima dice ``va tutto bene'', anche quando potrebbe ottenere di piu' per il se', o per un altro, ma a discapito di qualcuno o del se'. E' una cosa grande quando l'anima dice ``va tutto bene'', anche quando la situazione e' difficile da sopportare e affrontare.

Se l'anima fa' cio' di sua spontanea volonta' e naturalmente, senza essere oppressa e obbligata dall'Io, ne' con la scusa di superstizioni, ne' di regole o leggi, allora in cio' l'anima trovera' piu' rapidamente Dio.

Ad esempio, anche se il mondo sembra apparentemente impazzito, crudele, duro e ingiusto, soffermandosi uno puo' vedere che nel traffico le persone rispettano le altre macchine, che per strada la grande maggioranza delle volte non si e' disturbati dagli altri, che l'educazione ed istruzione ricevuta a scuola, anche se ancora lontana dalla perfezione, ha un senso perche' forma, rende cittadini della societa' e, da' strumenti intellettuali, che se coltivati in proprio, sono utili. E ancora, pur la societa' avendo strada da fare, e' comunque lodevole in alcune parti del mondo: ospedali, scuole, citta' dove non regna la violenza incontrollata (come potrebbe essere nella giungla), etc... E, infine, gioire della buona salute di cui si gode.

In altre parole, tutto questo potrebbe essere scontato ma non lo e' e gia' di questo si potrebbe essere, se non contenti, almeno comprensivi dello stato attuale.

Tuttavia, se si fa' questo discorso ad una persona sofferente, ad esempio, una persona che ha perso il lavoro e che quindi nutre un risentimento verso la societa', verso il suo superiore o alcuni dei suoi colleghi, allora non e' detto che quella persona possa recepirlo. Se non lo recepisce e si cerca di convincerla, allora, lei si difendera' e si allontanera' dal nostro impulso positivo iniziale. Solamente amandola veramente, e stando con lei, magari anche non dicendo niente, pregando nel proprio cuore che possa andare oltre la sua sofferenza, lei fara' il suo cammino e poi, un domani arrivera' ad una consapevolezza simile.

A volte il ``grazie'' dell'Anima e' un ``sacrificio'', perche' l'Anima aveva \emph{bisogno}, ma cio' che riceva materialmente non e' sufficiente.

Questi sacrifici, non sono arbitrari, repressivi od oppressivi. Sono sacrifici che l'Anima si trova a scegliere se compiere per mantenere il suo amore per gli altri o la salute del corpo. Sono sacrifici, che nel massimo dell'impegno e capacita', non hanno alternative piu' semplici e indolori.

Matematicamente, pensando al Dilemma del Prigioniero, esistono situazioni in cui l'unico modo per mantere l'ottimo, e' scegliere dei sacrifici personali. Nel dilemma del prigioniero\footnote{\url{https://en.wikipedia.org/wiki/Prisoner's\_dilemma}}, vedi tabella \ref{tabPrisonerDilemma} pag. \pageref{tabPrisonerDilemma}, ognuno ha un rendiconto personale maggiore se tradisce l'altro, e, inoltre, se entrambi collaborano il rendiconto non e' il massimo possibile per ognuno. Tuttavia, solo se ciascuno rischia di essere tradito, potra' non tradire l'altro, e ottenere l'ottimo dell'Anima piuttosto che della sua anima (social welfare, ovvero somma dei rendiconti di ogni giocatore).

\begin{center}
    \begin{table}
        \begin{tabular}{ |c|c|c| }
            \hline
            &  Supporta   &   Tradisce \\
            \hline
            Supporta   & (-1,-1) & (-3,0) \\
            \hline
            Tradisce & (0,-3) & (-2,-2) \\
            \hline
        \end{tabular}
        \caption{\label{tabPrisonerDilemma}Dilemma del prigioniero: due criminali che insieme hanno commesso un reato, vengono posti in due celle lontane e non comunicanti tra loro. Ad ognuno viene detto che se testimonia a sfavore dell'altro prigioniero (tradisce) non avra' nessuna pena se l'altro invece non lo tradisce, e avra' una pena di due anni se invece l'altro lo tradisce. Se entrambi si supportano a vicenda non tradendosi, allora' ognuno avra' una pena di un anno. Sopra e' riportata la tabella, come si usa in teoria dei giochi. La riga indica la scelta del prigioniero A, la colonna indica la scelta dell'altro prigioniero B. Ad esempio, la casella in alto a destra indica che il prigioniero A supporta il prigioniero B, mentre B tradisce A. In questo caso, A avra' 3 anni di prigione, mentre B nessuno. Se A e B si supportano a vicenda, la somma delle pene di entrambi e' di 2 anni di prigione. In tutti gli altri casi la somma e' maggiore. Quindi, il ``social welfare'', ovvero il benessere sociale, e' massimo se entrambi si supportano. Dal punto di vista egoistico, in media, se un giocatore tradisce ha $(0+(-2))/2 = -1$, un anno di prigione. Se invece supporta ha in media $((-1)+(-3))/2=-2$ anni di prigione. Quindi, probabilisticamente, dal punto di vista egoistico e' meno rischioso tradire. }

    \end{table}
\end{center}



\subsection{L'infinito}

Al contrario dell'anima, l'animo viene incontro all'Anima, orientandosi verso l'infinito, che, in rapporto alla sua forza, e' a distanza infinita. Ad esempio, se l'anima desidera sempre che ognuno sia ben nutrito e non abbia da patire la fame, allora questo ``sempre'' e ``ognuno'' nell'atto pratico e' difficile da realizzare, e soprattutto per una sola persona. Eppure, l'animo andra' con successo incontro all'anima quando spendera' se stesso per raggiungere il suo punto all'infinito\footnote{e, preferibilmente, non facendo solo atti simbolici, ma preferendo atti che danno veri risultati a quelli simbolici. Certo, se e' impossibile sul momento avere risultati, allora l'anima sara' soddisfatta degli atti simbolici}.


\subsection{Definizioni negative}

Fin'ora tutte le definizioni date sono positive. Per ognuna, si puo' dare una definizione speculare negativa, che nasce dal fatto che siamo esseri limitati e imperfetti. Le definizioni positive sono da prediligere, perche' costruire e' difficile ma costruendo e' piu' facile mantenere cio' che e' gia' stato costruito, invece, distruggere e' facile, ma e' anche piu' facile vanificare cio' che c'era gia' di buono.\\

Considerando un'anima $\mathcal{A}$, ella e' \emph{egoista} se include nell'unione $\mathcal{A}$ almeno un essere $A_i$, non per amarlo, ma solo per amare un $A_j$, con $j$ distinto da $i$ ($j\ne i$).

Di solito $A_j = \self(\mathcal{A})$, cioe' $\mathcal{A}$ considera altri esseri $A_i$ non per amarli ma per amare il se'.\\

Un'anima $\mathcal{A}=A_1+A_2+\cdots$ e' \emph{narcisista} se e' innamorata di un $A_i$, e quindi crede che i bisogni e desideri di tutti gli $A_j$ sono rappresentati dai bisogni e desideri di $A_i$, quando questo non e' vero. Ad esempio, se $A_i$ predilige cibi salati, mentre almeno un $A_j$, distinto da $A_i$, non li ama, allora $\mathcal{A}$ comunque pensera' che per $A_j$ e' un bene mangiare cibi salati, e magari cerchera' di convincere $A_j$ di cio'.

\subsection{Considerazioni geometriche}
L'unione $A+B$ e' qualcosa che trascende sia $A$ che $B$. Immaginiamola come l'unione dell'asse $X$ e dell'asse $Y$ che, da una dimensione, passa a due dimensioni, ed e' il piano.

\def\state{\textrm{state}}

Considerare una retta, che va' da $-\infty$ a $+\infty$, e' anche significativo se con $\state(A)$ indichiamo la sua posizione attuale nella retta e che rappresenta il suo \emph{stato}. Tanto piu' e' verso l'infinito positivo $+\infty$, tanto piu' $A$, complessivamente, sta' bene. Viceversa, considerando $-\infty$.

Per l'unione, $\state(A+B)=(\state(A),\state(B))$ e' lo stato dell'anima empatica. Esso dipende sia da $\state(A)$ che da $\state(B)$ ed e' un punto del piano. $\state(A)$ e $\state(B)$ possono essere correlati oppure no \footnote{se sono correlati, tanto $A$ sta' meglio tanto piu' (o meno) $B$ stara' bene}.\\

L'Io di $A$ e' il vettore che direziona $A$ lungo il suo asse. Il ``Noi'' di $A+B$ e' il vettore che direziona $(\state(A),\state(B))$ nel piano, e le sue componenti sono l'Io di $A$ e l'Io di $B$.

Se un Io ama, la sua direzione e' verso $+\infty$.

Un animo si puo' pensare come il modulo dell'accelerazione applicata all'anima. La direzione e' stabilita dalla somma dei vettori Io. L'animo e' infatti energia, desiderio. L'Io direziona tale desiderio.\\

Infine, Dio e' il ``motore immobile dell'universo''. Dio e' quel punto posto a $(+\infty, +\infty)$ che l'anima desidera raggiungere. E' anche quella sorgente di forza che da' l'energia ad ogni Io. L'Io, tanto piu' e' puro e capace, tanto piu' la spende e la direziona verso Lui, ed essa diventa l'accelerazione che muove l'anima verso Dio.\\

La Natura, entra nel discorso, compattificando lo spazio dell'anima. Tale spazio, non e' piu' uno spazio infinito, e' un sottospazio di misura finita, determinato dai vincoli della natura. Ad esempio, per rendere l'idea, un tale sottospazio e' $\state(A+B) \in \{\; (x,y) \;\; | \;\; \sqrt{x^2+y^2} \le 1 \;\}$, ovvero lo stato dell'anima puo' essere solo all'interno del cerchio di raggio 1, centrato nell'origine. Quindi, lo spazio dell'anima $A+B+N$, dove $N$ e' la Natura, e' uguale al cerchio unitario: $B(O,1)$, e non e' tutto il piano. (nota: non e' piu' corretto scrivere $A+B+N$, magari si dovrebbe scrivere $A+B\;\textrm{mod}\;N$).\\

Nelle trattazioni piu' classiche, si procede in maniera opposta a quanto fatto fin'ora: l'uomo incolto e' in un punto lontano dal centro $O$ dello spazio ($O=(0,0)$ nel piano). L'uomo che ama fa' tendere l'anima al centro, positivamente (nel piano, rimanendo nel quadrante positivo). Tanto piu' l'anima e' vicina al centro, tanto piu' sta' meglio.

In questa trattazione, Dio e' il centro $O$. L'animo e' un'accelerazione centripeta. $O$ rimane sempre infinitamente ``alto'': tanto piu' si ci avvicina al centro, tanto piu' spostamenti infinitesimali sono piu' difficili, piu' ricchi ed hanno piu' valore.

Visivamente, viene in mente il disegno della geometria iperbolica, dove piu' si ci allontana dal centro, piu' le cose sembrano rimpicciolirsi: link  \url{https://en.wikipedia.org/wiki/Poincar\%C3\%A9\_disk\_model}.\\


% Credo che le due trattazioni si possono vedere una come la geometria euclidea, dove piu' un oggetto e' distante dal centro, piu' ha spazio per ingrandirsi, e una come la geometria iperbolica, dove piu' un oggetto e' distante dal centro, piu' si rimpicciolisce. (piu' un punto e' distante dal centro, piu' e' piccolo), link: \url{https://en.wikipedia.org/wiki/Poincar\%C3\%A9\_disk\_model}.

Ritornando alla geometria usuale (ovvero Euclidea), il fatto che l'anima raggiunge Dio, quando dice ``tutto cio' che e' e' Sua volonta'', vedi par. \ref{loZero} pag. \pageref{loZero}, si puo' rendere dicendo che il suo sistema di riferimento e' tale che $\state(\anima{A})$ si trova nel primo quadrante. 

In fondo, che $\state(\anima{A})$ sia positivo in ogni componente, e' un fatto di definizione: dipende da dove si pone l'origine $O$.

L'anima soffre quando pone $O$ piu' verso l'infinito che verso la terra, e cosi' si trova sempre in difetto. L'anima e' forte, quando anche rispetto alle limitazioni fisiche della Natura, non desidera piu' di cio' che gia' ha e sta' ricevendo (dal se', dal resto del mondo e dalla natura).

Questo stato si raggiunge non imponendolo, ma quando l'anima raggiunge una vera e profonda consapevolezza che ``tutto serve Dio, e che ogni uomo, donna e atomo, anche quando apparentemente non sembra, sta' contribuendo al Suo grande lavoro''.

Dal punto di vista dell'animo, ogni spostamento che provoca dello stato dell'anima, verso la direzione del bene, per quanto piccolo, contribuira' a far raggiungere all'anima Dio.

\section{Appendice A}
\label{menteCrea}

Un uomo riceve segnali sensoriali dal mondo esterno. Tramite il suo cervello, elabora questi segnali e (inconsciamente e cosciamente) crea un modello che descrive e predice tutti gli stimoli che riceve. Ad esempio, in base alla sua esperienza, quando vedra' un zona luminosa, di colore rosso, che emana calore, la cataloghera' col concetto di "fuoco". Non si avvicinera' a questo ``fuoco'' perche' dentro di se predice che una tale azione avra' effetti dolorosi\footnote{Questa sua conoscenza deriva o da una esperienza che ha fatto da bambino, oppure da un ammonimento ricevuto dai genitori}.\\
La realta' che lui percepisce e' una realta' che lui sta' creando dentro di se. Ad ogni segnale sensoriale che riceve da' un significato, ad esempio, alcuni segnali luminosi saranno per lui delle ``forme'' e alcune forme le pensera' come ``oggetti''. Agli oggetti attribuira' delle proprieta'.  Quindi, ad esempio, se avesse piena coscienza dei suoi meccanismi cognitivi, potrebbe dire: ``dalla luce che vedo dai miei due occhi riesco a tracciare delle forme. In particolare, una forma che vedo e' compatta e ha una profondita'\footnote{la profondita' e' percepita dal fatto che ogni occhio riceve la luce da due punti differenti e da altri indizi, vedi \url{https://en.wikipedia.org/wiki/Depth\_perception}} quindi dico: ``e' un oggetto'', inoltre, noto anche le seguenti proprieta': e' tonda e grigia. Per tenerla in mano, avverto uno sforzo muscolare, quindi dico ``e' pesante''. Considerando tutto, dico ``esiste un oggetto tondo, grigio e pesante. Ho visto altri oggetti simili e ho imparato a chiamarli \emph{pietre}. Siccome, tutte le pietre che ho visto fin'ora le ho sempre ritrovate nel posto in cui le lasciavo, dico che qui dove mi trovo, esiste una pietra\footnote{In questa frase stiamo anche implicitamente considerando il suo concetto di spazio, che e' sempre un qualcosa che l'uomo crea dentro di se. Vedi \url{https://en.wikipedia.org/wiki/Spatial\_ability}}''.\\




