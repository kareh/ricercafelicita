\chapter{La ricerca di Dio con approccio scientifico}
\label{chapterDio}

\epigraph{
     La scienza d'Amore, oh, si! la parola risuona dolce all'anima mia, desidero soltanto questa scienza.
 }{S. Teresa di Lisieux}

Avere fede in un Dio vuol dire nutrire la profonda speranza che la Felicita' e la Pace esistono, e quando non sono apparenti credere che, perserverando nel bene, saranno di nuovo raggiunte. Questo, la cultura del mondo, non puo' dirlo. In essa, la felicita' dipende da condizioni esterne alla propria vita: dal denaro, dall'avere tanti amici, dal lavoro, dall'avere un partner, dal ricoprire una posizione di prestigio, dalle condizioni climatiche, dalle guerre, dai traumi della nostra storia personale, dalle persone che non hanno rispettato noi stessi e non ci hanno amato abbastanza per come volevamo ed avevamo bisogno, dalle persone che ancora non ci amano, dalla societa'. Nelle fede, tutto questo non e' svalutato. Ognuna delle cose elencate, ha la sua importanza, ma sono tutte secondarie all'unica esistenza necessaria e sufficiente: Dio.
Nelle comunita' religiose sane, per perserverare con piu' facilita' e camminare con piu' velocita', e' una grazia il potersi confrontare con e imparare da un maestro, aprirsi ad una maestra, confrontarsi con i fratelli e le sorelle di fede, imparare dai testi di chi ha gia' camminato molto prima di noi, e se mancasse tutto questo, applicare in ogni aspetto di se stessi dei principi posti a fondamento della vita.
Oggi giorno, la fede verso Dio non esiste piu', se non tra pochi. Si trascura la tematica di Dio, perche' e' la scienza che da' ogni risposta e si crede che Dio e' solo una superstizione per chi non sa' ragionare. Eppure, cosi' come la ragione e' una dote naturale che l'uomo puo' sviluppare ed adoperare per migliorare la sua vita, anche la fede in Dio e' una predisposizione naturale dell'uomo che egli puo' sviluppare ed adoperare per la ricerca della felicita'. Anche se apparentemente ragione e fede sembrano incompatibili, l'una puo' rafforzare l'altra e viceversa. Se la ragione stabilisce principii e direzioni, la fede da' la forza per rispettarli e per perseguirli. Se la fede riconosce bisogni e speranze umane, la ragione scruta l'insieme delle possibilita' per realizzarli o per comprendere che, seppur chi ha bisogno e' amato, non esiste soluzione migliore della realta' attuale.

La fede in Dio, vuol dire anche scegliere, tra gli infiniti modi di esistere, solo uno come privilegiato. Nel Cristianesimo e' quello di Gesu', del vivere solo e soltanto nell'amore disinteressato, in ogni condizione e situazione uno si trovi. Questo limita la propria liberta' personale, in quanto alcuni sentimenti saranno tenuti in quarantena, e solo quando si arrivera' a maturare una soluzione d'amore, potranno essere espressi. Tuttavia, non e' un perdere la propria vita, ma e', tramite un lavoro interiore e sacrifici piccoli o grandi, vivere delle gioie piu' grandi e luminose, radicarsi in una pace piu' vasta e sconfinata. E chi ha dimostrato che tali gioie e tale pace e' raggiungibile e' Gesu'. E chi lo dimostra oggi, e' chi lo ama, e riesce a vivere questa dimensione d'amore, nella propria vita, che sia ministeriale (preti, suore, ...) o che sia laica.

Se nel passato con la religione si e' esagerato ed e' stata corrotta ed opprimente, nel '900, con il progresso scientifico e tecnologico, l'uomo ha riscoperto la propria liberta'. Il nuovo benessere e le vittorie sulle malattie, hanno spinto con forza la societa' all'abbandono delle sue superstizioni e leggi repressive. Al contempo, pero', hanno portato con se' anche l'illusione del dominio dell'uomo sulla propria vita e sulla Natura. E' piu' facile oggi credere che la vita e' solo un fatto meccanico, di cui, quando ci va bene, noi siamo padroni e di cui, nella nostra onnipotenza possiamo disporre arbitrariamente, oppure, quando ci va male, di cui noi non possiamo fare nulla perche' schiacciati dalle forze della natura e della societa'. L'uomo adesso domina ogni cosa ed ogni aspetto della sua vita, eppure adesso e' gravemente piu' infelice di prima, adesso i problemi psicologici nei giovani e nella societa' sono sempre di piu' e piu' profondi\footnote{Il 19\% della popolazione americana soffre di ansia \url{https://adaa.org/understanding-anxiety/facts-statistics}}. La Natura e' vista solo come una risorsa da sfruttare e da combattere per sopravvivere. 

Cosi', si e' perso il dare ufficiale valore ai temi umani, alla sacralita' della vita, alla purezza dell'amore. Solo in pochi, i piu' sensibili e ``deboli'', i piu' provati dalla vita, si consolano pensando a queste cose con belle poesie, vergognandosi o nascondendosi dagli altri piu' ``forti''.
Si ricerca il piacere, ma non si sa' dove trovarlo, non si sa' cos'e'. Viene definito solamente tramite i sensi. Ma il ``senso dei sensi'' non viene definito. Si cercano scappatoie, nuove realta' virtuali, aumentate, tour ed esperienze trascendenti, ma al ritorno si e' sempre gli stessi. Si trascura il fatto che l'anima desidera ed ha bisogno dell'infinito, e qualsiasi raggiungimento, traguardo, sensazione, e' sempre superata da un desiderio e bisogno piu' grande.

Cio' che rimane nel mondo e' deludente, ne' piu' deludente del mondo superstizioso del passato, ne' meno. Oggi, ci sono solo effetti speciali, tecnologie all'ultimo grido, immagini ed esperienze mozza fiato, una dietro l'altra, che adoperiamo, incosciamente e cosciamente, per convincerci sempre che va bene rimanere tali e quali a come si e', che siamo perfetti, che al piu' sono gli altri a sbagliare, a non allinearsi al nostro modo di fare, che e' il mondo che ancora non ci capisce.

Tutte queste tecnologie, sono una piccola cosa di fronte alla vera forza che e' quella Sua, forza che e' comprensione, consolazione ed estatico nutrimento dell'anima, forza che Lui ha per tutti e che muove ogni atomo.

\subsection{Cosa segue}

Nel paragrafo \ref{nichilismoScientifico} pag. \pageref{nichilismoScientifico}, si parte da una visione atomistica e meccanicistica dell'Universo e si discute come da questo e' comunque necessaria e importante la figura di Dio. 

Dal paragrafo \ref{amareSe} pag. \pageref{amareSe}, si discute in maniera piu' semplice su cosa consiste avere fede in Dio. Si arrivera' a una definizione di Dio che fa leva sul discorso sviluppato in \ref{DioScientificamentePoesia} pag. \ref{DioScientificamentePoesia}.

Nell'ultimo paragrafo \ref{procAx} pag. \pageref{procAx}, si fara' una analisi quasi formale, in stile matematico dei concetti di anima, spirito e Dio. Questo paragrafo e' per chi ha dimestichezza con il linguaggio logico/matematico. Esso e' uno sforzo per riuscire a catturare il concetto di Dio partendo da concetti piu' primitivi, in maniera logica. 

\subsection{Avvertenze dottrinali}
L'esposizione della dottrina Cattolico-Cristiana e' fatta in ``best-effort'', al massimo della mia personale comprensione, senza alcuna pretesa di autorita' dottrinale. La mia tendenza e' quella di non divergere dalla dottrina, tuttavia, seguendo uno stile scientifico/filosofico che predilige uno sviluppo logico degli argomenti, e' possibile che alcuni punti sembrino divergenti dalla dottrina cattolica. Questa e' solo apparenza: se si prende l'esposizione e si parte dalle premesse e si segue il ragionamento, a volte molto sottile, ma pur sempre logico, si arrivera' a una conclusione che non contraddice la dottrina. D'altra parte, io non essendo santo, porto con me il mio peccato e le mie idee sono di certo influenzate dalla mia lontananza da Dio. Quindi, alcuni punti potrebbero proprio essere divergenti dalla dottrina cattolica. Nel tempo, sono maturato, ed ogni volta mi sono trovato sempre piu' vicino alla dottrina. Ad esempio, un punto molto difficile e' stato il capire perche' Gesu' e solo Gesu' e' considerato Dio, e non i Santi, che pure loro hanno dato la vita e hanno raggiunto una purezza aurea. Nel tempo, mi sono trovato a mio agio con questo mistero, ed e' sorta in me una spiegazione logica, che puo' essere condivisibile o no, ma considerando che alla base di tutto e' posto il principio della Carita', dell'amore disinteressato, non temo molto le apparenti divergenze del mio approccio scientifico/filosofico, perche' Gesu' stesso disse ``chi non e' contro di noi (la chiesa) e' per noi''\footnote{Mc 9,38 \url{https://bibbiaedu.it/CEI2008/nt/Mc/9/?sel=9,38\&vs=Mc\%209,38}}. E cosi' come per questo mistero della divinita' di Gesu', allo stesso modo per tutti gli argomenti trattati in questo capitolo.


\section{Dal nichilismo scientifico, a Dio}
\label{nichilismoScientifico}
In questo capitolo ci concentriamo sulla descrizione ``nichilistica'' di Dio e della fede, per affermare che, anche pur adottando una visione materialistica dell'Universo, la fede rimane quell'insieme di principi, pratiche ed entita' psicologiche proprie dell'essere umano che, se adoperato a fin di bene, e' supporto e fondamento della vita.

\subsection{La materia}

E' la vita mera materia che si trasforma secondo leggi fisiche inamovibili? Se si, non ha forse senso massimizzare il piacere e minimizzare il dolore, rispettando gli altri per non avere problemi nella societa'? 
In verita', la vita e' molto di piu'. Per capirlo, prima ragioniamo dicendo che se la vita fosse vissuta solo su un piano materiale, o al piu' intellettuale, in una realta' percepibile solo con i sensi propri, allora la vita non avrebbe alcun valore.
Infatti, tutta la materia in se non ha valore, e' un pugno di sabbia che si trasforma, brucia, si solidifica, evapora, e cosi' via. Giocando a The Falling Sand Game\footnote{\url{https://en.wikipedia.org/wiki/Falling-sand_game}}, \footnote{\url{https://www.youtube.com/watch?v=L4u7Zy_b868}} si puo' vedere come semplici leggi, reiterate, possono produrre mondi complessi, ma che nella loro essenza non sono nient'altro che natura morta. Anche le esperienze piu' mozzafiato ed estasianti, allora, non sarebbero altro che un'esperienza di un pugno di sabbia.
Noi siamo affezionati a noi stessi, ma se tutto e' per noi solo materia, non e' il nostro attaccamento alla vita solo un senso di importanza prodotto dai segnali elettrici del nostro cervello? Che senso avrebbe preferire il piacere al dolore, la vita alla morte, se non siamo altro che un pugno di sabbia? Che senso avrebbe, preoccuparsi di rimanere in salute, di non essere poveri, di non essere derisi, se tutto il nostro attaccamento alla vita non e' che un automatismo, neanche scelto da noi stessi?
Potremmo illuderci di valere qualcosa, piu' di un pugno di sabbia, riuscendo in grandi imprese, conquistando l'ammirazione degli altri, confortandoci con l'avere cio' che riteniamo importante, tuttavia, in cuor nostro sapremo sempre, che cio' che stiamo vivendo non e' che una finzione.

\subsubsection{La santita'}

In noi pero' esiste una via d'uscita da questo abisso nichilisitico privo di vita. In noi risiede la potenzialita' dell'essere divini, dell'essere figli di Dio, di essere santi, di essere veramente forti e grandi. Questa forza e grandezza non si conquista con le forze e le intelligenze umane che si vantano di dominare la materia morta.
Un riflesso della nostra potenzialita' divina, si vede in noi che avendo cognizione di cio' che ci circonda e, soprattutto, di noi stessi e degli altri uomini, abbiamo emozioni che ci fanno essere felici o tristi, carichi o demoralizzati. Abbiamo la liberta' di scegliere di che farcene della nostra cognizione e delle nostre emozioni. Possiamo direzionare la nostra vita verso la creazione di un paradiso terrestre, o abbandonare tutto e tutti e lasciare che la terra si trasformi in un inferno terrestre. Niente nell'Universo, per quanto potente, ha queste potenzialita'. Ad esempio, nessuna particella, pianeta o galassia, ha la minima cognizione di cosa sia essa stessa, di cosa e' cio' che la circonda. Quindi, agisce seguendo ciecamente le leggi della Natura, senza liberta' alcuna. Inoltre, senza emozioni non le importa se venga distrutta essa stessa o se i suoi simili vengano distrutti. Non conosce la gioia, sua, e di niente e nessun altro. \\
I santi sono uomini che hanno realizzato a pieno la loro potenzialita' di bene, hanno mosso la materia ed utilizzato ogni forza dell'Universo, per favorire la gioia in se stessi e negli altri.

Se fosse facile essere santi, non sarebbe una condizione di vita cosi' speciale. Essere santi e' difficile perche' per ricercare il bene, molte volte bisogna sacrificare il proprio piacere o benessere, per rischiare e investire in un piacere e bene superiore. Come nel Dilemma del Prigioniero\footnote{Vedi ``Prisoner's Dilemma'' \url{https://en.wikipedia.org/wiki/Prisoner\%27s\_dilemma}}, chi e' veramente disposto a sacrificarsi per il bene della collettivita'? Chi e' disposto a sacrificare un forte piacere presente per un benessere futuro?

La santita' e' cosi' contrassegnata dal sacrificio e dal raggiumento di una pace e di un bene superiore, quasi mai comunemente vivibile. Qui e' dove si fermano le parole. La piena santita' e' impossibile per me da descrivere a chi non ne ha fatto una minima esperienza, perche' in primo luogo, e' come descrivere l'orizzonte, il cielo, il mare e i colori a chi non ha mai visto. La santita' ed il ``regno dei cieli'' e' cosi' trascendente che e' impossibile da immaginare senza averne fatto esperienza. In secondo luogo, io ne ho fatto una proprio minima esperienza e ancora ho molto da camminare per poterla vivere in pieno. Solo Gesu', e' riuscito a descrivere pienamente la santita' all'uomo. Lui era santo ed ha condiviso la sua santita' a tutti non risparmiando se stesso. Solo in Gesu' si puo' intravedere la luce originale della santita'. Aveva tutto nella vita, ogni benedizione, forza e grazia, eppure, si e' messo in gioco totalmente per il bene, sapendo fin da principio cosa questo gli avrebbe costato. Non e' come un povero che non ha niente e parla contro i potenti e promette ogni bene a tutti gli altri poveri che vogliono dargli ascolto nella sua rivoluzione aggressiva contro il potere aggressivo. E', piuttosto, come un miliardario, che ha tutto, e decide di parlare contro il potere, inimicandosi i ricchi, e decide di vendere tutto per i poveri, sapendo di diventare povero. Consiglia a tutti di fare lo stesso per vivere di nuovo in pace sulla terra. E fa tutto questo sapendo che nessuno lo ascoltera', e che diventera' povero, e che diventatolo, anche i poveri stessi poi lo criticheranno ed insulteranno per essere diventato povero ed aver promesso un regno di pace. Fa tutto questo perche' ama chi non e' amato da nessuno, neanche da se stesso.\\
L'inespribilita' della santita' spiega anche perche' sono inutili e vane tutte le critiche contro la fede religiosa mosse da chi non ha fatto esperienza di un'autentica fede \footnote{Tralasciando le critiche storiche mosse contro gli sbagli compiuti da peccatori nell'amministrazione delle istituzione religiose, riconosciuti dagli ultimi papi dell'ultimo secolo.}. Ad esempio, un ateo, o un edonista, o uno scientista che non hanno mai vissuto lo stato di purezza dell'anima, non possono che criticare la fede che vedono loro, non la fede per come realmente e'. Quindi, si soffermeranno sui limiti della loro concezione della fede e sulle difficolta' che loro hanno con essa.\\
Nel testo che segue procederemo nella spiegazione della santita' e della figura di Dio, parlando con gli stessi concetti e termini cari a chi e' vicino a una forma mentis nichilistica/scientista. Questo nella speranza di mostrare che la fede e' una disciplina umana che porta a dei risultati veri, e non raggiungibili per altre vie, anche a chi pensa che tutto e' materia e che non c'e' nulla oltre la realta' dei sensi. Fermo restando che per quanto le spiegazioni possano essere chiare, solo una pratica autentica di fede potra' far vedere lo splendore del regno dei cieli.

Noi abbiamo la potenzialita' di essere divini, ovvero di essere santi, essere co-autori di Gesu' di un regno di pace, personale e collettiva. 
L'uomo, oltre ad avere cognizione di se stesso ed avere emozioni, puo' avere un fine, uno scopo. Essere santi vuol dire far coincidere il proprio fine con l'unico scopo di generare vita, in se stessi e negli altri, per se stessi, e per gli altri. Vuol dire adoperare ogni cognizione e ogni forza per la vita, e scegliere profondamente, nel proprio intimo, in ogni momento, di cambiare e andare verso una direzione che permetta di vivere e far vivere, tra la vasta scelta di emozioni che si possono provare, solo quelle che generano pace e gioia. Cio' non significa sorridere sempre ed essere sempre felici. La vita potra' richiedere momenti tristi o spiacevoli, ma saranno vissuti in maniera non opprimente, ``soffrendo in pace''\footnote{Santa Teresina (lt 63 a celina 4 aprile 1889)}. Non si ci abbandonera' allo sconforto, o all'odio, ma con mitezza e forza interiore, si fara' un piccolo passo, dopo l'altro, per poi un giorno risalire di nuovo a contemplare le meraviglie della vita.  \\
La fede consiste nel credere che vivere una vita santa sia una direzione naturale e piena di vita, non solo una bella ideologia. La fede consiste nel porre la massima realizzazione personale nell'essere come Dio, il santo dei santi, colui che e', per definizione, il bene in persona. Chi ha fede crede che l'uomo puo' essere santo per sua natura e sua libera scelta.


\subsubsection{Dio}

Resta da chiarire la santa figura di ``Dio''. L'uomo e' come una lente d'ingrandimento che convoglia pensieri, sentimenti, intenzioni verso un'unico centro. Questo centro e' il dio che adoriamo intimamente nei nostri cuori. Anche se noi non ne siamo consapevoli, anche se noi diciamo di non credere, interiormente, inconsciamente, crediamo sempre a qualcosa nella vita. E piu' che a ``qualcosa'' e' un ``qualcuno'': abbiamo sempre un'immagine di chi vorremmo e dovremmo essere, e di chi in parte riusciamo ad essere, di come sarebbe essere senza limiti, senza miserie, di come sarebbe essere in pace e felici. Quindi, la vera domanda non e' se dio esista, ma e' a quale dio crediamo.

Gesu' ha creduto al Dio che ama e solo ama, tutti. Ha amato anche chi non ha avvantaggiato di alcunche' la sua vita, anche chi gli era di piccolo o grande ostacolo, anche chi ha posto a rischio la sua vita. Lo ha amato in quanto lo ha ritenuto suo figlio, prezioso come la pupilla dei suoi occhi. Non c'e' una identita' di bene superiore a questa.

Quanto abbiamo finora detto puo' sembrare strano e far domandare: Dio e' solo un'immagine? Chi osserva dall'esterno una persona che ama Dio, vedra' il Beneamato come una figura che ella crea ed ama, e dira' che non esiste fisicamente. Cio' non deve soprendere. Consideriamo una persona come ``anima'', ovvero nei suoi aspetti psicologici. Bene, dovrebbe sorprendere di piu' il fatto che per un'anima, qualsiasi persona lei possa amare, che sia suo padre, sua madre, il suo compagno o compagna, suo figlio o figlia, e perfino lei stessa, e', per l'anima, una figura, un'immagine che lei stessa crea ed ama. Non esiste niente per un'anima all'infuori di ella stessa. Tutto cio' che vede ed esiste, per lei, e' creato al suo interno dai segnali elettrici del suo cervello, che sia un oggetto esterno, o una persona, o il se'.  Tutto e' nell'anima una sorta di ``immagine''. Queste immagini non sono create dall'anima arbitrariamente, ma rispettano un fine. Se dalle luci e dai colori, l'anima crea dentro di se la cognizione visiva di un muro e dice ``esiste un muro davanti a me'', lo fa per amare se stessa (il se'), cosi', ad esempio, da non sbattere andando avanti verso il muro. Se crea dentro di se la cognizione ``ecco una persona che voglio bene'', ed anche ``ecco accanto a lei un pericolo'', e dice ``stai attento'', lo fa per amare l'altro. Allora, se un'anima ama Dio, pur essendo per lei un'immagine, cosi' come tutte le altre cose e persone, lo fa per un fine. Lo fa per direzionare con impegno la sua vita verso l'identita' che profondamente ama essere per se e per gli altri. Se ama Gesu', lo fa per essere un uomo o una donna caritatevole, che ama disinteressatamente, tutti.

Amare una persona fisica sembrerebbe piu' facile di amare un Dio invisibile ed incorporeo. Infatti, cio' che l'anima crea si basa su i suoni, la luce, i colori, le informazioni tattili, e poi anche sulle parole delle persone amate. Tuttavia, un'anima puo' affinare la sua creazione interna di Dio tramite la preghiera, una vita giusta e sana, praticando l'amor proprio, e praticando l'amore verso gli altri, riuscendo a compiere piccoli e grandi sacrifici per il bene comune, o di chi ha bisogno. 

Infine, Dio ``immagine'' e ``direzione'' non e' solo una idea della mente del credente, ma e' una parte viva e concreta del credente. Qualsiasi immagine un'anima crea e ama, che sia una persona o un oggetto, e' parte di se stessa. Il se' e' una parte dell'anima che rappresenta lo stato fisico ed emotivo dell'anima stessa. Un'altra persona e' un'altra parte dell'anima che rappresenta lo stato dell'altra persona. Lo stesso per un oggetto. Poiche' quando l'anima crea l'immagine del se' o di un'altra anima non puo' far altro che far riferimento alla propria esperienza del corpo, della realta' e delle emozioni, anche lo stato di un'altra anima e' internamente descritto tramite le proprie esperienze. Una persona che non ha mai sperimentato la rabbia in vita sua, quando vede un'altra persona arrabbiata, non puo' capire cosa sta' accadendo nell'altra persona. La vedra' muoversi in maniera agitata ed energica, parlare a voce alta, ma non capira' esattamente cosa sta' vivendo. Se pero' la prima persona ha almeno una volta sperimentato la rabbia, quando vedra' nell'altra persona gli stessi segni che si sono manifestati in lei stessa quando era arrabbiata, pensera', anche inconsciamente, a quello che sta' vivendo l'altra persona. Tuttavia, il suo pensiero si rifara' alla sua esperienza. In altre parole, quando ``capiamo'' gli altri, non stiamo che proiettando nostre sensazioni fisiche ed emotive, presenti o passate, negli altri. La realta' e' sempre filtrata dalla nostra esperienza. Quando ``capiamo'' un altro, ci ``mettiamo nei panni'' dell'altro, ma come possiamo: siamo sempre noi stessi vestiti con l'abito dell'altro. Nella maggiorparte dei casi questo meccanismo funziona perche' siamo esseri simili e viviamo sensazioni ed emozioni simili. Quindi la mia esperienza di rabbia concide molto con l'esperienza di rabbia dell'altro.

Se l'anima compie, per lo piu' inconsciamente, ma anche consapevolmente, queste operazioni di ``mettersi nei panni'' in maniera non oggettiva, allora, in psicologia, si parla di ``proiezione''. Infatti, in maniera non oggettiva, l'anima pensa di mettersi nei panni dell'altra anima o dell'altro oggetto, ma le caratteristiche dell'altro che vede non sono che uno specchio del se, di lei stessa. Quindi, l'anima crede di parlare dell'altro, ma non sta' che parlando di lei stessa. Questo accade, ad esempio, quando un'anima e' arrabbiata, ma non vuole ammetterlo a se stessa o agli altri, e quando parla con un'altra persona l'accusa di essere arrabbiata.
Se, pero' l'anima si sforza di compiere in maniera oggettiva l'operazione di ``mettersi nei panni'' di un'altra anima, allora l'immagine dell'altra anima, diventa una parte di se, ma distinta dal se, una parte con proprie caratteristiche. Cosi' come un innamorato quando pensa all'amata si accorge di molti aspetti e caratteristiche diversi dal se e propri dell'amata, che prima non aveva mai notato, o cosi' come quando uno scienziato pensa all'universo e si accorge di aspetti che vanno al di la' dell'intuizione usuale della realta'. Un'anima per conoscere oggettivamente un'altra anima che ama, devere imparare ad essere l'altra anima. Deve dedicare una parte di se e farla diventare sempre piu' fedelmente l'altra anima.

Riassumendo tutto quanto detto: qualsiasi entita' esista, per un'anima e' si un'immagine, ma anche una parte di se, che se curata in maniera oggettiva, ha caratteristiche proprie e distinte dal se'.

Anche Dio e' allora una parte del credente, non e' solo un'immagine o un'idea. Potrebbe inizialmente essere una proiezione, e il credente potrebbe confondere cio' che e' e crede lui stesso, con cio' che e' Dio e crede Gesu'. Pensando che i suoi pensieri, soprattutto quelli piu' forti sono i pensieri di Gesu' o di Dio.
Quando pero' l'anima pensa a Dio piu' obbiettivamente, pensa a come sarebbe essere realizzati nella pace, e nel caso del Dio caritatevole di Gesu', a come sarebbe amare infinitamente chi ancora non e' realizzato allo stesso modo. Pensando in maniera oggettiva, pregando, mettendo in pratica la carita', sperimentando la propria idea di Dio e amore, confrontandosi con l'idea di Dio dei profeti, dei Santi, e dei propri fratelli, il fedele puo' conoscere sempre piu' come e' essere veramente realizzati, in pace, capaci, e come si ama veramente. Quindi, andando al di la' della propria concezione iniziale di Dio, un'anima puo' tendere a conoscere sempre piu' a fondo Dio. In questo tendere, si unisce sempre di piu' a Dio, perche' per conoscere, cambia se stessa. Per poter concepire oggettivamente cio' che ama, deve imparare ad essere come cio' che ama.

Cosi' un credente raggiunge la santita' quando 1. il dio a cui crede e' il Padre di Gesu', 2. quando non c'e' piu' differenza tra lui stesso e l'immagine di Dio a cui crede. Il santo e' ``unito a Dio'': e' una realizzazione in terra di Dio, e Dio e' descrivibile tramite la persona del santo stesso. Come Gesu'. 


A conferma di tutto quanto fin'ora detto, Sant'Agostino affermava di aver trovato Dio dentro se stesso, non fuori, nel mondo, nelle altre persone, o nelle idee. Dio e' nella santita' dell'anima che crea la materia e il tempo, il se e ogni creatura per amore.

\subsubsection{La persona di Dio}

L'uomo per eccellenza santo e' Gesu'. Gesu' ha portato a compimento quello che i suoi padri, dal tempo di Abramo, avevano coltivato dentro di se. I padri avevano ricercato la santita', il privilegiare l'Io che e' puro bene. Gesu' ha pienamente realizzato in se tale santita'. Perche'? Perche' non c'e' amore piu' grande di dare la vita per i propri amici, per gli ultimi e i dimenticati da tutti, senza vendicarsi o fuggire dai nemici persecutori, ed avendo dentro di se ogni forza e grazia.

Gesu' e' santo, e' il ``verbo di Dio incarnato'', e' perfettamente unito a Dio. La sua volonta' e' la volonta' del Padre e la volonta' del Padre e' la sua\footnote{``Io sono nel Padre e il Padre e' in me'' Nuovo testamento, Giovanni, capitolo 14, versetto 11}. Questa unione permette di dire che Gesu' e' Dio. Solo Gesu' e' Dio, e non gli altri santi, perche' Gesu' e' colui che ha descritto tramite la sua persona il Dio dell'amore caritatevole, e gli altri santi hanno imparato ad amare tramite lui questo Dio, e i santi che verranno faranno lo stesso o, se non lo faranno immediatamente, poi riconosceranno che il loro Dio e' lo stesso di quello di Gesu', e uniranno le loro forze alla sua chiesa. Quindi, e' sufficiente porre a capo della Chiesa una sola persona e un solo volto: Gesu'.

\subsubsection{Dio senza la persona Dio}
Si ci puo' chiedere perche' e' necessario pensare a un'altra entita', Dio. Non basta vivere la vita e pensarla cosi' come e'?
Se pratichiamo l'amore si, non e' necessario pensare a una entita' distinta da noi. Qualsiasi discorso si fa su Dio si potrebbe convertire in maniera laica parlando dell'amore e di virtu' che discendono dalla tensione all'amore, e viceversa. Ad esempio, ``ama il Signore, che e' misericordioso'' si potrebbe convertire dicendo ``per una vita vissuta nel bene, sii generoso, ama tutti anche oltre i loro sbagli, perdonando'', e viceversa.
Tuttavia, se pratichiamo l'amore, pensare a Dio, offre dei vantaggi. Oggettificando la semantica di una frase in una figura che e' una persona (Dio), molte espressioni diventano piu' compatte, molte diventano piu' intuitive. Per una piena potenzialita' di pensiero, e' meglio essere in grado di pensare in modo laico ma anche in maniera teologica. Oggi giorno, il concetto di Dio e' stato bandito dalla cultura comune, e ognuno colleziona le frasi di saggezza celebri che piu' gli piacciono. Siamo immersi in una costante educazione alla verita', ma tutte queste verita' non hanno un centro, come invece lo e' la persona di Dio. Ancor peggio, ognuno predilige il suo sottoinsieme di verita', vivendo nella sua bolla individualistica. Eppure, saper pensare in termini teologici garantisce una maggiore profondita' di pensiero ed una intuizione agevolata per molte questioni che riguardano la vita, affrontate tramite l'amore e il bene.

\subsubsection{Sull'onnipotenza}

L'onnipotenza di Dio e' cio' che distingue nettamente Dio dall'uomo. Tuttavia, tale onnipotenza va' intesa in senso spirituale, non in senso superstizioso. Dio non e' una entita' magica che trasforma il piombo in oro, Dio e' una realta' spirituale che scaturisce dalla forza del nostro spirito che rifiuta il male e muove la vita verso il bene, e dalla purezza e dal coraggio della nostra anima che riconosce il vero bene, e accetta di perseguirlo. Inteso cosi', Dio e' onnipotente perche' puo' ridarci vita li' dove ormai ogni speranza e' perduta, dove abbiamo fatto delle nostre case e ricchezze i nostri sepolcri, delle nostre sventure e miserie le barelle a cui rimaniamo affezionatamente ancorati.

Dio e' anche signore della natura. Come puo' essere spiegata questa affermazione? Usando il ``principio della materia morta'' e il ``principio di santità'', esposti prima, si puo' spiegare ogni affermazione della fede, se si e' sufficientemente distaccati e semplici. Ad esempio, riguardo l'onnipotenza, possiamo dire che ogni piccolo gesto che da' come risultato la vita, equivale ad utilizzare l'intero Universo a fin di bene. Raccogliere dell'acqua piovana per dissetarsi e dissetare, equivale a far evaporare interi mari, far piovere, per poi raccogliere l'acqua. E' vero che non siamo noi a far piovere, ma e' anche vero che nessun atomo ``morto'' sta' facendo piovere, semplicemente accade che piove e nessuno, nell'universo nichilistico, e' autore di cio'. Tuttavia, se il nostro spirito e' santo, e' solo il nostro spirito che sta' desiderando nell'universo che  piova affinche' possiamo dissetarci e possiamo dissetare gli altri. E' solo il nostro spirito che sta' ``dando vita'' alla materia, le sta' dando un fine, e la sta' cosi' veramente facendo esistere. Infine, se il nostro spirito fosse santo, saremmo uniti a Dio, come dice San Giovanni della Croce\footnote{Cantico Spirituale, Manoscritto B, San Giovanni della Croce}, quindi potremmo concludere che ``Dio sta facendo piovere per dissetare noi e gli altri''. (nota \footnote{Se non fossimo santi, ma avessimo fede, non ci glorieremmo del fatto che piove perche' solo noi, a differenza della materia, possiamo apprezzare tale evento, ma attribuiremmo tale evento sempre a Dio, nella speranza di un giorno essere uniti a Lui, e gioire con Lui di cio'.}).

Cosi' come nell'esempio del raccogliere l'acqua, cosi' come per ogni altra azione e avvenimento nella vita. Se il nostro spirito fosse interamente santo, ogni cosa nell'Universo avrebbe un senso e un fine al servizio della vita, tutto l'Universo si muoverebbe per la vita, e solo per la vita. Non e' facile da immaginare pensando alle difficolta', ai pericoli e ai disastri che avvengono a causa della Natura, nella Natura. Tuttavia, i Santi possiedono uno spirito che gli consente di non vedere nulla di contraddittorio nella creazione, e nella volonta' di Dio. E' difficile da immaginare, ma si potrebbe ragionare come segue: la nostra anima e' unita al corpo, e' un tutt'uno con esso. Il corpo, e' parte dell'Universo, e quindi non puo' che amare l'Universo, anche quando e' di ostacolo alla stessa vita. Noi, per vivere, amiamo il nostro corpo, e quindi anche l'Universo. Allora, per vivere, siamo legati alle leggi dell'universo, e quindi, proprio per vivere, a volte siamo soggetti a vivere situazioni di sacrificio fisico.

Per concludere e riassumere, chi vive in santita', e' unito a Dio, e ogni evento o azione, che di per se non ha significato per la materia, prende vita tramite lo spirito di Dio, e diventa un'azione che genera vita. Per tutto questo Dio e' onnipotente.


\subsubsection{Sulla fede trascendente}
\label{SullaFede}
Potremmo analizzare ogni aspetto della religione usando i due principi della materia morta e della santita', e tutto acquisirebbe senso razionale. Tuttavia, questo procedimento e' lento e macchinoso. Il punto principale e' chiedersi, a questo punto, se la ragione ha priorita' sull'anima o se l'anima ha priorita' sulla ragione. Essendo la ragione un prodotto della mente, e non la mente un prodotto della ragione\footnote{la ``ragione'' e' un prodotto della rete neurale del nostro cervello}, possiamo affermare la seconda. Allora, come per Dante, che quando arrivato in paradiso deve fare a meno della guida di Virgilio (la ragione), ma affidarsi totalmente al suo amore per Beatrice (la fede), il credente arriva a questo nuovo orizzonte, dove cerca il bene dell'anima come bene primario. 

Quindi, si smette di ragionare rigidamente con schemi fisici e dimostrabili, e si comincia a sentire, a intuire, a credere.

%\subsubsection{Sulla scienza}
%E' facile pensare che con la tecnologia e l'ingegno si possano risolvere tutti i problemi materiali, e credere che tutti i problemi del mondo si possano pure risolvere allo stesso modo. Tuttavia, la soluzione primaria, a fondamento d'ogni altra proposta, e' Gesu'. Gesu', oltre ad essere stato un uomo, e' una identita' eterna, risorta. Essere Gesu' vuol dire essere in grado di accettare il proprio destino, anche quando, per amore degli altri, richiede di sacrificare se stessi. Essere Gesu' vuol dire mai rinnegare l'amore. Cio' detto vale in ogni sfera della vita, da quella lavorativa, fino a quella relazionale, con il proprio amato e amata, con i propri fratelli e sorelle, con i propri genitori, con chiunque altro. Addirittura, vale anche tra le varie parti del proprio io psicologico.
%
%Solo se saremo in grado di vivere la nostra vita come Gesu', il mondo potra' essere in pace, perche' ognuno potra' vivere in pace, con se stesso, e con gli altri, per quanto le sfide della vita saranno numerose e difficili.
%Allora, ogni nuova scoperta tecnologica e scientifica, sara' si utile e sorprendente, ma, rimarra' un ``contorno raffinato'' in confronto al ``pane disceso dal cielo'', unico piatto principale e necessario.

\subsubsection{Sul soprannaturale}
Niente e' sopra la Natura nella Natura. Tutto e' Natura nella Natura. Pero' ognuno di noi e' anche anima, e' come un punto, di dimensione zero, che ha una facolta' particolare: la possibilita' di scegliere. Anche nella condizione piu' stretta, di un universo deterministico, di cui ogni cosa e' predeterminata, o non determinabile, rimane sempre la possibilita' di scegliere se essere ok con la rotta che sta' prendendo l'universo o no.  Questo e' veramente sopra la Natura. Niente nella Natura, che non abbia anima, ha questa facolta'. Un buco nero puo' influenzare grandemente il suo ambiente circostante, puo' inghiottire innumerevoli stelle, ma non lo fa per scelta. Ne' vuole o desidera essere chi e', ne' vuole o desidera essere chi non e'.

\subsubsection{Una nuova dimensione: l'anima}
Se vogliamo descrivere completamente la nostra vita umana, oltre a cio' che percepiamo come oggetti che interagiscono con altri oggetti (il nostro corpo, gli oggetti dell'ambiente, gli altri corpi), dobbiamo descrivere cio' che percepiamo come anima che interagisce con se stessa e con altre anime. L'anima e' il risultato di processi elettrochimici, ma questi processi, sono reali, complessi e, poiche' generano la vita, sono sacri. 

Percio', oltre lo spazio e il tempo, un essere umano vive in una ulteriore dimensione: lo spazio interiore dell'anima.

Questo e' un altro ragionamento per concludere che la vita spirituale e' soprannaturale, al di la' dello spazio e del tempo.

La psicosomatica dimostra che la dimensione dell'anima puo' avere effetti sulla dimensione corporale. Il viceversa e' intuitivo. Il corpo influisce sull'anima: se non mangiamo, l'anima si ritrova a disposizione poche energie.


\subsection{Sullo Spirito Santo}
Abbiamo detto che Dio e' il centro del nostro essere. Dio e' l'immagine perfetta a cui tendiamo. Questa immagine, non e' solo esistente nei ``cieli'', ma tanto piu' diventiamo santi, tanto piu' s'incarna in noi stessi. Tanto piu' diventiamo santi, tanto piu' la nostra anima si unisce a Dio, e diventa una cosa sola con Lui\footnote{Cantico Spirituale, Manoscritto B, San Giovanni della Croce}.

Lo Spirito Santo allora e' quella parte di Io che realizza Dio in se. Comunemente, il nostro Io non e' santo, pero' a volte in una determinata condizione, tempo e spazio e' puro bene. Allora, si dice che e' lo Spirito Santo che sta' agendo in noi stesso, sta' realizzando Dio in noi stessi. Realizza concretamente, in terra, quel centro focale del nostro essere che e' Dio Padre, nei cieli. 

Lo Spirito Santo e' anche cio' che consente di ascoltare Dio dentro di se o nell'altro, e cio' che permette di parlare con Lui. Infatti, lo Spirito Santo, che e' Dio, non e' lontano da trovare. Se pensiamo a un Io che trova pace e gioia nel respirare, nel non nutrire alcun male verso se stesso e verso alcun altro, gia' troviamo lo Spirito Santo. Se riuscissimo, nella preghiera, a trovare questo semplice ma profondo stato d'essere, e ripartire da tale stato, allora tutto in noi e nella nostra vita si trasformerebbe in presenza e manifestazione di Dio.

La missione di Gesu' e' stata quella di donare e dare in eredita' lo Spirito Santo a tutti.

\subsection{In pratica}

Tutto cio' fin'ora detto puo' sembrare molto astratto. In pratica, cosa chiede la religione? Cosa vuol dire credere in Dio?

Vuol dire credere che una realizzazione dell'uomo e' l'amore. L'uomo puo' essere amore, se si mette in cammino nell'ardua salita dell'imparare veramente ad amare. E lo e' stato in Gesu', e lo e' parzialmente in chi crede in lui, e lo e' pienamente nei suoi santi. Vuol dire volere che la propria realizzazione sia proprio quella dell'amore, e dell'amore di croce, che ama e solo ama, in ogni condizione, di fronte ad ogni ostacolo, verso qualunque conclusione comporti per se.

%Credere in Gesu' e in suo Padre, vuol dire rispettare un unico comandamento: vivere, amando e solo amando, tutti.
%Vivere: non e' cosi' facile, nel mondo di oggi, ne' lo e' stato nel mondo di ieri. Crescere, superare ogni giorno difficolta' e sfide.

Tutto il resto e' una declinazione di questa premessa: amare e solo amare. Ad esempio,
\begin{itemize}
    \item Amare se stessi
    \item Amare gli altri
    \item Conoscere se stessi
    \item Curare le proprie ferite inconscie derivanti dall'infanzia
    \item Elevarsi dal proprio naturale stato narcisistico ed egocentrico
    \item Rispettare delle regole, come buona norma e guida nel cammino di crescita spirituale
    \item Unirsi agli altri nel cammino spirituale, comune ed unico, formando un'unica Chiesa
    \item Rinunciare a se stessi, per gli altri
    \item Vedere il lavoro, nell'ottica globale, come servizio fatto alla societa', per quanto umile e piccolo
\end{itemize}


\section{Amare}
\label{amareSe}


\subsection{Se stessi}

\epigraph{
    Amerai il tuo prossimo \textbf{come te stesso}
 }{Gesu'}


Amare se stessi e' fondamentale, cosi' come amare gli altri. Cio' che si impara nel tempo, dalla psicologia e dalla religione, e' che, in realta', tutti pensiamo che gia' ci amiamo tanto, ma invece, non ci conosciamo veramente, non sappiamo amarci, e addiritura in parte, inconsciamente, non vogliamo amarci. 
Amare se stessi e' sentire come stiamo e, in questo sentire, voler stare bene, accettandoci per come siamo e, al contempo, migliorandoci per essere cio' che e' auspicabile essere e cio' che e' naturale diventare.

Come spiegano Eric Berne ed Alexander Lowen (vedi capitolo \ref{chapRiferimenti} pag. \pageref{chapRiferimenti}), in questo mondo difficile, chi piu', chi meno, riceve degli insegnamenti inconsci distorti dai suoi genitori, che a loro volta hanno ricevuto dai loro genitori, e cosi' via. Questi insegnamenti, ci allontanano dalla nostra vera natura. Nei casi piu' estremi, degenerano in psicopatologie. Nei casi comuni, sono insegnamenti accettati dalla societa' in cui si vive, e quindi, nessuno, se non scruta e scava dentro se stesso si accorge che non sono veri. Un esempio, e' la cultura del divertimento del modello Americano, dove chi non si diverte e' marchiato come strano ed emarginato dal gruppo, e dove il sesso o altre cose serie e che hanno un profondo impatto nella persona, sono considerati come piaceri superficiali.

L'istinto dell'Io e' quello di dire ``io ho ragione, perche' sono Io''. Se l'Io si allontana dal sentire il se' e gli altri, allora dira' di avere ragione, di essere importante e di avere molto bisogno anche quando questo non sara' cosi'. Se percio' l'Io non si sottopone ad allenamento con esperienze in cui si confronta con gli altri e con se stesso, costantemente e con perserveranza nel tempo, l'Io si cristallizzera', fara' ricorso a soluzioni fuorvianti, e portera' la persona piu' verso la tristezza e la morte che verso la gioia e la vita. Amera' immagini e poteri, personaggi e cose che non saranno al servizio delle sensazioni ed emozioni vere della sua stessa persona e degli altri.

Amare se stessi e' quindi, intanto capire chi siamo e cosa proviamo, veramente, e questo, anche se il nostro Io dice di sapere benissimo chi siamo, cosa vogliamo e se stiamo bene o male, non e' semplice, richiede tempo e tenacia. Riusciti in questa impresa, si potra' conoscere il vero piacere, la vera gioia, ma anche mettersi in guardia dai veri pericoli. Questo piacere e questi pericoli sono quelli che i nostri sogni ci ricordano instancabilmente ogni notte, e che sono troppo difficili da esprimere a parole, e che a fatica il nostro Io durante la veglia riesce a comprendere se non li ha conosciuti, studiati ed attenzionati negli anni. Questo piacere e questi pericoli, sono quelli veri, che vanno al di la' di tutti i desideri e sogni che il nostro Io proietta e promette durante la veglia. Promesse come la posizione sociale o lavorativa, economica o famigliare. Noi, in verita', valiamo molti ordini di grandezza in piu' rispetto a tutte queste cose.

Quando riconosciamo che con nessun nostro impegno, che con nessuna richezza o potenza, per quanto grande, riusciremo mai a soddisfare la nostra anima, perche' limitati di fronte a qualcosa di illimitato, e quando al contempo riconosceremo il valore di ogni cosa, per quanto piccola, che ci viene incontro o da noi stessi, dagli altri o dallo stesso mondo che sempre disprezziamo, allora comincieremo a metterci in cammino verso il veramente amare noi stessi.

Amare se stessi e' legato ad amare gli altri, in quanto, cosi' come il se' e' parte del Tutto, anche gli altri lo sono, e non si puo' veramente gioire se si ama una sola sua parte. Di questo parleremo in seguito.

\subsection{La non dipendenza dall'esterno}

Molte volte pensiamo che la causa della nostra mancata o possibile pace sia un'altra persona. Tuttavia, per quanto un'altra persona possa favorirci od ostacolarci, la pace risiede nel propriamente amare noi stessi. Nel momento in cui ci sentiamo veramente amati, non abbiamo piu' bisogno di altro. Le altre persone, se ci amano, diventano fonte di una maggiore felicita', ma non sono la condizione necessaria alla pace.
Questa semplice ma difficile verita' e' per lo piu' ignorata da tutti, e le persone se la prendono con le altre persone, per motivi piccoli o grandi, pretendendo che tutti e tutto si conformi alla loro vita, in misura maggiore o minore. Cosi' facendo pero', sprecano molta della loro preziosa energia nel cambiare cio' che non si puo' e non si deve cambiare con la forza (gli altri). Energia che avrebbero potuto impiegare per ascoltarsi, conoscersi ed amarsi meglio, e veramente soddisfare i loro bisogni e sogni. Ed anche se inesperti, gia' averci provato ed esserci riusciti solo un po' avrebbe migliorato moltissimo la loro condizione ed aperto nuovi orizzonti di vita.

Amarsi indipendentemente dagli altri, e' difficile, e a volte doloroso. Questo perche' da bambini impariamo a vivere in dipendenza dai genitori. Per qualsiasi nostro bisogno basta che piangevamo e tutto si risolveva automaticamente. Quando cresciamo la realta' non funziona piu' cosi', anche nelle nuove relazioni con le altre persone, e questo a volte non riusciamo a capirlo, gestirlo, sopportarlo. Dobbiamo inoltre stare di fronte al nostro narcisismo, che non permette di mettere da parte i nostri bisogni per dare spazio ai bisogni degli altri perche' ci fa vedere e credere profondamente perfetti e di un valore superiore a quello di qualsiasi altro. Quando, invece, siamo dei semplici esseri, con bisogni, difetti, a volte cattivi sentimenti, debolezze e sogni, come tutti gli altri esseri. Dobbiamo stare di fronte al nostro egocentrismo, che ci fa vedere come il centro di tutto, dove se prima non parliamo di noi stessi, non parlaremo ne' faremo parlare gli altri. Solo noi re, gli altri sudditi o nemici se si oppongono ai nostri piani. Sia il narcisismo che l'egocentrismo ci fa dipendere in maniera non sana dagli altri per stare bene. Senza l'altro con il narcisismo non possiamo affermare la nostra immagine elevata di noi stessi. Senza l'altro, con l'egocentrismo non avremo chi si sorbisce le nostre richieste e comandi, o chi prende in carico i nostri piani. Quindi paradossalmente, quando disprezziamo gli altri, per qualche ragione, li', ne siamo piu' dipendenti.

E' inutile dire ``io non sono cosi' '', ne' dire ``io, irrimediabilmente sono cosi', ed anche peggio' ''. La vera domanda e' ``l'altro come e' stato con me?''. Chiesto questo, facendo introspezione e mea culpa, sicuramente riconosceremo che non siamo perfetti. Il percorso di santita' inizia proprio dalla consapevolezza che non siamo santi, e mai potremmo esserlo in terra. Solo Dio padre e' santo, come diceva San Francesco. 
Cosa c'entra la santita' con l'amare se stessi? Il rimanere fermi nel proprio essere, e non crescere tendendo alla santita', e' cio' che ci rende dipendenti dall'esterno e da altre persone per ottenere la felicita'. Se rimaniamo narcisi, solo se siamo riveriti dagli altri e rispettati sempre ci sentiremo a nostro agio. Mai saremo liberi di essere cio' che siamo, nelle nostre debolezze, fuori dal coro, e dalle mode e gli usi dei tempi. Se rimaniamo egocentrici, non potremo mai contare sull'aiuto degli altri, e quando lo riceveremo non lo apprezzeremo, e quando non lo riceveremo considereremo gli altri estranei o nemici. Ad ogni modo, rimarremo deboli. Inoltre, per rimediare, rimpiazzeremo il nostro ego con potenze del mondo, come soldi o tecnologie, con persone che considereremo importanti, come capi, partner, vip e persone che ``stanno meglio di noi''. Ma questo non ci permettera' di amare noi stessi, perche' saremo schiavi di un altro Io, stando appresso ai suoi problemi. Solo l'Io di chi ci ama e sa amarsi indipendentemente dagli altri, potrebbe servirci, ma solo se saremo disposti ad entrare nella porta stretta e percorrere il cammino difficile della crescita, indispensabile per ricevere il suo amore.

\subsection{L'Altro}
\label{altrui}
\epigraph{
    Amerai \textbf{il tuo prossimo} come te stesso
 }{Gesu'}

Capito come amare se stessi, amare l'altro dovrebbe essere una conseguenza naturale. Infatti, tutto si racchiude nei seguenti versi

\centerpoemon{quanto stai bene.}
\begin{poem}
    Vivo,\\
    quando sto' bene.\\
    Amo,\\
    quando stai bene.\\
\end{poem}

Tuttavia, facciamo come esercizio quello di declinare cosa vuol dire, un po' nel dettaglio, amare l'altro.

Di norma \emph{siamo} solo per noi stessi. Se amiamo gli altri, pensiamo a loro, ci dedichiamo a loro, \emph{siamo} anche per loro. Le forme, i suoni, le sensazioni e i concetti acquistano un senso per l'altro. Non c'e' piu' solo il nostro corpo, il nostro se', c'e' anche il corpo dell'altro, cio' che lui sente, il suo se'. La nostra anima da' sostanza al nostro se' e al se' altrui, e ci fa' stare vicini alle emozioni ed alle sensazioni fisiche nostre e dell'altro. \\
Noi acquisiamo un particolare significato per l'altro, un significato che puo' non essere uguale a quello che a nostra volta diamo a noi stessi. A volte ci da' piu' importanza (ad esempio, una madre), a volte meno importanza. Tuttavia, se veramente \emph{siamo} anche per l'altro, comprenderemo il suo punto di vista. E il suo punto di vista sara' la Sua verita'. Se quel suono e' per lui dolce, sara' dolce. Non diremo ``lui sente quel suono come dolce, ma in realta' e' per me un normale suono''. Perche', se lo amiamo, cio' che e' suo e' anche nostro.\\
Viceversa, l'altro acquista un significato per noi. Se questo e' normale, poiche' tutti diamo un significato a tutto e a tutti, il divino si raggiunge quando il significato che gli diamo e' lo stesso, o migliore, di quello che lui da' a se stesso. Cioe' quando quello che vogliamo dall'altro non diventa un obbligo innaturale per lui, quando non pretendiamo, quando lo accettiamo e soffriamo segretamente per come e', senza emarginarlo dalla nostra vita. Ma possiamo fare di piu', possiamo vederlo migliore di come lui vede se stesso, riconoscendolo a volte come fratello, a volte come figlio, a volte come padre.

Amare l'altro vuol dire risolvere il prima possibile i nostri mali, o quanto meno, smettere di dolercene, per dare spazio alla pace ed al piacere di esistere dell'altro. Non importa quanto saremo bravi o splendidi agli occhi dell'altro. 
La quantita' di benessere e piacere che l'altro avra' sara' determinata dalla sua storia, da cio' che cerca nella vita, e da quanto noi siamo cio' che lui cerca.
Se e' un santo, ringraziera' anche se avra' ricevuto nel concreto poco, se e' in preda alle tempeste della vita disprezzera' la sua vita e di riflesso anche noi, ma proprio perche' la sua vita ha valore, dovremmo amarlo ancora di piu'.
Aver presente questo, permettera' di realmente contribuire al compimento del suo desiderio, e non nel compimento di altre cose che non c'entrano, come 1. il soddisfacimento di nostri desideri, 2. il compiacimento di suoi desideri superflui o non autentici, probabilmente per lusingarlo e farci belli ai suoi occhi, o per vederci noi stessi bravi.
La coppia desiderio-amore, trascende limitazioni fisiche. L'altro potrebbe essere felice anche solo del nostro raccogliere un fiore, e allo stesso modo, anche solo del fatto che noi desideriamo cio' che lui desidera. Viceversa, l'altro potrebbe non essere mai contento di nulla.
Non c'e' limite a cosa possiamo essere, dare e fare per l'altro. L'unico limite e' l'amore per noi stessi. Ad esempio, se sentiamo troppa fatica, vorremmo interrompere il lavoro che stavamo facendo per l'altro.
L'altro non deve alcunche' a noi per qualsiasi cosa abbiamo fatto e faremo per lui. Se lo facciamo, e' perche' il farlo ci gratifica, non per secondi fini. Se e' vero che facciamo qualcosa per l'altro e non per noi stessi, non ricercheremo premi o ricompense.

Ne' nulla e nessuno puo' imporci o allettarci falsamente di amare. Infatti, dovremmo amare solo se veramente stiamo bene con noi stessi\footnote{a meno che non ci siano emergenze o urgenze, e bisogna intervenire anche se non ci sentiamo di farlo}, e se veramente capiamo che potremmo non averne alcun guadagno, ne' emotivo, come, ad esempio, quando giochiamo con un bambino non per provare emozioni di gioco\footnote{giacche' un adulto preferisce altre attivita' per svagarsi}, che invece il bambino prova, ne' sensoriale, come cucinare e far compagnia all'altro anche se noi abbiamo gia' mangiato. Dovremmo amare solo se riusciamo in primo luogo a stare bene con noi stessi e con gli altri. Raggiunta questa pace, sentiremo una pace anche maggiore nel riuscire a condividerla.

Stare bene con noi stessi, non significa stare bene, in salute, ed in emozioni. Significa, che pur nella tempesta, stiamo andando nella giusta direzione, e possiamo dire ``Io sono''. Paradossalmente, anche se non stiamo bene in salute ed in emozioni, se non escludiamo l'altro, il desiderio di amarlo, sara' un forte richiamo e motivazione allo stare bene, questo ci dara' speranza, coraggio e forza. 
E' anche da aggiungere, che la vera forza non e' nel non avere debolezze, difetti e paure, ma nell'amare pur essendo deboli, difettosi e piccoli di fronte a cio' che prospetta la paura.

Possiamo desiderare qualcosa dall'altro. Ma questo desiderio sara' sano e grande e potra' realizzarsi solo se coincidera' con l'amore dell'altro. Se desidero un dolce da un pasticciere che incontro, solo se lui ha piacere di fare un dolce per me potro' gustare un vero dolce. Se non lo ha, per qualsiasi motivo, non sara' la stessa cosa. Potrei 1. proporgli del denaro o allettarlo con altri premi in cambio, 2. andare da un altro pasticciere\footnote{oppure armarmi di forza di volonta' e fare il dolce da me}, 3. dire che il mio desiderio e' fuori luogo. Tuttavia, nessuna delle tre soluzioni e' in realta' una soluzione. La vera soluzione e' dire: io ho questo desiderio, ma ho bisogno che anche lui desideri amarmi, fino ad allora il mio sara' un desiderio, non una realta'. Solo cosi', godro' della vita presente e viva, e conoscero' tutti i desideri miei che, inconsapevolmente, io, gli altri e la natura stanno gia' soddisfacendo.

\subsection{L'inconscio}

Molto spesso avremo la sensazione di aver capito come vivere, come amare noi stessi e gli altri, ma poi il giorno dopo, o un'ora dopo, o un attimo dopo, faremo gli stessi sbagli di prima. Anche se abbiamo la consapevolezza di come dovremmo essere, non lo siamo. Su questo gioca molto il nostro inconscio. Fino a quando non andremo a toccare le corde profonde che vibrando determinano il nostro essere, rimarremo sempre gli stessi. 

Il piu' delle volte rifiutiamo le nostre emozioni inconscie perche' dobbiamo vivere nella societa' e vogliamo farci rispettare dagli altri e, soprattutto, per seguire le direttive parentali che abbiamo ricevuto da piccoli e che, a loro volta, i nostri genitori hanno ricevuto da piccoli\footnote{Vedi il concetto del ``copione'', di Eric Berne in ``Ciao e poi...''}. Cosi', molte volte andiamo, inconsapevolmente, contro la nostra stessa natura per, impropriamente, uniformarci alla tradizione, alla morale, alle leggi, al gruppo, al partner, con la speranza cosi' di sopravvivere e vivere serenamente. Facciamo cio' giustamente. Non e' essendo trasgressivi, immorali, criminali ed asociali che vivremo meglio. La tradizione racchiude insegnamenti preziosi, la morale norme di convivenza pacifica, le leggi linee guida fondamentali per vivere tutti in un mondo complesso, il rispettare la cultura del gruppo serve a rispettare i singoli che vivono bene in quella cultura e ne traggono benefici, e cosi' facendo imparare anche noi a vivere bene nel gruppo, amare il partner serve a camminare felicemente curando l'affettivita' e l'intimita'. Percio', la cultura della ribellione e della trasgressione, non puo' pienamente risolvere i problemi personali che si hanno con ognuno degli aspetti del vivere elencati. L'unico modo e': 1. capire quali emozioni non riusciamo a vivere 2. scoprire come viverle ed imparare a viverle nel rispetto e con la collaborazione degli altri. Tutto cio' e' molto difficile, ma e' la vera soluzione, che porta immensi benefici. 

Essere a confronto con il proprio inconscio e' difficile. 
Quando pensiamo, chi ha scelto cosa stiamo pensando? Noi, diremo. Ma in realta', stiamo pensando e basta. Pensiamo e solo pensando poi diventiamo coscienti del nostro pensiero.
Quando viviamo un'emozione, chi ha deciso di provare quell'emozione e non altre, altrettanto valide emozioni, che potevamo comunque provare nella stessa situazione? Ad esempio, perche' proviamo rabbia, e non delusione (o viceversa)?
Allo stesso modo, chi ha deciso di farci sentire importanti dei bisogni, o desideri, che nello stesso momento e nella stessa situazione altre persone non ritengono importanti? Il punto e' che noi viviamo per lo piu' inconsciamente, \emph{pensiamo}, \emph{agiamo}, \emph{siamo} istintivamente. La coscienza e' una piccola parte del nostro essere, dove mettiamo a vaglio quello che stiamo facendo e ci sforziamo un po' a mantenere delle scelte coscienziose. Possiamo anche sforzarci molto, ma sarebbe un lavoro infruttuoso. La coscienza puo' conoscere molto poco di noi stessi, solo i maestri spirituali dopo una vita passata ad ascoltarsi ed a camminare rettamente capiscono i segreti della loro anima. Nell'atto pratico, soprattutto nella frenesia del mondo moderno, noi non capiamo praticamente nulla. Possiamo avere l'illusione naturale di avere il controllo della nostra vita e di sapere chi sono stati i colpevoli di eventi avversi, ma nella realta', viviamo costantemente al di la' dei nostri pensieri, dei nostri propositi, del nostro controllo, e cio' non dipende da quanto gli altri siano avversi. In conclusione, la nostra coscienza e' solo un riflesso debole della nostra completa volonta': l'inconscio.

Poiche' noi siamo il nostro inconscio, non basta pensare ``voglio essere cosi', piuttosto che cosi' '', ne' fare anche mille azioni verso la direzione scelta. Infatti, non facciamo alcuno sforzo per essere come siamo, che ci piaccia o no, invece, lo sforzo e' infinito nell'essere cio' che non siamo. L'unico modo per cambiare e' cercare i bisogni autentici, inconsci, che non riusciamo neanche a pensare. Non riusciamo a vederli correttamente, perche' altrimenti se avessimo chiare le idee e la visione loro, li avremmo gia' soddisfatti in pienezza con poco sforzo. In questa ricerca, si puo' procedere da soli, a mani nude, ma non si puo' che trovare la psicoterapia come ottimo strumento di conoscenza e lavoro interiore. E' anche da notare, che il lavoro interiore e' diverso dalla matematica e dalla tecnica: li' basta un buon libro, qui bisogna soffrire molto, perche' si intravede che si puo' vivere in maniera enormemente migliore se fossimo diversi, ma non possiamo perche' non lo siamo. Soffrire da soli e' molto piu' difficile rispetto a soffrire ascoltati da chi da' valore e significato alla nostra sofferenza.
``Psicoterapia'' vuol dire, cura dell'anima, e riguarda terapie realizzate con strumenti psicologici quali il colloquio, l'analisi interiore, il gruppo, ecc., per cambiare quei processi psicologici che sono causa di un malessere o di uno stile di vita controproducente, e connotato spesso da sintomi come ansia, depressione, fobie, ecc. (tratto da Wikipedia).
La psicoterapia, pero', non e' solo per chi con fatica vive una vita normale. E' strumento efficace anche se si vogliono raggiungere livelli di realizzazione superiori, se si vuole una vita piu' autentica e piena.
E' da notare che negli sport, atleti professionisti fanno un percorso di psicoterapia per superarsi. Ad esempio, ``Il tiro a volo è uno sport dove l’errore e' fatale e si entra in finale per un piattello in piu' o in meno. Anche un semplice battito di ciglia imprevisto, un pensiero che sfugge, l’emozione di un momento, possono rovinare una prestazione che sembrava perfetta.''\footnote{\urlOrig{https://www.igf-gestalt.it/wp-content/uploads/2013/07/Gestalt-Mental-Training-nel-Tiro-a-Volo-BERNARDI-tesi.pdf} ``Gestalt Mental Training nel Tiro a Volo. L'applicazione dei principi della Psicoterapia Gestalt nell'allenamento mentale con un atleta del tiro a volo''}.\\
Niccolo' Campriani campione di tiro a segno, dopo una delusione in un campionato in Cina e alcuni anni di empasse, ha superato dei suoi conflitti interiori con lo psicologo Edward Etzel, e raggiungendo un approccio diverso al tiro, piu' libero da suoi blocchi, ha vinto i campionati mondiali di Monaco nel 2010, le Olimpiadi nel 2012 e nel 2016.


%\subsubsection{L'infinito e la preghiera}
%
%Se i nostri desideri o bisogni sono autentici, necessari o, se non necessari, non superflui, allora Dio li soddisfera', anche se richiedono di spostare la Luna. Cio' avverra' senza richiedere sforzo, senza forzare alcun essere, e senza sovvertire alcunche' nella natura.
%
%Il punto di partenza e' ascoltare, conoscere e vivere il piu' possibile i propri desideri e bisogni. Infatti, apparentemente sappiamo cio' che vogliamo, ma in realta', siamo un cervello che poco conosce i bisogni del Se', un cervello che ascoltando, studiando ed amando il Se', puo' comprendere nel tempo sempre di piu' ed in profondita' i bisogni propri e degli altri.
%
%Per ascoltare, un modo naturale e' quello di lasciare totalmente libera la propria o altrui preghiera. Comprese richieste infinite, od incomprensibili, che pero' non siano orientate all'odio. Per parlare di quello che vorremmo o di cui abbiamo bisogno, e' a volte piu' naturale parlarne senza considerare limiti, sapendo che, almeno Lui, ascolta nel profondo di noi stessi, e comprende la natura del nostro desiderio, che sia la cosa realizzabile fisicamente cosi' come noi immaginiamo o che sia soltanto realizzabile nei cieli. In matematica, questo e' un modo di fare molto ricorrente e importante. Ad esempio, nell'algebra si dice ``sia $x$ la soluzione'', e poi si tratta $x$ come se fosse gia' concreta e disponibile, e studiando l'equazione si comprende sempre di piu' la natura del problema, per arrivare ad una soluzione che e' coerente con gli assiomi di partenza. Nella preghiera, l'assioma di partenza e' l'Amore disinteressato. Se non esiste soluzione, non e' perche' Dio non vuole accontentarci, ma perche' vuole mantenere vero l'assioma fondamentale dell'Amore per ogni suo figlio. 
%
%Fatto cio', sentito e maturato a fondo il desiderio o bisogno, si arriva al punto di essere pronti a mettersi in gioco, ad aprirsi agli altri e fidarsi, o ad essere se stessi anche quando gli altri sono diversi da noi. E, punto fondamentale, saremo veramente pronti quando considereremo che il ``fallimento'', la possibilita' che il desiderio puo' non realizzarsi in terra, fa' parte del normale sviluppo del nostro desiderio stesso. 
%
%Il fatto che non si realizza, non e' perche' la Natura o gli altri, conoscenti o estranei, si oppongono ad esso, piuttosto e' perche' nel nostro desiderio non abbiamo incluso i loro desideri. Non abbiamo incluso il desiderio di quel masso di stare dove e', e cosi' sembra che ostacola la strada, non abbiamo incluso il desiderio di quella persona di reclamare la nostra attenzione per qualcosa di importante per lei, anche se vorremmo dedicarci ad altro, di quell'altra di difendere le sue idee, a cui sta' dedicando da anni la sua vita, anche se lei non si sta' curando delle nostre idee, di quell'altra di appoggiarsi su di noi, anche se noi abbiamo bisogno di appoggiarci su qualcuno, di quell'altra ancora di scartarci per portare avanti i suoi liberi progetti, anche se avrebbe fatto molto bene a noi continuare nel nostro ambito ruolo.
%
%Ancora, molte altre volte non si realizza perche' nella nostra volonta', non abbiamo incluso noi stessi! Quindi la cosa ci va' male, perche' non desideriamo cio' che nel profondo vogliamo. Il nostro desiderio non e' autentico, e la nostra volonta' non ancora comprende e copre pienamente la verita' del nostro cuore. Cosi' molte volte, ci affatichiamo molto su false mete, falsi idoli, senza mai raggiungere la pace e gioia ambita, ma ottenendo soltando vittorie, di cui il giorno dopo dimentichiamo l'euforia della gloria momentanea.
%
%Quando sentiamo e maturiamo nell'intimo e nella sincerita' il nostro desiderio, ascoltiamo il desiderio altrui, non ci opponiamo ad esso, ma piuttosto, rispettiamolo perche' nato dalla libera volonta' di un essere pari a noi, e comprendiamolo perche' anche i desideri piu' neri sono rivolti a soddisfare i bisogni necessari e non necessari dell'altro\footnote{i desideri neri a differenza dei desideri luminosi tentano di soddisfare i bisogni in modo nocivo, per gli altri o per se stessi}, bisogni che abbiamo anche noi, e che, anche se non abbiamo, potrebbero naturalmente nascere in noi nel futuro; accogliamo la parte del desiderio altrui che e' anche nostra, distinguiamo e separiamo da noi la parte altrui che non e' nostra, e, infine, interagiamo col ``desiderio'' della materia e della Natura ascoltando ed impiegando il nostro corpo, nel piacere e nella fatica; facendo tutto questo, allora senza forzare ne' noi stessi, ne' gli altri, il desiderio si realizzera' certamente e in maniera grande. Perche', cio' che si realizzera' e' il Suo desiderio, che non conosce ostacoli e di cui tutto l'Universo gioisce. 
%
%A volte, realizzare il desiderio sara' molto faticoso. D'altronde, noi siamo esseri finiti con risorse limitate. Tuttavia, se mettiamo da parte il nostro ego, che dice ``io non ci riesco'' oppure ``io ci \emph{devo} riuscire, a \emph{tutti i costi}'', allora quello che rimarra' sara' il desiderio, e gia' sentirlo fino in fondo e non abbandonarlo, appaghera' l'anima. Sara' piacevole poi mettersi in moto, con le proprie forze per realizzare, per creare, per mantenere. Se risultera' impossibile raggiungere l'obbiettivo, sentiremo tuttavia che l'obbiettivo non sara' stato trascurato, e raggiungeremo una nuova serenita'. Se l'obbiettivo e' forte e necessario, ci fermeremo in pausa solo quando subentrera' la stanchezza e i limiti dati dalla fatica. Se ci sara' un rischio di un incidente che potrebbe danneggiarci, e se il bisogno che il desiderio vuole soddisfare sara' primario, allora sara' meglio correre il rischio, piuttosto che certamente morire nell'anima.
%

%\section{La Natura}
%
%Cosi' come l'energia ne' si crea ne' si distrugge, perche' l'energia ceduta da un corpo viene assorbita da un altro corpo, anche qualsiasi perdita piccola o grande o totale della propria vita e' riacquisita da un'altra persona o da altre persone o dalla Natura. Questo trasferimento di vita, avviene a volte volenti o a volte nolenti. In Cristo, si diventa sempre volenti di questi trasferimenti, che diventano doni, anche se non ricambiati, ma parliamo adesso in particolare dei trasferimenti verso la Natura. 
%
%La nostra anima ama il corpo suo e quello di chi altri ama. Il corpo e' fatto di materia, e' una parte della Natura. Il corpo rispetta e soggiace alle leggi dell'Universo perche', come parte della Natura, ama tutta la Natura, e per questo i suoi atomi sono sempre fedeli agli altri atomi. Quando ricevono energia fisica, l'assorbono, quando hanno energia la cedono agli altri atomi.
%Cosi' il corpo ha un peso e, senza altre forze, tende verso il centro della Terra. Il corpo acquisisce calore dal Sole e cede calore nella notte, se lontano da un fuoco. E cosi' via, il corpo interagisce con tutta la materia, cosi' come la materia interagisce con altra materia.
%
%A volte il corpo assorbe troppa energia, ad esempio, quando riceve troppa energia meccanica diciamo che abbiamo sbattuto contro qualcosa, e sentiamo dolore. A volte, l'energia termica ricevuta e' troppa, e sentiamo caldo o ci bruciamo, e se invece e' troppa quella ceduta sentiamo freddo. A volte, non abbiamo piu' energia per i nostri muscoli e per l'organismo per la fame e ci sentiamo deboli. Nei casi estremi, dobbiamo affrontare dei drammi, perche' l'amore del corpo verso la Natura, ci priva seriamente della nostra vita. In ogni caso, la nostra anima puo' pensare alla Pace, se si rende consapevole che Lei ama il corpo. Ogni cosa che le accade o deve affrontare nella Natura, e' dettata dal suo amore verso il corpo. Non e' quindi lei in castigo esigliata nella terra, piuttosto lei vive nella Natura e con la Natura, proprio perche' vive il suo forte desiderio di amare il corpo e quindi anche la Natura. 
%
%Il corpo, anche se spiritualmente non e' il fine dell'esistenza, e' il punto di partenza fondamentale. Non possiamo esprimerci senza parlare, senza usare le corde vocali, o senza fare dei segni muovendo il corpo. Non possiamo voler bene ad un'altra anima senza pensare al suo corpo, pensando a dove si trova, se sente freddo o caldo, se e' stanca o riposata, se e' vicina ai corpi di chi vuole bene o no. Ne' possiamo essere vicini a colei che soffre rimanendo nelle nostre stanze, ma muovendoci nello spazio per raggiungerla, attrezzandoci per affrontare le distanze, il freddo, la fame, il tempo, e una volta raggiunta, abbracciarla, ascoltare la sua voce e vedere cio' che vede.
%Solo con il corpo possiamo pregare o lodare Dio. Solo con il corpo possiamo vivere la pace e la gioia di Dio, e possiamo condividerle agli altri. E' vero anche che a volte la fede chiede di andare oltre il corpo, ma cio' non significa abbandonarlo, ma accentando i suoi limiti, continuare ad amare col nostro cuore.
%
%Dal punto di vista psicologico il corpo, e di conseguenza tutta la Natura, e' lo strumento fondamentale tramite cui l'anima ama, e' cosi' fondamentale che l'anima e' anche corpo e Natura: l'anima vivente non e' come pensavano gli antichi distaccata dal corpo e dalla Natura, l'anima ama cosi' tanto il corpo che e' un tutt'uno. Ella e' influenzata dal corpo e il corpo dall'anima, in un intreccio forte e stretto, \footnote{
%    \url{https://it.wikipedia.org/w/index.php?title=Unit\%C3\%A0\_psicofisica&oldid=121651506}
%    La psicoanalisi nella sua globalità è l'insieme degli studi che analizzano le relazioni tra la dinamica ormonica e la costituzione della personalità psichica. 
%}, il corpo e' influenzato dalla Natura e la Natura dal corpo.
%

\subsection{L'Io psicologico e Dio}

\label{DioScientificamentePoesia}
\label{DioScientificamentePsicologicamente}

Considera il concetto \emph{Io} in maniera astratta, non strettamente come te stesso, ma come unita' psicologica, unita' presente, in forma diversa, in te e in ognuno. Scientificamente, allora, definiamo Dio come l'Io di \mbox{Gesu'}, un Io che da' completa, infinita, pace e gioia all'\mbox{anima} che accoglie il suo amore. Fa questo, tramite solo la pace e la gioia di ognuno e di tutti, che siano amici o nemici. E questo in ogni tempo, luogo e situazione che l'anima vive e affronta. Non fa questo tramite l'infelicita' minima di alcuno o alcuna. 

Che un Io possa essere considerato da chi lo ama (da se stesso o da altri) come onnipotente, non e' una sorpresa. Infatti, la mente crea la realta' (vedi appendice \ref{menteCrea} pag. \ref{menteCrea}). Noi esperiamo una realta' che noi stessi creiamo. Tuttavia, la vera onnipotenza e' quella che da' pace e gioia all'anima propria e altrui. Per questo, solo un Io santo, che crea la vita in se e negli altri, tramite e solo tramite l'amore disinteressato, puo' essere definito come Dio.

Nell'atto pratico, Dio, puo' essere presente in misura maggiore o minore in se stessi e negli altri, a volte di piu' in alcuni momenti e situazioni, a volte di meno in altri, a seconda di quanto l'Io coincide con la definizione data di Dio.

Se in ognuno puo' esserci Dio, ci sono molteplici Dei allora? Anche se sembrano molteplici, essendo orientati all'Amore, tali Io sono concordi ed uniti, e percio', come in una musica note diverse compongono un'unica melodia, sono un unico Io, e cosi' come ciascuna membra di un corpo e' sacra, tanto quanto lo e' tutto il corpo, ciascun Io e' Dio, nella misura della sua santita'.

Un Io di norma non coincide tutto con Dio. Dio e' cosi in parte nei cieli, in parte in terra. Un Io puo' allora ricercare la sua completa unione con Dio. Infatti, esistendo l'anima con i suoi bisogni e con i suoi ``desideri necessari'' e' possibile ambire a quell'essere $x$ che la soddisfa e le da' gioia. Questo essere divino $x$ e' ricercabile e concretizzabile in se stessi, e amabile negli altri che lo ricercano e lo concretizzano in loro stessi. Come? Tramite la carita',  l'amore disinteressato e gratuito, il grazie per tale amore, e il vivere solo di questo. Tanto piu' si e' santi, o si ama una persona e tanto piu' e' santa, tanto piu' Dio esiste in terra, nella vita. Viceversa, tanto meno si vive tramite l'Amore, tanto piu' l'essere divino $x$ esiste solo nei cieli. 

Lo Spirito Santo e' la parte santa del nostro Io. Che gia' e' presente in quella parte che ci fa respirare tranquillamente. Questa e' una parte che da' pace (il respiro) e non chiede ne' toglie alcun che' agli altri, mantenendo cosi' la loro pace. Ogni persona ha poi delle doti che portano pace e gioia a se stesse e agli altri, questo e' sempre lo Spirito Santo.

Il Padre e' Dio nella completezza, e comprende la nostra parte santa terrena (Spirito Santo) e la parte che e' nei cieli ed e' ricercabile in terra. Conosceremo il Padre del tutto e saremo uniti a lui, una cosa sola con Lui, quando saremo santi in terra, o, quando ritornera' Gesu'.

Gesu' e' il Figlio. E' santo e completamente unito al Padre. 

Poiche' come abbiamo detto gli Io santi sono concordi ed uniti, possiamo dire che ogni altro santo cristiano nella storia e' stato lo stesso spirito di Gesu' che si e' manifestato in Terra pienamente. Ancora, per quanto un fedele sia peccatore, esistera' una sua percentuale non nulla del suo Io che e' pura ed e' amore, questo e' lo Spirito Santo. Cosi', Gesu' risorge in chi crede in lui, nel suo amore, e prende con se la croce della santita'.

Se tutti i santi hanno lo stesso spirito, che tra l'altro e' lo stesso di quello di Gesu', perche' solo Gesu' e' Dio? Dio, come gia' detto in precedenza e' la realizzazione nella santita' dell'uomo. Gesu' e' stato un uomo che si e' realizzato in un particolare modo, ovvero tramite l'amore caritatevole. Tutti i santi si sono realizzati nello stesso modo, tuttavia, il modo migliore per affermare qual'e' questo modo in cui si sono realizzati e' semplicemente dire che si sono realizzati nel modo di Gesu'. Dio e' sempre lo stesso, non cambia faccia, anche se cresce la moltitudine di santi che lo adorano. Per cui, il dio di ogni santo, e' sempre il dio di Gesu', ovvero Gesu' stesso.

%Il compito del fedele e' dare spazio dentro di se a Gesu', per trovarlo in se stesso. Quindi, un santo non si pone al centro dell'attenzione dei fedeli. Ciascuno e' libero e incoraggiato a trovare il proprio centro: Gesu'. Ciascuno e' incoraggiato a diventare santo lui stesso.
% Possiamo pensare anche ad altre motivazioni.
% Ad esempio, i santi sono concordi ed uniti, e quindi, nello spirito uguali. Se sono spiritualmente tutti uguali, quale uomo scegliamo per rappresentarli tutti? Naturalmente Gesu'. Dio e' uno affinche' sia l'anima del fedele non sia divisa nell'amare molteplici divinita', sia affinche' la comunita' dei fedeli non sia divisa. Percio' serve scegliere un unico rappresentate, Gesu'. Un santo non e' un nuovo Gesu', non ha nulla di nuovo o di diverso da dire rispetto a cio' che ha detto Gesu': l'amore e' fondamento di tutto. Piuttosto, un santo e' proprio Gesu', e' il suo spirito, e' lo Spirito Santo. Come dice San Paolo, noi siamo membra del corpo di Cristo, in misura maggiore o minore a seconda della santita' di ognuno.
%Un altra cosa da chiedersi e', che motivo avrebbe un uomo a riverlarsi come Dio, come ha fatto Gesu' a suo tempo? Ormai Gesu' ha rivelato tutto quello che c'era da rivelare, ovvero, che l'amore caritatevole e' a fondamento di tutto. Non c'e' bisogno di aggiungere piu' nulla. Quindi, che motivo avrebbe Dio a incarnarsi nuovamente se non per l'apocalisse e la proclamazione del regno dei cieli, che giungera' nel momento in cui tutti saranno convinti di Cristo, e tutti i convertiti saranno santi?

\subsection{La ricerca di se stessi e di Dio}
\label{laricerca}

Affrontare un percorso di crescita interiore, serve per qualsiasi fine. Ad esempio, migliorare e diventare piu' bravi nell'amore, nella famiglia, o nella scienza, nell'arte, nel lavoro, nella societa', nel proprio gruppo di amici. In generale, serve per migliorare.\\
Perche' migliorare? Non sono gia' abbastanza per quello che sono? Se c'e' qualcosa che non mi piace della vita, il lavoro o la disoccupazione, la solitudine o la centrifuga di troppe relazioni e contatti, l'essere legati ad un partner o il non approfondire mai una relazione, la sessualita', il non essere riconosciuti, l'ingiustizia, ... Se c'e' qualcosa che ci fa soffrire, allora c'e' spazio di miglioramento.\\
Quando vediamo il male in qualcosa, in realta' stiamo proiettando una nostra sofferenza in un'entita' esterna. Anche quando soffriamo perche' altri soffrono. Se ci prendessimo pienamente cura di noi stessi, non esisterebbe il male in niente. Se ci prendessimo cura della sofferenza che empaticamente abbiamo per la sofferenza altrui, pure smetteremmo di soffrire. Prendersi cura vuol dire fare qualcosa che crediamo valida, e se materialmente impossibile, comunque dare pieno spazio e sviluppo ad i nostri sentimenti, ad esempio tramite preghiere. A seconda della situazione e di come e' l'altro, anche lui guarirebbe un po' o molto.\\
Ma non siamo nati con il cervello gia' programmato\footnote{addirittura all'inizio i neonati non distiguono neanche le forme e, in pratica, e' per loro tutto un miscuglio psichedelico di cose mischiate fra loro. Poi la loro rete neurale, si evolve e comincia a creare forme, colori, etc...}. Amare se stessi e gli altri e' un arte che si impara strada facendo, e come ogni arte richiede tempo e dedizione. E' un'arte a scopo di lucro, che fa vivere meglio se stessi e gli altri. Per questo motivo ``migliorare'' e' importante. Come? In principio, e' un gran casino. Ci possiamo fare male e romperci qualcosa, abbiamo paure, gli altri ci danno fastidio, siamo insoddisfatti e ce la prendiamo con gli altri di questo. Tuttavia, basterebbe ``camminare naturalmente e respirare''\footnote{Vedi Alexander Lowen, Il Piacere} per essere contenti e per essere in grado di rendere contenti gli altri. Essere felici della vita, qui ed ora, essere appagati di semplicemente stare respirando in buona salute, fisica e psichica, capire che questo e' il bene piu' grande e condividerlo tutto.

La condivisione e' la porta che conduce a Dio. Dio e' amore puro, totale e incondizionato, per ogni essere vivente, e quindi per noi stessi e per tutti gli altri.
Essere amore per se stessi e gli altri e' difficile. In principio sembra banale: basta mettere da parte il proprio io, noi stessi, ed impegnarsi, sforzandosi, per se stessi e per l'altro. Tuttavia, dopo poco si ci stanca, e se si continuasse a farlo meccanicamente, o ``professionalmente'' come si dice nel mondo del lavoro, cio' sarebbe barare tanto quanto prendersi delle pillole per non sentire lo sforzo in una gara agonistica.
Il vero amore e' la carita', e la carita' non nasce da un obbligo esterno imposto da noi stessi o da altri. Nasce da una piena, propria, realizzazione, da un profondo soddisfacimento che vogliamo condividere.
Piuttosto che forzare l'ego, anche se a fin di bene, e' piu' piacevole e proficuo ascoltare, essere aperti e chiari, voler bene, piuttosto che forzare, imporre, svalutare, provocare noi stessi e gli altri. Non importa se si e' nel giusto. Le maniere forti sono piu' facili, e sembra che danno risultati immediati, ma allontanano, deludono, e feriscono. Sopportare lo stress, non sfociando nell'aggressivita', mantenendo la comprensione propria o dell'altro e' piu' difficile, ma da' risultati migliori e piu' duraturi. Infatti, la vita e' la ricerca costante del piacere e la minimizzazione del dolore. Quindi, chi parla il linguaggio del piacere, parlera' il linguaggio della vita. Non c'e' pericolo di sedurre falsamente con belle parole, perche' come dice Oscar Wilde nella favola dell'Usignolo e la Rosa\footnote{Oscar Wilde, ``Il principe felice e altri racconti''}, ``il vero amore e' silenzioso''. Bastano poche parole, in un lungo e piacevole silenzio.
La vita e' come camminare in un campo verde e vastissimo, con bei fiori. Se ce la prendiamo con noi stessi o con gli altri, e diciamo: ``qui non c'e' niente, corri, affaticati per trovare fiori!'', correndo e affaticandoci raccoglieremo piu' prontamente molti fiori, ma tutti quei fiori avranno perso i loro colori vivaci e luminosi. Se, invece, camminiamo contemplando il campo, e quando il Vento dolcemente lo suggerisce, cogliamo un fiore illuminato dal Sole, vedremo meraviglie, e quando il vento non soffiera' e il Sole manchera', sicuri che Dio e' con noi, non dispereremo. E quando il sereno ritornera', gioiremo di nuovo.

La ricerca di Dio, consiste nel migliorare con pazienza e senza sforzo nel tempo, sia materialmente (lavoro, salute, ...), sia psicologicamente, e condividere il piu' possibile il benessere e la conoscenza derivante con tutti. Si puo' vivere nella grazia di Dio in terra, l'anima puo' avvicinarsi ed essere toccata da Dio.
Si puo' fare, lavorando incessantemente sul proprio Io, rimpicciolendolo dove e' sovrabbondate e portandolo ai suoi limiti dove e' carente, purificandolo da desideri e tendenze che lo allontanano dalla meta, che in realta', soffermandosi, puo' riconoscere di essere superflue. \\
Chi si mette alla ricerca di Dio, pratica cio' che impara o pensa o crede, nel rispetto degli altri, nella continua ricerca, auto-critica. Fa' tesoro dei consigli, comodi o scomodi, di chi lo vuole bene. Dopo aver riconosciuto chi e' piu' bravo di lui, impara dai suoi aspetti positivi. Degli aspetti negativi degli altri, riconosce che lui stesso li ha, cerca sempre di migliorarli in lui, e quando molto tempo dopo sara' maturato, condividera' le sue soluzioni agli altri, esortandoli ad una via piu' luminosa. 
Fa' ricorso quando serve a chi e' piu' esperto di lui: genitori, zii, nonni, filosofi, ministri di Dio (preti, suore), psichiatri/psicologi di professione.
Infine, frequenta una o poche comunita': associazioni culturali, sportive, centri sociali, parrocchie, ambiti lavorativi. Un luogo dove si trova a proprio agio e dove si puo' slanciare con nuove sfide, e dove le attivita' sono la moneta con cui si interagisce con gli altri, dove cosi' nascono naturalmente relazioni interpersonali e si ha il privilegio e l'opportunita' di stare con propri simili. Solo amando anche gli altri si puo' migliorare nell'amare se stessi (e viceversa).

Questo e' il percorso verso Dio. Si puo' fare da soli. Si puo' fare amando ed essendo amati dal proprio partner, amando ed essendo amati dai propri figli, amando ed essando amati dal proprio maestro spirituale. Richiede, tempo, pazienza, tenacia e tutte le proprie capacita', anche quelle che non sappiamo di avere. E' un percorso epico, e tutte le imprese umane non sono che una metafora di questo percorso.
E' difficile sentirsi ``arrivati'' alla fine di un tale percorso. Per ogni passo, la strada si apre nuovamente con mille passi in piu' da poter percorrere, sognare, conquistare. Ma per quel poco che si sara' percorso, si ci sara' sforzati molto, si ci sentira' molto contenti, e si guardera' indietro sentendosi diversi.


\section{E nella piccolezza, eccoTi}
\label{DioPadreOnnipotenteDef}

Come fa Dio ad essere uomo in Gesu'?

In genere, l'uomo, insoddisfatto della sua vita, vuole cambiare tutto quanto e' intorno a lui, la natura, gli altri, se stesso, affinche' tutto ruoti intorno a lui e solo a lui. Il cammino di santita' verso Dio, invece, procede diversamente. Si scopre che nulla e' posto ad ostacolo. 

L'essere umano, santo, in unione con l'umanita' intera e con la natura, non piega la fisica cercando di renderla cio' che non e', ma piuttosto, da' vera pace e gioia a se stesso e a chi ha bisogno, in ogni condizione il suo corpo e la sua psiche si trovino. Fa questo in maniera terrena, adoperando forze ed energe fisiche, e per il resto, in maniera celeste, tramite forze ed energie ``spirituali''. Queste forze ed energie, sono segni consci e inconsci che il santo manifesta all'altro, e che dimostrano consciamente e inconsciamente che l'altro e' profondamente e veramente amato. Queste forze ed energie, sono il risultato di sentimenti, di intenzioni, di impegni, di azioni e pensieri che a affermano ``Ti Amo'' e null'altro affermano.

Un santo e' un essere che non si puo' comprendere con l'usuale logica della realta' quotidiana. In unione con il Padre, e' veramente onnipotente, ma non si puo' comprendere questo con l'usuale intuizione, anche se non contraddice alcuna legge fisica, ne' ne introduce altre. Lui accetta i limiti, l'umilta', le difficolta', la mortalita' di essere umani, e nell'accettare tutto cio' che l'egoismo e il narcisismo non puo' accettare, da' pace a se stesso e agli altri, anche nelle condizioni piu' serie, gravi e critiche. Questo, un uomo con neanche tutta l'energia dell'Universo potrebbe farlo.

Solo mettendo da parte il nostro egocentrismo e narcisismo, diventiamo in grado di vedere l'onnipotenza dell'amore. Di essere degli uomini e donne ed amare seppur limitati, deboli e feriti. Di essere umani, e pur deboli ed abbandonati, trovare pace ed anche gioia, nell'amare gli altri quanto noi stessi. Di essere umani, e pur potenti e ricchi, trovare serenita' e motivazione, nell'amare chi non e' amato e gli umili, senza guadagno, fino a investire e rischiare tutto quello che abbiamo, tutto quello che siamo.

Un santo e' un essere umano che nell'interezza Ama e solo Ama. Tutto di un santo e' Dio che Ama. Nessuno e' di per se' santo. I bambini sono egocentrici e narcisi, non perche' sono cattivi, ma perche' cominciano a sviluppare il proprio senso del se' e dell'io\footnote{\url{https://en.wikipedia.org/w/index.php?title=Egocentrism\&oldid=998343250} \url{https://en.wikipedia.org/w/index.php?title=Narcissism\&oldid=1011686661\#Required\_element\_within\_normal\_development}}, solo nel tempo crescono e maturano.  Ma anche nell'adolescenza e nella vita adulta, non si finisce mai di crescere e di mantere le qualita' raggiunte, di farsi carico di responsabilita' sempre maggiori e piu' complesse e di rimanervi fedele fino alla fine. Questa crescita e' benefica perche' fino a quando l'essere umano rimane alle prime armi con la propria ed altrui psiche, con le difficolta', gli stress e le fatiche, non superando i traumi infantili e dell'adolescenza, molte volte non riuscira' a volere cio' che il Suo cuore autenticamente desidera, ne' a godere e donare la pace e la gioia che la Sua anima desidera per se stesso e per agli altri.

L'uomo ha la possibilita', per sua vocazione, per sua soddisfazione e pace, di perseguire e attivamente coltivare le qualita' divine formalizzate dalla religione. 
Il mondo ``libero'' e', invece, pieno di mete facili ed illusorie, facili in confronto al vivere pienamente e solo nell'Amore. Se il successo del lavoro e' difficile, quanto alto e duro sembrera' ad un adulto dare e ``bruciare'' per almeno un'altra persona il proprio tempo, i propri obbiettivi, la propria fatica, il proprio orgoglio e autostima, il proprio respiro, il consumo dei propri organi? Per un'altra persona che nulla ci ha dato e dara'? Una persona che neppure conosciamo, e che di punto in bianco, spunta nelle nostre vite? E se divertimenti di semplici piaceri, anche se fatti ad arte, sembrano il paradiso e non se ne puo' fare a meno, cosa sara' allora vivere la Sua pace e gioia nei momenti di carita', nelle giornate fraterne con l'umanita'? Non e' neppure immaginabile. 

Nessuno nasce Santo. Ma tutti possono diventarlo e tendere alla Pace e Gioia, da vivere ora e per sempre.
Possiamo riassumere tutto dicendo:
\eal{&
    \lim_{\textrm{amore}\to\infty} \textrm{EssereUmano} = \textrm{Dio}
}
che si legge: ``il limite dell'essere umano per il suo amore che tende all'infinito e' Dio''.

La realizzazione dell'essere umano nell'Amore e' un'incarnazione di Dio. Ma tale realizzazione non si puo' pensare in termini normali e mondani. Seppur realizzato, un figlio di Dio non e' un super-uomo, per come immaginiamo che sia l'onnipotenza dei super-eroi. Se realizzato, Lui e il Padre sono pero' una cosa sola, e cio' che e' stabilito dal Padre, e' accettato dal figlio, e cio' che e' richiesto dal figlio, e' ascoltato dal Padre\footnote{``Io sono nel Padre e il Padre e' in me'' Nuovo testamento, Giovanni, capitolo 14, versetto 11}. Cosi', tutto si muove secondo la Sua volonta', che e' quella del Padre e quella del figlio, unica volonta'.

Un uomo cosi' Santo, realizzato nell'Amore, e' manifestazione di Dio Padre onnipotente quando, pur avendo raggiunto i suoi limiti fisici e materiali, non smette di amare ed ama ancora, bruciando il suo cuore. \\

Abbiamo detto piu' volte ``Amore'', con la A maiuscola. Ma, esattamente, di quale amore stiamo parlando? In piu', e' veramente impossibile descriverlo. Si puo' tentare, con poesie, con modelli matematici e psicologici, ma resterebbero solo belle parole e interessanti schemi. In piu', chiunque potrebbe pregiarsi di recitare tali parole, anche se lui stesso, in realta', e' lontano nella sua vita dal significato che sottendono. Ad un certo punto della storia umana, un uomo, veramente e totalmente realizzato nell'amore, con la sua vita e con il suo sangue ha scritto la definizione perfetta: ``Amare e' vivere, sentire e fare come io vivo, sento e faccio''.
A Lui, nulla della vita pesava, non perche' non faticosa, non dolorosa o terribile, ma perche' amava veramente se stesso, senza lacune, difetti o vizi, e di questo amore la sua anima era nutrita. Di questo amore, anche l'anima di chi lui ha amato, ed lo ha accolto in se, ha gioito e ne ha tratto guarigione.
Ed ha amato se stesso e gli altri, non risparmiando a se stesso, fatiche, pericoli, derisioni, umiliazioni e torture.

Tutto questo e' il cardine della dottrina Cristiana: Dio e' Gesu', l'Amore non e' il nostro, che e' tendenzialmente egocentrico e narciso, ne' e' quello scritto nei cioccolattini o nei film Americani, e', invece, quello di Cristo.

Percio' il nostro Io si realizza e si unisce totalmente a Dio, quando il nostro Io diviene conforme a quello di Gesu', ed agisce calato nella nostra vita, nella nostra missione.
Lo stesso vale per l'Io altrui, divenuto Gesu', e noi chiamati ad amarlo, ad essere uno con Lui.\\

E' opportuno spendere qualche parola sulla potenza di Dio e sui miracoli. In primis, le letture sacre e i miracoli sono da leggere in senso spirituale, non in senso superstizioso. In secondo luogo, il punto fondamentale, e' capire non tanto se Dio sovverte la Natura per salvare la nostra vita, ma se Dio ha la potenza di salvare la nostra vita. Dio e' in grado di condurre alla vera vita, alla Pace, alla vera estasi. Dio e' in grado di curarci da quei mali che affligono la nostra vita e, che proiettiamo in cose, persone e situazioni, ma invece sono il risultato del nostro Io che si e' bloccato per traumi, paure, illusioni o dipendenze, e forse si e' bloccato da cosi' tanto tempo che e' diventata la normalita' per noi. Si e' bloccato e non tende piu' all'infinito, non ama piu' veramente noi stessi o gli altri. In questa prospettiva, Dio opera veramente miracoli. Una sola parola di Gesu' riesce a cambiare la vita di chi crede in lui.
 

\subsection{C'e' una distinzione tra l'uomo e Dio?}
Prima si e' detto che Dio e' il limite dell'uomo all'infinito del suo amore, e che ``un Io di norma non coincide tutto con Dio''.
Quindi, Dio padre, che e' immagine del limite infinito, dell'essere che e' tutto Amore, e' un essere superiore e l'uomo un essere inferiore.
Questo e' un modo di vedere le cose al negativo. Un altro modo e' diametralmente opposto: Dio ci ama per come siamo, non solo quando siamo santi e sublimi. Ci ama quando siamo piccoli, semplici, uomini onesti, ma anche quando siamo deboli, e il nostro cuore e' contrito per cio' che di sbagliato abbiamo commesso. Ci ama per quello che siamo, e siamo uomini, non Dio. La santita' e' certamente un richiamo e vocazione per tutti gli uomini, ma non e' una costrizione. E' come quando si lavora con gioia e passione. Nella gioia si rispetta se stessi e gli altri, e cio' che si produce e' un ``effetto collaterale'' della gioia, puo' essere poco o molto, di qualita' o di bassa qualita'.  Alla base di tutto, il valore di ciascuno non e' quanto o cosa produce, ma e' la sua stessa esistenza.

Chi si mette nella via di Gesu', che conduce alla santita', e' come un bambino che vuole diventare come suo padre. Ammira suo padre e lo imita. Ma, essendo piccolo, sbaglia e a volte fa il contrario di quello che dovrebbe essere fatto. Suo padre lo ama, e darebbe la sua vita per lui. Lo conosce, e lo guida, lo incita, lo consola. Il bambino nel tempo, naturalmente, senza costrizione, cresce, impara, e diventa grande. Un giorno diventa come il padre, e non c'e' piu' differenza, a livello di maturita' e capacita', tra lui e il padre. Ma sono una cosa sola. Nessuno e' piu' grande dell'altro, e tutti sono unanimi, con una sola volonta'. Il figlio fa la volonta' del padre e il padre la volonta' del figlio.

Tutto questo spiega perche' e' buono che valgano le seguenti affermazioni:
\begin{enumerate}
    \item di per se' un uomo non e' Dio
    \item a volte un uomo puo' essere ispirato, sentirsi chiamato e puo' servire Dio in un particolare modo, in un particolare momento (nella dottrina si dice che e' lo Spirito Santo che agisce). Dire si e rispondere alla chiamata e' cio' che conduce alla santita'.
    \item nella via della santita', un uomo si unisce sempre piu' a Dio. Non per sentirsi superiore agli altri, ma per vivere da figlio di Dio, senza una vera differenza tra lui e il Padre, come Gesu'. E godendo di una nuova pace e forza, servira' gli altri per donargli la stessa gioia.
\end{enumerate}


\subsection{La preghiera}
La preghiera e' distaccarsi da tutto, lasciare lo spazio e il tempo, non prendere decisioni, non andare avanti ne' indietro. Qui, ascoltare il cuore e, lasciatolo libero di essere, in ogni sua parte, che agli occhi propri o umani puo' anche sembrare cattiva e malvagia, lasciare che Dio gli parli, lo lenisca li' dove e' ferito, lo alimenti ed esalti li' dove piace a Lui. Le parole di Dio sono dolci alla propria anima, ne' false ne' illusorie. L'anima puo' fidarsi completamente di Dio, di mostrarsi come e', nelle sue miserie e nelle sue altezze. Cosi', in questo dialogo e relazione tra l'anima e il cuore di un figlio\footnote{il fedele che prega} con il Padre, l'anima si rinnova e il cuore riprende forza, per poi mettere in pratica nella vita le promesse fatte nel segreto al suo Beneamato.

La preghiera puo' comportare un mettersi in discussione profondo, in dialogo con Dio. Come nei momenti di crisi, in cui tutto il nostro essere e' messo in discussione per necessita', cosi' nella preghieria, il fedele si mette totalmente in discussione, rievoca in lui emozioni abbandonate, spegne forze interiori infruttuose ed, al limite malevoli. Riaccende parti fruttuose che si erano arrese o assopite. Orchestra le sue varie parti, gli rida' vita ed unita', e lui stesso rideventa vivo ed uno. Infine, prova a lanciarsi nel tendere verso l'unione con Dio, e a fare un passo in avanti.

Nella preghiera, fatta anche con altri o per altri, l'Io lascia l'anima sua e altrui libera e si pone in ascolto. In questa dimensione, un'anima puo' fare chiarezza per se stessa ed esprimere, con speranza e fiducia, i suoi bisogni e desideri. 
L'Io, in unione con Dio, ascolta, non giudica e comprende l'anima, e potra' in futuro attivarsi e spendersi per lei.


\subsection{Sull'istituzione religiosa}
La Chiesa e' la palestra dell'anima. Un luogo dove si parla di amore, e lo si pratica, allenandosi, in proprio e con gli altri, nei vari gruppi offerti dalle parrocchie. Cosi' come le vere palestre, alcune piacciono di piu', altre piacciono di meno, ma cio' nonostante, non esiste luogo alternativo alla Chiesa dove si parla esplicitamente di Amore, inteso in senso non egoistico, e dove coscientemente lo si studia, pratica e ricerca. 

Le chiese Cristiane, hanno come riferimento la figura di Cristo, ed i loro ministranti (preti, suore, monaci, ...) assumono la responsabilita' di essere suoi discendenti, imparando da maestri che a loro volta hanno imparato da precedenti maestri, e cosi' via fino agli apostoli, per poi giungere a Cristo, primo, originale maestro. Il loro compito e' quello di tendere alla stessa santita' di Cristo, e quello di accompagnarci nella stessa direzione, nel corso della vita.

Perche' la Chiesa cattolica e non un'altra? Le istituzioni, in generale, cosi' come ogni prodotto e servizio umano, sono imperfette. Per capirne le ragioni, basta pensare che gia' e' difficile per i ``capi di famiglia'', i genitori\footnote{qui si da' una descrizione sul negativo per stressare le difficolta' dei vari ruoli}, gestire due o tre figli, e allo stesso modo e', in vari momenti, difficile per i figli sopportare i propri capi. 
Allora, sicuramente e' difficile per una persona curare una parrocchia a cui appartengono almeno un centinaio di persone, ed e' difficile per poche persone\footnote{si intende i ministri religiosi di piu' alto livello gerarchico} curare un'intera citta'. E, viceversa, e' difficile per un singolo relazionarsi con la ``macchina'' che porta avanti l'istituzione, dato che il singolo non puo' essere curato personalmente in maniera approfondita e totale. Il singolo e' uno tra le centinaia degli altri singoli che hanno bisogno di attenzione e di avere le proprie preferenze, aspettative ed urgenze prese in esame nella loro particolarita' ed individualita'.

Cosi' come ciascuno di noi, pur essendo imperfetto, ha dentro di se un'essenza e parte divina, anche le istituzioni, prodotto umano, hanno una parte di se che e' a servizio dell'uomo, che e' utile e meravigliosa. Questa parte va ricercata nella terra, e trovata, conosciuta, apprezzata. 
D'altro canto, le istituzioni sono necessarie, per potersi fidare di persone e ruoli ufficiali, standardizzati e certificati, per non cadere preda di truffe create da persone che, improvvisandosi, mettendosi un bel vestito e adoperando eccellenti strategie di marketing, poi, come sciacalli si avventano sulla fortuna delle persone senza curarsi della loro sorte\footnote{vedi, ad esempio, tutti i ``maghi'' che hanno truffato molte persone proponendo cure miracolose}.
Quindi, e' facile lamentarsi di un'istituzione, perche' si ci aspetta grandi cose da lei, come se fosse incarnazione di Dio padre onnipotente, ma tuttavia si ottengono risultati molto umani e a volte deludenti. 
Solo da Dio possiamo aspettarci risultati propri di Dio.
Non per questo gli ospedali sono inutili, le forze dell'ordine, le scuole e cosi' via. Sono ognuno una macchina a servizio di Dio, per quanto imperfetta e umana. Allo stesso modo la Chiesa Cattolica.

Tutto quanto detto e' una visione macroscopica delle istituzioni, e della Chiesa. A livello microscopico, bisogna cercare. Partendo dal proprio quartiere, da eventi di divulgazione rivolti ai giovani o tematici, da ritiri, ma anche da personaggi o luoghi famosi anche se piu' remoti e distanti da un rapporto personale. Bisogna cercare per trovare delle persone ed una comunita' con cui uno si trova bene\footnote{
    Nella mia esperienza, ho trovato un buon ambiente nel gruppo del Rinnovamento nello Spirito Santo e delle Suore Carmelitane Messaggere dello Spirito, ed in una parrocchia di un quartiere confinante al mio.
}.
Se si e' mossi da un autentico spirito di ricerca e di fratellanza con gli altri, si vivranno momenti piccoli ma unici e si condivideranno emozioni con la fiducia di essere rispettati ed ascoltati. Ogni rito, inoltre, se eseguito con fede, sara' espressione di cio' che ognuno ha dentro di se', e non di cio' che l'uomo invanamente cerca fuori da se' (superstizione). Infine, poco importeranno le sottigliezze come il dire che una legge formale e' giusta rispetto alla propria cultura o meno. Nessuno in una comunita' sana obblighera' a rispettare alcuna legge interiore. Non importera' stabilire se la Terra gira intorno al Sole o viceversa, quando cio' che si stara' insieme cercando, nelle profondita' del proprio essere, e' molto piu' importante.

Per quanto riguarda i ministri religiosi, esistono persone che hanno scelto di servire e lavorare per la Chiesa per seguire una vocazione autentica. Tra esse ci sono persone in gamba, che nel mondo normale, se lo avessero seguito, avrebbero trovato successo.
Anche se non si ha il privilegio di trovarle, una gia' basta per quantomeno rispettare un'intera istituzione. Infatti, loro credono nella Chiesa.

Se poi, si riconoscono i valori di un'istituzione, ma non si trovano persone valide, allora siamo noi stessi che dobbiamo scendere in campo per difenderli. Se, poi, riconosciamo che una regola o legge deve essere cambiata, allora dobbiamo prenderci la responsabilita' di promettere a noi stessi e agli altri che come pensiamo noi condurra' alla felicita' e alla serenita'. E quando le nostre promesse avranno portato danno, dobbiamo accogliere le colpe e le punizioni che ci verranno inflitte dalla vita e dagli altri. Tutto questo Gesu' l'ha fatto, e cosi' facendo ha cambiato la Chiesa dei suoi tempi. 
Inoltre, questo e' l'unico modo di procedere: se un'istituzione nasce da sani propositi e da persone valide, ma poi si corrompe nelle generazioni, e' fuorviante creare un'istituzione sostitutiva, succedera' nel tempo la stessa cosa. Invece, vanno individuate le cause della corruzione, e combattere per curarle.

\subsection{Sul peccato e sul perdono}

Il peccato e' ogni errore che nasce dal non essere centrati nell'amore caritatevole. Il perdono totale, introdotto da Gesu', riconosce che l'uomo non nasce perfetto, e che la santita' e' un cammino, una ricerca, fatta di errori e cadute. In altre parole, riconoscere il ``peccato'' e il ``perdono'', vuol dire porre l'uomo in un cammino di crescita interiore, che ha come meta Gesu', Dio fattosi uomo.

\subsection{Sulle leggi della religione}

Ogni legge di una religione ha un suo senso, se si e' disposti a comprenderlo. La cultura moderna occidentale, ha tanto criticato le leggi della religione, tuttavia se non si fa uno studio serio delle varie questioni, le critiche diventano superficiali, perche' ignorano il senso originario, naturale e sano di tali leggi.

Le leggi sono linee guida per non fare errori grossolani. Non sono la meta, che rimane Dio. Gesu' stesso esce fuori le linee guida della legge del suo tempo, non per interesse personale, ma sempre per fare del bene a tutta la comunita', e ad ognuno della comunita'. Fatti emblematici sono quelli delle guarigioni fatte di Sabato\footnote{\url{https://bibbiaedu.it/CEI2008/nt/Lc/14/} \url{https://bibbiaedu.it/CEI2008/nt/Lc/13/}}. Tuttavia, bisogna essere umili, e non cadere nell'illusione del perbenismo moderno, che con la formula ``sono una persona per bene'', ci autorizza poi ad esprimere giudizi morali e di vita a nostro arbitrio. I saggi del passato, impiegavano un'intera vita per capire la vita, ed anche se non disponevano della scienza e dell'istruzione moderna, sicuramente era andati molto in profondita' nello studio dell'anima e nel prodigarsi per gli altri. Quindi, i loro contributi, rimangono classici, e sempre attuabili, se letti nel loro senso umano e riconoscendo l'intenzione di far del bene.

All'osservazione ``io sono una persona per bene, non ho bisogno di prodigarmi nella via spirituale'', bisogna rispondere cosi': una propria tendenza non sana, anche se mai espressa, verra' espressa quando saremo messi alla prova, e li' non potremo dominarla. Ad esempio, se uno studente ha l'abitudine di trovare trucchi e sotterfugi per non studiare fino in fondo le sue materie, e cosi' sopravvive fino alla fine dei suoi studi, anche se non sta' facendo nulla di male a nessuno, se non a se stesso, quando poi si trovera' nel mondo del lavoro, e dovra' attenzionare fino in fondo una sua mansione, si trovera' in seria difficolta'. Oppure, chi sta' con gli amici solo per trascorrere momenti superficiali, senza mai aprirsi veramente, nell'andare del tempo si ritrovera' solo. O chi e' abituato a prosperare tramite favori, quando poi assumerera' una posizione di responsabilita' nel lavoro, sara' fortemente tentato a scendere a patti con metodi corrotti. E' anche vero che una persona forse non assumera' mai ruoli cosi' seri da dover richiedere una integrita' assoluta, tuttavia, la ragione del tendere alla santita', nasce dal fatto che la vita propria ed altrui ha un valore infinito, e cio' rende una necessita' riuscire a difenderla e lodarla tendendo alla perfezione.

Le leggi sono una linea guida per non commettere errori che comporteranno riparazioni dolorose, o che saranno irreparabili. Questo soprattutto nell'affrontare situazioni emotive, affettive e sociali di cui non siamo pratici e di cui non abbiamo sufficiente esperienza e conoscenza per prendere delle decisioni istintive.

La legge invita a rimanere su dei binari e sprona a crescere, a maturare lo spirito della legge per poter essere non piu' schiavi di essa, ma liberi e consapevoli.

Se e' vero che le leggi hanno un senso, bisogna anche essere tolleranti e comprensivi. Infatti, rispettare anche una sola legge e' un grande traguardo, ma l'errore piu' grave e' imporre a se stessi o ad agli altri di raggiungere tale traguardo. Se si rispetta la legge con l'obbligo e non con un proprio senso di dovere, allora sara' peggio di non averla rispettata. La Pace della religione si raggiunge attraverso il libero arbitrio, la fiducia e impegnandosi, sbagliando e poi colto l'errore e le sue conseguenze, rialzandosi ed ancora con fiducia impegnandosi.

Chi ha mille volte peccato, ma poi ha ascoltato la voce del Signore e ripreso con pentimento, con speranza e decisione il cammino, e' perdonato da Gesu'. Ma non come abbuono, o perche' il Cristianesimo e' una religione che lascia passare, ma perche' chi e' pentito, nonostante il peso di quello che si porta dentro e che riconosce sua colpa, nonostante non potra' mai piu' vivere la vita felice che poteva vivere se non avesse commesso l'errore compiuto, trova la forza di dire a se stesso ed al mondo intero ``ho sbagliato'', e di accettarne le conseguenze. 

%\section{Il Grazie}
%
%Una parte importante nell'arte dell'amare e' riconoscere ed apprezzare il poco, il niente, il silenzio.
%
%Se qualcuno ama, non significa che deve fare per forza qualcosa di buono, di utile, o di bello. Significa, prima di tutto che prova questo sentimento di amore. In formula, si potrebbe dire che ogni qualvolta ha a disposizione energie, risorse ed intelligenze, le impieghera' per fare qualcosa di utile e piacevole. Tuttavia, siamo tutti limitati fisicamente e quindi, quello che qualcuno puo' fare concretamente e' ben poco, sia per se stesso sia per gli altri.\\
%Allora, il suo sentimento di amore, i suoi sforzi e risultati valgono poco? No. E' cio' che nutre l'anima, se lo si apprezza, e lo si conserva e matura dentro di se'.
%
%Se noi amiamo chi ama, e' sufficiente lasciarsi nutrire dal suo sentimento per essere veramente felici di Lui o di Lei.
%
%E' chiaro che non di solo sentimento vive l'uomo. Ma, tutte le cose spicciole, come il lavoro e il cibo, derivano come conseguenza di quel sentimento, non viceversa.
%
%Il Grazie, diventa fondamento filosofico per risolvere molti problemi umani: il non ottenere un desiderio per come si era immaginato, o non ottenerlo affatto, le difficolta', le avversita', le malattie. Tramite il Grazie verso chi ci ha amato e ci ama, compresi noi stessi, se ci amiamo, compresi i religiosi di cui ci fidiamo e pregano per noi, compresi i religiosi Santi del passato, permette di uscire fuori dall'oppressione del non-essere, e ritornare ad accogliere ed accettare la vita, e lottare per essa affrontando le sue sfide.
%
%Dio e', se lo si ricerca, ed ama, quindi, anche nel nulla, nel silenzio, nel non avere, nel non essere. 
%
%Si puo' venire incontro a noi stessi ed agli uomini ed alle donne che ci amano, tramite il ``grazie'' e, nei casi piu' difficili, tramire il ``sacrificio''. Ringraziare e' apprezzare quello che qualcuno, qualcuna o alcuni sono per noi, compresi noi stessi, e in questo apprezzamento, riconoscendo che cio' che viene dato e' un dono, e che non viene dato di piu' non per cattiveria!, rimanere appagati di cio' che esiste. Sacrificarsi e' la forma piu' forte e difficile del ringraziare, in quanto cio' che si riceve non e' sufficiente per soddisfare i bisogni, e nonostante cio', si rimane comunque grati per cio' che gli altri e noi siamo. 
%
%Nel paragrafo \ref{loZero} a pagina \pageref{loZero}, si spiega matematicamente, usando il Dilemma del Prigioniero, che esistono situazioni in cui e' necessario un proprio sacrificio personale.
%

%\subsection{Sui rituali e sulla preghiera}
%Allenamento della fede: in ambiente controllato, sicuro, semplice. Provare a vivere con fede.
%
%Esempio: la preghiera, non dominio dell'anima tramite la volonta' e la mente, ma luogo sicuro dove esercitarsi a ``volare''.
%Esercitarsi a essere santi nelle parole, nel canto.
%Se non santi nella propria casa, nel proprio tempo, nel recitare o cantare frasi semplici e d'auspicio, come e' possibile essere santi nella vita?
%

%%%%%%%%%%%%%%%%%%%%%%

%Ad esempio, gli scienziati credono al Dio della verita' oggettiva, della ragione ed efficienza, e del progresso tecnologico. 
%Noi a quale Dio crediamo? E il nostro Dio e' un Dio che dopo le fatiche e i sacrifici della settimana ci consola, ci dona pace e gioia, oppure, e' un Dio che pretende solo il nostro sangue, e che per scusarsi dell'inquietudine e del caos che lascia nelle nostre vite, dice che la colpa non e' sua ma di tutte le altre persone, che non ci amano abbastanza, o nel modo adeguato o che non sono degne del suo amore?
%
%L'isomorfismo tra la realta' interiore e quella esteriore e' la sacralita'. Vivo <--> Il signore e' onnipotente, crea un universo e me lo dona.
%
%La verita' non si possiede, noi siamo posseduti dalla verita'. A quale verita' vuoi accedere? Per una verita' di carita', vivi in carita'.



\section{Procedimento assiomatico}
\label{procAx}

In questo paragrafo, procederemo in maniera semi formale, come se stessimo dando un modello matematico di tutto quanto esposto fino ad adesso. Questo, sia per poter essere piu' precisi in alcuni aspetti, sia per puro piacere artistico. Questo paragrafo, con i suoi sottoparagrafi, non e' necessario, e chi vuole lo puo' tranquillamente saltare.

Cio' che definisce, consciamente o inconsciamente, cio' che e' piacere e cio' che e' non-piacere (dolore), sia chiamato ``anima''.

Cio' che desidera, consciamente o incosciamente, che l'anima provi piacere e non dolore, sia chiamato ``animo'' o ``spirito''.

Diremo che un'anima e' ``empatica'' quando essa, come parte di se stessa, include l'anima altrui. E quindi, il piacere e il dolore altrui e' piacere e dolore suo. \\
Osservazione: questa definizione non e' simmetrica! Se un'anima A e' empatica verso un'anima B, non e' detto che B sia empatica verso A. A potrebbe amare B senza essere amata da B. \\
Osservazione: un anima A, empatica verso un'anima B, include B come sua sottoparte, tuttavia, ancor meglio si deve presupporre che A smette di essere A e che diventa una nuova entita' che comprende come sottoparti il \emph{se'} e l'\emph{altro}. Questa nuova entita' si puo' indicare con A+B. \\

Diremo che A ``ama'' B quando l'anima di A e' empatica con B, quando l'animo di A desidera che A+B provi piacere e non dolore, e quando A spende se stessa per realizzare questo desiderio. Per la precisione, non si puo' parlare di A che ama B, ma piuttosto di A+B che ama B. 

In generale, se un essere $A$ ama molteplici esseri, scriveremo l'unione $\mathcal{A}=A_1+A_2+\cdots$, dove ogni $A_i$ e' un essere dell'unione.

\def\self{\textrm{self}}
\def\other{\textrm{other}}

Se $A$ ama $B$, ma non vicersa, si hanno due unioni, quella di $A$ e quella di $B$ e vale $\mathcal{A} \supset \mathcal{B}$ (nota\footnote{E' l'inclusione insiemistica, $\mathcal{B}$ e' incluso in $\mathcal{A}$ quando ogni elemento di $\mathcal{B}$ e' un elemento di $\mathcal{A}$}).  Ovvero, cio' che $B$ ama lo ama $A$, ma non viceversa, e quindi $\mathcal{A} \neq \mathcal{B}$. Se $B$ ama pure $A$, allora ogni unione contiene l'altra e $\mathcal{A} = \mathcal{B}$ e quindi si parlera' di un'unica anima. 

In questo caso, la loro anima e' distinta ma equivalente.  Ognuno pero' la vive dal proprio punto di vista. Dal punto di vista di $A$, l'anima la indichiamo con $\mathcal{A}_A$ e definiamo $\textrm{self}(\mathcal{A}_A)=A$ come il \emph{se'} di $\mathcal{A}$, mentre un qualsiasi essere dell'unione e' chiamato \emph{altro} per $A$. Viceversa, dal punto di vista di $B$, si ha $\textrm{self}(\mathcal{A}_B)=B$.

Molte volte parleremo solo di $A+B$, ma quanto detto si potra' estendere automaticamente anche ad una unione qualsiasi $A_1+A_2+\cdots$. \\

\subsection{L'anima come insieme di eventi positivi}
\label{AnimaInsiemeEventi}
Inizialmente, abbiamo posto come principio il fatto che un'anima $A_i$ stabilisce cosa e', per lei stessa, il piacere e cosa il dolore. Questo si puo' modelizzare pensando che $A_i$ sia un insieme di eventi, come nella probabilita'. Ad esempio, $p\in A_i$ potrebbe essere l'evento p=``domani fara' bel tempo''. $A_i$ e' l'insieme di tutti quegli eventi, presenti, passati e futuri, che procurano non dolore all'anima. Diciamo ``non dolore'' piuttosto che piacere, perche', oltre agli eventi che generano piacere, anche la negazione degli eventi che generano dolore per $A_i$ devono essere inclusi in $A_i$. Ad esempio, l'evento ``correndo non inciampo e cado'' e' da essere incluso in $A_i$.


\subsection{L'animo}

\def\des{\textrm{des}}
\label{defDiDesideriDes}

\def\anima#1{\mathcal{#1}}

\def\spirit#1{\textrm{spirit}(#1)}
\def\Animo#1{\spirit{\anima{#1}}}

L'animo $\spirit{A_i}$ e' un'entita' che ha il fine ed il solo fine di realizzare tutti e soli gli eventi di $A_i$. Un approccio dell'animo di realizzare un evento, si puo' chiamare \emph{desiderio} di $A_i$. Se l'evento e' il \emph{cosa} realizzare, un desiderio e' il \emph{come} realizzare. L'analogia con l'informatica, e' una \emph{funzione}, che stabilisce cosa si vuole fare, e l'\emph{algoritmo} che stabilisce come implementare tale funzione. Poiche' l'essere $A_i$ non e' perfetto, un suo desiderio non e' sempre un modo perfetto di realizzare l'evento desiderato. A volte inoltre, un desiderio e' completamente mal posto, e potrebbe anche realizzare la negazione dell'evento voluto. Con l'analogia informatica, un desiderio e' da pensare come una \emph{macchina di Turing}, che $A_i$, nelle migliori delle sue facolta', crede realizzi la funzione voluta, ma che non e' detto che che sia proprio un algoritmo corretto per tale funzione.

Definiamo l'insieme $\des(A_i)$ dei \emph{desideri} di un essere $A_i$. $\spirit{A_i}$ puo' essere modelizzato ponendo proprio $\spirit{A_i}=\des(A_i)$. 

In realta', un evento puo' essere molto o poco necessario per un'anima. Potremmo allora distinguere una scala di desideri tra desideri necessari e desideri non essenziali. I desideri necessari, li definiremmo come \emph{bisogni}. Un desiderio sarebbe \emph{vano} quando l'animo $A_i$ crede che sia necessario, ma in realta' non lo e'. Tuttavia, per semplificare il discorso, non distingueremmo tra desideri necessari e non.

L'Animo $\Animo{A}$, dell'unione $\anima{A}=A_1+A_2+\cdots$, e' inteso come l'animo formato dai singoli animi $\spirit{A_1}$, $\spirit{A_2}$, $\cdots$, che cercando di cooperare in maniera \emph{coerente ed efficiente}, realizzano al meglio tutti gli eventi di $A_1,A_2,\cdots$, ovvero, realizzano $A_1+A_2+\cdots=A_1\cup A_2\cup \cdots$. L'animo $\Animo{A}$ non e' la semplice somma di ogni $\spirit{A_i}$, infatti, ciascuno per cooperare deve cambiare. Non e' possibile unire due macchine di Turing, o piu' semplicemente due programmi/due strategie/due piani, cosi' come si uniscono due insiemi. Una strategia potrebbe essere controproducendente per l'altra. Solo se due desideri sono coerenti tra loro possono essere perseguiti entrambi. Inoltre, anche ragioni di efficienza possono richiedere, nell'unione di due animi, il cambiamento di uno dei due o di entrambi.
\\

\subsection{Sulla ricorsione del Se'}
\label{ricorsioneSe}

L'anima $\anima{A}=A_1+A_2$, con $\self(\anima{A})=A_1$, e' un'anima che, amando se stessa, include essa stessa nell'unione. 

L'anima $\anima{A}$ si trova fisicamente in $A_1$. Se $A_2$ ama $A_1$, allora $\anima{A}$ si troverebbe pure in $A_2$. Ma per il momento, attenzioniamo il fatto che almeno si trova tutta in $A_1$.

$A_1$ allora contiene $\anima{A}=A_1+A_2$ e percio' vale 
\eal{&A_1=A_1+A_2=\anima{A}}
Cio' non e' intuitivo: $A_1$ contiene se stesso, come una matrioska. Ma in matematica, non e' cosi' strano ragionare ricorsivamente. Infatti, se proviamo, tutto rimane coerente: sostituendo $A_1$ nella parte destra di $A_1=A_1+A_2$, si ottiene $A_1=A_1+A_2+A_2$, e sostituendo di nuovo $A_1=A_1+A_2+A_2+A_2$, e cosi' ad libidum...

Se consideriamo il $+$ come unione insiemistica, allora $A_1=A_1+A_2+A_2+\cdots = A_1+A_2$. L'affermazione $A_1=A_1+A_2$ e' equivalente insiemisticamente all'affermazione $A_1 \supseteq A_2$. Questo e' coerente con le definizioni che abbiamo dato all'inizio di anima empatica che ama un'altra anima.

\subsection{Sull'unicita'}

L'anima e' una, anche se e' empatica ed ama altre anime. Come gia' detto, l'anima che ama altre anime e' una entita' distinta da come era prima quando non le amava. Ella e' uguale all'unione delle varie anime. 

All'apparenza se l'essere A ama B, e la sua anima diventa $A+B$, sembrerebbe che comunque $A+B$ risieda nel corpo di $A$, dato che l'anima e' il risultato di processi neurologici. Tuttavia, il corpo di A non e' piu' sufficiente per descrivere A+B. La percezione del dolore o del piacere di A+B, deriva dai sensi del corpo di A e del corpo di B. Se A+B ama B, allora, l'anima A+B desidera' il bene di B, ovvero desiderera' che B percepisca quanto piu' piacere e quanto meno dolore, che siano queste percezioni ``fisiche'' o ``psicologiche'' (es. emozioni). 

$A$ percepisce cio' che  $B$ percepisce tramite il vedere, sentire, ascoltare $B$, tramite il stare con $B$, tramite il parlare con $B$, tramite l'empatia.

Se, a fin di bene, $A+B$ ``possedesse'' il corpo di $B$ per percepire cio' che $B$ percepisce senza che $B$ lo desideri, parleremmo non di amore, ma di possesso\footnote{``possesso'' nel senso generico di violenza psicologica (es. stalking) o fisica}. Questo fa' capire che l'anima dell'essere $B$ deve essere concorde con l'anima di $A$, deve trarre quanto piu' piacere e quanto meno dolore dall'amore di $A$. Allora, $A+B$ vuol dire veramente l'unione dell'anima dell'essere $A$ e dell'anima dell'essere $B$. $B$ condivide ad $A$ le sue percezioni (e' aperto), e $A$ le elabora per $A+B$. 

Per questo, $A+B$ risiede fisicamente sia in $A$ sia in $B$. L'amore di $A+B$ verso $B$ non puo' sussistere senza $B$. 

Inoltre, si comincia a capire che se $B$ e' empatico e la sua unione e' ad esempio, $B+C$, allora $A+B+C$ e' l'unione risultante dall'unione $A+B$, cioe'
\eal{
    &\anima{B}=B+C\\
    &\anima{B}=B\;\;\;\;\;\;\;\;\textrm{Questo per \ref{ricorsioneSe} pag. \pageref{ricorsioneSe}}\\
    &\anima{A}=A+B=A+\anima{B}=A+B+C\\
}
Vale anche $\mathcal{A} \supseteq \mathcal{B}$, ovvero, l'unione $\anima{A}$ deve includere almeno l'unione di $B$. Quindi, ogni essere parte di $\mathcal{B}$ e' parte di $\mathcal{A}$. Non vale il viceversa. Questo, ad esempio, quando $A$ ama $B$ mentre $B$ non ama $A$. Se $\mathcal{A}=A+B+C+D$ e $\mathcal{B}=B+C$, $D$ ed $A$ stesso sono amati da $A$ ma non da $B$.

Anche se e' necessaria un'interazione fisica, l'anima non automaticamente ama gli altri solo per il mero fatto che gli altri esistono, vivono fisicamente ``fuori'' dal suo se' e stanno bene perche' se la cavano da soli. L'anima ama quando e' disposta ad impiegare se stessa per la gioia altrui, ad amarli come parte di se stessa. Quando, gioisce delle gioie altrui e soffre delle loro sofferenze. 

Per l'animo e' sufficiente amare la sua stessa anima e non altre anime ``fuori''. Amando la sua anima, amera' lei stessa (il se') e tutte le altre parti ``fuori'' e che fanno parte dell'unione dell'anima.

Se l'essere $\anima{A}_i=A_1+A_2+\cdots$ e' pero' innamorato in maniera sana di un essere $A_j$, con anima $\anima{A}_j$, e quindi di un anima fuori dal suo se' ($A_j\ne A_i$), per l'animo $\spirit{A_i}$ e' sufficiente amare solo l'anima di cui e' innamorato, cioe' $A_j$. Infatti, se l'anima $\anima{A}_j$ ama $\anima{A}_i$ (in cio' consiste la sanita' dell'innamoramento), allora $\spirit{A_i}$ amando $\anima{A}_j$ amera' $\anima{A}_i$.
Quindi, per l'animo $\spirit{A_i}$ e' sufficiente amare $A_j$ per amare tutta $\anima{A}_i$, inclusa lei stessa $A_i$ e tutti gli altri esseri.

Quindi, sia se un'animo ama la sua stessa anima, sia se e' innamorato di un'altra anima, possiamo dire che per l'animo e' sufficiente una sola \emph{Anima} da amare.

Esempio: chi e' innamorato di Gesu', se fara' piacere a qualcuno che ha veramente bisogno, lo fara' perche' in lui vedra' qualcuno che Gesu' stesso ama, e non per altro fine. Vedra' chi ha bisogno tanto importante quanto per Gesu' e' importante. E quando qualcuno amera' come Gesu', in lui vedra' un aspetto di Gesu' stesso.

Allo stesso modo, chi e' innamorato di un altro essere, vedra' nell'anima dell'essere la sua Anima, e fara' cose simili.

Gli animi possono essere molteplici. Ma dato che un animo che ama la sua Anima si allea solo con animi che la amano, l'insieme di tutti gli animi che la amano agira' coerentemente e concordemente, ognuno nei suoi limiti, e percio' l'effetto complessivo sara' quello di un'unico cuore che pulsa, di molte braccia che spingono i remi nella stessa direzione, e di molte menti che scrutano l'infinito ma che armoniosamente scelgono un'unica direzione. Che cio' poi sia realizzato con democrazia o gerarchia, poco importa. Se ognuno ha di mira il fine ultimo, ovvero il benessere e la gioia dell'Anima, qualsiasi organizzazione che gli animi sceglieranno sara' buona.

Un animo non puo' obbligare un'altro animo ad amare la sua Anima. Questo anche quando le azioni dell'altro animo sono potenzialmente nocive o dannose (se non vengono prese ed attuate misure di difesa).


\def\Dio{\mathcal{D}}

\subsection{Dio come Anima}
\label{PureSoulAsGod}
\label{SecondaDefinizioneDio}

Definiamo l'anima di Dio $\Dio$ come quell'anima che e' in imperturbata pace, al di sopra di ogni gioia.

Poniamo l'anima di Gesu' $\anima{G}$ come anima di Dio: $\Dio=\anima{G}$. L'anima di Gesu', per il suo amore caritatevole verso tutta l'umanita' e' $\anima{G}=u_1+u_2+\cdots$, ed e' composta da tutti gli esseri umani, che vivono nel presente, che hanno vissuto nel passato o che vivranno nel futuro. Dio come anima ama tutti gli esseri umani, tutti gli esseri domestici o selvatici amati da almeno un essere umano, e la natura amata dal corpo di tutti gli esseri umani $a_i$.

Un essere umano $a_i$ ama imperfettamente $\Dio$, ovvero $a_i \subset \Dio=u_1+u_2+\cdots$. Nel suo cammino di fede e santita', pero' tende ad amare $\Dio$

\eal{& \lim a_i = \Dio }

Fino a quando $a_i \ne \Dio$, l'anima non e' completamente unita a Dio, e per lei Dio e' in parte, $\Dio\setminus a_i$, un'immagine sacra che coltiva dentro di se, in parte, $a_i\cap \Dio$, e' cio' che lei e' nel bene. Quindi, nella dottrina cattolica, $\Dio$ e' il $\emph{Padre}$.

Un'anima santa $d$, che ama perfettamente tutta l'anima $\Dio$, e' tutta unita a Dio, e vale 
    \eal{&d=\Dio}

La sequenza di cambiamenti che $a_i$ compie su se stesso per amare $\Dio$ e' il cammino spirituale. La via di Gesu' e' quella  della carita'. Ogni passo e' compiuto nella direzione della carita', nella direzione di amare e solo amare ogni uomo, ogni bambino, ogni anziano.

Per un'anima $\Dio$ e' la meta che cercata e trovata nel silenzio della preghiera e nel calore del cuore, le permette  di fare il passo successivo. In altri termini, $\Dio$ rispecchia chi lei veramente e' nel bene, superata ogni insicurezza, malizia e contrasto interiore. $\Dio$ e' chi $a_i$ vorrebbe essere nel bene ma non riesce nel presente con le sue forze umane, e che un giorno sara' nell'eternita' o per grazia divina in terra.

Inoltre, un'anima raggiunge tale limite piu' velocemente se cammina insieme ad altre anime che ricercano lo stesso limite. Questo e' il senso della comunita' religiosa. Non siamo delle isole, ed il confronto, la sana competizione, i piccoli incitamenti, rimproveri e le lodi sono preziose. Infine, un'anima, raggiunto tale limite, rimarrebbe incompleta se anche le altre anime che ama e che la amano, non lo raggiungessero. Questo e' il senso della carita'.
Nella carita', non si fa dono di elemosina. Nella vera carita' si fa dono della propria vicinanza ed unione a Dio, si dona la cosa piu' importante e preziosa che si ha.

\subsubsection{Come virtu', e sul camminare insieme}
Il limite descritto sopra
\eal{& \lim a_i = \Dio }
che in forma poetica si puo' scrivere come
\eal{&\lim_{\textrm{amore}\to\infty} \textrm{EssereUmano} = \textrm{Dio}}
si puo' pensare anche come 
\eal{& \lim_{v_i\to \infty} a_i = d }
dove $v_i=(v_{ip},v_{if},v_{ig},v_{if_2},v_{im},v_{ic},v_{is},...)$, sono le virtu' di $a_i$: pazienza, fortezza, giustizia, fede, misericordia, carita', speranza, .... Cosi', Dio e' il limite di un'anima con tutte le sue virtu' portate all'infinito.


\subsection{Dio come Spirito}
\label{DioComeSpirito}

Ricordandoci la definizione di spirito e desiderio (\ref{defDiDesideriDes} pag. \pageref{defDiDesideriDes}), definiamo lo Spirito di Dio come lo spirito i cui desideri sono tutti e solo relativi all'anima di Dio $\Dio$. Non importa se i desideri sono perfetti, ovvero se ognuno e' la soluzione corretta che realizza il relativo evento voluto. Quello che importa e' che il desiderio sia \emph{centrato} sull'anima \emph{Dio}, ovvero che l'animo voglia il bene dell'anima.

    Dio come Spirito viene visto comunemente in maniera superstiziosa come un essere che sovverte le forze della Natura. Tuttavia, lo Spirito non ha bisogno di essere un supereroe dei fumetti. L'anima ha bisogno di essere amata, e per cio' e' sufficiente che lo Spirito spenda la sua forza e le sue risorse energetiche. L'onnipotenza dello Spirito, risiede nel fatto che tanto piu' un'anima e' santa, tanto piu' riconosce l'amore dello Spirito, di cio' si nutre, si ricorda della sua origine divina, ritrova la forza, il coraggio e la gioia di vivere, e cosi' realizza in se stessa la Pace.  Questa e' la vera potenza, la verita' dello Spirito che ama l'Anima, e l'autorita' dell'Anima che ha su se stessa di riconoscere tale amore e tramite questo trovare la pace e la gioia.

    D'altro canto, se lo spirito sta' spendendo veramente tutto se stesso, anche se l'anima amata non riconosce il suo amore, lui potra' essere soddisfatto dell'aver veramente amato, perche' avra' appagato il desiderio della sua anima di amare una parte di se, che e' l'anima amata. Il sacrificio fatto per amore porta sempre frutto. A volte, come per Mose' che non pote' vedere la terra promessa, i risultati non saranno visibili nell'immediato, ma sempre un seme sara' piantato, e anche dopo molto tempo, germogliera'.

    Infine, anche se uno spirito terreno non dispone di super-poteri, se di fronte ad un'anima che cerca Dio dice che ``Dio non esiste'' perche' nessuno ha i super-poteri, sbaglia. E' lui che ha la responsabilita' di mettersi in gioco affinche' l'anima raggiunga la pace e, cosi', lei possa dire ``Dio esiste''.

Gesu' e' Dio come Spirito per eccellenza. Lui non si e' tirato mai indietro e come Spirito ha amato fino alla fine tutti, spendendo, bruciando e sacrificando la sua intera vita per dimostrare il Suo amore.

 Nel paragrafo \ref{loZero} pag. \pageref{loZero}, si approfondisce come l'Anima puo' venire incontro allo spirito degli uomini e delle donne che la amano nella terra, tramite il ``grazie'' e, nei casi piu' difficili, tramire il ``sacrificio''. 


\subsection{Dio come Figlio}
Gesu' non da' una definizione teorica di Dio. Cio' che dice, e'. Lui e' le virtu' che ama e predica, lui e' l'uomo che ha gia' percorso la strada ed e' i limiti che professa. Non predica di amare leggi e principi, ma predica di amare Lui stesso. Questo e' vantaggioso perche', l'amante si fa' uguale all'amato, quindi amare Gesu', vuol dire diventare Gesu', incarnare con la propria vita il Suo amore.

Lo Spirito Santo e' l'eredita' spirituale di Gesu', di cui abbiamo la responsabilita' di coltivare, ascoltare e mettere in pratica, e di riconoscere, amare e seguire nei ministri della Chiesa ispirati da Dio, in noi stessi e negli altri. Con lo Spirito Santo, Gesu' non ci ha abbandonato sulla croce. $\Dio$ non resta una definizione astratta del tipo ``esiste la Pace, esiste la Gioia'', ne' una legge sterile ``tendi alle virtu'! E avrai Pace.''. $\Dio$ diventa manifesto, diventa una dimostrazione concreta che la Pace e la Gioia esiste, diventa una via concreta per raggiungerle. E' una via diretta e privilegiata per toccare con mano l'esistenza Dio.

Gesu' non e' una figura creata ad arte da un'istituzione.
L'anima $a_i$ che ricerca la santita', ricercando e scegliendo il bene per se e per gli altri, tende a diventare $\anima{D}$, un'anima tende a diventare una piccola incarnazione e manifestazione di Gesu' in terra. Tanto piu' tende al limite $\anima{D}$, tanto piu' Dio per lei e' concreto, non e' un ideale, o un oggetto superstizioso, un'immagine all'infuori di se stessa. Gesu' ha dimostrato che l'essere umano puo' amare tutti, in ogni condizione si trovino, fino alla fine della sua vita, e facendo cio' raggiungere la Pace dell'anima (il regno dei cieli). L'essere umano puo' amare $\anima{D}$, naturalmente, anche superando dolori e difficolta', e in cio' trovare il senso della sua vita e la vera pace e gioia. Dio, allora, non e' una figura creata ad arte da regole e norme religiose, e', piuttosto, il limite ed il centro piu' vero e vitale di ogni essere che ricerca il Bene. 

La Chiesa, nel suo centro, seppur nei secoli ha irrigidito la dottrina, ha comunque tramandato una figura di Gesu' veritiera: non si puo' raggiungere la Pace e non si puo' Amare se si predilige il proprio Io su quello degli altri. Non si puo' vivere in pace nella societa', se non si accetta di doversi sacrificare per il bene degli altri, per quanto gli altri possono apparire in errore ai nostri occhi. Non si puo' essere grandi, se non si e' disposti a consumare, impegnare e rischiare tutta la propria vita. Non si puo' trovare l'Amore, solo nell'appagamento di pulsioni biologiche governate da un'istinto narcisista.

\subsubsection{Nota sulla santita'}
$a_i$ raggiunge la sua perfezione e completezza $\Dio$, non quando riesce in imprese eclatanti, ma piuttosto quando $a_i$ riesce ad essere, nel bene, cio' che veramente e'. E non importa che $a_i$ venga riconosciuto santo o meno, la ricompensa per $a_i$ sara' di vivere quanto piu' possibile la Sua pace ed la Sua gioia, ed essere con Lui fonte di amore in terra.

\subsection{Il Se e l'Altro come corpo di Cristo}

Dio e' nell'infinito, nel limite di un'anima che tende a diventare Anima, ma e' anche nel piccolo.
 In ogni respiro che un uomo o una donna compie, lei ha la tendenza di scegliere, inconsciamente, di vivere e di amare. Anche se inconsciamente, ha la tendenza negativa di fare scelte non sane, comunque, una parte di lei che tende alla vita esiste. Questa parte, coordina i suoi respiri, fa battere il suo cuore, ascolta i suoi organi e sente dolore se qualcosa non va. Ed anche altruisticamente parlando, quando e' alla presenza di altri, non ne trova dolore (a meno che non abbia subito traumi), anzi, la presenza di altri genera in lei desideri di vita, magari inizialmente solo egoistici, ma pur sempre desideri di vita.
In un'anima $a_i$, cosi', si nasconde lo Spirito che genera vita e vuole la vita, si nasconde Dio. Nel tempo, $a_i$, diventando sempre piu' ferma nello scegliere il vivere e l'amare, tende sempre meglio ad essere e manifestare questo Dio, cosi' come abbiamo detto nella definizione di sopra \ref{SecondaDefinizioneDio} pag. \pageref{SecondaDefinizioneDio}.


Fin'ora abbiamo parlato poco del \emph{Se'}, che e' molto simile al concetto di Anima ed anche di Io di un essere\footnote{In realta', sono tutti la stessa cosa, ovvero, sono l'essere stesso, ma ogni concetto fa vedere l'essere sotto una prospettiva diversa}. Il Se' di un essere vivente e' quell'oggetto che lui ama quando, agendo da soggetto (Io), ama se stesso. L'\emph{Altro} e' analogamente l'oggetto che l'Io ama quando ama un altro essere. In termini piu' poetici, il Se' (o l'Altro) e' l'essere a priori di qualsiasi idea, giudizio o pensiero che l'essere od altri possono avere su se stesso (o sull'Altro). Il Se' e' l'essere considerato a priori di ogni scelta personale conscia dannosa e non ottimale, o di ogni scelta ormai diventata inconscia a seguito di una stratificazione di molte scelte conscie nel tempo, ma che l'essere, adesso, rinnega (sovrastrutture non sane), come ad esempio, morali repressive della societa' che l'essere ha introiettato dentro di se. 

Il se', se amiamo l'altro, non e' solo la propria persona (la ``carne'' in gergo Cristiano). L'altro diventa una parte del se'. L'altro e' il se' che non e' esperibile tramite i propri sensi. L'altro e' conoscibile solo tramite il sentimento d'amore, che porta ad un'altra dimensione, ad un'altra realta', che non e' quella dei propri sensi personali. 

Il Se' e' come un legno che e' pronto per essere modellato e mantenere una forma. Si potra' dire ``che sia un tavolo'' o ``che sia una sedia'', a seconda di quello che serve.  Inizialmente l'Anima di un essere umano non sa' che forma dare al proprio Se', l'essere non ha ancora una identita'. Ma quando dice ``mi voglio bene'', assumera' l'identita' di una persona ed anima che ama se stessa, in formula $a_i=a_i+\cdots$. Quando dice amo quelle altre anime, assumera' l'identita' di un'anima empatica che ama altre anime, in formula $a_i=a_i+a_1+a_2+\cdots$. 

 Ogni Se' e cosa esistente, ha una proprieta' divina che e' appunto la sua esistenza. Per quanto un frutto di un albero possa essere amaro o brutto, esso esiste, cosi' come e', e per quanto un essere vivente sia fastidioso o pericoloso, egli esiste e desidera essere cio' che e'. Allo stesso modo noi stessi e gli altri.  Il Se' e l'Altro, a priori di qualsiasi desiderio, scelta ed identita', e' sacro, e' il dono piu' prezioso ed importante che abbiamo. In ogni momento che respiriamo stiamo usufruendo dei suoi doni: la vista, l'udito e gli altri sensi, la forza e la coordinazione che ci fa' e lo fa' stare in piedi e camminare, il comprendere le parole, il pensare, il vivere le emozioni. Il Se' e l'Altro che ci ama, e' sempre pronto a tendersi e consumare i suoi organi per i nostri desideri. Anche se l'Altro non ci ama, comunque stiamo ammirando lo svolgersi della vita in lui, che, dovrebbe essere incantevole allo stesso modo di guardare le stelle e le nebulose dello spazio, od emozionante allo stesso modo di scappare da un temporale improvviso. A volte, ci scoraggiamo perche' non abbiamo ottenuto niente o non abbiamo cio' che desideriamo, ma guardando al Se' o al Se' di chi ci ama, come un nostro genitore od una sorella, possiamo renderci conto che abbiamo gia' tutto cio' di cui abbiamo bisogno.
Per quanto ci amiamo poco o molto, esistiamo e viviamo, e questo e' miracoloso e grande. Quindi, il Se', per questa proprieta', e' divino, e' figlio di Dio Padre. Gesu' ha gia' dimostrato al mondo, che anche il piu' umile di questa terra e' amato dal Signore ed e' suo figlio, ed anche i peccatori lo sono. Noi siamo figli di Dio, siamo originati da Dio, per quanto il nostro cuore con il peccato ci deturpi e deformi e ci allontani dalla nostra vera forma e vita. 

Noi, nascendo inesperti e bambini, non conosciamo il vero Se', e solo i saggi lo conoscono fino in fondo, e solo i santi lo amano completamente. Abbiamo sempre la necessita' di amarlo, di ascoltarlo, di conoscerlo piu' a fondo, ed allo stesso modo il Se' altrui.  Questo amore, deve andare al di la' del nostro Io, di cio' di cui siamo convinti, di cio' di cui sono convinti gli altri, al di la' della nostra storia personale, delle sconfitte e degli allori. L'Io devo porsi come umile servo del Se', che, ripeto, al di la' della nostra idea di noi stessi, e' divino e sacro, e' figlio di Colui che E'. Non conosceremo mai completamente il Se', ne' saremo mai in grado di amarlo pienamente. Ed amarlo non ci fara' grandi ai nostri ed altrui occhi. Forse anzi, per amarlo dovremmo accettare una vita molto piu' umile di quanto avevamo immaginato per noi, od una vita piu' tumultuosa e rischiosa, o piu' semplice. Anche per questo si capisce che il vero Se', proprio ed altrui, e' tanto alto e sacro, ed allo stesso tempo e' tanto umile e piccolo quanto un granello di sabbia. Il Se' e l'Altro, nella loro umilta' e piccolezza, sono figli di Dio. Non confondendo il Se' con l'Io, si puo' vedere che il Se' e' la vita di un essere, e per questo e' Dio stesso. Si deve avere ben presente che ogni Se', non solo il proprio, e soprattutto quelli non facilmente amabili, in alcuni loro aspetti brutti, difficili, pericolosi, costosi, fragili, sofferenti, sono amati dal Signore, cosi' come i poveri e malati, scartati dalla societa' dell'India, sono stati amati da Madre Teresa di Calcutta.
Considerando il nostro corpo come materia tangibile del Se', come dice San Paolo, esso e' il corpo stesso di Cristo. Con la comunione, ci uniamo al corpo di Cristo, e i dolori o le emozioni che proviamo, e' Cristo stesso a patirle o rallegrarsene \footnote{Prima lettera di San Paolo ai Corinzi, capitolo 10, versetti 16-17 \url{https://bibbiaedu.it/CEI2008/nt/1Cor/10/} ed anche Capitolo 12, ver. 12-26}: ``il calice della benedizione che noi benediciamo, non è forse comunione con il sangue di Cristo? E il pane che noi spezziamo, non è forse comunione con il corpo di Cristo? Poiché vi è un solo pane, noi siamo, benché molti, un solo corpo: tutti infatti partecipiamo all'unico pane. ''.
Quindi, esiste un unico Se': il corpo di Dio, e piu' concretamente, il corpo di Cristo.

Dio padre, rimane una figura trascendente, ma che rispecchia cio' che un fedele cristiano anela ed ama. Dio padre, e' lo specchio dell'amore del Se' che va al di la' dei limiti umani. Dire, ``Tu solo puoi, padre, salvare quest'anima'', e' un riconoscere umilmente i propri limiti nel fare del bene verso una persona, ed al contempo amare quella persona, anche se non si sa' fare e non si puo' fare niente di materiale, orientandosi verso il bene massimo per quella persona.

Definiamo matematicamente il Se', come l'insieme vuoto: $d=\{\}$. L'insieme vuoto, non contiene nulla, neanche il proprio Io, i propri bisogni o desideri, ne' i bisogni o desideri di altri. L'insieme vuoto ha la potenzialita', a seconda dei desideri o dei bisogni dell'Io o di Altri, di trasformarsi per contenere l'Io e gli Altri. Quindi, quando un'anima ama se stessa, l'insieme vuoto diventa un'altro insieme che contiene gli eventi dei suoi piaceri e dei suoi dolori, come gia' detto in \ref{AnimaInsiemeEventi} pag. \pageref{AnimaInsiemeEventi}. Quando un'anima ama ed e' amata da altre anime, diventa un altro insieme che contiene loro.
\eal{
    &\{\} \;\;\;\rightarrow\;\;\; a_i\;\;\;\rightarrow \;\;\; a_i+a_1 \;\;\; \rightarrow \;\;\; a_i+a_1+a_2+\cdots
}

Considerando il Se' come sacro e considerando un desiderio od una necessita' come un consumo del Se' proprio od altrui, meno si abusa di desideri meglio e', piu' si soddisfano i desideri in maniera naturale, senza sforzare se stessi o gli altri, meglio e'.


\subsection{Lo Spirito Santo}
\label{definizioneAnimo}

Abbiamo detto che un $a_i$ che ama $\anima{U}$ tende ad incarnare in se stesso Dio nei suoi aspetti positivi, che sviluppa man mano nella sua crescita, e di cui alcuni possono essere temporanei, mentre altri piu' duraturi. Nella dottrina Cattolica, $a_i$, nei suoi aspetti positivi e' chiamato Spirito Santo. Quindi, oltre al Padre trascendente e al Figlio Gesu' Cristo, Dio agisce in terra tramite lo Spirito Santo, dato a noi in eredita' da Gesu'. Lo Spirito Santo presente in $a_i$ rappresenta la parte di $a_i$ che riesce ad amare $\anima{U}$.

Che l'Io tende ad essere Dio stesso, non vuol dire che diventa un super eroe, perfetto ed intoccabile, al di sopra della natura, che non ha bisogno di nessuno, che e' al di sopra delle leggi, e che puo' fare cio' che vuole di ogni vita. 
Dire che l'Io tende a diventare Dio, vuol dire che tende ad essere speso ed indirizzato verso tutti i bisogni ed i desideri necessari dell'Anima. Nella sua vita ogni fase che passa e situazione che supera lo fa crescere e, sempre di piu', niente di lui/lei turba il se' o gli altri, coglie e comprende ogni desiderio piu' essenziale e vitale dell'Anima, e riesce a spendersi verso la realizzazione di tale desideri. Ama l'Anima nella sua completezza e nell'individualita' di ogni sua parte, ovvero ogni essere che la compone. Ogni azione e pensiero che muove, respiro ed intenzione che alimenta, e' volto a creare la vita in ogni essere, a mantenerla ed esaltarla. E tutto questo, nel rispetto dei limiti fisici e psicologici degli esseri che sta' amando, ovvero nel rispetto del corpo e della psiche degli esseri.

Facciamo un esempio. Se pensiamo ad un panetterie che fa' onestamente il suo lavoro, il suo punto all'infinito e' un Panetterie, ovvero, un uomo che non solo fa' onestamente e con cura il suo lavoro ma che lo fa' perche' ama se stesso ed ogni suo cliente. Idealmente sa' a chi piace quale tipo di pane e se avesse tempo farebbe un pane personale per ogni suo cliente, adatto in quel giorno alle circostanze che il cliente sta' vivendo. Finito il suo lavoro, e' in pace, e pur stanco, loda Dio per quello che gli dona, per i suoi fratelli e le sue sorelle, per i suoi figli, se ne ha, e, pur messo in difficolta' dal male che alberga irrisolto nei cuori altrui, e' in pace con tutti. 

Il fatto che un Io santo non abbia fisicamente capacita' infinite, non vuol dire che non sia manifestazione di Dio. Vuol dire che Dio Padre in Lui e' trascendente. Se quell'Io esiste veramente per l'Anima e direziona le sue energie per servirla e vederla gioire, allora cio' che sta' agendo non e' piu' un Io terreno ma e' Dio.
Ad esempio, se un insegnante si sveglia ogni mattina e con premura si dirige verso la scuola per il bene dei suoi alunni, allora loro, amandolo, vedranno in lui un riflesso della vera luce.
Che un Io muovi un solo piccolo passo verso l'infinito, non e' cosa per niente semplice da conquistare, ed ogni sforzo di se stesso, o di chi ama e lo ama, impiegato ad amare l'Anima, e' prezioso. Infatti, ogni sforzo vuol dire fisicamente e psicologicamente un consumo del corpo e della vita.
D'altra parte, pero', solo Dio puo' scegliere un Io per amare l'Anima in un certo tempo, per una certa durata e in un certo luogo e in che modo. Se un Io volesse essere bravo ma Dio non volesse amare tramite lui, allora non nascerebbe amore. E tale cosa non puo' essere forzata: nessun essere, neanche tutti gli esseri insieme, per quanto numerosi e capaci, possono amare l'Anima per loro volonta' egoistica o tramite cose frapposte tra loro stessi ed Ella, come ad esempio, tecnologie, possedimenti, beni e risultati meritevoli. 

Tutto cio' detto e premesso, ci sono due dimensioni su cui si puo' lavorare: una dell'anima che e' un costante definire, affermare e rifinire cio' che e' buono, cio' e' piacevole, cio' che e' vero. Una dell'animo che e' un costante capire, allenarsi e fare per realizzare e mantenere la salute ed il piacere dell'anima.

\subsection{Lo zero}
\label{loZero}
Un'anima puo' venire incontro agli animi, stabilendo che cio' che e' stato gia' raggiunto e' buono e non chiedendo piu' di cio' che sta' ricevendo: ``Grazie. Tutto cio' che e', e' buono, e' Sua volonta'.''.

In pratica, l'anima, paradossalmente, raggiunge Dio quando si rende conto che il tutto e' gia' buono, sacro e prezioso.

Anche se sembra facile a dire a parole, questo ``rendersi conto'' e' costoso, ed a volte richiede sacrifici. E' essere come un imprenditore che avendo tutta la potenzialita' di far accrescere il capitale della sua azienda, decide di non andare oltre, in nome di un bene piu' grande, ad esempio, perche' gia' il fatturato e' piu' che buono e non serve inquinare oltre l'ambiente, o sottoporre i lavoratori ad ulteriore stress, od impiegare nuove persone per lavori banali.

E' essere come un santo che, pur avendo una ferita che lo attanaglia, non si abbatte, ne' rinnega Dio, e continua ad avere una serenita' superiore e ad essere a disposizione degli altri.

Quindi, e' una cosa grande quando l'anima dice ``va tutto bene'', anche quando potrebbe ottenere di piu' per il se', o per un altro, ma a discapito di qualcuno o del se'. E' una cosa grande quando l'anima dice ``va tutto bene'', anche quando la situazione e' difficile da sopportare e affrontare.

Se l'anima fa' cio' di sua spontanea volonta' e naturalmente, senza essere oppressa e obbligata dall'Io, ne' con la scusa di superstizioni, ne' di regole o leggi, allora in cio' l'anima trovera' piu' rapidamente Dio.

Ad esempio, anche se il mondo sembra apparentemente impazzito, crudele, duro e ingiusto, soffermandosi uno puo' vedere che nel traffico le persone rispettano le altre macchine, che per strada la grande maggioranza delle volte non si e' disturbati dagli altri, che l'educazione ed istruzione ricevuta a scuola, anche se ancora lontana dalla perfezione, ha un senso perche' forma, rende cittadini della societa' e, da' strumenti intellettuali, che se coltivati in proprio, sono utili. E ancora, pur la societa' avendo strada da fare, e' comunque lodevole in alcune parti del mondo: ospedali, scuole, citta' dove non regna la violenza incontrollata (come potrebbe essere nella giungla), etc... E, infine, gioire della buona salute di cui si gode.

In altre parole, tutto questo potrebbe essere scontato ma non lo e' e gia' di questo si potrebbe essere, se non contenti, almeno comprensivi dello stato attuale.

Tuttavia, se si fa' questo discorso ad una persona sofferente, ad esempio, una persona che ha perso il lavoro e che quindi nutre un risentimento verso la societa', verso il suo superiore o alcuni dei suoi colleghi, allora non e' detto che quella persona possa recepirlo. Se non lo recepisce e si cerca di convincerla, allora, lei si difendera' e si allontanera' dal nostro impulso positivo iniziale. Solamente amandola veramente, e stando con lei, magari anche non dicendo niente, pregando nel proprio cuore che possa andare oltre la sua sofferenza, lei fara' il suo cammino e poi, un domani arrivera' ad una consapevolezza simile.

A volte il ``grazie'' dell'Anima e' un ``sacrificio'', perche' l'Anima aveva \emph{bisogno}, ma cio' che riceve materialmente non e' sufficiente.

Questi sacrifici, non sono arbitrari, repressivi od oppressivi. Sono sacrifici che l'Anima si trova a scegliere se compiere per mantenere il suo amore per gli altri o la salute del corpo. Sono sacrifici, che nel massimo dell'impegno e capacita', non hanno alternative piu' semplici e indolori.

Matematicamente, pensando al Dilemma del Prigioniero, esistono situazioni in cui l'unico modo per mantere l'ottimo, e' scegliere dei sacrifici personali. Nel dilemma del prigioniero\footnote{\url{https://en.wikipedia.org/wiki/Prisoner's\_dilemma}}, vedi tabella \ref{tabPrisonerDilemma} pag. \pageref{tabPrisonerDilemma}, ognuno ha un rendiconto personale maggiore se tradisce l'altro, e, inoltre, se entrambi collaborano il rendiconto non e' il massimo possibile per ognuno. Tuttavia, solo se ciascuno rischia di essere tradito, potra' non tradire l'altro, e ottenere l'ottimo dell'Anima piuttosto che della sua anima (social welfare, ovvero somma dei rendiconti di ogni giocatore).

\begin{center}
    \begin{table}
        \begin{tabular}{ |c|c|c| }
            \hline
            &  Supporta   &   Tradisce \\
            \hline
            Supporta   & (-1,-1) & (-3,0) \\
            \hline
            Tradisce & (0,-3) & (-2,-2) \\
            \hline
        \end{tabular}
        \caption{\label{tabPrisonerDilemma}Dilemma del prigioniero: due criminali che insieme hanno commesso un reato, vengono posti in due celle lontane e non comunicanti tra loro. Ad ognuno viene detto che se testimonia a sfavore dell'altro prigioniero (tradisce) non avra' nessuna pena se l'altro invece non lo tradisce, e avra' una pena di due anni se invece l'altro lo tradisce. Se entrambi si supportano a vicenda non tradendosi, allora' ognuno avra' una pena di un anno. Sopra e' riportata la tabella, come si usa in teoria dei giochi. La riga indica la scelta del prigioniero A, la colonna indica la scelta dell'altro prigioniero B. Ad esempio, la casella in alto a destra indica che il prigioniero A supporta il prigioniero B, mentre B tradisce A. In questo caso, A avra' 3 anni di prigione, mentre B nessuno. Se A e B si supportano a vicenda, la somma delle pene di entrambi e' di 2 anni di prigione. In tutti gli altri casi la somma e' maggiore. Quindi, il ``social welfare'', ovvero il benessere sociale, e' massimo se entrambi si supportano. Dal punto di vista egoistico, in media, se un giocatore tradisce ha $(0+(-2))/2 = -1$, un anno di prigione. Se invece supporta ha in media $((-1)+(-3))/2=-2$ anni di prigione. Quindi, probabilisticamente, dal punto di vista egoistico e' meno rischioso tradire. }

    \end{table}
\end{center}

\pagebreak

\subsection{Dio come realizzazione vera di ogni desiderio}
\label{FormulazionePuntualeDiDio}

Diamo una definizione che si rifa' alla discussione \ref{DioScientificamentePsicologicamente} pag. \pageref{DioScientificamentePsicologicamente}. Questa definizione non usa il concetto di limite, ma e' un modo alternativo ed equivalente, e sotto certi aspetti piu' concreto, del dire che Dio e' quell'essere il cui amore e' infinito.

Prima definiamo l'insieme $\textrm{Des}(a)$ (gia' definito sopra a pagina \pageref{defDiDesideriDes}) e per assioma poniamo che e' infinito:
    \begin{align*}
        &\textrm{Des}(a) = \textrm{insieme dei desideri dell'anima } a\\
        &\textrm{Des}(a)\;\textrm{ e' infinito }
    \end{align*}
    poi diamo la seguente definizione: Dio e' quell'unico essere $D$ tale che
    \begin{align*}
        &\forall a \in \textrm{Anima}\;\;\forall b \in \textrm{Des}(a)\;\;b \textrm{ e' soddisfatto da }D
    \end{align*}
    Nota: desiderio ``soddisfatto'' non vuol dire che si realizza alla lettera, ma vuol dire che il bisogno dell'anima che sottende tale desiderio e' soddisfatto o lenito a tal punto da poterne fare a meno, se non e' lecito che sia realizzato.\\

    \textbf{Teorema.} Vale la seguente:
    \begin{align*}
        &\textrm{Gesu'}\in \{ \;U\in\textrm{Uomo}\;|\;U=\textrm{Dio}\;\}
    \end{align*}
    dove l'uguaglianza $U=D$ e' intesa in senso spirituale. Anche un paralitico $U$, puo' essere santo!

    \textbf{Corollario. } Dio esiste, in quanto esiste Gesu'.

    A parole: esiste almeno un uomo che e' Dio, ovvero il cui amore per se stesso e gli altri, in questa natura e vita terrena, e' infinito. Tale uomo e' Gesu'. \emph{E'}, e non \emph{e' stato}, perche', spiritualmente, Gesu' e' risorto. Altri uomini il cui amore e' infinito, ugualmente a quello di Dio, sono i santi. I santi percio', hanno pienamente lo stesso spirito di Gesu' e spiritualmente vale $U=D=\textrm{Gesu'}$. La loro anima ha trovato felicita' e pace nell'essere uguali a Gesu'. Essendo uguali a Gesu', sono uguali a Dio, ma di cio' non se ne vantano, e rimandano ogni anima solo a Gesu', affinche' ogni anima non sia confusa nella ricerca dell'\emph{unico} Dio\footnote{Se spiritualmente $U_1=D$ e $U_2=D$, allora $U_1=U_2$, quindi, nell'equivalenza, esiste un solo $U$ tale che $U=\textrm{Dio}$. Si puo' scegliere Gesu' come rappresentante della classe d'equivalenza $\{ \;U\in\textrm{Uomo}\;|\;U=\textrm{Dio}\;\}$. Per questo, si puo' dire che Gesu' e' l'unico Dio, e che ogni santo, non e' che una incarnazione dello spirito santo di Gesu'}.

%\subsubsection{Divertendoci con le formule...}
%\label{SecondaDefDivertendociConLeFormule}
%
%Possiamo porre un desiderio $b\in \textrm{Des}(a)$ come una funzione $b:\Omega\longrightarrow \mathbb{R}$ che valuta se l'universo nello stato $u\in \Omega$, dove $\Omega$ e' lo spazio di fase\footnote{il concetto di spazio di fase e' stato gia' accennato in \ref{extInt} pag. \pageref{extInt}. E' l'insieme di tutti i possibili stati dell'universo. Cioe' tutti i possibili istanti di tutte le possibili storie in cui l'universo si puo' sviluppare}, e' realizzato. $b(u)$ tende a $0$, tanto piu' $u$ realizza $b$.
%Dio dispone di una funzione di scelta $C$, tale che $C(\Omega)$ e' il sottoinsieme massimo di $\Omega$ i cui stati sono quelli in cui \emph{ogni} $a$ vive in pace, ovvero, per ogni $b$ di ogni $a$, la funzione ristretta $b_{|C(\Omega)}$ diventa costante e uguale a $0$, o in altri termini $b(C(\Omega))=\{0\}$. Cosi', Dio e' quell'unico essere $D$ tale che, la sua funzione di scelta $C$ su $\Omega$ e' tale che
%    \begin{align*}
%        &\forall a \in \textrm{Anima}\;\;\forall b \in \textrm{Des}(a)\;\;b(C(\Omega))= \{0\}
%    \end{align*}
%Nota che, da questa definizione, avendo posto ${\forall a\in\textrm{Anima}}$, discende che $C(\Omega)$ deve includere solo stati in cui $a$ vive con carita'.
%
\subsection{Definizioni negative}

Fin'ora tutte le definizioni date sono positive. Per ognuna, si puo' dare una definizione speculare negativa, che nasce dal fatto che siamo esseri limitati e imperfetti, e che $a_i$ di norma non coincide con il limite $\Dio$. Le definizioni positive sono da prediligere, perche' costruire e' difficile ma costruendo e' piu' facile mantenere cio' che e' gia' stato costruito, invece, distruggere e' facile, ma e' anche piu' facile vanificare cio' che c'era gia' di buono.\\

Considerando un'anima $\mathcal{A}$, ella e' \emph{egoista} se include nell'unione $\mathcal{A}$ almeno un essere $A_i$, non per amarlo, ma solo per amare un $A_j$, con $j$ distinto da $i$ ($j\ne i$).

Di solito $A_j = \self(\mathcal{A})$, cioe' $\mathcal{A}$ considera altri esseri $A_i$ non per amarli ma per amare il se'.\\

Un'anima $\mathcal{A}=A_1+A_2+\cdots$ e' \emph{narcisista} se e' innamorata di un $A_i$, e quindi crede che i bisogni e desideri di tutti gli $A_j$ sono rappresentati dai bisogni e desideri di $A_i$, quando questo non e' vero. Ad esempio, se $A_i$ predilige cibi salati, mentre almeno un $A_j$, distinto da $A_i$, non li ama, allora $\mathcal{A}$ comunque pensera' che per $A_j$ e' un bene mangiare cibi salati, e magari cerchera' di convincere $A_j$ di cio'.


\subsection{Futuri sviluppi}
\label{FuturiSviluppiProcAssiomatico}
In futuro, sarebbe bello trattare l'argomento su Dio, esplorando altre definizioni oltre quella del limite infinito.
Quella che segue e' una bozza.

	\textbf{Idea. } Sia $\textit{Anima}$ l'insieme di tutti gli esseri, compresi quelli che stanno nei cieli\footnote{Per capire cosa si intende con ``che stanno nei cieli'', vedere poesia a pagina \pageref{DioScientificamentePoesia}} e che stanno in terra. Per due esseri $E,A$, sia l'amore di $E$ per $A$, una funzione 
    \[\textrm{amore}_E(A):\textrm{Stato}\longrightarrow\textrm{Stato}\]
    che porta l'anima $A$ da uno stato, es. tristezza, ad un altro stato, es. felicita'. Allora, definiamo Dio $D$, come quell'unico essere tale che
    \begin{align}
        &\exists ! D \in \textrm{Anima}\;\forall A \in \textrm{Anima}\;\;\nonumber\\
        &\qquad\textrm{amore}_D(A): \textrm{Pace}\cup\textrm{Sofferenza} \longrightarrow \textrm{Pace}
    \end{align}
    Poniamo $\Diomath=D$, e riscriviamo succintamente le precendenti proposizioni:
    \begin{align*}
        &\textrm{Pace}\cup\textrm{Sofferenza} \stackrel{\textrm{amore}_{\Diomath}}{\longrightarrow} \textrm{Pace}
    \end{align*}

\section{Dall'esterno verso l'interno}
\label{extInt}
Con serenita', lasciamo per un momento ogni proposito come se fosse realizzato, e osserviamo tutto cio' che puo' essere osservato e con cui, almeno con l'immaginazione, possiamo interagire. Questo tutto e' una cosa sola. ``Tutto'' include ogni cosa, quindi, non c'e' niente oltre al tutto.
Questa cosa, nel complesso, cambia. \emph{Prima} e' in un modo, \emph{dopo} e' in un altro modo. Nella piu' assoluta generalita', possiamo immaginare come se cambiasse di colore. La tavolozza dei colori possibili ha una gradazione infinita, cosi' che puo' cambiare in infiniti modi. Ogni volta che cambia, diciamo che e' passato del \emph{tempo}. Cambia da colore a colore, non a caso, ma seguendo una regolarita'. 

La prima regolarita' e' questa: il tutto e' divisibile, si puo' pensare composto da una molteplicita' di elementi indivisibili: gli atomi\footnote{o piu' precisamente, le \emph{particelle}}.
Quanti sono? Sono molti o pochi? Non c'e' motivo di dire ``pochi'' o ``molti'', quanti atomi sono  e' semplicemente un numero: $10^{80}$, 10 elevato ad 80, ovvero 1 seguito da 80 zeri, cioe' cento milioni di miliardi di miliardi di miliardi di miliardi di miliardi di miliardi di miliardi di miliardi. 
Ora questo numero non e' ne' tanto spaventevole, ne' tanto affascinante. Infatti, la Natura non ha il concetto di proporzioni umane, di poco o molto. Siamo noi, nella nostra esperienza che diamo una proporzione alle cose, e diciamo ``quella scala ha molti gradini, sara' faticosa salirla''. 
I tantissimi atomi non esistono in proprio ne' per fare qualcosa di bello, ne' di opprimente.
Concludiamo il discorso dicendo che il numero di atomi e' calcolato dagli scienziati con metodi ed esperimenti molto raffinati, basati sulla traiettoria delle stelle, delle galassie, sul colore delle stelle, e tanti altri parametri.

La seconda regolarita' e' che ogni atomo ha una \emph{posizione} e una \emph{velocita'}. In genere pensiamo a qualcosa \emph{posto} dentro \emph{qualcosa}. Cio' in cui sono posti gli atomi, sia chiamato \emph{spazio}. Se l'atomo si trova in una posizione $P$, la sua velocita' indica in quale posizione $Q$ si trovera' in un istante successivo. 

Se consideriamo le posizioni e le velocita' di tutti gli atomi come un singolo oggetto (matematico), chiamato \emph{configurazione nello spazio di fase}\footnote{\url{https://en.wikipedia.org/wiki/Phase\_space}}, e assegnamo un colore distinto ad ogni configurazione distinta, allora capiamo la prima frase ``...esiste solo una cosa. Questa cosa cambia di colore in colore...''.

E noi? Siamo come le stelle: formazioni naturali, spostanee. Rispetto alle stelle, eccelliamo in complessita': abbiamo milioni di cellule, di diversi tipi, che gia' di per se' complesse, formano organi ancora piu' complessi, ognuno che soddisfa una regolarita': il cuore pompa il sangue, il fegato lo purifica, i polmoni assorbono ossigeno, etc...

La cosa piu' difficile da accettare e' questa: non c'e' niente e nessuno che dirige o vuole la sintonia con cui lavorano i nostri organi e gli organi degli altri. Potremmo dire che noi stessi lo vogliamo, tuttavia, non completamente. Per ferite psicologiche subite nell'infanzia, che forse neanche sappiamo di avere e per errori commessi nella vita e per i loro conseguenti traumi, neanche noi stessi vogliamo completamente, in maniera perfettamente sana, la vita per noi e per gli altri. A volte scegliamo la vita, a volte la morte, in maniera inconscia o conscia. La fede afferma che l'unico che ci ha amato completamente e' Gesu', perche' lui era santo, fin dalla nascita. L'unico che ci puo' amare completamente e' il Suo spirito che ci ha lasciato in eredita', se noi gli diamo spazio e lo coltiviamo in noi stessi. Se non amiamo, \emph{la vita e' materia che si aggrega, trasforma, disgrega continuamente, meccanicamente, senza alcuno scopo o volonta'}. Se amiamo, se facciamo nostro l'amore e lo coltiviamo, la vita diventa poesia, piacere, a volte dolore -per il forte desiderio del piacere fin'ora vissuto e che sembra svanito-, combattimento per il ristabilimento del piacere, e ancora poesia. 

Cosi', se ami, come Gesu', decidi di che fartene di tutti quei $10^{80}$ atomi e di tutte quelle cellule, quel sangue, muscoli e nervi che formano le anime di chi ami (compreso tu stesso / te stessa)\footnote{l'anima e' vista non come cosa incorporea, ma come un tutt'uno con il corpo. Non c'e' anima senza corpo, ma anche corpo senza anima! Il corpo e' in se' una cosa morta, e' l'anima e' cio' che lo rende vivo. Ma, in vita, l'anima, non puo' essere intesa come distaccata, separata, anche di un solo atomo, dal corpo}, e al contempo, allo stesso modo e' chi ti ama come Gesu' che decide per loro. Se e' ``amore'' nel senso classico, allora queste decisioni portarenno al Bene. Gli effetti di queste decisioni, saranno sottili ed invisibili, ma saranno cio' che alimentera' veramente la Tua vita.

In questo Universo d'amore, allora, alzandoti presto la mattina, potrai dire ``sorgi Sole, riscalda le nostre membra'', e un'intensissima fusione nucleare illuminera' il cielo con raggi di luce che riscalderanno senza bruciare.



\section{Appendice A}
\label{menteCrea}

Un uomo riceve segnali sensoriali dal mondo esterno. Tramite il suo cervello, elabora questi segnali e (inconsciamente e cosciamente) crea un modello che descrive e predice tutti gli stimoli che riceve. Ad esempio, in base alla sua esperienza, quando vedra' un zona luminosa, di colore rosso, che emana calore, la cataloghera' col concetto di "fuoco". Non si avvicinera' a questo ``fuoco'' perche' dentro di se predice che una tale azione avra' effetti dolorosi\footnote{Questa sua conoscenza deriva o da una esperienza che ha fatto da bambino, oppure da un ammonimento ricevuto dai genitori}.\\
La realta' che lui percepisce e' una realta' che lui sta' creando. Ad ogni segnale sensoriale che riceve da' un significato, ad esempio, alcuni segnali luminosi saranno per lui delle ``forme'' e alcune forme le pensera' come ``oggetti''. Agli oggetti attribuira' delle proprieta'.  Quindi, ad esempio, se avesse piena coscienza dei suoi meccanismi cognitivi, potrebbe dire: ``dalla luce che vedo dai miei due occhi riesco a tracciare delle forme. In particolare, una forma che vedo e' compatta e ha una profondita'\footnote{la profondita' e' percepita dal fatto che ogni occhio riceve la luce da due punti differenti e da altri indizi, vedi \url{https://en.wikipedia.org/wiki/Depth\_perception}} quindi dico: ``e' un oggetto'', inoltre, noto anche le seguenti proprieta': e' tonda e grigia. Per tenerla in mano, avverto uno sforzo muscolare, quindi dico ``e' pesante''. Considerando tutto, dico ``esiste un oggetto tondo, grigio e pesante. Ho visto altri oggetti simili e ho imparato a chiamarli \emph{pietre}. Siccome, tutte le pietre che ho visto fin'ora le ho sempre ritrovate nel posto in cui le lasciavo, dico che qui dove mi trovo, esiste una pietra\footnote{In questa frase stiamo anche implicitamente considerando il suo concetto di spazio, che e' sempre un qualcosa che l'uomo crea. Vedi \url{https://en.wikipedia.org/wiki/Spatial\_ability}}''.\\




