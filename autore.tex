
\chapter{Note sull'autore}

Ho iniziato a ragionare sulle tematiche di questo testo dopo un confronto con un praticante Sufista, che ha studiato filosofia, questo all'incirca nel 2012. Seguo un percorso di psicoterapia dal 2010, questo a seguito di alcuni blocchi che sono affiorati dentro di me. La psicoterapia mi e' piaciuta dall'inizio perche' mi ha dato l'occasione di dedicarmi ai miei sogni autentici e non a cose aliene a me. Con la psicoterapia ho migliorato la mia vita, mentre con la filosofia e la pratica centrata sull'amore mi sono avvicinato piu' semplicemente e piu' direttamente alla Verita'. \\

La verita' di cui parlo altro non puo' essere che le conclusioni sulla vita tratte dalla mia esperienza\footnote{in realta', anche la mia tradizione, ovvero l'esperienza tramandatami dalla mia famiglia e da altri uomini e donne che ho amato}, dal mio percorso psicoterapeutico e dai miei studi.

Questa verita' e' stata distillata e scritta cercando, praticando, sbagliando, pentendomi, frenandomi, e ancora cercando, praticando, ... Questo in una associazione culturale (in cui ho passato la mia adolescenza), in un centro sociale, nella famiglia, nel lavoro, e di recente in comunita' in una parrochia (Gen 2021). Da quando ho iniziato coscientemente questo percorso, sono passati 7 anni (dal 2014). 

Non sono santo, ovvero non rispetto sempre cio' che ho detto, non sono sempre in pace con me stesso, gli altri e la vita. Tuttavia, ogni volta che ritorno in una situazione pesante, mi trovo piu' resistente e l'affronto meglio, ed ogni volta che ne esco ritrovo che quanto credo e' vero e, se non vero, trovo nuovi spunti per arrivare ad una nuova formulazione della verita' piu' semplice e bella. Quindi, posso dire che credo in quello che ho scritto, e che applico cio' che credo nella mia vita per quanto posso.

Quando ho trovato la calma o la forza nel tempo libero di sedermi e rappresentare quello che avevo scoperto e quello che ormai avevo assodato, scrivevo. Molte volte anche se mi sarebbe piaciuto scrivere una bella idea, o correggere il testo ho rimandato perche' non consideravo l'idea abbastanza matura. L'idea era solo una bella idea, ma era piu' vanita' che verita'. Cosi' facendo, ho iniziato a scrivere nel Novembre del 2019 (oggi e' il \finishDate).\\
Cio' che e' scritto e' condiviso nell'augurio che o possa dire a voce aperta cio' che gia' credi o che possa essere spunto di riflessione. Nello specifico, che si possa 
\begin{enumerate}
    \item intuire che la felicita', quella vera, che conoscevamo quando eravamo bambini, puo' esistere, totalmente e globalmente, anche in questo mondo complesso, anche nelle situazioni difficili;
    \item che si possa ritrovare il concetto di Dio, quello dell'amore e della fede vera, calato nella cultura moderna scientifica, senza alcuna contraddizione con la visione scientifica del mondo. Sia la scienza, sia la fede hanno qualcosa da dire all'uomo;
    \item che si possa capire che il mondo cambia nel momento in cui cambiamo noi stessi (e non in altro modo), e che non ci sono poteri piu' forti, come il capitalismo, che possano bloccare l'altruismo.
\end{enumerate}

Infine, sono nato nel sud Italia all'incirca nel 1990 da un bravo padre e da una affettuosa madre.

Ho studiato matematica, lavoro come programmatore su sistemi Linux.

\subsection{La farfalla}

La farfalla in copertina e' il disegno di una formula matematica chiamata ``butterfly curve''. Si puo' disegnare con Gnuplot con il seguente codice:
\begin{verbatim}
set polar
set style line 5 lt rgb "#ADD8E6" lw 3
plot exp(sin(t)) - 2*cos(4*t) + \
        (sin(1/24*(2*t-pi)))**5 ls 5
pause -1
\end{verbatim}
\url{https://en.wikipedia.org/wiki/Butterfly\_curve\_(transcendental)}
