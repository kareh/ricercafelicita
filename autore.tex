
\chapter{Note sull'autore}

Ho studiato matematica e informatica. Nel 2012, dopo un confronto con un praticante Sufista, laureato in filosofia, ho iniziato ad esplorare il mondo della spiritualita', ovvero della cura della propria interiorita' per crescere nell'amore caritatevole. Ho seguito un percorso di psicoterapia per piu' di 9 anni. Con la psicoterapia ho avuto gli strumenti per lavorare sulle carenze della mia vita, mentre con il ragionamento filosofico e la pratica spirituale ho avuto la motivazione per compiere questo lavoro. Questo, rispetto a tutto quanto mi aveva fornito il mondo, come l'istruzione, l'universita', il lavoro e le amicizie, mi ha avvicinato piu' semplicemente e piu' direttamente alla Verita'.

La verita' di cui parlo altro non puo' essere che le conclusioni sulla vita tratte dalla mia esperienza e dall'esperienza di altri uomini e donne che amo.

Non sono santo, ovvero non sempre ho la forza di aderire a cio' che scrivo, non sono sempre in pace con me stesso, gli altri e la vita. Tuttavia, ogni volta che ritorno in una situazione pesante, non mi perdo, non odio la vita, so perche' resistere, e piano piano mi risollevo. Ed ogni volta che ne esco ritrovo che quanto ho scritto e' vero, o trovo nuovi spunti per arrivare ad una formulazione della verita' piu' semplice e bella. 

Quando ho trovato la calma o la forza nel tempo libero di sedermi e rappresentare quello che avevo scoperto e quello che ormai avevo assodato, scrivevo. Molte volte anche se mi sarebbe piaciuto scrivere una bella idea, o correggere il testo ho rimandato perche' non consideravo l'idea abbastanza matura. L'idea era solo una bella idea, ma era piu' vanita' che verita'. 

Cio' che e' scritto e' condiviso nell'augurio che o possa dire a voce aperta cio' che gia' nel tuo cuore credi o che possa essere spunto di riflessione per arrivare a tue conclusioni piu' profonde. Nello specifico, che tu possa 
\begin{enumerate}
    \item intuire che la felicita', quella vera, che conoscevamo quando eravamo bambini, puo' esistere, totalmente e globalmente, anche in questo mondo complesso, anche nelle situazioni difficili;
    \item che e' infinitamente grande e non vi e' cosa piu' grande di spendersi, rischiare ed amare fino alla fine, per dare la felicita' a se stessi ed agli altri;
    \item che si possa ritrovare il concetto di Dio, quello dell'amore e della fede vera, calato nella cultura moderna scientifica, senza alcuna contraddizione con la visione scientifica del mondo. Sia la scienza, sia la fede sono strumento per l'uomo che ama;
    \item che si possa capire che il mondo cambia nel momento in cui cambiamo noi stessi e non in altro modo, e che non ci sono poteri piu' forti, come il capitalismo, che possano bloccare l'altruismo.
\end{enumerate}


\subsection{La farfalla}

La farfalla in copertina e' il disegno di una formula matematica chiamata ``butterfly curve''. Si puo' disegnare con Gnuplot con il seguente codice:
\begin{verbatim}
set polar
set style line 5 lt rgb "#ADD8E6" lw 3
plot exp(sin(t)) - 2*cos(4*t) + \
        (sin(1/24*(2*t-pi)))**5 ls 5
pause -1
\end{verbatim}
\url{https://en.wikipedia.org/wiki/Butterfly\_curve\_(transcendental)}
