\chapter{Sul lavoro altruistico nell'era del capitalismo}
\label{altruismoLavoro}

Il lavoro onesto, per quanto snaturato nel mondo capitalistico, puo' essere vissuto e compiuto per essere in Pace con stessi e con il resto della societa'. 

In un mondo ideale, il lavoro sarebbe piu' leggero, interessante e scorrevole. Con la tecnologia, occorre molto poco sforzo per ottenere risultati sorprendenti. Con le tecnologie attuali e' facile immaginare che in una o due generazioni, tutti potremmo vivere senza morire di fame. Abbiamo infatti a disposizione campi agricoli vastissimi, che resistono a funghi, insetti e malattie. Possiamo trasportare con facilita' tonnellate di materiale da un posto a un altro. E, se volessimo, potremmo fare tutto questo usando energia rinnovabile e non inquinante. Nel mondo attuale, invece, siamo tutti in perenne crisi. Milioni di persone muoiono di fame, e quelle che sopravvivono sono stressate dalla punta dei capelli fino ai piedi per non finire per strada a mendicare. I posti di lavoro sono pochi e siamo in perenne competizione con noi stessi e forse anche con gli altri.

Vale pero' la pena di non arrendersi e mettercela tutta per stare bene. Infatti, oltre a salvaguardare il proprio sostentamento, chi lavora rende un servizio agli altri, perche' non deve chiedere a loro piccoli o grandi sacrifici per aiutare se stesso. Inoltre, chi lavora onestamente e', indipendentemente dalla mansione che svolge, una persona che sta' amando tutta la societa', e il suo lavoro e' importante al pari del lavoro considerato piu' importante.

Anche quando si svolge un lavoro che ha significato per pochi, in realta' si sta collaborando con ogni persona del pianeta al lavoro universale svolto da tutta la societa'. 
In passato, nelle piccole societa' come le tribu', cio' era evidente. Infatti, ognuno era indispensabile a tutti gli altri e, anche se la mansione svolta era umile, come ad esempio badare al gregge mentre il resto della tribu' cacciava,  se non fosse stata svolta, questo avrebbe costituito un danno per tutti.
Oggigiorno, non e' piu' così scontato pensare in questo modo, soprattutto perche' i modelli culturali premiano ruoli lavorativi di prestigio, come ad esempio il ruolo dell'ingegnere, del medico, dell'imprenditore, e mettono in ombra lavori piu' umili, come ad esempio il manovale, il commesso, il cameriere. 
Ma se lo spazzino non raccogliesse la spazzattura dai cassonetti, dopo sette giorni ci sarebbe la peste per tutta la citta'. Ed e' grazie alla costanza e alla dedizione, che la vite di ferro viene fabbricata dalle mani dell'operaio. Se l'operaio e' in interiormente in pace, compie semplicemente e diligentemente il suo lavoro, e questo oltre a dare a lui stesso la soddisfazione di svolgere bene un buon lavoro, lo pone allo stesso livello di qualsiasi altro tipo di lavoratore. 
Infatti, anche se chiunque altro avrebbe potuto imparare in breve la semplice tecnica che l'operaio adopera per il suo lavoro, lui, quando fabbrica le viti, e' li' ad impiegare il suo tempo, piuttosto che a dedicarsi a svaghi e passioni, e' li' concentrato e a sopportare la fatica, e tutto questo per quelle viti. 
Ed anche se in se' non serve a molto una vite ad una persona qualsiasi, serve a tutti, in quanto, sara' usato per fabbricare un prodotto che noi compreremo. 
Noi, allora, con le nostre abitudini consumiste dove prediligiamo prodotti industriali a prodotti artigianali per il loro minor prezzo e migliore tecnologia, stiamo in realta' chiedendo a gran voce che esista quell'operaio che lavori giornalmente nella fabbrica.
Quindi, il lavoro dell'operaio non e' un mero atto meccanico sostituibile con qualsiasi altro atto meccanico equivalente. E' come quando un genitore aspetta la figlia all'uscita da scuola. Qualunque altro guidatore avrebbe potuto prenderla ed accompagnarla a casa, ma quel genitore, per sua figlia, era li'. 
L'operaio, se e' interiormente in pace con se stesso e con gli altri, era li' per fabbricare quei piccoli artefatti, e dargli la forma e consistenza che la societa' vuole. E noi, se siamo interiormente in pace, riconosciamo l'importanza dell'operaio e del suo lavoro al pari di come la figlia ama essere accompagnata a casa dal proprio genitore e non da una persona qualsiasi.


Un'azienda e' una macchina che ha come unico scopo quello di mantere e aumentare i suoi profitti. Per quanto con il marketing essa si mostri al resto della societa' come disponibile ed impegnata verso i bisogni e gli interessi dell'uomo, e', in verita', di poco valore per il resto del globo\footnote{Anche se produce qualcosa di utile, ci sono decine di altre aziende che producono la stessa cosa. Inoltre, quelle che producono in maniera pulita, nel rispetto dell'ambiente, dell'uomo e dei lavoratori, hanno molto meno potere nel mercato. Quindi, le normali aziende non sono cosi'.}.

Un lavoratore e' di poco valore per il resto del mondo?
No! Fintantoche' il lavoro e' onesto, ovvero non lede ne' il suo benessere ne' il benessere delle altre persone.\\
Infatti, l'essere in grado di non dover chiedere agli altri per pensare a se' vuol dire avere la potenzialita' di essere felici senza sentire il bisogno di obbligare gli altri in alcun modo. Gli altri sono liberi di regalare o meno averi, di essere simpatici o freddi, di essere educati o rozzi, e quindi di vivere e scegliere la propria vita.\\
Il ``non chiedere'' potrebbe non sembrare un grande atto di altruismo. Tuttavia, bisogna vederlo in rapporto alla grandezza della popolazione: se tutti cominciassero a pretendere un favore anche piccolo in una popolazione grande, il risultato sarebbe molto fastidioso, difficile da sopportare o da ostacolo alla vita.

Se 10 persone al giorno per strada chiedessero 1 euro, a fine mese avremmo 300 euro in meno. Se tanti chiedessero un piccolo favore ogni giorno, come badare ai loro figli, fare un lavoro per aggiustare la loro casa, riparare il loro computer, prestare a loro la macchina, ospitarli in casa, allora non avremmo tempo e risorse per noi stessi e per chi amiamo.

Quindi, chi lavora onestamente puo' ``non chiedere''. Se fa cosi', il suo lavoro e' altruistico e puo' sentirsi una parte preziosa della societa', in quanto le fa' onore.

\subsection{Sui lavori utili}
Un lavoro e' utile quando altri hanno bisogno che venga svolto, e tanto piu' hanno realmente bisogno tanto piu' viene considerato utile. Ci sono poi lavori utili, come chi raccoglie alle 5 di mattina la spazzatura dai cassonetti, che sono utili ma che non vengono considerati tali, di questo non parleremo nel successivo paragrafo.

Tanto piu' un lavoro e' utile, tanto piu' richiede responsabilita', e in verita', sacrificio, da parte di chi lo compie. Ci si puo' sacrificare senza sforzo, senza indebitarsi o indebitare gli altri, anche solo di un grazie, solamente se si ama se stessi e gli altri, se si e' in Pace con se stessi e con il resto della societa'.

Chi vuole fare un lavoro ``importante'', puo' capire se veramente vuole, se apprezza essere disposto a fare un lavoro onesto anche se non utile, che come spiegato prima e' gia' perfetto, e inoltre, e' molto piu' \emph{leggero}\footnote{e' tanto piu' leggero tanto piu' la vita altrui non dipende dal risultato del proprio lavoro. Un gioielliere, non fa' un lavoro critico. Se sbaglia, al piu' il gioiello vera' brutto, e al piu' dovra' venderlo a minor prezzo. Un team di ingegneri che sbaglia un calcolo per la progettazione di un aereo, rischia di farlo precipitare con il pilota e altre persone a bordo. Vedi precipitazioni degli aerei Boeing 737 MAX \url{https://en.wikipedia.org/wiki/Boeing\_737\_MAX\_groundings}}. 
E se, considerando il rischio che comporta il lavoro piu' importante, e consapevole delle possibili conseguenze negative in caso di fallimento, capisce che puo' farlo e si sente di sacrificare una parte di se per questo. Cosi' sara' disposto ad accettare, se la societa' lo ritiene adeguato, un lavoro piu' importante.

In verita', queste sembrano parole idealiste. Nel mondo, piu' il lavoro e' considerato non utile, piu' si guadagna meno, con la scusa che chi fa' i lavori piu' difficili e' piu' importante o bravo.

Tuttavia, come insegnano Gesu', Gandhi e altri, non bisogna aspettare che il mondo cambi: prima cambiamo noi, senza pretendere alcun cambiamento negli altri, essendo solo felici ed aperti ad un loro cambiamento, e se non ci riusciamo, pregando Dio di esserlo.

Questa e' la vera rivoluzione, e costa tantissimo: la propria vita. Perche', nessuno elogiera' chi si mettera' in questo cammino, la societa' non lo riconoscera', ne' lo eleggera' Papa. Colei che si mettera' in questo cammino, avra' meno privilegi, meno diritti. Saranno solo forze spese. 

L'unica ricompensa sara' quella di Dio, sara' quella di poter essere veramente in Pace con se stessi e con la societa', non escludendo nessuna persona, di qualsivoglia ceto o condizione.

\subsection{--Paragrafo in corso d'opera--}
Quando un lavoratore rispetta e crede in tutte le leggi della societa', lavora onestamente nel rispetto di tutti, egli e' al pari di ogni altro lavoratore. La ragione e' che, inanzitutto, per vivere egli non cerca via traverse, che danneggerebbero in piccola o grande misura l'altro. Inoltre, il servizio di codesto lavoratore, essendo richiesto dal datore di lavoro e dai clienti, e', come gia' spiegato prima, utile ad una parte grande o piccola della societa', e quindi, ad almeno una persona.
  Il premio per pensare in questo modo, non e' da cercare nel ritorno economico. Il mercato e' ingiusto, perche' a parita' di utilita' e bonta' di un servizio, alcuni lavori sono pagati di piu', altri di meno. Mentre una grossa azienda rivende il suo servizio a dieci o cento volte tanto il suo effettivo costo di produzione, un lavoratore autonomo non puo' comportarsi allo stesso modo: i costi che deve sostenere sono molto piu' alti e la sua clientela infinitamente piu' ristretta. Eppure, l'idraulico che ripara il tubo dell'acqua che perde, svolge un lavoro tanto utile quanto quello dell'azienda che vende un'automobile.
Tuttavia, accettando il mondo cosi' come e' e tentando di cambiarlo solo senza la forza, in maniera non violenta, lentamente nel tempo, consente di sacrificarsi per lavorare, alle condizioni attuali del mercato, e vedere il proprio sacrificio come un atto di amore verso la societa'. Infatti, ogni lavoratore e' al pari di ogni lavoratore, e porta avanti la societa' nel suo piccolo.

Un lavoratore, così come un'azienda, guadagnando nel tempo dispone di un capitale. Tuttavia, egli e' piu' libero ed ha piu' potere di spendere il suo capitale per il bene collettivo. Quando lo utilizza per soddisfare le proprie necessita' e desideri premia i lavoratori che gli offrono dei servizi, e a catena i lavoratori che offrono i servizi a questi ultimi.

Tuttavia, il lavoratore ha piu' liberta' e potere,perche' puo' scegliere di prediligere dei prodotti e dei servizi al posto di altri, per rispettare dei principi, come ad esempio, comprare presso una bottega locale piuttosto che presso un supermercato, comprare frutta e verdura della propria regione, usare prodotti biologici e che rispettano gli animali, fino a prodotti fair trade e simili.


.........


Poi, ci sono le persone che effettivamente fanno i lavori, si sacrificano e rischiano, e quelle che invece, ci mettono solo una parte di loro stessi, con molta cautela e sicurezza che anche se le cose andranno male, tuttavia, non gli succedera' nulla di che' (nota \footnote{Un caso estremo e' ``Amianto, una storia operaia. Alberto Prunetti''}). Ma anche di questo non parleremo.
