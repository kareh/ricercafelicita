\chapter{L'importanza del lavoro umile nell'era del capitalismo}

\label{altruismoLavoro}

Il lavoro onesto, per quanto snaturato nel mondo capitalistico, puo' essere vissuto e compiuto per stare in Pace con stessi e con il resto della societa'. 

In un mondo ideale, il lavoro sarebbe piu' leggero, interessante e scorrevole. Con la tecnologia, occorre molto poco sforzo per ottenere risultati sorprendenti. Con le tecnologie attuali e' facile immaginare che in una o due generazioni, tutti potremmo vivere senza morire di fame. Abbiamo infatti a disposizione campi agricoli vastissimi, che resistono a funghi, insetti e malattie. Possiamo trasportare con facilita' tonnellate di materiale da un posto a un altro. E, se volessimo, potremmo fare tutto questo usando energia rinnovabile e non inquinante. Nel mondo attuale, invece, siamo tutti in perenne crisi. Milioni di persone muoiono di fame, e quelle che sopravvivono sono stressate dalla punta dei capelli fino ai piedi per non finire per strada a mendicare. I posti di lavoro sono pochi e siamo in perenne competizione con noi stessi e con gli altri.

Vale pero' la pena di non arrendersi e mettercela tutta per stare bene. Infatti, oltre a salvaguardare il proprio sostentamento, chi lavora rende anche un servizio agli altri, perche' non deve chiedere a loro piccoli o grandi sacrifici per aiutare se stesso. 
L'essere autonomi, vuol dire avere la potenzialita' di stare in salute, vivere momenti lieti con amici, e seguire le proprie passioni senza sentire il bisogno di obbligare gli altri in alcun modo. Gli altri rimangono liberi di regalare o meno averi, di essere simpatici o freddi, di essere educati o rozzi, e quindi di vivere e scegliere la propria vita.
Quindi, voler bene a se stessi senza pretendere attenzioni dagli altri e', oltre che utile, altruistico. 

Inoltre, chi lavora onestamente e', indipendentemente dalla mansione che svolge, una persona che sta' amando tutta la societa'. Anche quando svolge un lavoro che ha significato per pochi, in realta' sta collaborando con ogni persona del pianeta al lavoro universale svolto da tutta la societa'. 
In passato, nelle piccole societa' come le tribu', cio' era evidente. Infatti, ognuno era indispensabile a tutti gli altri e, anche se la mansione svolta era umile, come ad esempio badare al gregge mentre il resto della tribu' cacciava,  se non fosse stata svolta, questo avrebbe costituito un danno per tutti.
Oggigiorno, non e' piu' così scontato pensare in questo modo, soprattutto perche' i modelli culturali premiano ruoli lavorativi di prestigio, come ad esempio il ruolo dell'ingegnere, del medico, dell'imprenditore, e mettono in ombra lavori piu' umili, come ad esempio il manovale, il commesso, il cameriere. 
Ma se lo spazzino non raccogliesse la spazzattura dai cassonetti, dopo sette giorni ci sarebbe la peste per tutta la citta'. Ed e' grazie alla costanza e alla dedizione, che la vite di ferro viene fabbricata dalle mani dell'operaio. Se l'operaio e' in interiormente in pace, compie semplicemente e diligentemente il suo lavoro, e questo oltre a dare a lui stesso la soddisfazione di svolgere bene un buon lavoro, lo pone allo stesso livello di qualsiasi altro tipo di lavoratore. 
Infatti, anche se chiunque altro avrebbe potuto imparare la tecnica che l'operaio adopera per il suo lavoro, lui, quando fabbrica le viti, e' li' ad impiegare il suo tempo, piuttosto che a dedicarsi a svaghi e passioni, e' li' concentrato, a sopportare la fatica, e tutto questo per quelle viti. 
Ed anche se in se' non serve a molto una vite ad una persona qualsiasi, serve a tutti, in quanto, sara' usata per fabbricare un prodotto che noi compreremo. 
Noi, allora, con le nostre abitudini consumiste dove prediligiamo prodotti industriali a prodotti artigianali per il loro minor prezzo e migliore tecnologia, stiamo in realta' chiedendo a gran voce che esista quell'operaio che lavori giornalmente nella fabbrica.
Quindi, il lavoro dell'operaio non e' un mero atto meccanico sostituibile con qualsiasi altro atto meccanico equivalente. E' come quando un genitore aspetta la figlia all'uscita da scuola. Qualunque altro autista fidato potrebbe prenderla ed accompagnarla a casa, ma quel genitore, per sua figlia, e' li'. 
L'operaio, se e' interiormente in pace con se stesso e con gli altri, e' li' per fabbricare quei piccoli artefatti, e dargli la forma e consistenza che la societa' vuole. E noi, se siamo interiormente in pace, riconosciamo l'importanza dell'operaio e del suo lavoro al pari di come la figlia ama essere accompagnata a casa dal proprio genitore e non da una persona qualsiasi.

Quando un lavoratore rispetta e crede in tutte le leggi della societa', lavora onestamente nel rispetto di tutti, egli e' al pari di ogni altro lavoratore. La ragione e' che, inanzitutto, per vivere egli non cerca via traverse, che danneggerebbero in piccola o grande misura l'altro. Inoltre, il servizio del lavoratore, essendo richiesto dal datore di lavoro e dai clienti, e', come gia' spiegato prima, utile ad una parte grande o piccola della societa', e quindi, ad almeno una persona.
  Il premio per pensare in questo modo, non e' da cercare nel ritorno economico. Il mercato e' ingiusto, perche' a parita' di utilita' e bonta' di un servizio, alcuni lavori sono pagati di piu', altri di meno. Mentre una grossa azienda rivende il suo servizio a dieci o cento volte tanto il suo effettivo costo di produzione, un lavoratore autonomo non puo' comportarsi allo stesso modo: i costi che deve sostenere sono molto piu' alti e la sua clientela infinitamente piu' ristretta. Eppure, l'idraulico che ripara il tubo dell'acqua che perde, svolge un lavoro tanto utile quanto quello dell'imprenditore di una fabbrica di tubi per l'impianto idrico di casa.
Tuttavia, accettare il mondo cosi' come e', e cambiarlo senza la forza, in maniera non violenta, lentamente nel tempo, consente di sacrificarsi per lavorare, alle condizioni attuali del mercato, e vedere il proprio sacrificio come un atto di amore verso la societa'. Il lavoro attuale presenta vari inconvenienti e difficolta', come paghe non proporzionate agli sforzi reali, incomprensioni da parte dei colleghi, poca lungimiranza e sensibilita' dei propri superiori e routine lavorativa non rispettosa dei propri ritmi biologici. Quando il lavoratore accetta il mondo attuale come atto d'amore, diventa in grado di ben sopportare questi inconvenienti, perche' vede non piu' solo i propri vantaggi e svantaggi, i propri dolori e piaceri, ma vede la societa' nella globalita' che avanza, e trova la forza e la gioia interiore di amarla. 
E' cosi' che ogni lavoratore e' al pari di ogni lavoratore, ed ama l'intera societa' insieme a tutti gli altri lavoratori.

\subsection{Sui lavori utili}
Un lavoro e' considerato utile quando altri sentono il bisogno che venga svolto. Tanto piu' un lavoro e' utile, tanto piu' richiede responsabilita', e in verita', sacrificio, da parte di chi lo compie. 

1. Ci si puo' sacrificare senza sforzo. Per far questo bisogna amare se stessi e gli altri, essere in Pace con se stessi e con il resto della societa'. In tal modo, ogni sacrificio non pesera' e non ci si pentira' in seguito di cio' che comportera'.

2. Chi vuole fare un lavoro molto utile, puo' capire se veramente vuole solo se apprezza anche un lavoro onesto, indipendentemente dal fatto che sia utile o meno. Un lavoro onesto ma ``inutile'', come spiegato prima nel capitolo \ref{altruismoLavoro}, e' ottimo e perfetto, ed inoltre, e' molto piu' \emph{leggero}\footnote{e' tanto piu' leggero tanto piu' la vita altrui non dipende dal risultato del proprio lavoro. Un gioielliere, non fa' un lavoro critico. Se sbaglia, al piu' il gioiello vera' brutto, e al piu' dovra' venderlo a minor prezzo. Un team di ingegneri che sbaglia un calcolo per la progettazione di un aereo, rischia di farlo precipitare con il pilota e altre persone a bordo. Vedi precipitazioni degli aerei Boeing 737 MAX \url{https://en.wikipedia.org/wiki/Boeing\_737\_MAX\_groundings}}. 
E se, consapevole del rischio che comporta il lavoro piu' importante e delle possibili conseguenze negative in caso di fallimento, capisce che puo' farlo e si sente di sacrificare una parte di se per questo, allora sara' disposto ad accettare un lavoro piu' importante che la societa' gli offre.

In verita', queste sembrano parole idealiste. Nel mondo, piu' il lavoro e' considerato non utile, piu' si guadagna meno, con la scusa che chi fa' i lavori piu' difficili e' piu' importante o bravo.

Tuttavia, come insegnano Gesu', Gandhi e altri, non bisogna aspettare che il mondo cambi: prima cambiamo noi, senza pretendere alcun cambiamento negli altri, essendo solo felici ed aperti ad un loro cambiamento, e se non ci riusciamo, pregando Dio di esserlo.

Questa e' la vera rivoluzione, e costa tantissimo: la propria vita. Perche', nessuno elogiera' chi si mettera' in questo cammino, la societa' non lo riconoscera', ne' lo eleggera' Papa. Colui che si mettera' in questo cammino, avra' meno privilegi, meno diritti. Saranno solo forze spese. 

L'unica ricompensa sara' quella di Dio, sara' quella di poter essere veramente in Pace con se stessi e con la societa', non escludendo nessuna persona, di qualsivoglia ceto o condizione.

\subsection{Sui consumatori}

Un consumatore, cosi' come un'azienda, dispone di un capitale. Tuttavia, egli e' piu' libero ed ha piu' potere di spendere il suo capitale per il bene collettivo. Quando lo utilizza per soddisfare le proprie necessita' e desideri premia i lavoratori che gli offrono dei servizi, e a catena i lavoratori che offrono i servizi a questi ultimi. In aggiunta, puo' scegliere di prediligere dei prodotti e dei servizi al posto di altri, per rispettare dei principi, come ad esempio, comprare presso una bottega locale piuttosto che presso un supermercato, usare prodotti biologici, prediligere prodotti e servizi che garantiscono il rispetto dei diritti dei lavoratori. Pagando il costo aggiuntivo che richiede il rispetto di buoni principi, il mondo cambia veramente, anche se di un infinitesimo. Se non viene pagato questo costo aggiuntivo, verra' comunque il tempo in cui altri lo pagheranno al posto nostro contro la loro volonta', come i paesi del terzo mondo, o in cui noi stessi lo pagheremo, come negli episodi dei disastri ambientali.
E' da aggiungere che agire solo singolarmente non puo' cambiare la situazione. Per questo e' poi opportuno unirsi ad altri che hanno gli stessi fini ed intenzioni. Solo cosi' potra' poi nascere una vera forza politica, sana, pacifica, che cambiera' il mondo.

\subsection{Esempio ideale}
Per dare concretezza a quanto fin'ora detto, facciamo l'esempio del lavoro come strumento per la ricerca della vera vita. Oltre a realizzare una mera logica di sussistenza, andando a lavorare si ha la possibilita': di stare con altre persone, che come noi cercano di sopravvivere impegnandosi onestamente e che sono al di fuori del nostro cerchio di amicizie; di rinforzare la fiducia in noi stessi tramite i piccoli quotidiani successi; di trasmettere le proprie tecniche e i propri trucchi ai colleghi; di sopportare chi ci e' antipatico, per rendersi conto che, in realta', il suo modo di vivere ci interessa sotto certi aspetti, e che ne' continuando ad essere come noi siamo, ne' essendo esattamente come lui e', arriveremo alla Meta; capito cio', cambiando nel tempo, ritrovare la pace di stare con lui; di impegnarsi su problemi che solo noi sappiamo o dobbiamo risolvere; di ricorrere a tutta la propria esperienza, resistenza e astuzia; di vivere la paura del fallimento, e poi ritrovare fede, speranza, e soffrire tanto impegno, e pazienza e ancora impegno, e poi arrivare al momento fatidico della messa in atto; nel caso di successo, condividere la gioia con i colleghi, ornarsi dei complimenti dei superiori, nel caso di fallimento, riconoscere la bonta' del lavoro fatto, trarre esperienza dagli errori commessi, e ritrovare il coraggio dai consigli di chi ci vuole bene, per poi rimettersi in gioco nel lavoro. Infine, andando a lavorare si ha la possibilita' di tornare a casa, dimenticandosi della veste del dovere indossata tutta la giornata, e apprezzare la ricerca delle semplici gioie della vita, come un piatto caldo e nutriente, mangiato in compagnia di chi vogliamo bene.

Per tutto questo, chi ricerca la vera vita lavorando, non si tira indietro dai sacrifici che il lavoro comporta e si alza anche la ventesima mattina, dopo averlo fatto gia' 19 giorni nello stesso mese, e va a lavorare, nonostante, quel giorno, farebbe piu' bene per il suo fegato fare una semplice passeggiata per comprare la frutta o stringere la mano del suo amore e guardare il cielo. Chi ricerca la vera vita, non si vede schiavo ne' accetta il mondo come perfetto. Comprende il suo cuore e quello degli altri e capisce che la societa' e' il risultato degli egoismi e delle eccellenze di ognuno. Sa' quindi che non puo' assentarsi perfino la ventesima mattina, perche'
\begin{enumerate}
    \item il suo capo o manager non ha cognizione delle esigenze personali di ogni suo dipendente, e averla non e' facile e sarebbe molto impegnativo per chiunque. Inoltre, e' gia' impegnato ad affrontare le problematiche di produzione aziendali, e quindi, anche desiderandolo, non potrebbe dedicarsi serenamente ad ognuno.
    \item Perche', il vero lavoratore sa che non ha cognizione completa dei piani dei suoi capi e di quello che stanno facendo i colleghi, delle difficolta' che sta' affrontando l'azienda e di quello che vuole fare per il prossimo futuro.
   Forse ha una visione di cio', ma non precisa, quindi non puo' concludere correttamente, autonomamente, se e' giusto fare qualcosa a riguardo del suo lavoro o no. Nello specifico, non puo' decidere se assentarsi o ritardare a suo arbitrio. Inoltre, anche se la sua assenza e' ininfluente, non puo' ragionare individualmente, perche' se poi ogni altro lavoratore facesse cosi', piu' spesso mancherebbe la persona che proprio quel giorno serve, e cosi' la produttivita' calerebbe.
    \item Perche' le aziende clienti dell'azienda del lavoratore, ad ogni minimo calo di qualita' od aumento del costo passerebbero subito ad altre aziende, perche', altrimenti, non riuscerebbero a soddisfare le richieste dei loro stessi clienti. Di conseguenza, l'azienda del lavoratore diventerebbe piu' debole nel mercato e le aziende competitrici ne approfitterebbero per superarla e godere dei benefici che ha goduto quando era avanti a loro, e che, seguendo una logica capitalista, non ha mai condiviso con loro.

\end{enumerate}
Il vero lavoratore, quindi, pur facendo parte del meccanismo capitalista che non lascia un pieno respiro alla sua vita,  ne comprende le ragioni umane, e considerando gli uomini di tutte le aziende, come suoi fratelli, non si ribella a loro. Sopportando la condizione presente della societa', gioisce delle piccole e semplici gioie della vita, e nel suo piccolo, si impegna per il benessere interiore dei suoi amici e delle persone che incontra,  sognando che aumenti pure per ogni membro della collettivita' cosi' che lavorare non sara' piu' cosi' pesante per l'uomo.

Il lavoro non e' il fine, e' un mezzo per mettere a fuoco la ricerca della vera vita. Esistono molti altri modi per ricercarla, come l'arte, lo sport, la politica, la religione, la scienza, la famiglia. Tutti, seppure diversi, hanno gli stessi ingredienti: 1. apprezzare cio' che si e' e gli altri sono, 2. impegnarsi per dare a tutti, 3. non dare immotivata priorita' a nessuno, compresi se stessi, al costo di rinunciare a molti privilegi.


