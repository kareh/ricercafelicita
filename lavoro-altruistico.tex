\chapter{Sul lavoro altruistico nell'era del capitalismo}
\label{altruismoLavoro}

Il lavoro onesto, per quanto snaturato nel mondo capitalistico, puo' essere vissuto e compiuto anche in maniera altruistica. Chi lavora onestamente, puo' stare in Pace con la societa'. Vediamo perche'. \\

Il lavoro e' una risorsa che permette di non dover chiedere agli altri. Chi lavora non deve chiedere agli altri per mangiare o per curarsi, o per far mangiare e curare chi vuole.  Cosi' gli altri rimangono liberi di dedicarsi a se stessi e a chi vogliono. 

In un mondo ideale, il lavoro sarebbe molto piu' leggero, interessante e scorrevole per tutti. Infatti, la tecnologia insegna che in queste Terre occorre molto poco sforzo per ottenere risultati sorprendenti. Ad esempio, per muovere molti massi con una pala meccanica.

Con le tecnologie attuali e' facile immaginare che in una o due generazioni, magari con un po' meno di sovrappolazione, tutti potremmo vivere senza morire di fame.
Abbiamo infatti a disposizione campi agricoli vastissimi, che resistono a funghi, insetti e malattie. Possiamo trasportare con facilita' tonnellate di materiale da un posto a un altro. E, se volessimo, potremmo fare tutto questo usando energia rinnovabile e non inquinante.

Nel mondo attuale, invece, siamo tutti in perenne crisi. Milioni di persone muoiono di fame, e quelle che non ci muoiono sono stressate dalla punta dei capelli fino ai piedi per non finire per strada a mendicare. I posti di lavoro sono pochi e siamo in perenne competizione con noi stessi e forse anche con gli altri.

Vale pero' la pena di non arrendersi e mettercela tutta per stare bene. Trovato un lavoro, dopo un po', sorge la domanda: a che serve?

Un'azienda e' una macchina che ha come unico scopo quello di mantere e aumentare i suoi profitti. Per quanto con il marketing essa si mostri al resto della societa' come disponibile ed impegnata verso i bisogni e gli interessi dell'uomo, e', in verita', di poco valore per il resto del globo\footnote{Anche se produce qualcosa di utile, ci sono decine di altre aziende che producono la stessa cosa. Inoltre, quelle che producono in maniera pulita, nel rispetto dell'ambiente, dell'uomo e dei lavoratori, hanno molto meno potere nel mercato. Quindi, le normali aziende non sono cosi'.}.

Un lavoratore e' di poco valore per il resto del mondo?
No! Fintantoche' il lavoro e' onesto, ovvero non lede ne' il suo benessere ne' il benessere delle altre persone.\\
Infatti, l'essere in grado di non dover chiedere agli altri per pensare a se' vuol dire avere la potenzialita' di essere felici senza sentire il bisogno di obbligare gli altri in alcun modo. Gli altri sono liberi di regalare o meno averi, di essere simpatici o freddi, di essere educati o rozzi, e quindi di vivere e scegliere la propria vita.\\
Il ``non chiedere'' potrebbe non sembrare un grande atto di altruismo. Tuttavia, bisogna vederlo in rapporto alla grandezza della popolazione: se tutti cominciassero a pretendere un favore anche piccolo in una popolazione grande, il risultato sarebbe molto fastidioso, difficile da sopportare o da ostacolo alla vita.

Questo lo possiamo immaginare anche pensando a quanto sarebbe fastidioso e imbarazzante se dovessimo essere sempre gentili ed educati con una qualsiasi persona estranea che per strada ci incontra e decide di salutarci. Magari in un paese con pochi abitanti potrebbe essere anche piacevole, ma in una citta', ogni giorno, in un momento qualsiasi, diventerebbe innaturale e controproducente.
Oppure anche, se 10 persone al giorno per strada chiedessero 1 euro, a fine mese avremmo 300 euro in meno.\\

Quindi, chi lavora onestamente puo' ``non chiedere''. Se fa cosi', il suo lavoro e' altruistico e puo' sentirsi una parte preziosa della societa', in quanto le fa' onore.

\subsection{Sui lavori utili}
Un lavoro e' utile quando altri hanno bisogno che venga svolto, e tanto piu' hanno realmente bisogno tanto piu' viene considerato utile. Ci sono poi lavori utili, come chi raccoglie alle 5 di mattina la spazzatura dai cassonetti, che sono utili ma che non vengono considerati tali, ma di questo non parleremo. 

Tanto piu' un lavoro e' utile, tanto piu' richiede responsabilita', e in verita', sacrificio, da parte di chi lo compie.

Poi, ci sono le persone che effettivamente fanno i lavori, si sacrificano e rischiano, e quelle che invece, ci mettono solo una parte di loro stessi, con molta cautela e sicurezza che anche se le cose andranno male, tuttavia, non gli succedera' nulla di che' (nota \footnote{Un caso estremo e' ``Amianto, una storia operaia. Alberto Prunetti''}). Ma anche di questo non parleremo.

Il punto da affrontare e' che: tanto piu' un lavoro e' utile, tanto piu' richiede responsabilita' e sacrificio, vero. Ci si puo' sacrificare senza sforzo, senza indebitarsi o indebitare gli altri, anche solo di un grazie, solamente se si ama se stessi e gli altri, se si e' in Pace con se stessi e con il resto della societa'.

Per fare un lavoro ``importante'', e' quindi necessario prima essere disposti a fare un lavoro onesto anche se non utile, che come spiegato prima e' gia' perfetto, e inoltre, e' molto piu' \emph{leggero}\footnote{e' tanto piu' leggero tanto piu' la vita altrui non dipende dal risultato del proprio lavoro. Un gioielliere, non fa' un lavoro essenziale. Se sbaglia, al piu' il gioiello vera' brutto, e al piu' dovra' venderlo a minor prezzo. Un team di ingegneri che sbaglia un calcolo per la progettazione di un aereo, rischia di farlo precipitare con il pilota e altre persone a bordo. Vedi precipitazioni degli aerei Boeing 737 MAX \url{https://en.wikipedia.org/wiki/Boeing\_737\_MAX\_groundings}}. 
Poi, se verra', il lavoro si puo' trasformare in qualche ruolo utile alla propria azienda, e poi al resto del mondo.

In verita', queste sembrano parole idealiste. Nel mondo, piu' il lavoro e' considerato non utile, piu' si guadagna meno, con la scusa che chi fa' i lavori piu' difficili e' piu' importante o bravo.

Tuttavia, come insegnano Gesu', Gandhi e altri, non bisogna aspettare che il mondo cambi: prima cambiamo noi, senza pretendere alcun cambiamento negli altri, essendo solo felici ed aperti ad un loro cambiamento, e se non ci riusciamo, pregando Dio di esserlo.

Questa e' la vera rivoluzione, e costa tantissimo: la propria vita. Perche', nessuno elogiera' chi si mettera' in questo cammino, la societa' non lo riconoscera', ne' lo eleggera' Papa. Colei che si mettera' in questo cammino, avra' meno privilegi, meno diritti. Saranno solo forze spese. 

L'unica ricompensa sara' quella di Dio, sara' quella di poter essere veramente in Pace con se stessi e con la societa', non escludendo nessuna persona, di qualsivoglia ceto o condizione.


