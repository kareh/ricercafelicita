\chapter{L'importanza del lavoro umile nell'era del capitalismo}
\label{altruismoLavoro}

Il lavoro onesto, per quanto snaturato nel mondo capitalistico, puo' essere vissuto e compiuto per essere in Pace con stessi e con il resto della societa'. 

In un mondo ideale, il lavoro sarebbe piu' leggero, interessante e scorrevole. Con la tecnologia, occorre molto poco sforzo per ottenere risultati sorprendenti. Con le tecnologie attuali e' facile immaginare che in una o due generazioni, tutti potremmo vivere senza morire di fame. Abbiamo infatti a disposizione campi agricoli vastissimi, che resistono a funghi, insetti e malattie. Possiamo trasportare con facilita' tonnellate di materiale da un posto a un altro. E, se volessimo, potremmo fare tutto questo usando energia rinnovabile e non inquinante. Nel mondo attuale, invece, siamo tutti in perenne crisi. Milioni di persone muoiono di fame, e quelle che sopravvivono sono stressate dalla punta dei capelli fino ai piedi per non finire per strada a mendicare. I posti di lavoro sono pochi e siamo in perenne competizione con noi stessi e forse anche con gli altri.

Vale pero' la pena di non arrendersi e mettercela tutta per stare bene. Infatti, oltre a salvaguardare il proprio sostentamento, chi lavora rende anche un servizio agli altri, perche' non deve chiedere a loro piccoli o grandi sacrifici per aiutare se stesso. 
L'essere autonomi, vuol dire avere la potenzialita' di stare in salute, vivere momenti lieti con amici, e seguire le proprie passioni senza sentire il bisogno di obbligare gli altri in alcun modo. Gli altri rimangono liberi di regalare o meno averi, di essere simpatici o freddi, di essere educati o rozzi, e quindi di vivere e scegliere la propria vita.
Quindi, voler bene a se stessi senza pretendere attenzioni dagli altri e', oltre che utile, altruistico. 

Inoltre, chi lavora onestamente e', indipendentemente dalla mansione che svolge, una persona che sta' amando tutta la societa'. Anche quando svolge un lavoro che ha significato per pochi, in realta' sta collaborando con ogni persona del pianeta al lavoro universale svolto da tutta la societa'. 
In passato, nelle piccole societa' come le tribu', cio' era evidente. Infatti, ognuno era indispensabile a tutti gli altri e, anche se la mansione svolta era umile, come ad esempio badare al gregge mentre il resto della tribu' cacciava,  se non fosse stata svolta, questo avrebbe costituito un danno per tutti.
Oggigiorno, non e' piu' così scontato pensare in questo modo, soprattutto perche' i modelli culturali premiano ruoli lavorativi di prestigio, come ad esempio il ruolo dell'ingegnere, del medico, dell'imprenditore, e mettono in ombra lavori piu' umili, come ad esempio il manovale, il commesso, il cameriere. 
Ma se lo spazzino non raccogliesse la spazzattura dai cassonetti, dopo sette giorni ci sarebbe la peste per tutta la citta'. Ed e' grazie alla costanza e alla dedizione, che la vite di ferro viene fabbricata dalle mani dell'operaio. Se l'operaio e' in interiormente in pace, compie semplicemente e diligentemente il suo lavoro, e questo oltre a dare a lui stesso la soddisfazione di svolgere bene un buon lavoro, lo pone allo stesso livello di qualsiasi altro tipo di lavoratore. 
Infatti, anche se chiunque altro avrebbe potuto imparare in breve la semplice tecnica che l'operaio adopera per il suo lavoro, lui, quando fabbrica le viti, e' li' ad impiegare il suo tempo, piuttosto che a dedicarsi a svaghi e passioni, e' li' concentrato e a sopportare la fatica, e tutto questo per quelle viti. 
Ed anche se in se' non serve a molto una vite ad una persona qualsiasi, serve a tutti, in quanto, sara' usata per fabbricare un prodotto che noi compreremo. 
Noi, allora, con le nostre abitudini consumiste dove prediligiamo prodotti industriali a prodotti artigianali per il loro minor prezzo e migliore tecnologia, stiamo in realta' chiedendo a gran voce che esista quell'operaio che lavori giornalmente nella fabbrica.
Quindi, il lavoro dell'operaio non e' un mero atto meccanico sostituibile con qualsiasi altro atto meccanico equivalente. E' come quando un genitore aspetta la figlia all'uscita da scuola. Qualunque altro guidatore avrebbe potuto prenderla ed accompagnarla a casa, ma quel genitore, per sua figlia, era li'. 
L'operaio, se e' interiormente in pace con se stesso e con gli altri, era li' per fabbricare quei piccoli artefatti, e dargli la forma e consistenza che la societa' vuole. E noi, se siamo interiormente in pace, riconosciamo l'importanza dell'operaio e del suo lavoro al pari di come la figlia ama essere accompagnata a casa dal proprio genitore e non da una persona qualsiasi.

Quando un lavoratore rispetta e crede in tutte le leggi della societa', lavora onestamente nel rispetto di tutti, egli e' al pari di ogni altro lavoratore. La ragione e' che, inanzitutto, per vivere egli non cerca via traverse, che danneggerebbero in piccola o grande misura l'altro. Inoltre, il servizio del lavoratore, essendo richiesto dal datore di lavoro e dai clienti, e', come gia' spiegato prima, utile ad una parte grande o piccola della societa', e quindi, ad almeno una persona.
  Il premio per pensare in questo modo, non e' da cercare nel ritorno economico. Il mercato e' ingiusto, perche' a parita' di utilita' e bonta' di un servizio, alcuni lavori sono pagati di piu', altri di meno. Mentre una grossa azienda rivende il suo servizio a dieci o cento volte tanto il suo effettivo costo di produzione, un lavoratore autonomo non puo' comportarsi allo stesso modo: i costi che deve sostenere sono molto piu' alti e la sua clientela infinitamente piu' ristretta. Eppure, l'idraulico che ripara il tubo dell'acqua che perde, svolge un lavoro tanto utile quanto quello dell'azienda che vende un'automobile.
Tuttavia, accettare il mondo cosi' come e' e cambiarlo senza la forza, in maniera non violenta, lentamente nel tempo, consente di sacrificarsi per lavorare, alle condizioni attuali del mercato, e vedere il proprio sacrificio come un atto di amore verso la societa'. Il lavoro attuale presenta vari inconvenienti e difficolta', come paghe non proporzionate agli sforzi reali, incomprensioni da parte dei colleghi, poca lungimiranza e sensibilita' dei propri superiori e routine lavorativa non rispettosa dei propri ritmi biologici. Quando il lavoratore accetta il mondo attuale come atto d'amore, diventa in grado di ben sopportare questi inconvenienti, perche' vede non piu' solo i propri vantaggi e svantaggi, i propri dolori e piaceri, ma vede la societa' nella globalita' che avanza, e trova la forza e la gioia interiore di amarla. 
E' cosi' che ogni lavoratore e' al pari di ogni lavoratore, ed ama l'intera societa' insieme a tutti gli altri lavoratori.

\subsection{Sui lavori utili}
Un lavoro e' utile quando altri hanno bisogno che venga svolto, e tanto piu' hanno realmente bisogno tanto piu' viene considerato utile. Ci sono poi lavori utili, come chi raccoglie alle 5 di mattina la spazzatura dai cassonetti, che sono utili ma che non vengono considerati tali, di questo non parleremo nel successivo paragrafo.

Tanto piu' un lavoro e' utile, tanto piu' richiede responsabilita', e in verita', sacrificio, da parte di chi lo compie. Ci si puo' sacrificare senza sforzo, senza indebitarsi o indebitare gli altri, anche solo di un grazie, solamente se si ama se stessi e gli altri, se si e' in Pace con se stessi e con il resto della societa'.

Chi vuole fare un lavoro ``importante'' puo' capire se veramente vuole solo se apprezza essere disposto a fare un lavoro onesto, indipendentemente dal fatto che sia utile o meno. Un lavoro onesto ma ``inutile'', come spiegato prima e' ottimo e perfetto, ed inoltre, e' molto piu' \emph{leggero}\footnote{e' tanto piu' leggero tanto piu' la vita altrui non dipende dal risultato del proprio lavoro. Un gioielliere, non fa' un lavoro critico. Se sbaglia, al piu' il gioiello vera' brutto, e al piu' dovra' venderlo a minor prezzo. Un team di ingegneri che sbaglia un calcolo per la progettazione di un aereo, rischia di farlo precipitare con il pilota e altre persone a bordo. Vedi precipitazioni degli aerei Boeing 737 MAX \url{https://en.wikipedia.org/wiki/Boeing\_737\_MAX\_groundings}}. 
E se, consapevole del rischio che comporta il lavoro piu' importante e delle possibili conseguenze negative in caso di fallimento, capisce che puo' farlo e si sente di sacrificare una parte di se per questo, allora sara' disposto ad accettare un lavoro piu' importante che la societa' gli offre.

In verita', queste sembrano parole idealiste. Nel mondo, piu' il lavoro e' considerato non utile, piu' si guadagna meno, con la scusa che chi fa' i lavori piu' difficili e' piu' importante o bravo.

Tuttavia, come insegnano Gesu', Gandhi e altri, non bisogna aspettare che il mondo cambi: prima cambiamo noi, senza pretendere alcun cambiamento negli altri, essendo solo felici ed aperti ad un loro cambiamento, e se non ci riusciamo, pregando Dio di esserlo.

Questa e' la vera rivoluzione, e costa tantissimo: la propria vita. Perche', nessuno elogiera' chi si mettera' in questo cammino, la societa' non lo riconoscera', ne' lo eleggera' Papa. Colui che si mettera' in questo cammino, avra' meno privilegi, meno diritti. Saranno solo forze spese. 

L'unica ricompensa sara' quella di Dio, sara' quella di poter essere veramente in Pace con se stessi e con la societa', non escludendo nessuna persona, di qualsivoglia ceto o condizione.

\subsection{Sui consumatori}

Un consumatore, cosi' come un'azienda, dispone di un capitale. Tuttavia, egli e' piu' libero ed ha piu' potere di spendere il suo capitale per il bene collettivo. Quando lo utilizza per soddisfare le proprie necessita' e desideri premia i lavoratori che gli offrono dei servizi, e a catena i lavoratori che offrono i servizi a questi ultimi. In aggiunta, puo' scegliere di prediligere dei prodotti e dei servizi al posto di altri, per rispettare dei principi, come ad esempio, comprare presso una bottega locale piuttosto che presso un supermercato, comprare frutta e verdura della propria regione, usare prodotti biologici e che rispettano gli animali, fino a prodotti fair trade e simili. Pagando il costo aggiuntivo che richiede il rispetto di buoni principi, il mondo cambia veramente, anche se di un infinitesimo. Se non viene pagato questo costo aggiuntivo, verra' comunque il tempo in cui altri lo pagheranno al posto nostro contro la loro volonta', come i paesi del terzo mondo, o in cui noi stessi lo pagheremo, come negli episodi dei disastri ambientali.
