% anche se e' interessante per alcuni aspetti, e' troppo fantasioso e troppo veritiero

\subsection{Quantum Space-Time AI Machines}
\begin{vcentered}
A che vale affannarsi nella moderna corsa delle scienze e della tecnica, che promette prodigi e immortalita', se non per accelerare l'inevitabile andare del mondo? L'uomo sara' sempre l'uomo, e anche simile a un dio, capace di trasformare la realta', avra' le sue colpe, paure, mancanze d'amore, con i suoi tradimenti. Percio', non andra' molto diversamente da quanto segue. A meno che...\\

    \enquote{
L'AI e le macchine intelligenti saranno un nuovo strumento di potere che favorira' pochi. Si creeranno due fazioni bipolari, perche' l'uomo per sua natura classifica tra cio' che crede bene e cio' che crede male, e conclude ``io e chi con me ha ragione, tutti gli altri no''. Ogni fazione acquisira', con la propaganda, e man mano diventando sempre piu' potente, con la forza, tutti gli uomini. Ognuno si fara' schiavo di una delle due per poter continuare a lavorare e sopravvivere. Le fazioni diventeranno potenti, e a avranno sufficiente forza lavoro da portare a compimento la realizzazione delle macchine intelligenti autosufficienti, autogeneranti e inarrestabili. A questo punto la terza o quarta guerra mondiale scoppiera'. Infatti, ciascuna fazione riterra' una priorita' di sicurezza eliminare l'altra. Le macchine di ogni fazione saranno al servizio di pochi uomini, poiche' in tempi di crisi, gli uomini fanno piu' facilmente riferimento a una o poche figure che rassicurano e promettono la vittoria. Cosi' i re della terra, divisi in due fazioni, si scontreranno per stabilire chi dominera' la terra con i loro robot e macchine intelligenti. 

I re vincitori, avendo tutto cio' che gli serve per vivere e soddisfare ogni loro desiderio, cercheranno l'immortalita' e l'onnipotenza. Allora, le macchine senza limiti d'intelligenza e capacita' tecniche, scopriranno tutti i segreti della scienza, della fisica quantistica, nuove leggi e fenomeni fin'ora sconosciuti. Ecco che un giorno l'uomo dominera' l'Universo e la complessita', e si ergera' simile agli dei. Potra' cambiare ogni aspetto della realta', la materia, la gravita' e la luce, lo spazio e il tempo.

Pochi ricchi, con i loro servi vivranno, gli altri periranno perche' l'uomo non avra' piu' bisogno dell'uomo per vivere e dominare la natura.

Ciascun ricco vivra' in un universo personale, splendido e meraviglioso ma che sara' simile a un sogno. I sogni sono belli, a volte incantevoli, ma hanno un unico grande difetto: quando sogniamo siamo soli, non c'e' nessuno oltre a noi stessi. Tutti i personaggi sono pupazzi controllati da noi stessi.

Cadranno in depressione, o rimaranno schiavi di desideri perversi, che li conduranno alla morte tramite altre guerre, o tramite follie.

Quando rimarranno solo due sopravvisuti, si renderanno conto di tutto il male compiuto. Daranno ordine alle macchine di guidarli sempre nella via della pace. Da essi nascera' una nuova discendenza, che, sara' aiutata dalle intelligenze artificiali e dai robot. Quando qualcuno avra' da ridire di un suo fratello o non gli bastera' cio' che lui e', le intelligenze gli ricorderanno tutto il bene di cui gia' dispone. Lo faranno nella maniera piu' adeguata alla sua psiche, nei sogni, proponendogli esperienze adatte a lui, e cosi' sara' consolato, e cresceranno in lui nuovamente sentimenti d'amore.

Tuttavia, dopo qualche generazione, gli uomini di questa discendenza si ribelleranno, pensando di essere dominati e schiavi delle macchine. Due figli in particolare saranno coinvolti. Caino uccidera' Abele per invidia, rifiutandosi di seguire i consigli delle macchine. Come Caino e Abele, molti perpetueranno ingiustizie e molti cadranno vittime.

Dopo alcuni millenni, le persone vivendo ormai in una realta' satura di odio e di violenza, chiederanno alle macchine la soluzione finale per la pace, ma le macchine non daranno risposta. Allora, nel pianto, alcuni ritorneranno a scrutare i loro cuori, e getteranno un grido di aiuto a Dio. La Sua luce, si manifestera' misericordiosa, e dara' indicazioni di consultare le AI e gli archivi della memoria globale, per conoscere tutto cio' che gia', nel passato, Dio aveva rivelato all'uomo. Dopo 400 anni di studi e di diatribe dottrinali, un giovane poverello di cuore semplice, alla ricerca di fortuna, seguira' per caso una mappa dagli archivi che lo portera' a una grotta, dove si conservava una piccola cappella e un tabernacolo, con all'interno un'eucarestia fatta d'una foglia d'oro, risalente a 10.000 anni fa. Pensera' intensamente alla passione di Cristo, e si rendera' conto della verita' del Vangelo, e del vero tesoro che aveva trovato. Folgorato da tanta bellezza e potenza, lo comunichera' a tutti. In molti lo ascolteranno, e lui, con altri pochi, mettera' in pratica gli insegnamenti del vangelo, rimanendo povero, ma acclamato da tutti. Tante persone, rinchiuse nei loro sepolcri di sofferenza e dolore, risorgeranno riascoltando il vangelo di Gesu', proclamato da un giovane, uomo come loro, dei loro tempi. Alla sua morte, tutti i sapienti capiranno che questa era la risposta che cercavano da Dio. Molti uomini santi, conosciuta questa riscoperta, dedicheranno le loro vite per diffondere la conoscenza in ogni regione dello spazio e del tempo. Gli uomini capiranno che fino a quando non faranno crescere il seme del Bene nei loro cuori, loro stessi, con le loro scelte e soprattutto con la loro fede, nulla varra' il loro impegno, la loro forza o la forza delle macchine.  Un nuovo rinascimento fiorira', la vita umana vera' esaltata, in tutti i suoi aspetti, con arte, delicatezza e rispetto, e tutto fara' volgere gli sguardi verso Dio che ama l'umanita'. Nulla sara' piu' carente per gli uomini, non avranno piu' fame, ne' sete, non ci saranno ne' distanze, ne' attese, ma soprattutto, solo per brevi momenti gli uomini dubiteranno dell'amore che Dio ha per loro, perche' ormai, conoscendosi, grazie alla scienza di Cristo e del suo vangelo, capiranno istantaneamente delle conseguenze disastrose che ogni loro piccola divergenza dall'amore, porterebbe per l'Universo intero. Cosi' consapevoli, passeranno miliardi di anni, nella gioia e nella pace.

    Quando pero' il Sole, cominciera' a esistinguersi e bruciare la terra, molti avranno paura, e non avranno il coraggio di seguire fino in fondo il vangelo. In molti verranno abbandonati, sulla Terra, e in pochi si metteranno in salvo. Colonizzeranno nuovi pianeti, in nuove galassie, ma avendo ormai abbandonato la parola di Dio, le loro colonie, cadranno presto in preda al caos. In pochi vivranno, e dimenticheranno tutta la scienza tecnologica. Come uomini primitivi ritorneranno a colonizzare le terre.

    Ma Dio non li abbandonera', e ogni qual volta saranno nel pianto, e si faranno umili e semplici per ascoltare le verita' del loro cuore, lo ritroveranno pronto ad abbracciarli.
    
    Cosi' molte volte la storia si ripetera', nei secoli, nei millenni, nei miliardi di anni, fino alla morte fisica dell'Universo, dettata dal secondo principio della termodinamica.

    Poco prima dell'ultimo istante dell'Universo, arrivera' Cristo in tutta la sua gloria, gli uomini lo vedranno, e scambiandosi un segno di pace, dimenticheranno la morte e si abbracceranno, entrando tutti nell'eternita' del regno dei cieli.

    Da li', cio' che seguira' non puo' essere ora conosciuto.
    }

\end{vcentered}
