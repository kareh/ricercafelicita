\chapter{Calabella}

Giovanni, aveva da tre mesi compiuto sei anni e giocava nelle strade di Calabella con i suoi compagnetti. Giocavano ad acchiappa-acchiappa, e si rincorrevano divertiti scorazzando qua e la' nel paese. Tornato a casa, da sua madre che preparava la cena, noto' qualcosa di strano. Sua madre era silenziosa, e rispondeva succintamente quando lui le chiedeva qualcosa. Finita la cena, sua madre disse a Giovanni: \enquote{Giua', figghio mio. Ti debbo dire na cosa} \\
Giovanni: \enquote{Cosa ma'? Cosa m'avesse a dire?}\\
Sua madre comincio' a piangere. Non riusciva a parlare. Giovanni non l'aveva mai vista cosi' abbattuta.\\
Ripresasi un attimo, disse: \enquote{Tuo padre e' morto. Gli e' venuta na' bronchite mentre era in campagna}\\

Da quel momento, la vita di Giovanni cambio'. Non era piu' il bambino spensierato di una volta, adesso era diventato lui l'uomo di casa. Doveva andare a lavorare nei campi come un uomo grande perche' altrimenti lui, sua madre e suo fratello piu' piccolo non avrebbero avuto di che mangiare. Questo perche' in quel tempo le donne non lavoravano. Veniva pagato due soldi pero', perche' era solo un bambino.\\
Il lunedi' mattina si alzava presto alle 5 per poi andare a piedi in una campagna distante un'ora di cammino. Li' lavorava e badava alle mucche, che erano diventate le sue amiche. Altre volte zappava la terra.\\
  Era molto difficile per un bambino stare tutta una giornata da solo, senza compagnia. Dei contadini che lo pagavano aveva paura. Questi contadini erano stati maltrattati da piccoli perche' erano poveri, erano cresciuti ed erano rimasti poveri, quindi credevano che per questo la gente del paese li considerasse poco e che i loro padroni avessero il diritto di fargli fare lavori pesanti. Insomma, non erano per niente felici e adesso che c'era un bambino ai loro comandi, cercavano di rimediare alla loro infelicita' facendo i padroni bastardi del bambino: appena sbagliava in qualche cosa, anche stupida, lo picchiavano e lo insultavano. Quindi, Giovanni era contento quando la giornata la passava da solo con le mucche.\\
 Alle volte tornava la sera a casa da sua madre, suo immenso conforto e ragione di vita, che piangeva quando sapeva che era stato ferito da quegli uomini che pensando di essere miseri lo diventavano davvero. Sua madre si sentiva anche in colpa di mandare Giovanni a lavorare.\\
 Altre volte, il bambino, calato il sole, dormiva in una casetta in campagna. Li' aveva paura dei serpenti e del buio. Allora, lasciava una candela accesa e pregava Dio fino a che non si addormentava.\\

Un giorno un uomo si propose a sua madre e lei, anche solo per smettere di mandare suo figlio a lavorare, quasi accetto'. Giovanni si oppose con tutte le forze. Allora, sua madre, per non dargli un dispiacere lascio' perdere.\\

I mesi passavano e pian piano in Giovanni si fece largo questa convinzione: il mondo e' un posto terribile, ma Dio mi vuole bene. Diventero' ricco e non soffriro' piu' la fame, il freddo e le tirannie.\\
  Passarono gli anni e ormai Giovanni si era fatto un bel giovanotto. Aveva 18 anni. Ormai veniva pagato meglio. Era passato alla muratura e tutti lo chiamavano perche' sapevano che era bravo e si fidavano di lui.\\

Un giorno ad una festa del paese si innamoro' di una ragazza. La ragazza era figlia di una famiglia benestante, era molto intelligente e carina. Aveva tante sorelle e alcuni fratelli. \\

A quel tempo, se due ragazzi non si sposavano, non potevano frequentarsi. O meglio, anche se erano fidanzati, potevano vedersi ma solo se in compagnia entrambi dai genitori o di almeno due fratelli di ognuno. Allora, Giovanni e Nicoletta, per vedersi si davano degli appuntamenti segreti.\\
 Una volta, passo' da dove erano loro la zia di Nicoletta. Li vide e Nicoletta vide lei. Nicoletta allora ebbe una forte paura. I suoi genitori l'avrebbero rimproverata aspramente e tutto il paese si sarebbe preso gioco di loro. Siccome la zia la voleva bene, tiro' avanti con il suo panaru cantando una canzoncina stupida, facendo la finta scema, come se non avesse visto niente.\\

Passarono dei mesi. Era da un po' che al paese arrivavano dicerie di come si viveva meglio all'estero. Nella capitale dei Galli, le strade erano larghissime, nei bar c'erano sempre uova sode sgusciate e crossaint caldi e burrosi, e nei mercati si trovava ogni bene. Inoltre, i lavoratori venivano pagati molto bene.\\

Questa era l'occasione che Giovanni aspettava. Anche se ormai si era fatto una vita sua in paese, credeva che la vita era troppo misera e che gli uomini non erano destinati a vivere di stenti e di fatiche. Decise di partire con suo fratello piu' piccolo, di 15 anni.\\
Nella terra dei Galli, inizialmente vissero in una bettola e lavoravano sodo. Il clima era molto piu' freddo rispetto al clima di Calabella, pero' i due fratelli non si scoraggiavano. \\
Quando Giovanni ricevette il primo stipendio, ne spedi' meta' a sua madre. Adesso stava guadagnando davvero.\\

Col tempo i capi di lavoro videro che Giovanni diventava sempre piu' bravo e svelto nel lavoro. Giovanni, infatti, cercava sempre di parlare con chi era piu' bravo di lui e lavorando a fianco a loro, imparava tutti i trucchi e le tecniche.\\

Una volta, alcuni colleghi muratori si presero di invidia. Giovanni era un semplice immigrato e in un anno gia' guadagnava piu' di loro. Decisero di fargliela pagare. Giovanni, sereno com'era nel suo lavoro, aveva le orecchie aperte. La mattina, senti' che c'erano dei discorsi strani. Non li capi', pero' poi chiese a un suo amico che volevano dire. Il suo amico, allora gli confido' in segretezza che volevano picchiarlo la sera stessa.\\
Per salvarsi, Giovanni escogito' uno strategemma. Disse al suo capo: stasera finiro' un capolavoro di muratura, vieni a vedermi. Il suo capo rise, ma poi, siccome gli piaceva il ragazzo, decise di andarci.\\

Quella sera c'era freddo, e Giovanni era rimasto nel cantiere. Non aveva smesso di lavorare per un solo istante, tanto che aveva fatto quello che avrebbe fatto in due giorni. A un certo punto' senti' degli insulti, poi delle voci sempre piu' forti. Erano venuti quegli stupidi a volersi riscattare, affossando un uomo piu' bravo di loro. Giovanni, mantenne il sangue freddo e continuo' a lavorare. Proprio in quel momento, venne il capo di Giovanni.\\
Sentendo quelle voci, si mise in una parte riparata del cantiere, ma in modo che potesse vedere quello che stava succedendo. I muratori, si avvicinarono a Giovanni e senza troppi convenevoli cominciarano a picchiarlo. Il capo allora, capita la situazione, ando' da loro e grido': basta, disgraziati. Con me farete i conti domani.\\
Cosi', Giovanni si salvo', e due dei muratori vennero licenziati.\\

Quando Giovanni tornava a Calabella, si incontrava con la sua ragazza. Dopo qualche anno, i due si sposarono. Un anno dopo, Nicoletta era incinta. Innamorata di Giovanni, decise di emigrare con lui e stabilirsi nella terra dei Galli. Nacque Teresa. Una bella bambina dai capelli ricci. \\
  Nicoletta dovette superare non poche difficolta'. Infatti, non conosceva la lingua dei Galli e, soprattutto, non conosceva nessuno li'. Non era come nel paese dove c'erano i suoi fratelli e sorelle e gli zii e i nonni che l'aiutavano. Quindi, partorire per lei fu una difficile avventura. \\

A questo punto Giovanni, con i risparmi che aveva messo da parte, viveva in una bella casa in affitto. Amava sua figlia che era uno splendore, e sua moglie lo seguiva e lo aiutava nelle difficolta' quotidiane e anche negli affari. Si, negli affari, perche' Giovanni con un suo amico aveva comprato un terreno e stava costruendo e vendendo una casa di suo pugno, e stava cosi' guadagnando parecchio.\\

Tuttavia, Giovanni era cresciuto solo fin da bambino, senza nessuno che gli spiegasse come funzionava il mondo e lo seguisse nella sua crescita. E si sentiva ancora solo, con nessuno che l'aiutasse.\\
Gli amici non erano abbastanza fedeli. Ad esempio, Giovanni voleva costruire una villa in un terreno che considerava favorevole. Il suo socio, inizialmente volle partecipare. E Giovanni inizio' a darsi gran verso per far andare bene il lavoro di questa villa. Poi il suo socio' volle smettere e non investi' piu' nell'affare.\\
Successe una cosa simile con un altro amico. Voleva spostarsi di paese e consiglio' all'amico di venire con lui. Nel nuovo paese avrebbero potuto fare buoni affari, a patto che l'amico seguisse le sue indicazioni su quale casa affittarsi e che accettasse il compenso che Giovanni voleva riservargli. All'inizio l'amico penso' che fosse una buona offerta, ma poi decise di andare altrove.\\
Rimase con questa sensazione fino a quando sua figlia compi' la maggiore eta', ovvero 14 anni dopo. In quell'anno, il suo secondo figlio, Giulio, compi' 13 anni. Esattamente un mese dopo, prese una valigia e parti' verso paesi remoti, augurando buona fortuna a Giulio.\\

Nessuno ebbe molte notizie di Giovanni. La gente inizialmente penso' che nei paesi remoti era felice perche' c'erano tante persone povere e lui, che era stato povero, si trovava a suo agio. C'erano anche tante belle donne in quei paesi e la gente penso' anche maliziosamente che, convertendosi alla religione locale, si sposo' e si risposo' numerose volte.\\

Nicoletta, rimase sola, ma forte d'animo si diede da fare. Si concentrava nella cucina ed era molto brava. Con alimenti semplici e naturali, preparava piatti saporiti e abbondanti. \\
Nella citta' fece alcune amicizie che rimasero per lungo tempo. Quando si incontrava con le amiche, si raccontavano le proprie storie, le difficolta' che avevano con i mariti e le difficolta' che avevano nell'essere donne nella societa'. Chi con il marito, chi sola, chi vedova, ognuna aveva le proprie difficolta'. Parlando non si sentivano sole ad affrontarle.\\

Giovanni nel suo peregrinare scriveva piccole poesie ispirato dalla Natura e ascoltava tante musiche nella radio. Vedeva posti e paesaggi dove la Natura si manifestava in tutto il suo splendore. Mangiava e beveva cibi sempre nuovi e gustosi della miglior qualita'. Ad esempio, quando si trovava in zone di mare, andava a parlare con i pescatori e, dopo aver parlato a lungo, comprava il pesce appena pescato. Ricercava e comprava tesori particolari. Ad esempio, una volta compro' un cavallo bianco puro sangue. Con il cavallo ando' sulle montagne dell'Anatolia e compro' da un vecchio eremita un piatto sacro d'oro decorato.\\

Dopo 10 anni, Giovanni continuava a sentirsi solo. Le persone attorno a lui erano povere, e lui, adesso che era ricco, non poteva ricevere aiuto da loro. Cosi', ricordandosi dei bei momenti trascorsi da Nicoletta, ritorno' in Europa. Sperando che lei lo perdonasse.\\

Quando si presento' alla porta di Nicoletta, le disse: \enquote{In questi anni ho provato tutti i piaceri della vita. Pero' ho continuato ad affrontare le difficolta', come alcuni briganti, da solo. So' che tu mi hai sempre rispettato e voluto bene. Mi dispiace che ti ho lasciato sola, ma come tu sai io sono cresciuto fin da piccolo senza nessuno.}\\

Nicoletta le disse: \enquote{Prima te ne sei andato, e ora mi dici che hai capito che ti volevo bene. Hai proprio una bella faccia. Credi forse che tu sei il centro dell'universo e la Luna e il Sole sorgono e tramontano per vedere sorridere te, mentre lasciano nell'ombra gli altri che hai abbandonato a se stessi?} \\
\enquote{Non credi forse che in questa terra, gli astri splendono anche per illuminare i volti delle altre persone? E che le persone hanno i loro dolori e hanno sogni da voler realizzare come te?}\\
Non mi e' piu' gradita la tua presenza.\\

Giovanni si rabbuio'. Prese la sua chevrolet e ando' via.\\

Giovanni ricomincio' a girovagare, non pensando pero' piu' a nessuno: ne' a sua moglie, ne' ai suoi figli, ne' ai suoi fratelli, ne' ai suoi parenti. E comincio' a spendere tanto. Ogni tre giorni cambiava paese. Di giorno camminava da solo tra le campagne con i vestiti buoni. E di sera andava nei ristoranti con piu' stelle.\\
Leggeva poeti decadenti e filosofi che dicevano: \enquote{Il vero uomo e' colui che rimane nelle sue idee, non si piega e che nonostante tutti non lo capiscano e nonostante con forza cerchi di convincerli, non si arrende e rimane beato nella sua solitudine.}\\

Dopo 6 mesi, giunse nelle suo camminare presso una vecchia casa di campagna. La casa, per quanto normale e non lussuosa, dava un'impressione piacevole. Poi, il pozzo col secchio che penzolava e la corda avvolta in spire regolari. La zappa e gli attrezzi posati su muro in fila e dritti. Un orticello di zucchine, patate, fagioli. Un albero di fichi.\\
Giovanni era stato da ragazzo nel paese al quale apparteva questa campagna. In questo paese, aveva abitato un suo zio. Forse in una casa come questa.\\
Elevato dall'atmosfera in cui si trovava, Giovanni si avvicino' e disse: \enquote{C'e' nessuno?}\\

Dalla casa usci' un vecchietto. Lui rispose: \enquote{Prego, accomodati}\\
In quella regione povera, non era costume ricevere una tale ospitalita', da un estraneo, in maniera cosi' diretta. Tuttavia, il modo, per quanto povero, in cui il vecchio aveva parlato, non mostrava alcun segno di imposizione. Era un'offerta, piuttosto che un ordine.\\
Giovanni entro'.\\
E disse: \enquote{Da quanto tempo vivi qui?}\\
\enquote{Da abbastanza. Da quando Tino e Roberta erano piccoli.}\\
\enquote{Chi erano, i tuoi figli?}\\
\enquote{No.}\\
\enquote{E tua moglie?}\\
Il vecchietto sorrise.\\
\enquote{E' morta?}\\
\enquote{Non ho mai avuto una moglie.}\\

Il vecchio si alzo', e prese una caraffa con dell'acqua fredda con lo zenzero.\\
I due bevvero. La bevanda li' desto' dal torpore estivo.\\

E' ricco e si e' goduto la vita, penso' Giovanni.\\
\enquote{Cosa hai visto nel mondo?} gli chiese.\\
\enquote{I macchinari della fabbrica dove ho lavorato. Alcuni disegni di due ingegnieri quando sono venuti in fabbrica. La villa del paese. Il mio orto}\\
Giovanni rimase sconcertato. Pensava che un uomo cosi' tranquillo e disponibile, non poteva che essere ricco. Invece, era stato un operaio e non aveva visto nella sua vita che il suo piccolo paese.\\
Quasi si penti' di essere entrato in quella casa.\\
Sorseggio l'ultimo sorso della bevanda e poi disse: \enquote{Io sono nato povero, e' stato difficile essere bambino per me. Poi, me la sono cavata e grazie alle mie innate doti e fortuna, sono ora ricco}\\
Il vecchio non rispose.\\
Giovanni insistette: \enquote{Io ho cominciato a comprare e vendere case e ville. Ho fatto grandi affari. Ho girato il mondo.}\\
Giovanni poi disse com'erano belli i posti che aveva visitato.\\
Il vecchio, non colpito, sorrise gentilmente. Non provava ne' entusiasmo ne' sentimenti negativi per quello che Giovanni aveva detto. Poi, visto che Giovanni non diceva piu' niente, guardo' fuori. Decise di alzarsi e si avvio' verso la cucina. \enquote{E' gia' un po' tardi. C'e' da cucinare.}\\

Giovanni, non capiva quali sentimenti stesse provando. Come poteva il vecchio non avere alcun desiderio per quello che Giovanni aveva avuto nella vita? Questa cosa lo turbava. E vari pensieri negativi si affollavano nella sua testa. Pensava che il vecchio era uno stupido. Che non capiva le cose di valore.\\

Il vecchio disse: \enquote{Se apparecchi la tavola e poi lavi le stoviglie, ti ricompensero' con un sacco di patate.}\\
Giovanni si mise a ridere. Si alzo' e avviandosi verso la porta disse: \enquote{Ho di meglio nel mio albergo. E se tu lavorassi per me come muratore, ti pagherei veramente. Non con patate. Grazie del pensiero, ad ogni modo}\\

Il vecchio si chiamava Agostino.\\

Giovanni, dopo quell'incontro smise di camminare in campagna tutto il giorno. E stette di piu' in paese. Li' guardava in una villa dei bambini giocare.\\

Una notte fece un sogno particolare. Si trovava nell'orto di Agostino. Tutto era deserto. Un deserto che si estendeva per chilometri e chilometri. Un paesaggio tetro e triste, scuro, un silenzio definitivo.\\
Poi noto' una pila di patate ammucchiate in un angolo dell'orto. Si avvicino' e le osservo'. Erano patate. Poi ne prese una la sbuccio'. La mise sopra una roccia. Poi continuo' cosi' anche con le altre patate.\\
Quando poso' l'ultima patata, si senti' meglio.\\
Riposandosi, poi decise di andare. Si volto' un attimo indietro per guardare le patate. Luccicavano di un colore intenso. Poi una folata di vento ne fece cadere una. Urto' su una roccia e udi' un rumore metallico.\\
Capi' che si erano trasformate in oro.\\
Si sveglio'.\\

La mattina seguente, tanto strano era stato il sogno, tanto strano fu' il comportamento di Giovanni. Giovanni penso': \enquote{Agostino mi sta' nascondendo qualcosa. Sicuramente e' ricco. Mi vuole anche bene, quindi, se lavoro da lui avro' una bella ricompensa}. E con questi pensieri si reco' da Agostino e gli propose di lavorare un po' da lui.\\

Stette un mese. Zappava l'orto. Faceva compere. Lavava i piatti.\\
Di tanto in tanto veniva qualche persona a trovare Agostino. Le persone si trovano a proprio agio con Agostino e gli parlava delle proprie cose. Agostino parlava con loro.\\
Poi, Agostino ordinava a Giovanni di fare alcune commissioni per queste persone: andare a comprargli un libro in libreria, aggiustare un loro paio di scarpe, o comprargli del pesce.\\
Giovanni non capiva il perche' di queste cose. Non ci guadagnava niente Agostino. Ne' Agostino era religioso e sperava nel paradiso per le sue buone azioni.\\

Alla fine del mese, Giovanni ripenso' alla sua ricompensa. Ormai si era convinto che Agostino avesse qualcosa di particolare, e pensava che molto probabilmente avesse ereditato da un suo nonno un grosso patrimonio.\\
Agostino non aveva studiato, pero' aveva una sua libreria con una decina di libri. Tra questi, si potevano notare un grosso libro su tecniche agricole, un romanzo di Oscar Wilde, un saggio di fisica e uno di psicologia.\\
Non si lamentava mai. Ne' quando batteva un ginocchio per sbaglio su un gambo del tavolo, ne' quando, raramente, una persona che veniva a trovarlo gridava perche' quello che Agostino gli aveva detto la volta precedente si era rivelato non fruttuoso.\\
Inoltre, Giovanni si trovava meglio rispetto a prima, che girovagava ogni giorno per la campagna senza sapere dove andare.\\
Agostino, vedendo che Giovanni stava tenendo in mano le chiavi della sua Chevrolet, gli disse: \enquote{
E' trascorso un mese. Avevo intuito che avevi deciso di rimanere solo un po' di tempo. Se pero' ti senti, puoi rimanere quanto vuoi.
}
Cosi', Giovanni decise accetto l'invito e stette ancora.\\

10 anni dopo, Giovanni disse ad Agostino: \enquote{vorrei il sacco di patate che mi avevi promesso quando ci siamo conosciuti}.\\
Agostino, rimase calmo, continuo' con quello che stava facendo, poi mise un punto e disse: \enquote{aspetta, lo vado a prendere}. E ritorno' con un sacco di patate.\\
Giovanni, lo ringrazio' cortesemente, prese il sacco, non guardo' neppure il suo interno, e sapendo che Agostino gia' sapeva, prese le chiavi della sua Chevrolet, e parti'.\\

Giovanni, arrivato a destinazione, vendette la sua auto e compro' un'Ape e compro' anche diversi cesti di frutta. Allesti' l'Ape con tutti questi cesti in maniera aggraziata e rimase in piedi.\\

Dopo un anno, era ora un fruttivendolo conosciuto nel paese.\\
Vendeva frutta normale e verdura normale, di buona qualita'. Tuttavia, metteva una notevole attenzione nella scelta della mercanzia dai fornitori. Nei giorni prima dell'acquisto, nelle sue passeggiate all'alba e alla sera, ripensava a quello che si era detto con alcuni clienti. Un signore aveva detto che alcune mele erano risultate troppo mature. Una signora che le era piaciuta una mela piuttosto acerba.

Dopo tutti gli anni passati a servire Agostino, queste osservazioni, per quanto banali, provocavano ora un'orchestra di sentimenti e pensieri in Giovanni.

Allora, Giovanni assaggiava a campione le sue mele e studiava su alcuni libri di gastronomia biologica. Non si limitava a leggere e scegliere una verita' piuttosto che un'altra. Lasciava che quello che leggeva rimanesse in lui a riposare.
Infine, in biblioteca, leggeva a grandi linee alcuni dei libri che sapeva erano piaciuti ai suoi clienti.

Giorni dopo, questi studi germogliavano in lui sotto forma di nuove idee: ``le mele di tipo X sono piu' adatte nella stagione Y per chi sta' passando dei momenti stressanti col proprio partner''.

Cosi', quando si trovava dai fornitori a comprare la sua merce, sapeva bene cosa scegliere.

Quando, arrivava un cliente, era sereno e disponibile. Non pensava a vendere, perche' gia', con quello che aveva messo da parte quando era imprenditore edile, aveva una sufficiente pensione. Ne' voleva parlare dei suoi studi.\\
Aspettava che il cliente guardasse tranquillo la merce. Se il cliente prendeva troppo tempo, continuava a sbrigare delle faccende non finite: sistemava alcune casse di merce ad esempio. Poi, quando il cliente si era deciso, valutava la sua scelta.

Non sempre era d'accordo con quello che sceglieva il cliente. Ma era sempre gentile e non contraddiceva la volonta' dell'altro. Tuttavia, non abbandonava quello che credeva, e con abilita' quasi filosofica discuteva con il cliente.

Andava poi a finire che il cliente discuteva anche di altre cose: di quanto gli sembrava poco gratificante per lui stare con la moglie che si lamentava sempre, di quanto era stancante badare ai due figli, etc...

Quando finivano di discutere, il cliente poi era in grado di scegliere con decisione la sua frutta e verdura. E il piu' delle volte concordava con quello che aveva pensato Giovanni.

Infine, la mattina, quando si svegliava, andava a curare il suo orto, dove aveva inizialmente piantato le patate che gli aveva dato Agostino.\\

Un giorno, noto' una donna che lo osservava mentre lavorava. Il terzo giorno di fila, la invito' ad andare con lui al mercato ortofrutticolo.

La donna non parlo' mai. Non pronuncio' nemmeno un suono. Era gentile e aggraziata e teneva sempre un velo, da dove non si vedeva chi fosse.\\
La donna lo segui' durante la giornata, e poi la sera, quando Giovanni stava per recarsi a casa per cenare e riposare,  fece cenno a qualcuno dietro un'angolo. Subito dopo sbuco' un piccolo ragazzo dell'eta di 13 anni.\\
Giovanni lo saluto' e si incammino' verso casa, il ragazzino lo seguiva.\\
Poi, Giovanni voltandosi disse: \enquote{mi dispiace che ti ho lasciato sola tutti questi anni. E ti ringrazio per darmi la fiducia di insegnare a nostro nipote}.\\
Lei tolse il velo. Era Nicoletta. Sorridendo disse: \enquote{sei cambiato. Mi fa piacere.}\\

Dopo un mese, il ragazzo chiese a suo nonno: perche' lavori cosi' tanto per vendere della semplice frutta e verdura?\\
Lui rispose: \enquote{mangiare non significa solo masticare e ingoiare, e vendere del mangiare non significa solo scambiare chili in euro.}\\
Poi aggiunse: \enquote{ho imparato ad apprezzare le cose semplici e fatte bene. Anche i miei clienti le apprezzano: quando vanno da un fruttivendolo, chiedono la frutta e portano a casa della frutta. Quando vengono da me, parlano di quello che gli piace e non gli piace, e se ne vanno con qualcosa che gli piace. Anche se poi molti se ne dimenticano subito dopo averle consumate. Ma se Dio mi ha ordinato di vivere in questa terra per servire i suoi uomini cosi', io ubbidisco.}\\

Nicoletta ritorno' a prendere il ragazzo. E poi, ogni mese ando' a trovare Giovanni. Giovanni, per lei, conservava gli ortaggi migliori che raccoglieva dall'orto.\\

Una mattina Giovanni trovo' una patata strana. Poi, a casa, separandola dalle altre, la mise a bollire. Quando, la tiro' fuori, con meraviglia vide che era d'oro.

Prese la patata e la mise in un forno per molte ore. Il fuoco era alimentato da molta legna e carbone.
Cosi', fuse la patata, e ricavo' due orecchini per Nicoletta e una chiave, con relativi doppioni, che sostituiva quella che usava per guidare la sua Ape da fruttivendolo.\\

Nicoletta, si commosse quando ricevette gli orecchini. Non perche' erano d'oro, ma perche' Giovanni non le aveva mai fatto dei regali che per lui erano importanti.\\

Da quel giorno, Nicoletta torno' a vivere con Giovanni. Poi, pian piano, Giovanni e Nicoletta vendettero tutti i loro capitali, tranne gli orecchini e la chiave. Con i ricavati della vendita, finanziarono segretamente, una associazione culturale per il paese in cui abitavano.

L'associazione nasceva dall'amara constatazione di un genitore che gli aveva detto che ormai i bambini e i ragazzi stavano sempre al computer o alla televisione, e neanche parlavano piu' tra di loro, ne' ascoltavano piu' i grandi.

In questa associazione, venivano proiettati i cortometraggi piu' belli di Youtube, selezionati da varie associazioni importanti nel mondo. Venivano consigliati e dati in prestito libri d'autore. E venivano finanziati, dove possibile, dei volontari che facevano doposcuola ai ragazzi e li seguivano nello studio.

I soldi, poi, Nicoletta e Giovanni, li spendavano, per dare, tramite l'associazione, borse di studio ai ragazzi che volevano continuare con l'universita'.\\

Cosi' Giovanni, non ebbe alcun problema con la sua chiave d'oro e con gli orecchini di Nicoletta: nessun ladro penso' mai di trovare alcunche' nella casa di un fruttivendolo, e gli hacker, quando entravano nel suo conto-corrente, non trovavano nulla, dato che era stato speso tutto per l'associazione del paese.\\

Loro nipote andava bene a scuola, usciva sempre con i compagni ed era felice. Quindi, erano sicuri che non avrebbe avuto problemi nella vita.

Decisero di lasciargli in eredita' i libri che leggeva Giovanni e le ricette di cucina di Nicoletta. E pregarano ogni sera per lui.\\

Cosi', Giovanni e Nicoletta vissero a lungo, felici.
